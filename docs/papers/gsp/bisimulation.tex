% !TEX root = main.tex

\section{Correctness of the Implementation}
\label{sec:simulation}

We now prove that  \igsp\ is a correct implementation 
of \gsp. We recall in \figref{fig:coherence-operation} the requirements stated in~\cite{DBLP:conf/ecoop/BurckhardtLPF15}
for the operations provided by the data types 
\statetype\ and \deltatype. 
Formally,  the relation $\_\triangleleft \_$ associates 
 delta and state objects with sequences of updates: $\adelta\triangleleft\anupdseq$
 (similarly, $\astate\triangleleft\anupdseq$)  means  that $\delta$ (correspondingly, $\astate$) 
 is a compact representation of $\anupdseq$. Then, it is also assumed that 
$\astate\triangleleft\anupdseq$ implies  $\iread \aread \astate  = \rvalue \aread  \anupdseq$ for any $\aread$.
Building on the above relation, we define under which conditions a
\igsp\ system is an implementation of a \gsp\ system.

\begin{figure}[t]
{\small
\[
\begin{array}{l@{\quad\qquad}l@{\quad\qquad}l}
  \mathax{\triangemptydelta\ \ }{\ \ \emptydelta \triangleleft \epsilon}{} 
  &
  \mathrule{\triangappend}
  		{\adelta \triangleleft \anupdseq}
		{\iappend \adelta \anupd  \triangleleft \anupdseq \cdot \tupdate}
  &
 %  \mathrule{\triangreduce}
 % 		{\adelta[_1] \triangleleft \anupdseq[_1] \ldots \adelta[_n] \triangleleft \anupdseq[_n]}
 %		{\ireduce {\adelta[_1] \cdots \adelta[_n]} \triangleleft \anupdseq[_1] \cdots \anupdseq[_n]}
  \mathrule{\triangread}
  		{\astate \triangleleft \anupdseq}
		{\iread r \astate  = \rvalue r  \anupdseq}  
\\[15pt]
  \mathax{\trianginitialstate\ \ }{\ \ \initialstate \triangleleft \epsilon}{} 
  &
  \mathrule{\triangapply}
  		{\astate \triangleleft \anupdseq \quad \adelta[_1] \triangleleft \anupdseq[_1] \ldots \adelta[_n] \triangleleft \anupdseq[_n]}
		{\iapply \astate {\adelta[_1]\cdots\adelta[_n]}  \triangleleft \anupdseq \cdot\anupdseq[_1] \cdots \anupdseq[_n]}
  & 

   \mathrule{\triangreduce}
   		{\adelta[_1] \triangleleft \anupdseq[_1] \ldots \adelta[_n] \triangleleft \anupdseq[_n]}
		{\ireduce {\adelta[_1] \cdots \adelta[_n]} \triangleleft \anupdseq[_1] \cdots \anupdseq[_n]}
%\\
%&
%\mathax{\triangapplyemp\ \ }{\ \ \iapply \astate {\emptydelta}  \triangleleft \astate}{} 
  
%&
%\mathax{\triangreduceemp\ \ }{\ \  \ireduce{\epsilon} \triangleleft \emptydelta}{} 
  
%  \mathrule{\triangread}
 % 		{\astate \triangleleft \anupdseq}
%		{\iread r \astate  = \rvalue r  \anupdseq}  

%\\
%  \multicolumn{3}{c}{\mathrule{\triangremove}
%  		{\adelta[_1] \triangleleft \anupdseq[_1]\ \ldots\ \adelta[_n] \triangleleft \anupdseq[_n]}
%		{\iremove{\adelta[_1] \cdots \adelta[_n]}{\adelta[_n]} \triangleleft \anupdseq[_1] \cdots \anupdseq[_{n-1}]}  }
\end{array}
\]
}
\caption{Coherence requirements for $\deltatype$ and $\statetype$ operators}
\label{fig:coherence-operation}
\end{figure}

%\begin{itemize}
%	\item On $\mathit{Delta} \times \mathit{Update}^*$, let $\triangleleft$ be the smallest relation such that (1) $\emptydelta \triangleleft []$, and
%(2) $d \triangleleft a$ implies $append(d,u) \triangleleft a \cdot u$ for all updates $u$, and (3) $d_1 \triangleleft a_1 \ldots d_n \triangleleft a_n$ implies $reduce(d_1 \ldots d_n) \triangleleft a_1 \cdots a_n$ and (4) $d_1 \triangleleft a_1 \ldots d_n \triangleleft a_n$ implies $\remove{d_1 \ldots d_{n}}{d_n} \triangleleft a_1 \cdots a_{n-1}$. \marginpar{El (4) lo agregamos nosotros. ¿Se aclara? ¿C\'omo?}
%  
%\item On $State \times Update^*$, let $\triangleleft$ be the smallest relation such that (1) $\initialstate \triangleleft []$, and
%(2) $s \triangleleft a \land \ d_1 \triangleleft a_1 \land \ \ldots \ \land \ d_n \triangleleft a_n$ implies $apply(s,d_1 \ldots d_n) \triangleleft a \cdot a_1 \cdots a_n$.
%\end{itemize}
%
% 


\begin{definition} \label{def:implementation}
Let  $\systemterm = \abstsyst[@][@][\Absclient_j]$ be a \tgspcalculus\ system
 such that $\Absclient_l={\abstcliJ[@][m]}$ for all
$l\in\{0,\ldots,m\}$, and  $\isystemterm = \concsyst$  a \igsp\  system such that  $\iserver = \iserverins[@][\amxrf[_s]]$ and $\iclient_l = \addid{\iclientinstJ}[{\cid{_l}}]$.
 We say {\em $\isystemterm$ implements $\systemterm$} if the following conditions hold:

\begin{enumerate}
	\item \label{prop_stateserver} $\astate \triangleleft \flatten{\queuemessage}$;

	\item \label{prop_state_known} $\astate_l \triangleleft \flatten {\queuemessage[0 .. \tknown_l - 1]}$;
	
	\item \label{prop_transactions} ${\transactionbuffer}_l\triangleleft{\ttransactionbuffer}_l$;
	
	\item \label{prop_pending}  
		$\ireduce{\igetdeltas{\pending_l}\cdot {\pushbuffer}_l} \triangleleft \flatten{\tpending_l}$;

%		\item \label{prop_pending} $\pending_l = \aroundtuplei[1]\ldots\aroundtuplei[h]$ and 
%		$\ireduce{\adelta_1\cdots\adelta_h\cdot {\pushbuffer}_l} \triangleleft \flatten{\tpending_l}$.

%	\item \label{prop_sent} If $\cid_l \in \dom\inserver$ and $\cid_l \in \dom\outserver$ then ${\pushbuffer}_l \ \triangleleft \ {\tsent}_t$  $\tsent_l$ = ${\tsent}_h \cdot {\tsent}_t$ and and $ \ireduce{\igetdeltas{\inserver(l)}} \triangleleft \ {\tsent}_h$
	\item \label{prop_sent}  
%	\begin{itemize}
%	   \item   
	   
	   if $\cid_l \in \dom\inserver$ and $\inclient \cdot\outserver(\cid_l) \neq \agssegpair[{\astate[']}][{\amxrf}]\cdot\aseqseg$ %, $\cid_l \in \dom\outserver$ 
	   then 
	   
	   $ \ireduce{\igetdeltas{\inserver(\cid_l)}\cdot {\pushbuffer}_l} \triangleleft \ \flatten{{\tsent}_l}$;
	 
%	 Otherwise,
%	     %($\cid_l \not\in\dom\inserver$), 
%	      ${\pushbuffer}_l \triangleleft \ \flatten{{\tsent}_l}$.
%			\henote{esto no vale: si partis con un sistema en donde vale la anterior y hay un drop, no vale ni la anterior ni esta}

%	\end{itemize}
%	\item \label{prop_sent}  
%	\begin{itemize}
%	   \item   $\cid_l \in \dom\inserver$%, $\cid_l \in \dom\outserver$ 
%	   and
%	 $ \ireduce{\igetdeltas{\inserver(\cid_l)}\cdot {\pushbuffer}_l} \triangleleft \ \flatten{{\tsent}_l}$; or
%	   \item  $\cid_l \not\in\dom\inserver,  {\pushbuffer}_l \triangleleft \ \flatten{{\tsent}_l}$
%			\henote{esto no vale}
%
%	\end{itemize}
%	


%	\item  \label{prop_size_buffersent} $m_l = n_l$ 
	
\item \label{prop_inclient} if $\cid_l \in \dom{\outserver}$ then either

    \begin{enumerate}[i.]
   % \item  ${\inclient}_l \cdot \outserver(\cid_l)  = \agssegpair[{\astate[']}]\cdot\agssegpairi[1] \cdots \agssegpairi[h]$
    % and  
    % $\iapply{\astate[']}{\ireduce{\adelta[_1]\cdots\adelta[_{h}]}}\triangleleft \flatten{\queuemessage} )$; or 
     \item ${\inclient}_l \cdot \outserver(\cid_l)  = \epsilon$, $\tknown_l = |\queuemessage|$; % and $\treceivebuffer_l =0$;

     \item ${\inclient}_l = \agsprefpair[{\adelta[']}]\cdot\aseqseg$,
      	$\ireduce{\igetdeltas{{\inclient}_l}}\triangleleft \flatten{ \queuemessage[\tknown_l..\tknown_l+\treceivebuffer_l-1]}$, and
      
     	 $\ireduce{\igetdeltas{\outserver(\cid_l)}}\triangleleft 
      	 \flatten{\queuemessage[\tknown_l+\treceivebuffer_l..|\queuemessage| - 1]}$;
  
     \item ${\inclient}_l = \agssegpair[{\astate[']}][{\amxrf[_0]}]\cdot\aseqseg$,      
  		 there exists $t$ s.t. $\tknown_l\leq t \leq \tknown_l+\treceivebuffer_l$ s.t. $\astate['] \triangleleft \ \flatten{\queuemessage[0 .. t - 1]}$, 
    
    		 $\ireduce{\igetdeltas{\aseqseg}}\triangleleft \ \flatten{ \queuemessage[t..\tknown_l+\treceivebuffer_l-1]}$ and 
     
     		$\ireduce{\igetdeltas{\outserver(\cid_l) }}\triangleleft \
       				\flatten{\queuemessage[\tknown_l+\treceivebuffer_l..|\queuemessage| - 1]}$; or
      
     \item ${\inclient}_l = \epsilon$,  $\outserver(\cid_l) = \agssegpair[{\astate[']}]\cdot \aseqseg$ there exists $t \geq \tknown_l+\treceivebuffer_l $ s.t.
		      $\astate['] \triangleleft \ \flatten{\queuemessage[0 .. t - 1]}$ %and $\tknown_l+\treceivebuffer_l \leq t$; 
     
       		$\ireduce{\igetdeltas{\aseqseg}}\triangleleft \ \flatten{ \queuemessage[t..|\queuemessage| - 1]}$;
    \end{enumerate}	
	
        \item \label{prop_deltas} for all $\amxrf$ s.t $\amxrf = \amxrf[_s]$ or $\agssegpair[\_]\in\inclient_l\cdot\outserver(\cid_l)$, %          \begin{enumerate}[i.]
   %       	\item $\amxrf(\cid_l)\leq\irounds_l$,
		%\item  
		for all $\aroundtuple[@][@][\adelta'] \in\pending_l$
        if $\irounds\leq\amxrf(\cid_l)$ then $\adelta\triangleleft \flatten{\queuemessage[x..x']}$ and  $\adelta'\triangleleft \flatten{\queuemessage[y..y']}$
        with $y'\leq x'$.
    \chnote{aca nos dijeron que el guion es un typo y deberia ser delta, en realidad es delta o state.}
	%\end{enumerate} 
%       \item \label{prop_desc} if $\cid_l \not\in \dom{\outserver}$ then for all $\aroundtuple[@][@][\adelta'] \in\pending_l$
%        if $\irounds\leq\amxrf[_s](\cid_l)$ then $\adelta\triangleleft \flatten{\queuemessage[x..x']}$ and  $\adelta'\triangleleft \flatten{\queuemessage[y..y']}$
%        with $y'\leq x'$

\end{enumerate}
	
\end{definition}


The  first three conditions are self-explanatory. Condition (\ref{prop_pending})  states that the pending blocks in $\flatten{\tpending_l}$  
correspond either to  rounds in the  pending list  $\pending_l$ or  to blocks ready to 
be sent, \ie, in ${\pushbuffer}_l$.
By condition (\ref{prop_sent}), if a client is synchronised with the store 
(\ie, $\cid_l \in \dom\inserver$ and $\inclient \cdot\outserver(\cid_l) \neq \agssegpair[{\astate[']}][{\amxrf}]\cdot\aseqseg$) 
then  all  blocks  in  the sending list $\tsent_l$  are either rounds 
that have been sent, i.e., in $\inserver(\cid_l)$, or ready blocks in ${\pushbuffer}_l$.
Condition (\ref{prop_inclient}) establishes the relation between the received messages in both models. Basically, 
the local replica 
is complete  when there are no segments for the client ($i$). %${\inclient}_l \cdot \outserver(\cid_l)  = \epsilon$). 
When the first received segment is a delta object ($ii$),
 the  content in the input buffer $\inclient_l$ corresponds to the received messages in $\queuemessage[\tknown_l..\tknown_l+\treceivebuffer_l-1]$
and the the  output buffer $\outserver(\cid_l)$ contains the  updates in the 
sequence $\queuemessage[\tknown_l+\treceivebuffer_l..|\queuemessage| - 1]$. 
In the remaining two  cases, the first segment  contains a state. 
When the segment is  
in the input buffer of the client ($iii$), 
the received state $\astate[']$ corresponds to a prefix of the sequence $\queuemessage$ 
whose length lies in between of the updates already received by the client in the 
idealised model, i.e., 
$\queuemessage[0 .. t - 1]$ with  $\tknown\leq t \leq \tknown_l+\treceivebuffer_l$, while the remaining
conditions are analogous to the previous case.  Differently, when the 
first segment is still on the output buffer of the store ($iv$),  $\astate[']$  corresponds to a
 prefix that contains at least all  updates in $\queuemessage$ already known to the client, 
 \ie, $t \geq \tknown_l+\treceivebuffer_l $.    
 
 Condition (\ref{prop_deltas}) states that in  any segment $\agssegpair$  sent by the 
 store, $\adelta$ corresponds to a contiguous sequence of updates in $\queuemessage$, i.e., $\queuemessage[x..x']$. 
 Moreover, all confirmed rounds are also within the prefix $\queuemessage[0..x']$.


We  now show that \igsp\ is a correct implementation of \gsp\ by proving that 
%
%all behaviours of $\isystemterm$  are allowed by $\systemterm$ 
%whenever $\isystemterm$  implements $\systemterm$. Technically,  we prove that 
%
$\systemterm$ weakly simulates $\isystemterm$ 
when
 $\isystemterm$  implements $\systemterm$.
 We use standard simulation %notion of simulation 
 but technically we take into account the fact that
 \gsp\ associates a fresh event identifier to each update while \igsp\ does not. Take  $\rightarrow = \tr{\tau} \bigcup_{\cid\in\idset} \arroi{\tau}$,
 $\Arro$ as the reflexive and transitive closure of $\rightarrow$, \ie $\Arro = \rightarrow^*$, and 
 $\Arroi{\lambda} \ = \ \Arro;\arroi{\lambda};\Arro$.
 
 

\begin{definition}[Simulation] $\mathcal{R}$ is an {\em implementation simulation} if for all 
$(\isystemterm,\systemterm)\in\mathcal{R}$ we have:
\begin{enumerate}
   \item If $\isystemterm\arro{\updatetran{\anupd}}\isystemterm[']$ then  $\exists\systemterm['],\vertice[w]$ s.t.  $\systemterm\Arroi{\updatetranaux{\anupd^{\vertice[w][]}}}\systemterm[']$ and $(\isystemterm['],\systemterm['])\in\mathcal{R}$;
   \item If $\isystemterm\arro{\lambda}\isystemterm[']$ and $\lambda\neq \updatetran{\anupd},\tau$ then  $\exists\systemterm[']$  s.t.  $\systemterm\Arroi{\lambda}\systemterm[']$ and $(\isystemterm['],\systemterm['])\in\mathcal{R}$;
   \item If $\isystemterm\tr{}\isystemterm[']$  then  $\exists\systemterm[']$  s.t.  $\systemterm\Arro\systemterm[']$ and $(\isystemterm['],\systemterm['])\in\mathcal{R}$;
\end{enumerate} 
\end{definition}

As usual, we write $\isystemterm\precsim\systemterm$ if there exists a {simulation} $\mathcal{R}$ s.t.  
$(\isystemterm,\systemterm)\in\mathcal{R}$.

%
%Next theorem states one of most important result of the paper, saying that abstract GSP protocol and the implementation of GSP are weak bisimulation equivalent (or weakly bisimilar).
%
%We use $\approx$ for standard weak bisimulation.

\begin{theorem}
\label{thm:simulation}
 If  $\isystemterm$ implements $\systemterm$, then $\isystemterm\precsim\systemterm$. 
\end{theorem}

\begin{proof}
We show that $\mathcal{R} = \{ (\isystemterm, \systemterm) \ |\ \implements{\isystemterm{}}{\systemterm} \}$ is a simulation.
\end{proof}

We remark that  $\mathcal{R}^{-1}$ is not a simulation because the implementation cannot mimic  the 
behaviour in which a client   have completed two 
consecutive blocks (i.e., two \pushcmd\ commands) without sending the first block. In \gsp\, it is still possible to interleave the two 
blocks with blocks sent by other clients but in \igsp\  they are treated as atomic
because they  will be sent as a unique $\delta$ object. 