% !TEX root = main.tex

\section{Conclusions}
We have proposed a formal model for 
the Global Sequence Protocol and its proposed implementation. We use our formal model 
to provide  a 
simplified proof (that relies on standard  simulation) that the proposed implementation is 
correct. 
We remark that our proof does not require to exhibit an auxiliary state for the simulation  and 
that several 
invariants are trivially ensured by the definition of the model (e.g., the fact that clients have
a consistent view of the global sequence) and the well-formed conditions imposed over systems.  
We have formally studied the consistency guarantees ensured by the model by relying on the 
 operational semantics of the calculus to incrementally compute (a relaxed version of) 
abstract histories. We  have also shown how  \gsp\ can be used to formally study programming patterns, 
like synchronous update operations, that provide stronger consistency guarantees at the expenses of efficiency 
and availability. We plan to use the \gsp\ calculus as a formal basis for 
developing programming techniques to enable the fine-tuning of consistency levels 
in applications. 
