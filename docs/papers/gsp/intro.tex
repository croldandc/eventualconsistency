% !TEX root = main.tex

\section{Introduction}

Cloud infrastructures provide data storages that are virtually unlimited, elastic (i.e., scalable at run-time), 
robust   highly available and partition tolerant by 
replicating data over multiple servers. A client may  perform update and 
read operations over any of these replicas and the  store is responsible for keeping 
the  replicas synchronised. 
However, it is known (CAP theorem~\cite{CAP}) that one system cannot provide at the same time availability, 
partition tolerance, and consistency, but has to drop one of these properties. Today's popular 
data storages, such as Dynamo~\cite{DeCandia:2007:DAH:1294261.1294281} and  Cassandra~\cite{lakshman2010cassandra}, favour 
availability over strong consistency and provide weaker versions called \emph{eventual consistency}.
Roughly, eventual consistency ensures that, although data consistency can be at time violated, at some point it will be restored
and the different replicas will eventually converge to the same state~\cite{DBLP:journals/cacm/BailisG13}.
The  storages adopt different strategies to achieve eventual consistency and the choice  
impacts on the  guarantees provided by the system, i.e., on the kind of 
inconsistencies or anomalies that are allowed to happen. For instance, a storage may
resolve automatically conflicts introduced be concurrent updates (e.g., by using timestamps or causality) or  
may leave the problem to applications that read the database (like in Cassandra).
In this way, the consistency model supported by a data store becomes crucial when writing applications. 

Consequently,  there has been a growing interest on establishing  programming abstractions to help developers 
to deal with eventual consistent stores. For instance, commutative replicated data types~\cite{DBLP:conf/sss/ShapiroPBZ11}  and 
cloud types~\cite{burckhardt2012cloud} provide programmers with suitable data type abstractions that encapsulate issues related 
to eventual consistency. Resent proposals advocate declarative approaches to 
eventual consistency programming,  e.g.,  
to automatically select the consistency level required from a store given 
a consistency contract ~\cite{sivaramakrishnan2015declarative}
or to prove that  a given consistency level is adequate for preserving some data invariant~\cite{gotsman2016cause}.
With similar aims,
% Another recent proposal  aimed at helping programmers to reason about applications 
%running on top of replicated stores is the  
the Global Sequence Protocol (\gsp)~\cite{DBLP:conf/ecoop/BurckhardtLPF15} proposes  an operational model  
 to reason about applications 
running on top of replicated stores. Basically, the state of a store is represented as the sequence of updates that 
have led to it. Clients have their own copy of the state which they operate upon: each read and write operation has immediate
effect over the local state and the system propagates changes to make all replicas consistent using 
a reliable total order broadcast  protocol (\textsc{rtob}). The \textsc{rtob} protocol guarantees that all messages
are delivered  in the same total order to all clients. Replicas rely on the order generated  by \textsc{rtob} to 
converge to the same state. In the very basic model, called core \gsp, each client interacts with its local state 
  by performing read and write operations. Albeit simple, this model  introduces some subtleties when programming 
  because  it does not ensure read stability (i.e., two successive reads may 
return different values) nor atomicity of several updates (i.e., different clients may partially observe 
a sequence of updates). To overcome these limitations, 
 three synchronisation primitives, namely \pullcmd, \pushcmd\ and \confcmd, 
 allow programmers to control the propagation of changes. % and the updating of the local state. 
It has been shown that this model can be implemented so to  
consider a reduced representation of sequences of updates by using of state and delta objects and to handle communciation
failures.The idealised model and its implementation,
have been defined in terms of a reference implementation. 

We propose here a
formal account for both the idealised model and the  implementation as process calculi: the 
\gsp\ and \igsp\ calculi (\secref{sec:gsp} and \secref{sec:igsp}). We show
that the implementation is correct by showing that the behaviour of a program
running over \igsp\ can be also observed over the idealised model. Technically, 
we show that each \igsp\ system can be simulated by the corresponding 
\gsp\  system (\secref{sec:simulation}). Then, we study and prove the consistency guarantees ensured by 
\gsp.  
We rely on the characterisation of consistency guarantees in terms of abstract histories proposed in~\cite{DBLP:journals/ftpl/Burckhardt14}.
Abstract histories capture the visibility relation between actions and the arbitration of updates in the system. Then, 
a wide-spectrum of consistency models can be characterised in terms of these two relations. 
In \secref{sec:properties-gsp}, 
we show how to operationally associate abstract histories to concrete computations and 
prove that \gsp\ enjoys the properties  \textit{Monotonic Read},
 \textit{Causal Visibility}
and	{\em Consistent prefix} among others. Finally, in \secref{sec:transactions} we study the  
extension of \gsp\ with synchronous write operations, which ensures strong consistency. 
%Actually, we prove that \gsp\ with synchronous updates enjoy the single order property.


