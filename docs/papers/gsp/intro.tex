% !TEX root = main.tex

\section{Introduction}

Cloud infrastructures provide data storages that are virtually unlimited, elastic (i.e., scalable at run-time), robust (which is achieved by using replicas),  highly available and partition tolerant. It is known (CAP theorem~\cite{CAP}) that one system cannot provide at the same time availability, partition tolerance, and consistency, but has to drop one of these properties. Cloud data stores typically relax consistency, while providing a weaker version called \emph{eventual consistency}. Eventual consistency ensures that, although data consistency can be at time violated, at some point it will be restored.
% if no new updates are made to that data. 

%



{ %
%A promising future direction is to provide a theory of testing of general purpose application in cloud systems, and support the engineering of services that support the integration of applications and stores (e.g., interfaces between applications and weak stores that provide to the formers stronger properties than those guaranteed by the latters). 

