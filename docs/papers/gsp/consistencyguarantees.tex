% !TEX root = main.tex
\section{Consistency Guarantees}
\label{sec:properties-gsp}

In this section we study the consistency properties 
offered by \gsp. We rely on the characterisation of properties in 
terms of abstract executions~\cite{DBLP:conf/esop/BurckhardtLFS12},
execution histories enriched with information about visibility and 
arbitration of actions. 

%\chnote{Aca de repente se hablan de eventos en lugar de acciones o transitions. Tal vez deber�amos hablar de eventos desde el principio o bien decir algo que los eventos estan asociados con las transiciones que puede realizar gsp} 
%
%We shall introduce a series of store-level consistency guarantees and then we shall show which 
%are captured by application written in GSP. We start identifying three kinds of relations between actions of update and read:�




%We extend the GSP language with a new term which capture the relations amount operation in our system. 

\begin{definition} Let $\systemterm$ be a well-formed \gsp\ system, an abstract history 
for $\systemterm$ is a tuple 
$ \environmentterm =  \environment{\systemterm}{\op}{\so}{\vis}{\arb}$
%\[
%    \begin{array}{l@{\quad}r@{\;::=\;}l}
%			 (\textsc{Abstract Exec.}) & \environmentterm &  \environment{\systemterm}{\op}{\so}{\vis}{\arb} \\
%	    \end{array}
%  \]
where:
 \begin{itemize}
     \item $\op: \verticesets\rightarrow\readset\cup\updatesets$ maps events to operations;
     \item $\sse: \idset\rightarrow\verticesets$ associates events with a session (\ie, a client);
     \item $\so\subseteq \verticesets\times\verticesets$ describes the  order of operations  within a session;
     \item $\vis\subseteq\verticesets\times\verticesets$ indicates whether the effects of an update  are visible to a read; 
     \item $\arb\subseteq\verticesets\times\verticesets$ resolves concurrent update conflicts.
 \end{itemize}
 \end{definition}
 
We write $\_\rest{\_}$ for  function/relation restriction. For a given abstract history 
$ \environmentterm$, we write $\mathbb{U}$ (similarly, $\mathbb{R}$) for the codomain 
restriction of $\op$ to $\updatesets$ (correspondingly, $\readset$), \ie, 
$\mathbb{U} = \{ \vertice\ |\ \vertice{}\in\op, \op(\vertice) \in \updatesets\}$ 
($\mathbb{R} = \{ \vertice\ |\ \vertice{}\in\op, \op(\vertice) \in \readset\}$). 
 
 
 
 \begin{definition}[Well-formed history]
Let $\systemterm = \Absclient_0\bigpar \ldots\bigpar \Absclient_m\bigpar\queuemessage$ be 
a \gsp\ system where
$\Absclient_l = \tclienti{\tprogram_l}{\tknown_l}{\tpending_l}{\ttransactionbuffer_l}{\tsent_l}{\treceivebuffer_l}[\cid_l][\trounds_l]$. 
 An  history 
$ \environmentterm =  \environment{\systemterm}{\op}{\so}{\vis}{\arb}$  
is well-formed if the following conditions hold:

 \begin{enumerate}
     \item for all $\cid\in\dom\sse$, $\sse(\cid) \subseteq \dom\op$;
     \item $(\vertice,\vertice[w])\in\so$ then exist $\cid\in\dom\sse$ s.t. $\{\vertice,\vertice[w]\}\in\sse(\cid)$.
     \item for all $\cid\in\dom\sse$, $\so\relrestriccion{\sse(\cid)}$ is a total order; %\chnote{que significa esto?}
     \item $\vis\subseteq\mathbb{U}\times\mathbb{R}$;
     \item $\arb\subseteq\mathbb{U}\times\mathbb{U}$ is a prefix order.%, then $\{\op(\vertice[@][_1]),\op(\vertice[@][_2])\}\subseteq\updatesets$;
     \item $(\vertice,\vertice[w])\in \arb$  iff 
     	\begin{itemize} 
	\item $\flatten{\queuemessage}[i] = \anupd^{\vertice}$ and $\flatten{\queuemessage}[j] = \anupd^{\vertice[w]}$ and $i< j$; or
     	\item $\anupd^{\vertice}\in \flatten{\queuemessage}$ and $\anupd^{\vertice[w]}\in\flatten{\tsent_l\cdot\ttransactionbuffer_l}$;
	\end{itemize}
     \item if $\flatten{\tsent_l\cdot\ttransactionbuffer_l}[i] = \anupd^{\vertice}$ and $\flatten{\tsent_l\cdot\ttransactionbuffer_l}[j] = \anupd^{\vertice[w]}$ and $i< j$, then
     $(\vertice,\vertice[w])\in \so$ and $\{\vertice,\vertice[w]\}\in\sse({\cid_l})$.
 \end{enumerate}
\end{definition}

The above conditions ensure that events in $\sse$ are associated with an operation by $\op$ (1). Besides, $\so$ 
only relates 
 events 
belonging to the same session (2), which are totally ordered within each session (3). Differently from 
the definition in~\cite{DBLP:journals/ftpl/Burckhardt14}, we restrict visibility to keep track of dependencies between updates and read events(4).
 We do not require $\arb$ to be a total order but instead to be 
a prefix order (5). In this way we allow  updates in different replicas not to be arbitrated until they do not reach the global store yet. 
The remaining two conditions require the abstract history to be consistent with the state of the system.

%\begin{itemize}
% \item {\em Session Order} (\so) describes the sequential order in which operations executed within a session (i.e., a single thread of 
% execution) are performed. This relation is a total order over the events associated with operations belonging to the same session.
% Hereafter, we will 
% associate a session with a \gsp\ client.  
%
% \item{\em Visibility} (\vis) .
%
% \item{\em Arbitration} (\arb) describes the resolution strategy for concurrent update conflicts. This is a total order over update events.
%
% \item{\em Same Session} (\sse) ...
%
%\end{itemize}
%
%
%%	 
%%Let $\environmentterm$, a new term, where $\systemterm$ represents our system introduced in Definition 1.1, $\op$ is a mapping of vertices to actions, $\so$ is a session order relation defined from vertices to relations of vertices $\verticesets$ $\times$ ($\verticesets$ $\times$ $\verticesets$) and $\vis$,$\arb$ are visibility and arbitration relation.
%%
%%
%

Rules in  \figref{fig:comp-abst-executions} provides an operational way to associate abstract executions 
with \gsp\ computations. Rules  \ruleName{a-update} and  \ruleName{a-read} add new events to the history
and corresponds to the execution of a read or update operation by a client. In
both cases $\op$ is extended with a new event $\vertice$ (\ie, $\vertice \notin dom(\op)$), which is associated 
with the corresponding operation (either $\aread$ or $\anupd$). The new event $\vertice$ is added to the 
corresponding  session $\cid$, and $\so$ is updated to make $\vertice$  the maximal event for the session $\cid$.
Rule  \ruleName{a-update}  amends $\arb$ by capturing the fact that all updates that are already in the 
global state have take place before the new event. Rule \ruleName{a-read} instead augments $\vis$ with the pairs
associating the new event with all events that are seen by the read action, namely, the local  view of 
the global state $\queuemessage[0..\tknown_{\cid}-1]$ and the local  buffers 
$\tpending_{\cid}$ and $\ttransactionbuffer_{\cid}$. Rule \ruleName{a-arb} handles the changes in the 
state of the global store (due to a send transition in one client) and  amends $\arb$ by (i) arbitrating the new 
events exactly in the way in which they are added to the store (\ie, $\{(\vertice[@][_i],\vertice[@][_j]) \ |\ i,j\in \{0,\ldots,n\}, i<j\}$)
and registering that they are arbitrated before any other concurrent update in the remaining 
clients (\ie, $
		\ (\{\vertice[@][_0],\ldots,\vertice[@][_n]\}\times 
		\{ \vertice[w]\ \ |\ \  \vertice[w]\in\mathbb{U} \
		     \forall\anupd. \anupd^{\vertice[w]}\not\in\queuemessage\cdot\anupdseq\}$).
The remaining transitions of the system are considered as internal changes that do not 
affect the history and are handled by rule \ruleName{a-int}.

\begin{lemma} \label{lemma:prop-wf} Let  $\environmentterm$ be a well-formed history. If 
 $\environmentterm\arroi{\action}\environmentterm'$, then $\environmentterm'$ is well-formed.
\end{lemma}

%%reglas
 \begin{figure}[t]
 \[
 \begin{array}{l}
\mathrule{a-update}
	{
	\begin{array}{p{\linewidth}}
		$\systemterm \arroi{\updatetran{u}} \systemterm' 
		\qquad\ \  \vertice \notin dom(\op)
		\qquad\ \ \op' = \op\upd \vertice u
		\hfill \sse' = \sse\upd \cid {\sse(\cid)\cup\{\vertice\}}
		 $\\[2pt]
		 $
		 \hspace{1cm}
		 \so' = \so\cup (\sse(\cid)\times\{\vertice\})
		 \hfill
		 \arb' = \arb \cup (\{ \vertice[w] \ |\ \udec[@][{\vertice[w]}]\in \queuemessage \}\times\{\vertice\})
		 \hspace{1cm}
	$\end{array}
	}
	{\environment{\systemterm}{\op}{\so}{\vis}{\arb} 
		  \arroi{\updatetran{u}}  
		  \environment{\systemterm'}{\op'}{\so'}{\vis}{\arb'}
	}
\\[18pt]	
\mathrule{a-read}
	{ 
	\begin{array}{p{\linewidth}}
		 $\systemterm \arroi{\readtran{r}} \systemterm' 
		 \qquad\ \  \vertice \notin  dom(\op) 
		 \qquad\ \  \op' = \op\upd \vertice r
		 \hfill  \sse' = \sse\upd \cid {\sse(\cid)\cup\{\vertice\}}
		 $\\[2pt]
		 $
		 \so' = \so\cup (\sse(\cid)\times\{\vertice\})
		 \hfill  \vis' = \vis \cup (\{\vertice[w] \ |\  \udec[@][{\vertice[w]}] \in \queuemessage[0..\tknown_{\,\cid}-1] \cdot \tpending_{\cid} \cdot \ttransactionbuffer_{\,\cid} \}\times\{\vertice\})
		 $
	\end{array}
	 }
	 {\environment{\systemterm}{\op}{\so}{\vis}{\arb} 
	   \arroi{\readtran{\udec[r]}} \environment{\systemterm'}{\op'}{\so'}{\vis'}{\arb}[\sse']}
\\[25pt]	
\mathrule{a-arb}
	{ 
	\begin{array}{p{\linewidth}}
		$\queuemessage \ \bigpar\ \Absclient \arroi{\lambda} \queuemessage\cdot \anupdseq \ \bigpar\ \Absclient' 
		%\hspace{2cm} \queuemessage\neq\queuemessage'
		\hspace{2.2cm} \lambda \neq  \updatetran{u},\readtran{r}
		\hfill \anupdseq=\ublock[{\udec[{\anupd[_0]}][{\vertice[@][_0]}]}\cdots{\udec[{\anupd[_n]}][{\vertice[@][_n]}]}]
		 $\\[2pt]
%		 \arb' = \arb \cup \{ (\vertice, \vertice[w])\ | \ \ell = |\queuemessage'|-1 \land \udec \in \queuemessage'[0..\ell-1] \land \udec[@][{\vertice[w]}]= \queuemessage'[\ell]\}
		$\arb' = \arb \cup
		\{(\vertice[@][_i],\vertice[@][_j]) \ |\ i,j\in \{0,\ldots,n\}, i<j\}
		$
		\\
		$\hfill\cup 
		\ (\{\vertice[@][_0],\ldots,\vertice[@][_n]\}\times 
		\{ \vertice[w]\ \ |\ \  \vertice[w]\in\mathbb{U}, \
		     \forall\anupd. \anupd^{\vertice[w]}\not\in\queuemessage\cdot\anupdseq\})
		  $   
	\end{array}
	 }
	 {\environment{\queuemessage\ \bigpar\ \Absclient}{\op}{\so}{\vis}{\arb} 
	 \arroi{\lambda}	   
	 \environment{\queuemessage\cdot \anupdseq\ \bigpar\ \Absclient'}{\op}{\so}{\vis}{\arb'}}
\\[18pt]
\mathrule{a-int}
	{ 
	\begin{array}{l}
		\queuemessage \bigpar \Absclient \arroi{\lambda} \queuemessage \bigpar \Absclient' 
		\qquad \lambda \neq  \updatetran{u},\readtran{r}
	\end{array}
	 }
	 {\environment{\queuemessage \bigpar \Absclient}{\op}{\so}{\vis}{\arb} 
	   \arroi{\lambda} 
	   \environment{\queuemessage\bigpar \Absclient'}{\op}{\so}{\vis}{\arb}}
\end{array}
 \]
 \caption{Computation of abstract executions} 
 \label{fig:comp-abst-executions}
 \end{figure}
 
 
 
%  \paragraph{Notation.} Given a session order relation $\so$ from client $i$ and a vertex $v$, we shall write $\soby{\so}{v}$  meaning that $\soby{(\mathcal{V}, \mathcal{R})}{v} = (\mathcal{V} \ \cup \ \{v\}, \mathcal{R}\ \cup \ \{(x,v) / x \in \verticesets\})$. We shall refer to an update action on the queue message as $\updateinqueuemessage{n}{i}$.The arbitration relation $\arb$ is defined as $\{ (v,w) / \{v \mapsto \updateinqueuemessage{m}{h}\} \in \op \land \ \{w \mapsto \updateinqueuemessage{n}{i} \} \in \op \land \ m < n \}$. A transition $\arroi{\alpha}$ denotes the fact that action $\alpha$ is perfomed by client $i$.
 

%\subsection{Ordering Guarantees}

We use  histories to analyse the ordering guarantees 
offered by the \gsp\ model. We focus on the characterisation
provided in~\cite{DBLP:journals/ftpl/Burckhardt14} (Due to space limitation, 
we refer the intersted reader to oC.
% now prove what ordering guarantees are assured by GSP language and what do not. 


\begin{theorem}  \label{theorem:properties}
	If $\environment{\systemterm}{\emptyset}{\emptyset}{\emptyset}{\emptyset}[\emptyset] \arro{}^*\ 
		\environment{\systemterm'}{\op}{\so}{\vis}{\arb}$ 
	then

	\begin{enumerate}  [(1)]
	    	\item
		\label{theorem:read-my-writes}
	    	\textit{Read My Writes}:  $\restUR\so  \subseteq  \vis$

		\item 
		\label{theorem:monotonic-read}
		\textit{Monotonic Read}:  $\restUR{\vis;\so}  \subseteq  \vis$.
		
		\item
		\label{thm:no-circular-causality}
		\textit{No Circular Causality}: $(\so \cup \vis ) ^+$ is acyclic.

		\item
		\label{thm:causal-visibility}
		\textit{Causal Visibility}: $\restUR{(\so \cup \vis ) ^{+} } \subseteq \vis$. 

		\item 
		\label{thm:causal-arbitration}
		\textit{Causal Arbitration}: $\restUU{((\so \cup \vis) ^{+} \setminus \so)}\  \subseteq\ \arb$. 
		
		\item
		\label{thm:consistent-prefix}
		\textit{Consistent prefix}: $\arb;(\vis\setminus\sse)\  \subseteq\ \vis$. 
		
	\end{enumerate}
\end{theorem}



%\begin{theorem}[\textsc{Read My Writes}]
%\label{theorem:read-my-writes}
%% $\textsc{\small{SO}}_R$ the se\cond component of the relation $\textsc{\small{SO}}$ and $\textsc{\small{VIS}}$ a visibility relation, if
% If $\environment{\systemterm}{\emptyset}{\emptyset}{\emptyset}{\emptyset}[\emptyset] \arro{}^*\ 
%     \environment{\systemterm'}{\op}{\so}{\vis}{\arb}$
%    then $\restUR\so  \subseteq  \vis$
%\end{theorem}
%
%\begin{theorem}[\textsc{Monotonic Read}]
%\label{theorem:monotonic-read}
%%Let $\textsc{\small{SO}}_R$ the se\cond component of the relation $\textsc{\small{SO}}$ and $\textsc{\small{VIS}}$ a visibility relation,
%If $\environment{\systemterm}{\emptyset}{\emptyset}{\emptyset}{\emptyset}[\emptyset] \tr{} ^*\ \environment{\systemterm'}{\op}{\so}{\vis}{\arb}$ 
%then $\restUR{\vis;\so}  \subseteq  \vis$.
%\end{theorem}
%
%\begin{theorem}[\textsc{No Circular Causality}]
%\label{thm:no-circular-causality}
%If $\environment{\systemterm}{\emptyset}{\emptyset}{\emptyset}{\emptyset}[\emptyset] \tr{} ^*\ \environment{\systemterm'}{\op}{\so}{\vis}{\arb}$ 
%then 
%%Let $\textsc{\small{SO}}_R$ the se\cond component of the relation $\textsc{\small{SO}}$ and $\textsc{\small{VIS}}$ a visibility relation, if $\environment{\systemterm_0}{\emptyset}{\emptyset}{\emptyset}{\emptyset} \arro{} ^*\ \environment{\systemterm}{\op}{\so}{\vis}{\arb}$ then 
%$(\so \cup \vis ) ^+$ is acyclic.
%\end{theorem}
%
%\begin{theorem}[\textsc{Causal Visibility}]
%\label{thm:causal-visibility}
%If $\environment{\systemterm}{\emptyset}{\emptyset}{\emptyset}{\emptyset}[\emptyset] \tr{} ^*\ \environment{\systemterm'}{\op}{\so}{\vis}{\arb}$ 
%then
%%Let $\textsc{\small{SO}}_R$ the se\cond component of the relation $\textsc{\small{SO}}$ and $\textsc{\small{VIS}}$ a visibility relation, 
%$\restUR{(\so \cup \vis ) ^{+} } \subseteq \vis$. 
%\end{theorem}
%
%\begin{theorem}[\textsc{Causal Arbitration}]
%\label{thm:causal-arbitration}
% If $\environment{\systemterm}{\emptyset}{\emptyset}{\emptyset}{\emptyset}[\emptyset] \tr{} ^*\ \environment{\systemterm'}{\op}{\so}{\vis}{\arb}$ 
%then
%$\restUU{((\so \cup \vis) ^{+} \setminus \so)}\  \subseteq\ \arb$. 
%\end{theorem}
%
%\begin{theorem}[\textsc{Consistent prefix}]
%\label{thm:consistent-prefix}
%If $\environment{\systemterm}{\emptyset}{\emptyset}{\emptyset}{\emptyset}[\emptyset] \tr{} ^*\ \environment{\systemterm'}{\op}{\so}{\vis}{\arb}$ 
%then
%$\arb;(\vis\setminus\sse)\  \subseteq\ \vis$. 
%\end{theorem}
%

The following example shows that the \gsp\ model exhibits the Dekker anomaly, hence it does 
not enjoy {\em sequential consistency}~\cite{DBLP:journals/ftpl/Burckhardt14}.

\begin{example}[Dekker anomaly] 
\label{no-consistent-prefix}
Consider the following system consisting of two clients and the empty store $\systemterm = \epsilon \ \bigpar\ \Absclient[_1]  \ \bigpar\ \Absclient[_2]$ where
\[
\begin{array}{lll}
   \Absclient[_1] & = & {\anabstcli[{}][{\updcmd[{\anupd_1}];\treadins{y}{r_1}}][0][\epsilon][\epsilon][\epsilon][0][0]}_{\cid_1}\\
   \Absclient[_2] & = & {\anabstcli[{}][{\updcmd[{\anupd_2}];\treadins{y}{r_2}[Q]}][0][\epsilon][\epsilon][\epsilon][0][0]}_{\cid_2}\\
\end{array}
\]



By using rule \ruleName{a-update}, we have the following transition
\[
	{\environment{\systemterm}{\emptyset}{\emptyset}{\emptyset} {\emptyset}[\emptyset]
		  \arro{\updatetranaux{\anupd_1^{\vertice[v][_1]}}}[\cid_1]
		  \environment{\epsilon \ \bigpar\ \Absclient[_1]'  \ \bigpar\ \Absclient[_2]}{\op_1}{\emptyset}{\emptyset}{\emptyset}[\sse_1]
	}
\]
with $\op_1 = \{\asoc{\vertice[v][_1]} \anupd_1\}$,  $\sse_1 = \{\asoc{\cid_1}{\{\vertice [v][_1]}\}\}$ and
\[
   \Absclient[_1]'  =  {\anabstcli[{}][{\treadins{y}{r_2}}][0][\epsilon][{\anupd_1}^{\vertice[v][_1]}][\epsilon][0][0]}_{\cid_1}
\]
Similarly, 
\[
	{\environment{\systemterm}{\emptyset}{\emptyset}{\emptyset} {\emptyset}[\emptyset]
		  \arro{\updatetranaux{\anupd_1^{\vertice[v][_2]}}}[\cid_2]
		  \environment{\epsilon \ \bigpar\ \Absclient[_1]'  \ \bigpar\ \Absclient[_2]'}{\op_2}{\emptyset}{\emptyset}{\emptyset}[\sse_2]
	}
\]
with $\op_2 = \{\asoc{\vertice[v][_1]},{\anupd_1},\asoc{\vertice[v][_2]} {\anupd_2}\}$,  
$\sse_2 = \{\asoc{\cid_1}{\{\vertice [v][_1]}\}, \asoc{\cid_2}{\{\vertice [v][_2]}\}\}$ and
\[
   \Absclient[_2]'  =  {\anabstcli[{}][{\treadins{y}{r_2}[Q]}][0][\epsilon][{\anupd_2}^{\vertice[v][_2]}][\epsilon][0][0]}_{\cid_2}
\]
Then, by using rule \ruleName{a-read},
\[ 
	\environment{\epsilon\ \bigpar\ \Absclient[_1]'  \ \bigpar\ \Absclient[_2]'}{\op_2}{\emptyset}{\emptyset}{\emptyset}[\sse_2]
	 \arro{\readtran{\udec[r_1][{\vertice[v][_3]}]}}[\cid_1]
	\environment{\epsilon \ \bigpar\ \Absclient[_1]''  \ \bigpar\ \Absclient[_2]'}{\op_3}{\so_1}{\vis_1}{\emptyset}[\sse_3]
\]
with $\op_3 = \{\asoc{\vertice[v][_1]} {\anupd_1},\asoc{\vertice[v][_2]} {\anupd_2},\asoc{\vertice[v][_3]} {r_1}\}$,  
$\sse_3 = \{\asoc{\cid_1}{\{\vertice [v][_1],\vertice [v][_3]\}},\asoc{\cid_2}{\{\vertice [v][_2]\}}\}$,
$\so_1 = \vis_1 = \{(\vertice[v][_1],\vertice[v][_3])\}$,
$   \Absclient[_1]''  =  {\anabstcli[{}][P\subst{x}{v}][0][\epsilon][{\anupd_1}^{\vertice[v][_1]}][\epsilon][0][0]}_{\cid_1}$,
and $\iread{r_1}{{\anupd_1}^{\vertice[v][_1]}}=v$.
And similarly,
\[ 
	\environment{\epsilon \ \bigpar\ \Absclient[_1]''  \ \bigpar\ \Absclient[_2]'}{\op_3}{\so_1}{\vis_1}{\emptyset}[\sse_3]
	 \arro{\readtran{\udec[r_2][{\vertice[v][_4]}]}}[\cid_2]
	\environment{\epsilon \ \bigpar\ \Absclient[_1]''  \ \bigpar\ \Absclient[_2]''}{\op_4}{\so_2}{\vis_2}{\emptyset}[\sse_4]
\]
with $\op_4 = \{\asoc{\vertice[v][_1]} {\anupd_1},\asoc{\vertice[v][_2]} {\anupd_2},\asoc{\vertice[v][_3]} {r_1}, \asoc{\vertice[v][_4]} {r_4}\}$,  
$\sse_4 = \{\asoc{\cid_1}{\{\vertice [v][_1],\vertice [v][_3]\}},\asoc{\cid_2}{\{\vertice [v][_2],\vertice [v][_4]\}}\}$,
$\so_2 = \vis_2 = \{(\vertice[v][_1],\vertice[v][_3]), (\vertice[v][_2],\vertice[v][_4])\}$, 
$   \Absclient[_1]''  =  {\anabstcli[{}][Q\subst{y}{v'}][0][\epsilon][{\anupd_1}^{\vertice[v][_1]}][\epsilon][0][0]}_{\cid_1}$ and
 $\iread{r_2}{{\anupd_2}^{\vertice[v][_2]}}=v'$.
 Since $\iread{r_1}{{\anupd_1}^{\vertice[v][_1]}}=v$ and $\iread{r_2}{{\anupd_2}^{\vertice[v][_2]}}=v'$, 
  none of the clients reads the update performed by the other.

\end{example}
