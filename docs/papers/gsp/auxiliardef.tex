% !TEX root = main.tex

 
 
 \section{Functions for implementing datatypes}
 \subsection{Auxiliar Function}

Before describing the implementation of GSP, we will introduce the following auxiliars functions which will be used when we define the operational semantic.

\footnotesize
\ttfamily


\begin{flushleft}
\specfunction{append}{\gssegmenttype}{$\roundtype^*$}{\gssegmenttype} \\
\append{\gssegmentins{$\delta$}{\maxround}}{$\epsilon$} = \gssegmentins{$\delta$}{\maxround} \\
\append{\gssegmentins{$\delta$}{\maxround}}{$\headerround$:\tailround} = \append{\textless\reduce{$\delta$ $\cdot$ $\delta_0$ $\cdot$ $\epsilon$}, $\update{\maxround}{b_0}{n_0}$\textgreater}{\tailround} 
\end{flushleft}

\begin{flushleft}
\specfunction{apply}{\gsprefixtype}{\gssegmenttype}{\gsprefixtype} \\
\apply{$\gsprefixins{\state}{\maxround}$}{\gssegmentins{$\delta$}{$\maxround'$}} =  \gsprefixins{\applyplus{\state}{$\delta$ $\cdot$ $\epsilon$}}{\maxround[$\maxround'$]}  \\
\end{flushleft}

\begin{flushleft}
\specfunctiononeparameter{receivedrounds}{($\partialfunction{\idset}{\textless \dominserver \textgreater^*}$)}{$\roundtype^*$}\\
receivedrounds($\undefined$) = $\epsilon$ \\
receivedrounds($\partialfunction{b}{\textless n_0,\ \delta_0 \textgreater} \cdot\ f$) = \textless $b$, \ $n_0$, \ $\delta_0$\textgreater\ $\cdot$ receivedrounds($f$)
\end{flushleft}

\begin{flushleft}
\henote{arreglar las lineas comentadas}
\chnote{hecho}
\specfunctionforparameters{curstate}{\statetype}{$\roundtype^*$}{\deltatype}{\deltatype}{\statetype} \\
$curstate(\state, \pending, \pushbuffer, \transactionbuffer$) = $\applyplus{\state}{getdeltas(\pending) \cdot\pushbuffer\cdot \transactionbuffer}$
\end{flushleft}


\begin{flushleft}
\specfunctiononeparameter{getdeltas}{$\roundtype^*$}{$\deltatype^*$} \\
getdeltas($\epsilon$) = $\epsilon$ \\
getdeltas(\textless $n_0$,$\delta_0$ \textgreater $\cdot$ $\delta$) = $\delta_0$ $\cdot$ getdeltas($\delta$)\\
\end{flushleft}

\begin{flushleft}
\specfunctiononeparameter{remove}{$\deltatype^*$}{$\deltatype^*$} \\
remove(rs,$\epsilon$) = $\epsilon$ \\
remove(rs, xs) = $(foldr (\backslash x\ rec\ ys\to h \ (filter \ (/$=$x) ys))\ id) xs \ rs$\\
\end{flushleft}


\begin{flushleft}
\specfunctionthreeparameters{notify}{$\idset^*$}{($\partialfunction{\idset}{\gssegmenttype^* \cup \gsprefixtype}$)}{$\gssegmenttype$}{$(\partialfunction{\idset}{\gssegmenttype^* \cup \gsprefixtype})$}\\
notify($\{\cid\}$, $\outserver$, gs) = $\outserver[\cid \mapsto gss \cdot gs]$\\
notify($\cid \cdot bs$, $\outserver$, gs) = $\notify{bs}{\outserver}{gs}[\cid \mapsto gss \cdot gs]$ 
\end{flushleft}

\begin{flushleft}
\specfunction{\cleannamefun}{$\idset^*$}{$(\partialfunction{\idset}{\dominserver}$)}{$(\partialfunction{\idset}{\dominserver})$}\\
\cleannamefun($\{\cid\}$, $\outserver$, gs) = $\inserver[\cid \mapsto \emptyset]$\\
\cleannamefun($\cid \cdot bs$, $\outserver$, gs) = $\clean{bs}{\inserver[\cid \mapsto \emptyset]$}
\end{flushleft}


\normalfont
\normalsize

\paragraph{Notation.} Let $f$ and $g$ be a partial function, we define the update operator $\_[\_]$ such that
  $dom(f[g])  = dom(f) \cup dom(g)$ and 
 \[
   \begin{array}{l@{\ = \ \Bigg\{}l}
     f[g](x) & 
     \begin{array}{ll}
        f(x) & \mbox{if }\ x\not\in dom(g) \wedge x\in dom(f) \\
        g(x) & \mbox{if }\ x\in dom(g)\\
        \mathit{\undefined}  & \mbox{Otherwise} 
     \end{array}
   \end{array}
 \]    

 We write $[x_1 \mapsto y_1, \ldots, x_n \mapsto y_n]$ for the partial function $f$ such that $dom(f)=\{x_1,\ldots,x_n\}$ and $f(x_i)=y_i$; $A \setminus B$ to denote the usual difference of sets.



The functions \textbf{reduce} and \textbf{apply} are abstract and depend on the data model used. 

