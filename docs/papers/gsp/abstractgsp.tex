% !TEX root = main.tex


\section{Global Sequence Protocol Calculus}
\label{sec:gsp}

%\henote{check store-store-etc.}


\subsection{Syntax}

Clients interact with a store by performing operations in $\updatesets\cup\readset$: 
an element in $\updatesets$ denotes an update operation, while one in $\readset$ 
stands for a read operation. 
None operation can simultaneously read and update a store, therefore we assume
 $\updatesets \cap \readset= \emptyset$.
 We write 
 $\anupd, \anupd['], \anupd[''],\ldots$ for  updates  and 
$\aread, \aread['], \aread[''],\ldots$ for reads. 

The  state of a store is 
represented by a sequence of updates. 
For technical 
convenience (particularly in \secref{sec:properties-gsp}), we 
distinguish  different executions of the same operation. Formally, stores 
 associate each update with a fresh event identifier.
We assume a set $\verticesets$ of event identifiers  $\vertice$, $\vertice[@][_0]$, \ldots, $\vertice[@][']$,$\ldots$ 
and write $\udec$ for the update  $\anupd$ associated with the event $\vertice$.
%We sometimes omit the decoration when it is not relevant.  \henote{Is this true?}

We  use $\anupdseq$ to denote sequences of decorated updates and
$\ublock[\anupdseq]$ for an atomic block of updates. 
%(We reserve the term 
%transaction for a different notion addressed in \secref{sec:transactions}). 
We write  $\blockseq$  for a sequence of blocks. We denote the empty sequence
with  $\emptysequence$
%  defined by the following grammar
% \[\tpending,{\tpending[']} ::= \emptysequence \ |\  [\alpha] \cdot \tpending\]
% \changed{}{A sequence $[\alpha] \cdot \tpending]$ stands for ....}, and we 
% use $\emptysequence$ to denote the empty sequence, \ie, the sequence of length 0.
and use the usual operations on sequences such as $\blockseq{[i]}$ to denote  
 the $i$-th  element of $\blockseq$, $\blockseq{[i..j]}$
for the subsequence of $\blockseq$ from position $i$ to $j$, $|\blockseq|$ for the length and 
$\blockseq\setminus\blockseq[']$ for the relative complement of $\blockseq$
in $\blockseq[']$. Additionally, 
 $\flatten{\blockseq}$ stands for the plain sequence a of  updates in $\blockseq$ (i.e., without 
 any separation in blocks). 
%is such that $\flatten{\emptysequence} = \emptysequence$, and 
%    $\flatten{\ublock\cdot\blockseq} = \ttransactionbuffer\cdot\flatten\blockseq$.

We rely on the countable sets $\varset$ of program variables $x,x',\ldots$ 
and $\idset$ of client identifiers $\cid,\cid',\ldots,\cid_1,\ldots$.



 
 \begin{definition}[GSP Language] 
The set of \tgspcalculus\ terms is given by the grammar in \figref{fig:syntax-gsp}.
\end{definition}


 \begin{figure}[t]
 \[
\begin{array}{l}
    \begin{array}{l@{\ }rcl}
		 \rName{naturals}
		 &
		 \multicolumn 3 l {\treceivebuffer, \tknown, \trounds \in \nat}
		 \\
    		 \rName{update} 
		 &
		 \updatesets
		 &{ =  }&
		 \{ \anupd, \anupd['], \ldots,  \anupd[_0],\ldots\}
		 \\
   		 \rName{read} 
		 &
		 \readset
		 &{ =  }&
		 \{ \aread, \aread['], \ldots,  \aread[_0],\ldots\} 
		  \\
   		 \rName{event} 
		 &
		 \verticesets
		 &{ =  }&
		 \{\vertice, \vertice[@]['], \ldots, \vertice[@][_0],\ldots\} 
		 \\
		 \rName{var}
		 &
		  \varset &{ =  }&
		 \{x,x',\ldots,x_0,\ldots\}
		 \\
		 \rName{ids}
		 &
		  \idset &{ =  }&
		 \{\cid,\cidj,\cid['],\ldots,\cid[_0],\ldots\}
\end{array}
\qquad
    \begin{array}{l@{\ }r@{}l}
    \\
		 \rName{upd seq}
		 &
		\anupdseq 
		 \ ::=\ 
			&
		 \emptysequence \ |\ \udec\cdot\anupdseq 
		 \\
		 \rName{block seq}	
		 &	 
		 \blockseq
		 \ ::=\ 
		&
		 \emptysequence\ |\ \ublock\cdot\blockseq	 
			\\
		\rName{system} 
		& \systemterm\ ::= \ 
		& \queuemessage \, \bigpar\, \Absclient
		 \\     	  \rName{store} 
			 & \queuemessage \ ::=\ 
			 &  \blockseq 
			 \\
      		\rName{client} 
			& \Absclient \ ::=\ 
			& 0 \;|\;�\tclient{\tprogram}{\tknown}{\blockseq}{\ttransactionbuffer}{\blockseq}{\treceivebuffer} \;|\;
       				  \Absclient\, \bigpar\, \Absclient %\hfill \text{with}\  n \in \nat
      \end{array}
      \\[30pt]
      \begin{array}{ll}
     		 \rName{program} 
			 \ \tprogram \ ::=\ &
			  \tupdateins \;|\; \treadins{x}{\aread} \;|\; \tpullins \;|\;
			   \\
			  & \tpushins \;|\;
			 \tconfirmedins{x} 
			%\;|\; \pwhile{\cond}{\tprogram}  \;|\;
			%\\
			%&
			%& \pifte
     \end{array}
\end{array}
  \]
  \vskip -10pt
\caption{Syntax of the \gsp\ calculus}
\label{fig:syntax-gsp}
\end{figure}


A \gsp\ system $\systemterm$ consists of  a store and zero or more  clients.
The global store $\queuemessage$ is completely defined by its state, which 
consists of  a sequence of  blocks.
The term $\tclienti{\tprogram}{\tknown}{\tpending}{\ttransactionbuffer}{\tsent}{\treceivebuffer}[\cid][n]$ 
stands for a client identified by $\cid$ and  engaged on the execution of the 
program $P$. 
The remaining elements 
are used to describe the state of the local replica: $\ttransactionbuffer$ contains the  updates 
that have been made locally and are  part of an unfinished block; $\tsent$ models the 
communication buffer, which keeps all blocks sent by the client but not received by the 
global store; 
$\tpending$ is the pending buffer, which contains all  completed blocks  that are 
unconfirmed by the global store.
For simplicity, we do not  have an explicit replica of the global store in each client;
 we use instead a natural number $\tknown$ to indicate the portion  of the global 
state that is known to the client.
Specifically, the client  $\cid$ knows the 
sequence $\queuemessage[0..\tknown-1]$. 
Similarly,  $\treceivebuffer$ indicates the number  updates received by the client
that have not been  added to the local replica, i.e., 
the client has received the updates contained in the segment $\queuemessage[\tknown..\tknown+\treceivebuffer-1]$.
%Value $\trounds$ counts the number of update operations executed by the client. 


A program $P$ is built as a sequence of  operations that interacts with 
the store: $\readcmd,  \updcmd, \pullcmd,  \pushcmd, \confcmd$
 (we postpone their description until \secref{sec:gsp-sos}). 
 %Conditional and loop statements contain expressions 
%$\cond, \cond', \ldots$ written in some language that we left unspecified. We only assume   
%such language to be equipped with a valuation function  that associates each expression $\cond$  
%with a value $v$ in some domain $\valueset$, written $\eval{\cond}{v}$.  
A program $\plet{x}{\ldots}{P}$ introduces a bound variable whose  scope is $P$. The definition of 
free variables of a program is  standard. We say that a process $P$ is {\em closed} when it does not contain  
 free variables. 
We keep the language for programs simple. We remark that this 
choice does not affect the results presented in this paper. Actually, we could just  have  characterised 
 the behaviour of programs as a labelled transition system, 
but we prefer to have a syntax through the presentation. 

%\henote{agregar algun comentario sobre otras estructuras de control}



\begin{definition}[Well-formedness]
\label{def:wf-gsp}  A \gsp\ system $\systemterm =  \Absclient_0\bigpar \ldots\bigpar \Absclient_m\bigpar\queuemessage$  where 
$\Absclient_l = \tclienti{\tprogram_l}{\tknown_l}{\tpending_l}{\ttransactionbuffer_l}{\tsent_l}{\treceivebuffer_l}[\cid_l][\trounds_l]$ 
for all  $l\in \{0,\ldots,m\}$  is {\em well-formed} if the  following conditions hold 
\begin{enumerate}
  \item \label{wf-gsp-id} $\cid_{l} \neq \cid_{l'}$ for all $l\neq l'$;
  \item \label{wf-gsp-server} $k_l+j_l\leq \length \queuemessage$ for all $l$;
  \item \label{wf-gsp-pending}  $\tpending_l = \ublock[{\anupdseq[_1]}]\cdots\ublock[{\anupdseq[_p]}]\cdot\tsent_l$ and 
  for all ${\sf 1} \leq x < y \leq{\sf p}$ there exists $x',y'$ s.t. $\queuemessage[x'] = \ublock[{\anupdseq}_x]$, $\queuemessage[y'] =\ublock[{\anupdseq}_y]$ and 
  $\tknown_l\leq x' < y'$; and
%  \item if $\ublock \in {\tpending_l}\cdot {\tsent_l}$ then $\cid = \cid_l$, \ie,  blocks in the local queues of 
%  a client $l$ are  all decorated with the identifier $\cid_l$.
%  \item $\cid \in {\cid_0,\ldots, \cid_n} $ for all $\ublock \in S$, \henote{Pensar bien, tal vez no es necesaria}
  \item \label{wf-gsp-send} $\anupdseq = \flatten{\queuemessage\cdot\tsent_0\cdots\tsent_m\cdot\ttransactionbuffer_0\cdots\ttransactionbuffer_m}$,
  if $\anupdseq{[x]}=\udec$, $\anupdseq{[y]}=\udec[@][{\vertice[@][']}]$ and $x \neq y$ then $\vertice \neq \vertice[@][']$.
\end{enumerate}
\end{definition}
We require identifiers to univocally identify clients (1) and every local state  to be consistent with the global store, i.e., 
 a client can see at most every messages in the store (2),  all unconfirmed blocks in $\tpending_l$ are  either in  the 
   communication buffer $\tsent_l$ or  in the unseen part of the global store  $\ublock[{\anupdseq[_1]}]\cdots\ublock[{\anupdseq[_p]}]$ (3).
 Moreover, an event identifier is associated with a unique update in the system (4). 
Hereafter, we assume every \gsp\ system to be well-formed.

%\changed{}{
%\begin{example} Show an example, for instance a server with two clients with different information.
%\end{example}
%}

\subsection{Operational Semantics}
\label{sec:gsp-sos}

The operational semantics of \tgspcalculus\ is given by a labelled transition system 
over well-formed terms, quotiented by the structural equivalence $\equiv$ defined as the least equivalence 
such that  $\bigpar$ is associative, commutative and has $0$ as neutral element.
The set of actions is given by the following grammar:
\[ 
\begin{array}{l}
  \action\ ::=\ \tau \ |\  \readtran{r} \ | \  \updatetran{u} \ | \ \pulltran \ | \ \pushtran \ | \ \confirmedtran \qquad
%  \actbyc \ ::=\   (\lambda,\cid)
\end{array}
\]
As usual,  $\tau$ stands for an internal, 
 unobservable action, while the remaining ones correspond to the interaction of a client with  
the  store. 
A  label  $(\lambda,\cid)$ indicates that  
 the client $\cid$ performs the action $\lambda$.  We  write  $\arro{\lambda}$ instead of $\tr{(\lambda,\cid)}$. 


\begin{figure*}[t]
{\small \[
 \begin{array}{l}
\mathrule{update}
         { \vertice \ \mbox{fresh}
         }
 	{
	\tsystem{\tclienti{\tupdateins}{\tknown}{\tpending}{\ttransactionbuffer}{\tsent}{\treceivebuffer}} 
	\ \ \,\arro{\updatetran{u}} \ \ \,
	\tsystem{\tclienti{\tprogram}{\tknown}{\tpending}{\ttransactionbuffer \cdot \tupdate}{\tsent}{\treceivebuffer}}
	}

\\[15pt]
\mathax{push}
	{
	\tsystem{\tclienti{\tpushins}{\tknown}{\tpending}{\ttransactionbuffer}{\tsent}{\treceivebuffer} }
	\ \ \, \arro{\pushtran}\ \ \,
	\tsystem{\tclienti{\tprogram}{\tknown}{\tpending \cdot \ublock}{\emptysequence}{\tsent \cdot \ublock}{\treceivebuffer}[@][{\trounds+1}]}
	}

\\[8pt]
\mathax{send}
	{
	\tsystem{\tclient{\tprogram}{\tknown}{\tpending}{\ttransactionbuffer}{\tsenthead \cdot \tsent}{\treceivebuffer}} 
	\ \arro{\tau}\ \tsystem{\tclient{\tprogram}{\tknown}{\tpending}{\ttransactionbuffer}{\tsent}{\treceivebuffer}}[\queuemessage \cdot \tsenthead]
	}

\\[7pt]
\mathrule{receive}
	{\tknown + \treceivebuffer< \text{\textbar} S \text{\textbar} } 
	{
	\tsystem{\tclient{\tprogram}{\tknown}{\tpending}{\ttransactionbuffer}{\tsent}{\treceivebuffer}} 
	\ \ \,\arro{\tau} \ \ \,
	\tsystem{\tclienti{\tprogram}{\tknown}{\tpending}{\ttransactionbuffer}{\tsent}{\treceivebuffer+1}}
	}
\\[15pt]
\mathax{pull}
	{
	\tsystem{\tclienti{\tpullins}{\tknown}{\tpending}{\ttransactionbuffer}{\tsent}{\treceivebuffer}} 
	\ \ \, \arro{\pulltran}\ \ \,
	\tsystem{\tclienti{\tprogram}{\tknown+\treceivebuffer}{\tpending \setminus \queuemessage[\tknown .. \tknown + \treceivebuffer {\ -1}]}
		{\ttransactionbuffer}{\tsent}{0}}
	}
\\[8pt]
\mathrule{read}
  	{
	\rvalue{r}{\flatten {\queuemessage[0..\tknown-1] \cdot \tpending} \cdot \ttransactionbuffer} = v
	}	
	{
	\tsystem{\tclienti{\treadins{x}{r}}{\tknown}{\tpending}{\ttransactionbuffer}{\tsent}{\treceivebuffer}}
	\  \, \arro{\readtran{\aread}} \ 
	\tsystem{\tclienti{{\tprogram}\subst{x}{v}}{\tknown}{\tpending}{\ttransactionbuffer}{\tsent}{\treceivebuffer}}
	}


\\[12pt]
\mathrule{confirm}
 %	{\eval{(\tpending \cdot  \ttransactionbuffer == \emptysequence)}  v}
	{v = {(\tpending \cdot  \ttransactionbuffer == \emptysequence)}}
	{
	\tsystem{\tclienti{\tconfirmedins{x}}{\tknown}{\tpending}{\ttransactionbuffer}{\tsent}{\treceivebuffer}} 
	\arro{\confirmedtran} 
	\tsystem{\tclienti{{\tprogram}\subst{x}{v}}
		{\tknown}{\tpending}{\ttransactionbuffer}{\tsent}{\treceivebuffer}}
	}
	
%\\[25pt]
%\mathrule{while-true}
%	{\eval \cond {true}}
%	{
%	\tsystem{\tclienti{\pwhile{\cond}{\tprogram}[Q]}{\tknown}{\tpending}{\ttransactionbuffer}{\tsent}{\treceivebuffer}}
%	\arro{\tau} 
%	\tsystem{\tclienti{\tprogram;\pwhile{\cond}{\tprogram}[Q]}{\tknown}{\tpending}{\ttransactionbuffer}{\tsent}{\treceivebuffer+1}}
%	}
%
%\\[25pt]
%\mathrule{while-false}
%	{\eval \cond {false}}
%	{
%	\tsystem{\tclienti{\pwhile{\cond}{\tprogram}[Q]}{\tknown}{\tpending}{\ttransactionbuffer}{\tsent}{\treceivebuffer}} 
%	\arro{\tau} 
%	\tsystem{\tclienti{Q}{\tknown}{\tpending}{\ttransactionbuffer}{\tsent}{\treceivebuffer+1}}
%	}
%	\\
%\henote{\mbox{Agregar reglas para if}}
 \end{array}
\]}
\vskip-10pt
\caption{Operational semantics for \tgspcalculus}
\label{fig:OS-tgsp}
\end{figure*}


We now comment on the inference rules in \figref{fig:OS-tgsp}.
When a client performs an update (rule \textsc{update}),  the change has only local effects:
the sequence of  local updates $\ttransactionbuffer$ is extended with the 
operation $\anupd$ decorated with a globally fresh identifier $\vertice$. We remark that 
decorations are used for technical reasons but they are 
 operationally irrelevant (see \secref{sec:properties-gsp}).

A client propagates its local changes to the global store by executing $\pushcmd$ (rule \textsc{push}):
all local changes in $\ttransactionbuffer$ will be transmitted as a block $\ublock$, \ie, as an atomic unit.
 Nevertheless, these changes are not  made  available immediately
 at the global store because of the asynchronous communication model. 
 In fact, the new block  $\ublock$ is added to the communication buffer $\tsent$, which 
contains all  blocks that  have not reached the global store.  Also,  
 $\ublock$ is  added  to the  pending messages $\tpending$.
Rule \textsc{send}  stands for  a  block  that finally reaches the global store. Conversely, 
rule \textsc{receive} models the reception of
 a new update. The received update is not   immediately added to the local replica. Actually, 
 each client  explicitly refreshes its local view  by executing  {\pullcmd} (rule \textsc{pull}).  
 At this time, all  previously received updates $\treceivebuffer$ are incorporated to the 
 local copy (\ie, $\tknown$ is changed to $\tknown+\treceivebuffer$). Additionally, 
 all pending updates  in the new fragment $\queuemessage[k..k+\treceivebuffer-1]$ 
 are remove from 
 %All pending blocks in 
 $\tpending$.
 % that are in the new fragment 
 %$\queuemessage[k..k+\treceivebuffer-1]$ are removed from  $\tpending$. 
% 


The semantics of operations is  defined abstractly by the interpretation function
 $\rvaluename:\readset\times\updatesets^*\rightarrow  \valueset$, 
\ie,  a function  that takes a read operation and a sequence of updates and returns 
a  value in some domain $\valueset$.  
A read operation $\aread$ is evaluated over the local state of the client (rule \textsc{read}), 
i.e., 
the known prefix of the global store $\queuemessage[0..\tknown-1]$ and the local updates in 
 $\tpending$ and $\ttransactionbuffer$. The  value $v$ is bound to the 
 variable $x$, and hence all free occurrences of $x$ in the 
 continuation $P$  are substituted by $v$.
 A client may perform $\confcmd$  to check whether its executed updates are already 
 in the global store: this operation  
  returns true only when the local buffers  $\tpending$ and $\ttransactionbuffer$ are both empty
  (rule \textsc{confirm}).
 %
%Remaining rules are standard.
%$\textsc{(t-while-trye)}$ and $\textsc{(t-while-false)}$ are standard. Rule $\textsc{(t-receive)}$ states that a counter $\treceivebuffer$ is incremented (by 1). Finally, the last one rule, $\textsc{(t-process)}$, moves updates from sent queue to the message queue.  

%In the next sections  will  make use of following operations on sequences: 
%\begin{itemize}
%    \item Relative complement:  
%    %Let $\tpending$ and $\ttransactionbuffer$ be sequence, we write $\setminus$ as the 
%    %relative complement of $\tpending$ in $\ttransactionbuffer$, \ie,
%     $\blockseq \setminus {\blockseq[']} = [\ublock | u_i \in \blockseq \wedge\ u_i\notin  {\blockseq[']} ]$. 
%     \henote{fix}
%    \item Flatten: $\flatten{\emptysequence} = \emptysequence$, and 
%    $\flatten{\ublock\cdot\blockseq} = \ttransactionbuffer\cdot\flatten\blockseq$.
%    \item subsequence: ${\ttransactionbuffer}[j.. k]$ with $j,k\in\nat$.
%    \item length: $\length\_$
%\end{itemize}


We remark that the operational semantics preserves well-formedness.

\begin{lemma}\label{lemma:gsp-wf} If $\systemterm$ is well-formed and $\systemterm\arroi{\action}\systemterm[']$, then $\systemterm[']$ is well-formed.
\end{lemma}