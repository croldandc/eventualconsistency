% !TEX root = main.tex


\section{Global Sequence Protocol Calculus}
\label{sec:gsp}

\henote{check store-store-etc.}


\subsection{Syntax}

Clients interact with a store by performing operations in $\updatesets\cup\readset$,
where the elements in $\updatesets$ denote update operations and those in $\readset$ 
stand for read operations. 
None operation can simultaneously read and update a store, therefore we assume
 $\updatesets \cap \readset= \emptyset$.
 We write 
 $\anupd, \anupd['], \anupd[''],\ldots$ for  updates in $\updatesets$  and 
$\aread, \aread['], \aread[''],\ldots$ for reads in $\readset$. 

The  state of a store is 
represented by sequences of update operations. 
For technical 
convenience (particularly in \secref{sec:properties-gsp}), we find useful to 
distinguish  different executions of the same operation. Formally, stores 
 associate each update with a fresh event identifier.
We assume a set $\verticesets$ of event identifiers  $\vertice$, $\vertice[@][_0]$, \ldots, $\vertice[@][']$,$\ldots$ 
and write $\udec$ for the update  $\anupd$ associated with the event $\vertice$.
We sometimes omit the decoration when it is not relevant.  \henote{Is this true?}

We  use $\anupdseq$ to denote sequences of decorated updates and
$\ublock$ for an atomic block of updates. 
(We reserve the term 
transaction for a different notion addressed in \secref{sec:transactions}). 
We write  $\blockseq$  for sequences of blocks and use  $\emptysequence$ for the empty sequence.
%  defined by the following grammar
% \[\tpending,{\tpending[']} ::= \emptysequence \ |\  [\alpha] \cdot \tpending\]
% \changed{}{A sequence $[\alpha] \cdot \tpending]$ stands for ....}, and we 
% use $\emptysequence$ to denote the empty sequence, \ie, the sequence of length 0.
We use usual operations on sequences such as $\blockseq{[i]}$ to denote  
 the $i$-th  element of $\blockseq$, $\blockseq{[i..j]}$
for the sequence of elements from position $i$ to $j$, $|\blockseq|$ for the length and 
$\blockseq\setminus\blockseq[']$ for the relative complement. We will use
 $\flatten{\blockseq}$ to obtain a plain sequence of updates by 
forgetting  the structure of constituent blocks, if any.
%is such that $\flatten{\emptysequence} = \emptysequence$, and 
%    $\flatten{\ublock\cdot\blockseq} = \ttransactionbuffer\cdot\flatten\blockseq$.

We rely on countable sets $\varset$ of program variables $x,x',\ldots$ 
and $\idset$ of client identifiers $\cid,\cid',\ldots,\cid_1,\ldots$.



 
 \begin{figure}
 \[
\begin{array}{l}
    \begin{array}{l@{\ }rcl}
		 \rName{naturals}
		 &
		 \multicolumn 3 l {\treceivebuffer, \tknown, \trounds \in \nat}
		 \\
    		 \rName{update} 
		 &
		 \updatesets
		 &{ =  }&
		 \{ \anupd, \anupd['], \ldots,  \anupd[_0],\ldots\}
		 \\
   		 \rName{read} 
		 &
		 \readset
		 &{ =  }&
		 \{ \aread, \aread['], \ldots,  \aread[_0],\ldots\} 
		  \\
   		 \rName{event} 
		 &
		 \verticesets
		 &{ =  }&
		 \{\vertice, \vertice[@]['], \ldots, \vertice[@][_0],\ldots\} 
		 \\
		 \rName{var}
		 &
		  \varset &{ =  }&
		 \{x,x',\ldots,x_0,\ldots\}
		 \\
		 \rName{ids}
		 &
		  \idset &{ =  }&
		 \{\cid,\cidj,\cid['],\ldots,\cid[_0],\ldots\}
\end{array}
\qquad
    \begin{array}{l@{\ }r@{}l}
    \\
		 \rName{upd seq}
		 &
		\anupdseq 
		 \ ::=\ 
			&
		 \emptysequence \ |\ \udec\cdot\anupdseq 
		 \\
		 \rName{block seq}	
		 &	 
		 \blockseq
		 \ ::=\ 
		&
		 \emptysequence\ |\ \ublock\cdot\blockseq	 
			\\
		\rName{system} 
		& \systemterm\ ::= \ 
		& \queuemessage \, \bigpar\, \Absclient
		 \\     	  \rName{store} 
			 & \queuemessage \ ::=\ 
			 &  \blockseq 
			 \\
      		\rName{client} 
			& \Absclient \ ::=\ 
			& 0 \;|\;�\tclient{\tprogram}{\tknown}{\blockseq}{\ttransactionbuffer}{\blockseq}{\treceivebuffer} \;|\;
       				  \Absclient\, \bigpar\, \Absclient %\hfill \text{with}\  n \in \nat
      \end{array}
      \\[40pt]
      \begin{array}{ll}
     		 \rName{program} 
			 \ \tprogram \ ::=\ &
			  \tupdateins \;|\; \treadins{x}{\aread} \;|\; \tpullins \;|\;
			   \\
			  & \tpushins \;|\;
			 \tconfirmedins{x} 
			%\;|\; \pwhile{\cond}{\tprogram}  \;|\;
			%\\
			%&
			%& \pifte
     \end{array}
\end{array}
  \]
\caption{Syntax of the \gsp\ calculus}
\label{fig:syntax-gsp}
\end{figure}

 \begin{definition}[GSP Language] 
The set of \tgspcalculus\ terms is given by the grammar in \figref{fig:syntax-gsp}
\end{definition}

A \gsp\ system $\systemterm$ consists of  a store and zero or more  clients.
The state of a global store $\queuemessage$ is represented by a sequence of  blocks.
The term $\tclienti{\tprogram}{\tknown}{\tpending}{\ttransactionbuffer}{\tsent}{\treceivebuffer}[\cid][n]$ 
stands for a client identified by $\cid$ and is engaged on the execution of the 
program $P$. 
The remaining elements 
are used to describe the state of the local replica: $\ttransactionbuffer$ contains the  updates 
that have been made locally and are  part of an unfinished block; $\tsent$ models the 
communication buffer, which keeps all blocks sent by the client but not received by the 
global store; 
$\tpending$ is the pending buffer, which contains all blocks completed locally but 
unconfirmed by the global store.
For simplicity, we do not explicitly replicate the portion of the global store that 
each client knows; we use instead a natural number $\tknown$ corresponding to 
 the length of the prefix of the global store known by the client. Specifically, the replicated 
 part of the global store known by $\cid$ is the 
sequence $\queuemessage[0..\tknown-1]$. Similarly, the number $\treceivebuffer$ identifies 
the new updates known by the client that needs to be added to the local replica, i.e., 
the client knows the new updates contained in the segment $\queuemessage[\tknown..\tknown+\treceivebuffer-1]$.
Value $\trounds$ counts the number of update operations executed by the client. 


A programs $P$ run by a client is  sequence of  primitives that handle the  
interacting with 
the store: $\readcmd,  \updcmd, \pullcmd,  \pushcmd, \confcmd$
 (we postpone their description until \secref{sec:gsp-sos}). 
 %Conditional and loop statements contain expressions 
%$\cond, \cond', \ldots$ written in some language that we left unspecified. We only assume   
%such language to be equipped with a valuation function  that associates each expression $\cond$  
%with a value $v$ in some domain $\valueset$, written $\eval{\cond}{v}$.  
A program $\plet{x}{\ldots}{P}$ introduces a bound variable whose  scope is $P$. The definition of 
free variables of a program is  standard. We say that a process $P$ is {\em closed} when it does not contain  
 free variables. 

We keep the language for programs simple. We remark that this 
choice do not affect the results presented in this paper. Actually, we could have just characterised 
 the behaviour of programs in terms of labelled transition systems, 
but we prefer to have a syntax for presentation. 

%\henote{agregar algun comentario sobre otras estructuras de control}



\begin{definition}[Well-formedness]
A \gsp\ system $\systemterm =  \Absclient_0\bigpar \ldots\bigpar \Absclient_m\bigpar\queuemessage$  with 
$\Absclient_l = \tclienti{\tprogram_l}{\tknown_l}{\tpending_l}{\ttransactionbuffer_l}{\tsent_l}{\treceivebuffer_l}[\cid_l][\trounds_l]$ 
for all  $l\in \{0,\ldots,m\}$  is {\em well-formed} if the  following conditions hold 
\begin{enumerate}
  \item $\cid_{l} \neq \cid_{l'}$ for all $l\neq l'$;
  \item $k_l+j_l\leq \length \queuemessage$ for all $l$;
  \item $\tpending_l = \ublock[{\anupdseq[_1]}]\cdots\ublock[{\anupdseq[_p]}]\cdot\tsent_l$ and 
  for all ${\sf 1} \leq x < y \leq{\sf p}$ there exists $x',y'$ s.t. $\queuemessage[x'] = \ublock[{\anupdseq}_x]$, $\queuemessage[y'] =\ublock[{\anupdseq}_y]$ and 
  $\tknown_l\leq x' < y'$; and
%  \item if $\ublock \in {\tpending_l}\cdot {\tsent_l}$ then $\cid = \cid_l$, \ie,  blocks in the local queues of 
%  a client $l$ are  all decorated with the identifier $\cid_l$.
%  \item $\cid \in {\cid_0,\ldots, \cid_n} $ for all $\ublock \in S$, \henote{Pensar bien, tal vez no es necesaria}
  \item $\anupdseq = \flatten{\queuemessage\cdot\tsent_0\cdots\tsent_m\cdot\ttransactionbuffer_0\cdots\ttransactionbuffer_m}$,
  if $\anupdseq{[x]}=\udec$, $\anupdseq{[y]}=\udec[@][{\vertice[@][']}]$ and $x \neq y$ then $\vertice \neq \vertice[@][']$.
\end{enumerate}
\end{definition}
Condition (1) ensures that all   clients have different identifiers. Condition (1) states that each client can see at most every messages in the store, while 
(3) ensures that all unconfirmed blocks in $\tpending_l$ are  either in the store  ($\ublock[{\anupdseq[_1]}]\cdots\ublock[{\anupdseq[_p]}]$) or in  the 
   communication buffer $\tsent_l$. Morevoer,  messages are kept in the relative order in which they have been generated. Finally, (4) ensures
   that an event  is associated with a unique update operation. 
    
   

Hereafter, we assume any system to be well-formed.

%\changed{}{
%\begin{example} Show an example, for instance a server with two clients with different information.
%\end{example}
%}

\subsection{Operational Semantics}
\label{sec:gsp-sos}

The operational semantics of \tgspcalculus\ is given by a labeled transition system 
over well-formed terms, quotiented by the structural equivalence $\equiv$ defined as the least equivalence 
such that  $\bigpar$ is associative, commutative and has $0$ as neutral element.
Transitions are be labelled by pairs $(\mu,\cid)$ where $\mu$ is the action performed 
by the client $\cid$. The set of actions is given by the following grammar:
\[ 
\begin{array}{r@{\ ::= \ }l}
  \action & \tau \ |\  \readtran{r} \ | \  \updatetran{u} \ | \ \pulltran \ | \ \pushtran \ | \ \confirmedtran \\
  \actbyc &  (\lambda,\cid)
\end{array}
\]
As usual,  $\tau$ stands for an internal, unobservable action, while the remaining actions allow 
a client to update, read from and synchronise with a global store. 
In what follows we will write labelled transitions $\arro{\mu}$ instead of $\tr{(\mu,\cid)}$. 


\begin{figure*}[t]
{\small \[
 \begin{array}{l}
\mathrule{update}
         { \vertice \ \mbox{fresh}
         }
 	{
	\tsystem{\tclienti{\tupdateins}{\tknown}{\tpending}{\ttransactionbuffer}{\tsent}{\treceivebuffer}} 
	\ \ \,\arro{\updatetran{u}} \ \ \,
	\tsystem{\tclienti{\tprogram}{\tknown}{\tpending}{\ttransactionbuffer \cdot \tupdate}{\tsent}{\treceivebuffer}}
	}

\\[20pt]
\mathax{push}
	{
	\tsystem{\tclienti{\tpushins}{\tknown}{\tpending}{\ttransactionbuffer}{\tsent}{\treceivebuffer} }
	\ \ \, \arro{\pushtran}\ \ \,
	\tsystem{\tclienti{\tprogram}{\tknown}{\tpending \cdot \ublock}{\emptysequence}{\tsent \cdot \ublock}{\treceivebuffer}[@][{\trounds+1}]}
	}

\\[12pt]
\mathax{send}
	{
	\tsystem{\tclient{\tprogram}{\tknown}{\tpending}{\ttransactionbuffer}{\tsenthead \cdot \tsent}{\treceivebuffer}} 
	\ \arro{\tau}\ \tsystem{\tclient{\tprogram}{\tknown}{\tpending}{\ttransactionbuffer}{\tsent}{\treceivebuffer}}[\queuemessage \cdot \tsenthead]
	}

\\[10pt]
\mathrule{receive}
	{\tknown + \treceivebuffer< \text{\textbar} S \text{\textbar} } 
	{
	\tsystem{\tclient{\tprogram}{\tknown}{\tpending}{\ttransactionbuffer}{\tsent}{\treceivebuffer}} 
	\ \ \,\arro{\tau} \ \ \,
	\tsystem{\tclienti{\tprogram}{\tknown}{\tpending}{\ttransactionbuffer}{\tsent}{\treceivebuffer+1}}
	}
\\[18pt]
\mathax{pull}
	{
	\tsystem{\tclienti{\tpullins}{\tknown}{\tpending}{\ttransactionbuffer}{\tsent}{\treceivebuffer}} 
	\ \ \, \arro{\pulltran}\ \ \,
	\tsystem{\tclienti{\tprogram}{\tknown+\treceivebuffer}{\tpending \setminus \queuemessage[\tknown .. \tknown + \treceivebuffer {\ -1}]}
		{\ttransactionbuffer}{\tsent}{0}}
	}
\\[12pt]
\mathrule{read}
  	{
	\rvalue{r}{\flatten {\queuemessage[0..\tknown-1] \cdot \tpending} \cdot \ttransactionbuffer} = v
	}	
	{
	\tsystem{\tclienti{\treadins{x}{r}}{\tknown}{\tpending}{\ttransactionbuffer}{\tsent}{\treceivebuffer}}
	\  \, \arro{\readtran{\aread}} \ 
	\tsystem{\tclienti{{\tprogram}\subst{x}{v}}{\tknown}{\tpending}{\ttransactionbuffer}{\tsent}{\treceivebuffer}}
	}


\\[18pt]
\mathrule{confirm}
 %	{\eval{(\tpending \cdot  \ttransactionbuffer == \emptysequence)}  v}
	{v = {(\tpending \cdot  \ttransactionbuffer == \emptysequence)}}
	{
	\tsystem{\tclienti{\tconfirmedins{x}}{\tknown}{\tpending}{\ttransactionbuffer}{\tsent}{\treceivebuffer}} 
	\arro{\confirmedtran} 
	\tsystem{\tclienti{{\tprogram}\subst{x}{v}}
		{\tknown}{\tpending}{\ttransactionbuffer}{\tsent}{\treceivebuffer}}
	}
	
%\\[25pt]
%\mathrule{while-true}
%	{\eval \cond {true}}
%	{
%	\tsystem{\tclienti{\pwhile{\cond}{\tprogram}[Q]}{\tknown}{\tpending}{\ttransactionbuffer}{\tsent}{\treceivebuffer}}
%	\arro{\tau} 
%	\tsystem{\tclienti{\tprogram;\pwhile{\cond}{\tprogram}[Q]}{\tknown}{\tpending}{\ttransactionbuffer}{\tsent}{\treceivebuffer+1}}
%	}
%
%\\[25pt]
%\mathrule{while-false}
%	{\eval \cond {false}}
%	{
%	\tsystem{\tclienti{\pwhile{\cond}{\tprogram}[Q]}{\tknown}{\tpending}{\ttransactionbuffer}{\tsent}{\treceivebuffer}} 
%	\arro{\tau} 
%	\tsystem{\tclienti{Q}{\tknown}{\tpending}{\ttransactionbuffer}{\tsent}{\treceivebuffer+1}}
%	}
%	\\
%\henote{\mbox{Agregar reglas para if}}
 \end{array}
\]}
\caption{Operational semantics for \tgspcalculus}
\label{fig:OS-tgsp}
\end{figure*}


We now comment on the inference rules in \figref{fig:OS-tgsp}, which define the operational semantics of \tgspcalculus.
When a client performs an update (rule \textsc{update}),  the changes have only local effects:
the sequence of  local updates $\ttransactionbuffer$ is extended with the 
update $\anupd$ decorated with a globally fresh identifier $\vertice$. We remark that 
decorations are used for technical reasons but they are 
 operationally irrelevant (see \secref{sec:properties-gsp}).

A client propagates its local changes to the global store by executing $\pushcmd$ (rule \textsc{push}):
all local changes in $\ttransactionbuffer$ will be transmitted as a block $\ublock$, \ie, as an atomic unit.
 Nevertheless, these changes are not  made immediately available 
 at the global store because of the asynchronous communication model. 
 In fact, the new block  $\ublock$ is added to the communication buffer $\tsent$, which 
contains all sent blocks that still have not reached the global store.  Also,  
 $\ublock$ is  added  to the  sequence of pending messages $\tpending$ to 
  be used  until the block is finally added to the store to evaluate locally in the subsequent read. 

Rule \textsc{send}  stands for  a  block  that finally reaches the global store. Conversely, 
rule \textsc{receive} models the reception of
 a new update. The received update is not immediately incorporated to the local replica. Actually, 
 each client  explicitly refreshes its local view of the global store by executing  {\pullcmd} (rule \textsc{pull}).  
 At this time, the  previously received updates $\treceivebuffer$ are incorporated to the 
 local copy (\ie, $\tknown$ is changed to $\tknown+\treceivebuffer$). Additionally, 
 all pending updates  in the new fragment $\queuemessage[k..k+\treceivebuffer-1]$ 
 are remove from 
 %All pending blocks in 
 $\tpending$.
 % that are in the new fragment 
 %$\queuemessage[k..k+\treceivebuffer-1]$ are removed from  $\tpending$. 
% 


The semantics of operations is  defined abstractly by the interpretation function
 $\rvaluename:\readset\times\updatesets^*\rightarrow  \valueset$, 
\ie,  a function  that takes a read operation and a sequence of updates and returns 
a  value in some domain $\valueset$.  
A read operation $\aread$ is evaluated over the local state of the client (rule \textsc{read}), 
i.e., 
the known prefix of the global store $\queuemessage[0..\tknown-1]$ and the local updates in 
 $\tpending$ and $\ttransactionbuffer$. The  value $v$ is bound to the 
 variable $x$, and hence all free occurrences of $x$ in the 
 continuation $P$  are substituted by $v$.
 %
 A client may perform $\confcmd$  to check whether its executed updates have been already 
 applied to the global store: this operation  
  returns true only when the local buffers  $\tpending$ and $\ttransactionbuffer$ are both empty.
 %
%Remaining rules are standard.
%$\textsc{(t-while-trye)}$ and $\textsc{(t-while-false)}$ are standard. Rule $\textsc{(t-receive)}$ states that a counter $\treceivebuffer$ is incremented (by 1). Finally, the last one rule, $\textsc{(t-process)}$, moves updates from sent queue to the message queue.  

%In the next sections  will  make use of following operations on sequences: 
%\begin{itemize}
%    \item Relative complement:  
%    %Let $\tpending$ and $\ttransactionbuffer$ be sequence, we write $\setminus$ as the 
%    %relative complement of $\tpending$ in $\ttransactionbuffer$, \ie,
%     $\blockseq \setminus {\blockseq[']} = [\ublock | u_i \in \blockseq \wedge\ u_i\notin  {\blockseq[']} ]$. 
%     \henote{fix}
%    \item Flatten: $\flatten{\emptysequence} = \emptysequence$, and 
%    $\flatten{\ublock\cdot\blockseq} = \ttransactionbuffer\cdot\flatten\blockseq$.
%    \item subsequence: ${\ttransactionbuffer}[j.. k]$ with $j,k\in\nat$.
%    \item length: $\length\_$
%\end{itemize}


We remark that the operational semantics preserves well-formedness.

\begin{lemma} If $\systemterm$ is well-formed and $\systemterm\arroi{\action}\systemterm[']$, then $\systemterm[']$ is well-formed.
\end{lemma}