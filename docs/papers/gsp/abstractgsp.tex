% !TEX root = main.tex


\section{Global Sequence Protocol Calculus}
\label{sec:gsp}

\subsection{Syntax}

We rely on countable sets $\varset$ of program variables $x,x',\ldots$ 
and $\idset$ of client identifiers $\cid,\cid',\ldots,\cid_1,\ldots$.
Clients interact with a storage by performing operations in $= \updatesets\cup\readset$,
where the elements in $\updatesets$ denote update operations and those in $\readset$ 
stand for read operations. 
None operation can simultaneously read and update a store, therefore we assume
 $\updatesets \cap \readset= \emptyset$.
 We write 
 $u, u', u'',\ldots$ for  updates in $\updatesets$  and 
$r, r', r'',\ldots$ for reads in $\readset$. 

The global state of a store and the local  replicas at each client are 
represented by sequences of update operations. 
For technical 
convenience (particularly in \secref{sec:properties-gsp}), we find useful to 
distinguish  different executions of the same operation. Formally, stores 
will associate each update with a fresh event identifier.
We assume a set $\verticesets$ of event identifiers  $\vertice$, $\vertice_0$, \ldots, $\vertice'$,$\ldots$ 
and write $\udec$ for the update operation $u$ corresponding to the event $\vertice$.
We sometimes omit the decoration when it is irrelevant for the context.  

We  use $\ttransactionbuffer$ to denote sequences of decorate updates and
$\ublock$ for an atomic block of updates performed by a client. 
(We reserve the term 
transaction for a different notion addressed in \secref{sec:transactions}). 
We use  $\tpending$ and $\tsent$ 
for sequences of blocks.
Formally,
\[ \ttransactionbuffer := \epsilon \ |\ \udec\cdot\ttransactionbuffer 
    \qquad\qquad 
    \tpending,{\tsent}:= \epsilon\ |\ \ublock\cdot\tpending\]
where  $\epsilon$ is the empty sequence.
%  defined by the following grammar
% \[\tpending,{\tpending[']} ::= \epsilon \ |\  [\alpha] \cdot \tpending\]
% \changed{}{A sequence $[\alpha] \cdot \tpending]$ stands for ....}, and we 
% use $\epsilon$ to denote the empty sequence, \ie, the sequence of length 0.
 
In the next sections  will  make use of following operations on sequences: 
\begin{itemize}
    \item Relative complement:  
    %Let $\tpending$ and $\ttransactionbuffer$ be sequence, we write $\setminus$ as the 
    %relative complement of $\tpending$ in $\ttransactionbuffer$, \ie,
     $\blockseq \setminus {\blockseq[']} = [\ublock | u_i \in \blockseq \wedge\ u_i\notin  {\blockseq[']} ]$. 
     \henote{fix}
    \item Flatten: $\flatten{\epsilon} = \epsilon$, and 
    $\flatten{\ublock\cdot\blockseq} = \ttransactionbuffer\cdot\flatten\blockseq$.
    \item subsequence: ${\ttransactionbuffer}[j.. k]$ with $j,k\in\nat$.
    \item length: $\length\_$
\end{itemize}



 \begin{definition}[GSP Language] 
The set of \tgspcalculus\ terms is given by the following grammar:
 
  \[
    \begin{array}{l@{\quad}r@{}l}
		 \rName{store} 
			 & \queuemessage \ ::=\ 
			 &  \blockseq 
			 \\
		 \rName{program} 
			 & \tprogram \ ::=\ 
			 & \tupdateins \;|\; \treadins{x}{r} \;|\; \tpullins \;|\; \tpushins \;|\;
			 \\
			&   
			& \tconfirmedins{x} \;|\; \pwhile{\cond}{\tprogram}  \;|\;
			\\
			&
			& \pifte
			\\
      		\rName{client} 
			& \Absclient \ ::=\ 
			& 0 \;|\;�\tclient{\tprogram}{n}{\blockseq}{\ttransactionbuffer}{\blockseq}{n} \;|\;
       				  \Absclient\bigpar \Absclient \hfill \text{with}\  \tknown, \treceivebuffer\ \in \nat
			\\
		\rName{system} 
		& \systemterm\ ::= \ 
		& \queuemessage \bigpar \Absclient
      \end{array}
  \]
\end{definition}

The state of a global store $\queuemessage$ is represented by a sequence of  blocks.
Programs $P$ run by clients are written in a simple sequential language enriched with  primitives 
for interacting with 
the store: $\readcmd,  \updcmd, \pullcmd,  \pushcmd, \confcmd$
 (we postpone their description until \secref{sec:gsp-sos}). Conditional and loop statements contain expressions 
$\cond, \cond', \ldots$ written in some language that we left unspecified. We only assume   
such language to be equipped with a valuation function  that associates each expression $\cond$  
with a value $v$ in some domain $\valueset$, written $\eval{\cond}{v}$.  
A program $\plet{x}{\ldots}{P}$ introduces a bound variable whose  scope is $P$. The definition of 
free variables of a program is  standard. We say that a process $P$ is {\em closed} when it does not contain  
 free variables.


The term $\tclienti{\tprogram}{\tknown}{\tpending}{\ttransactionbuffer}{\tsent}{\treceivebuffer}$ 
stands for a client that has identifier $\cid$ and it is engaged in the execution of the 
program $P$. 
The remaining elements 
describe the state of the local replica: $\ttransactionbuffer$ contains the  updates 
that have been made locally and are  part of an unfinished block; $\tsent$ contains the locally 
completed blocks that have not reached the global store; 
$\tpending$ are the locally completed blocks that await confirmation about 
their successful application to the global storage.
 For simplicity, we avoid to explicitly replicate the known 
part of the global state in each client, instead we use natural numbers for identifying the 
portion of the global state that is known by each client. Specifically, 
$k$ identifies the portion of the global 
store that is replicated locally, i.e., $\queuemessage[0..k-1]$; and $j$ stands for the number of new updates 
in the global store that have been received and needs to be added to the local replica, i.e., $\queuemessage[k..k+j-1]$.
A system $\systemterm$ is composed by a store and a  set of clients. 

A system $\systemterm =  \Absclient_0\bigpar \ldots\bigpar \Absclient_n\bigpar\queuemessage$  with 
$\Absclient_l = \tclienti{\tprogram_l}{\tknown_l}{\tpending_l}{\ttransactionbuffer_l}{\tsent_l}{\treceivebuffer_l}[\cid_l]$ 
for all $0\leq j \leq n$  is {\em well-formed} if it 
satisfies the following conditions: 
\begin{itemize}
  \item $\cid_l \neq \cid_k$ for all $j\neq k$, \ie,  clients have different identifiers;
  \item $k_l+j_l\leq \length \queuemessage$ for all $l$, \ie,  a client see at most every messages in the storage;
%  \item if $\ublock \in {\tpending_l}\cdot {\tsent_l}$ then $\cid = \cid_l$, \ie,  blocks in the local queues of 
%  a client $l$ are  all decorated with the identifier $\cid_l$.
%  \item $\cid \in {\cid_0,\ldots, \cid_n} $ for all $\ublock \in S$, \henote{Pensar bien, tal vez no es necesaria}
\end{itemize}
\changed{}{
\begin{example} Show an example, for instance a server with two clients with different information.
\end{example}
}

\subsection{Operational Semantics}
\label{sec:gsp-sos}

The operational semantics of \tgspcalculus\ is given by a labeled transition system 
over well-formed terms, quotiented by the structural congruence $\equiv$ defined as the least congruence 
such that  $\bigpar$ is associative, commutative and has $0$ as neutral element.
Transitions will be labelled by pairs $(\mu,\cid)$ where $\mu$ is the action performed 
by the client $\cid$. The set of actions is defined by the following grammar:
\[ 
\begin{array}{r@{\ ::= \ }l}
  \action & \tau \ |\  \readtran{r} \ | \  \updatetran{u} \ | \ \pulltran \ | \ \pushtran \ | \ \confirmedtran \\
  \actbyc &  (\mu,\cid)
\end{array}
\]
As usual,  $\tau$ stands for an internal, unobservable action, while the remaining actions allow 
a client to update to, read from and synchronise with a global store. 
In what follows we will write labelled transitions $\arro{\mu}$ instead of $\tr{(\mu,\cid)}$. 


\begin{figure*}[tp]
 \[
 \begin{array}{l}
\mathrule{update}
         { \vertice \ \mbox{fresh}
         }
 	{
	\tsystem{\tclienti{\tupdateins}{\tknown}{\tpending}{\ttransactionbuffer}{\tsent}{\treceivebuffer}} 
	\arro{\updatetran{u}} 
	\tsystem{\tclienti{\tprogram}{\tknown}{\tpending}{\ttransactionbuffer \cdot \tupdate}{\tsent}{\treceivebuffer}}
	}

\\[25pt]
\mathax{push}
	{
	\tsystem{\tclienti{\tpushins}{\tknown}{\tpending}{\ttransactionbuffer}{\tsent}{\treceivebuffer} }
	\arro{\pushtran} 
	\tsystem{\tclienti{\tprogram}{\tknown}{\tpending \cdot [\ttransactionbuffer]}{\epsilon}{\tsent \cdot [\ttransactionbuffer]}{\treceivebuffer}[n+1]}
	}

\\[25pt]
\mathax{send}
	{
	\tsystem{\tclient{\tprogram}{\tknown}{\tpending}{\ttransactionbuffer}{[\tsenthead] \cdot \tsent}{\treceivebuffer}} 
	\arro{\tau} \tsystem{\tclient{\tprogram}{\tknown}{\tpending}{\ttransactionbuffer}{\tsent}{\treceivebuffer}}[\queuemessage \cdot [\tsenthead]]
	}

\\[25pt]
\mathrule{receive}
	{\tknown + \treceivebuffer< \text{\textbar} S \text{\textbar} } 
	{
	\tsystem{\tclient{\tprogram}{\tknown}{\tpending}{\ttransactionbuffer}{\tsent}{\treceivebuffer}} 
	\arro{\tau} 
	\tsystem{\tclienti{\tprogram}{\tknown}{\tpending}{\ttransactionbuffer}{\tsent}{\treceivebuffer+1}}
	}


\\[25pt]
\mathax{pull}
	{
	\tsystem{\tclienti{\tpullins}{\tknown}{\tpending}{\ttransactionbuffer}{\tsent}{\treceivebuffer}} 
	\arro{\pulltran}
	\tsystem{\tclienti{\tprogram}{\tknown+\treceivebuffer}{\tpending \setminus \queuemessage[\tknown .. \tknown + \treceivebuffer {\ -1}]}
		{\ttransactionbuffer}{\tsent}{0}}
	}

\\[25pt]
\mathrule{read}
  	{
	\rvalue{r}{\flatten {\queuemessage[0..\tknown-1] \cdot \tpending} \cdot \ttransactionbuffer} = v
	}	
	{
	\tsystem{\tclienti{\treadins{x}{r}}{\tknown}{\tpending}{\ttransactionbuffer}{\tsent}{\treceivebuffer}}
	\arro{\readtran{r}} 
	\tsystem{\tclienti{{\tprogram}\subst{x}{v}}{\tknown}{\tpending}{\ttransactionbuffer}{\tsent}{\treceivebuffer}}
	}


\\[25pt]
\mathrule{confirm}
 	{\eval{(\tpending \cdot  \ttransactionbuffer == \epsilon)}  v}
	{
	\tsystem{\tclienti{\tconfirmedins{x}}{\tknown}{\tpending}{\ttransactionbuffer}{\tsent}{\treceivebuffer}} 
	\arro{\confirmedtran} 
	\tsystem{\tclienti{{\tprogram}\subst{x}{v}}
		{\tknown}{\tpending}{\ttransactionbuffer}{\tsent}{\treceivebuffer}}
	}
	
\\[25pt]
\mathrule{while-true}
	{\eval \cond {true}}
	{
	\tsystem{\tclienti{\pwhile{\cond}{\tprogram}[Q]}{\tknown}{\tpending}{\ttransactionbuffer}{\tsent}{\treceivebuffer}}
	\arro{\tau} 
	\tsystem{\tclienti{\tprogram;\pwhile{\cond}{\tprogram}[Q]}{\tknown}{\tpending}{\ttransactionbuffer}{\tsent}{\treceivebuffer+1}}
	}

\\[25pt]
\mathrule{while-false}
	{\eval \cond {false}}
	{
	\tsystem{\tclienti{\pwhile{\cond}{\tprogram}[Q]}{\tknown}{\tpending}{\ttransactionbuffer}{\tsent}{\treceivebuffer}} 
	\arro{\tau} 
	\tsystem{\tclienti{Q}{\tknown}{\tpending}{\ttransactionbuffer}{\tsent}{\treceivebuffer+1}}
	}
	\\
\henote{\mbox{Agregar reglas para if}}
 \end{array}
\]
\caption{Operational semantics for \tgspcalculus}
\label{fig:OS-tgsp}
\end{figure*}


We now comment on the inference rules in \figref{fig:OS-tgsp}, which define the operational semantics of \tgspcalculus.
When a client performs an update $u$ (rule \textsc{update}), the change has effect only locally. This is achieved
 by extending the local sequence of updates $\alpha$
with the operation $u$ decorated with a globally fresh identifier $\vertice$.  
%
A client invokes  the primitive $\pushcmd$ to  propagate its local changes to the global store. All changes in the local sequence $\alpha$ are
 transmitted as an atomic unit, \ie, as a block $\ublock$ (rule \textsc{push}). Nevertheless, those changes are not  available immediately
 at the global storage because of the asynchronous nature of the communication model. This is modelled by using the communication buffer $\tsent$, which 
contains all those blocks sent by the client that still have not reached the global store.  Note also that the 
completed block $\ublock$ is added  to the local sequence $\tpending$, which will be temporarily used
  in subsequent reads until  the client updates its view of the global store. 
% 
Rule \textsc{send} models the fact that a sent block is finally incorporated to the global state. Conversely, rule \textsc{receive} models a client 
receiving a message containing a new update incorporated to the global store. Note that the update is not immediately incorporated to the local copy. The received messages will 
be processed when the client executes the operation pull (rule \textsc{pull}).  At this moment, the local view of the global store considers the 
$j$ blocks previously received. In addition, all temporary blocks in $\tpending$ that are already in the newly discovered fragment of the 
store $\queuemessage[k..k+j-1]$ are removed from  $\tpending$. 
% 
The behaviour of a client performing a reading operation $r$  is defined by rule textsc{read}.  The underlying idea is that the 
semantics of operations is abstractly defined by an interpretation function $\rvaluename:\readset\times\updatesets^*\rightarrow  \valueset$, 
\ie,  a function  that takes a read operation and a sequence of updates, and returns the value that results from applying all the updates in the sequence to the initial state of the store. Note that the operation $r$ is evaluated over the client's local view of the global state, that is, 
the known part of the server state $\queuemessage[0..\tknown-1]$ and the local updates in 
 $\tpending$ and $\ttransactionbuffer$. The obtained value $v$ is bound to the variable $x$, and hence all free occurrences of $x$ in the 
 continuation $P$  are substituted by $v$.
 %
 Primitive $\confcmd$ allows a client to check whether the local updates have been already applied to the global state. In fact, this  
 primitive returns true when the sequences  containing local changes $\tpending$ and $\ttransactionbuffer$ are both empty.
 %
Remaining rules are standard.
%$\textsc{(t-while-trye)}$ and $\textsc{(t-while-false)}$ are standard. Rule $\textsc{(t-receive)}$ states that a counter $\treceivebuffer$ is incremented (by 1). Finally, the last one rule, $\textsc{(t-process)}$, moves updates from sent queue to the message queue.  

