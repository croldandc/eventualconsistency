% !TEX root = main.tex
\section{Implementation of {\gsp}}

The \tgspcalculus\ model describes an idealised system that abstracts  
away from several  implementation details such as non-optimised 
representation of the state and unreliable communication. 
This section presents a formal model for the implementation   
proposed in\henote{~\cite{}}. 


\subsection{Syntax}
The implementation of \gsp\, 
 relies on a  compact representation for 
states and updates. Their precise definition highly depends
on the datatype of the values handled by the store,
%. \henote{agregar ejemplo?}. 
but they are 
%In order to keep the description of the model as general as possible, they have been
characterised  in terms of 
two abstract types:  $\statetype$ and $\deltatype$, which provides
 the following  operations\henote{in~\cite{}}:

{\small
\[
\begin{array}{lll}
%	\textbf{const} & 
	\emptydelta & : \deltatype 
	\\
%	\textbf{function} & 
	\iappendname & : \partialfunction{\deltatype \times \updatetype}{\deltatype} 
	\\
%	\textbf{function} & 
	\ireducename & : \partialfunction{\deltatype^*}{\deltatype} \\
\end{array}
\qquad\quad
\begin{array}{ll}
%	\textbf{const} & 
	\initialstate & : \statetype 
	\\
%	\textbf{function} & 
	\iapplyname & : \partialfunction{\statetype \times \deltatype^*}{\statetype} 
	\\
%	\textbf{function} & 
	\ireadname & : \partialfunction{\readtype \times \statetype}{\valuetype} 
	\\
\end{array}
\] }

Constants $\emptydelta$ and  $\initialstate$  denote the 
empty elements in their respective types. An object
$\adelta \in \deltatype$ describes the effects of a sequence of updates and 
is built  by either appending an update  to an existing delta ($\iappendname$) 
or  combining together several deltas  ($\ireducename$).
Operation $\ireadname$ is the 
interpretation function for read operations (\ie, the implementation counterpart of  $\rvalue{\_}{\_}$  in the 
idealised model) and $\iapplyname$ corresponds to state transformations.
%
%We shall use  $\astate,\astate['],\ldots, \astate[{_1}],\ldots$ to denote values of  type $\statetype$ and 
%$\adelta,\adelta['],\ldots, \adelta[{_1}],\ldots$ for values of type $\pushbuffertype$.

Clients and the global store exchange $\adelta$ objects to communicate changes. As 
each single $\adelta$ may correspond to several update operations, clients
send each $\adelta$  accompanied by its own identifier and a sequence number $\irounds$. 
Precisely, clients send rounds, i.e.  triples $\around = \aroundtuple$. Differently,
the global store sends segments $\aseg = \agssegpair$, in which a $\adelta$ is accompanied
by a function $\amxrf\in\idset\rightarrow \nat$ stating that $\adelta$ corresponds to 
the changes sent by each client $\cid$ until the round $\amxrf(\cid)$. To 
deal with crashes and recovery, the server may send segments of the form
$\agsprefpair$, communicating a state instead of a delta object. 

 \begin{definition}[GSP Language] 
The set of \igsp\ terms is given by the grammar in \figref{fig:syntax-igsp}.
\end{definition}
 \begin{figure}[t]
 \[
\begin{array}{l}
    \begin{array}{l@{\ }rcl}
	\rName{state}
	&
	\multicolumn 3 l {\astate,\astate['],\ldots, \astate[{_1}],\ldots \ \in\  \statetype}
	\\
    	\rName{delta} 
	&
	\multicolumn 3 l {\adelta,\adelta['],\ldots, \adelta[{_1}],\ldots \ \in\  \deltatype}
       \\
       \rName{round}
       &
       \around &::= & �\aroundtuple
        \\
       \rName{rnd Seq}
       &
       \aseqround &::= & \epsilon \ |\ \around\cdot\aseqround
	\\
	\rName{max rnd}
       &
       \multicolumn 3 l {\amxrf,\amxrf['],\ldots, \amxrf[{_1}],\ldots \,  \in \  \idset\rightarrow \nat}
        \\
       \rName{segment}
       &
       \aseg & ::= &\agssegpair  \; |\;  \agsprefpair 
%        \\
%       &
%       {\inclient} &::= &  \aseqseg%{{\agsseg}^* \cup \agspref}
%       \\
%       & 
%       \outclient &\in & {\aseqround}

\end{array}
\quad
    \begin{array}{l@{\ }rcl}
       \rName{sgmt seq}
       &
       \aseqseg &  \ ::=\  &\epsilon \ |\ \aseg\cdot\aseqseg
\\       \rName{in srv}
       &
       \inserver &\ \in\ & \partialfunction{\idset}\aseqround%\outclient
       \\
       \rName{out srv}
       & 
       \outserver\ &\ \in\ & \partialfunction{\idset}\aseqseg%\inclient
       \\
 	\rName{system} & \isystemterm & \ ::=\  & \iserver\ \bigpar\ \iclient \\
	\rName{server} & \iserver &\ ::=\  & \server{\astate}{\inserver}{\outserver} \\
	\rName{clients}& \iclient &\ ::=\  &\zero \;|\; 
	 						%\client{P}{<\astate,\pendingtype,\pushbuffertype,\transactionbuffertype,\receivebuffertype,n,\inqueue,\outqueue>} 
							\addid{\iclientsyntax}
							\;|\; \ 
							\iclient \bigpar \iclient  \\
      \end{array}
\end{array}
  \]
\caption{Syntax of the \igsp\ calculus}
\label{fig:syntax-igsp}
\end{figure}

As for \gsp, a system is composed by a global store $\iserver$ and possibly many clients $\iclient$. 
A global store is modelled by a tuple $\server{\astate}{\inserver}{\outserver}$ containing a state $\astate$, 
a function $\amxrf$ to track of the changes received from the clients and already apply to the state,
$\inserver$ and $\outserver$ stands for the communication buffers. There are two dedicated buffers for 
 each client $\cid$: $\inserver(\cid)$ contains the  rounds received from $\cid$,  and $\outserver(\cid)$ has
  the  segments sent to $\cid$. 
  


%In addition,  we will use partial functions  $\amxrf,\amxrf['],\ldots, \amxrf[{_1}],\ldots$ in  $\idset\rightarrow \nat$
%assigning clients with a natural number denoting \henote{the number of the last round exchanged with ...}
%
%%Let $\idset$ be a set of clients' name ranged over by $b$, $b_1$, $b_2$, $\ldots$; 
%
%The global state of the store is now represented by a pair $\gsprefixtype$ be a set over sets \statetype\ $\times$ $\idset$ $\rightarrow$ $\nat$, \gssegmenttype\ be a set operation over sets \deltatype $\times$ $\idset$ $\rightarrow$ $\nat$ and \roundtype\ the set over sets $\idset$ $\times$ $\nat$ $\times$ \deltatype; $\emptygssegment$\ is a fresh element of \gssegmenttype.
%%
%\henote{Explain gsspref... round}
%
%%Furthermore, we assume the following parametric functions defined over the abstract data types above borrowed from :
%
%
%%Let $\pendingtype$, $\sigma$ be sequence defined:
%
%%$\pendingtype$,$\sigma$ := $\epsilon$ \text{\textbar} [$\delta$]$\cdot$ $\pendingtype$
% 
%
%We assume the following countable sets of persisted state $\gsprefixtype$ ranged over by $\agspref$, $\agspref_1$, $\ldots$; 
%
%%\begin{figure}[t]
%%\[
%%   \begin{array}{l@{\quad}r@{\;::=\;}l}
%%	\rName{system} & \isystemterm &  \iserver\ \bigpar\ \iclient \\
%%	\rName{server} & \iserver & \server{\astate}{\inserver}{\outserver} \\
%%	\rName{clients}& \iclient & \zero \;|\; 
%%	 						%\client{P}{<\astate,\pendingtype,\pushbuffertype,\transactionbuffertype,\receivebuffertype,n,\inqueue,\outqueue>} 
%%							\addid{\iclientsyntax}
%%							\;|\; \ 
%%							\iclient \bigpar \iclient  \\
%%%			 (\textsc{program}) & P & \zero \;|\; \readins{x}{r} \;|\; \updateins{u} \;|\; 	\pushins \;|\; \pullins
%%     \end{array}
%%\]
%%\end{figure}
%%
%
% \begin{definition}[Implementation of GSP] 
%	The syntax of clients and server is given by the following grammar 
%   \end{definition}
% 
% \henote{
%The implementation of a system is a server and clients interacting concurrently. A server will be represented with a persisted state denoted by $\persistedstate$, input messages $\inserver$ and output messages $\outserver$.
%We will refer to clients, as tuple of program $P$ and set of states and sequences $E$. A program client stores a state ($E.\known$), a pending queue ($E.\pending$),i.e., updates sent to the server without confirmation of their reception, a push buffer ($E.\pushbuffer$) which holds updates that were pushed by the client but have not been sent to the server, a transaction buffer ($E.\transactionbuffer$) which holds updates for sending with another ones, a receive buffer ($E.\receivebuffertype$) for updates which were sent by the server, the number of round sent ($E.\nround$), a input message queue ($E.\inclient$) and a output message queue ($E.\outclient$).
%}

\subsection{Operational Semantics}

The operational semantics of \gspcalculus\ is defined by a labeled transition system 
over well-formed terms, up-to the structural congruence.

%The labeled transition system considers the following actions:
%\[ 
%\begin{array}{r@{\ ::= \ }l}
%  \alpha & \tau \ | \ \readtran{r} \ | \  \updatetran{u} \ | \ \pulltran \ | \ \pushtran \ | \ \confirmedtran 
%  \\
%\end{array}
%\]



% \[
% \begin{array}{l}
%   \hspace{-.3cm} \textsc{CLIENTS}\\
%%		\mathax{update}{\clientr{\updateins{u}}{\stateclient} \arro{\updatetran{u}} \clientr{P}{\update{E}{\transactionbuffer}{\appendplus{E.\transactionbuffer}{u}}}} \\[15pt]
%%		
%	\mathax{update}
%		{\begin{array}{l}
%		\addid{\iclientinst[\tupdateins]} \ \bigpar \ \isystemterm
%		\arro{\updatetran{u}} 
%		\\[5pt]
%		\hspace{5.5cm}
%		{\addid{\iclientinst[P][@][@][@][\iappend{\transactionbuffer}{u}]} \ \bigpar \   \isystemterm}
%		\end{array}}
%\\
%% 		\mathax{push}{\clientr{\pushins}{\stateclient} \arro{\pushtran} \clientr{P}{\updatethree{E}{\pushbuffer}{\reduce{E.\pushbuffer\cdot E.\transactionbuffer}}{\transactionbuffer}{\epsilon}{\nround}{E.\nround+1}  }} \\[25pt]
%	\mathax{push}
%		{\begin{array}{l}
%		\addid{\iclientinst[\tpushins]} \ \bigpar \ \isystemterm
%		\arro{\pushtran} 
%		\\[5pt]
%		\hspace{5cm}
%		{\addid{\iclientinst[P][@][@][\ireduce{\pushbuffer\cdot \transactionbuffer}][\epsilon][@][n+1]}\ \bigpar \ \isystemterm}
%		\end{array}}
%%\updatethree{E}{\pushbuffer}{\appendplus{\pushbuffer}{\transactionbuffer}}{\transactionbuffer}{\epsilon}{\nround}{\nround+1}
%\\
%%	 {\clientr{P}{\stateclient} \arro{\sendtran} \clientr{P}{\updatethree{E}{\pending}{E.\pending \cdot E.\pushbuffer}{\pushbuffer}{\epsilon}{\outclient(i)}{\outclient(i) \cdot \round{i}{E.\nround}{E.\pushbuffer}}}}
%
%	 \mathrule{send}
%	         {\pushbuffer \neq \epsilon}
% 		{\begin{array}{l}
%		\addid{\iclientinst} \ \bigpar \ \isystemterm
%		\arro{\tau} 
%		\\[5pt]
%		\hspace{5.3cm}
%		\addid{\iclientinst[@][@][\pending \cdot \pushbuffer][\epsilon][@][@][@][@][\outclient \cdot {\aroundtuple[\cid][\nround][\pushbuffer]}]}\ \bigpar \ \isystemterm
%	 	\end{array}
%		}	
%\\
%%		\mathrule{receive}{E.\inclient = \gs_0 \cdot \gs_t}{\clientr{P}{\stateclient} \arro{\receive} \clientr{P}{\updatetwo{E}{\receivebuffer}{E.\receivebuffer \cdot \gs_0)}{\inclient}{\gs_t)}}}\\[25pt]
%	\mathax{receive}
% 		{\begin{array}{l}
%		\addid{\iclientinst[@][@][@][@][@][@][@][\aseg \cdot \inclient]} \ \bigpar \ \isystemterm
%		\arro{\tau} 
%		\\[5pt]
%		\hspace{6.4cm}
%		\addid{\iclientinst[@][@][@][@][@][\receivebuffer \cdot \aseg]}\ \bigpar \ \isystemterm
%	 	\end{array}
%		}	
%\\
%
%%		\mathrule{pull}{E.\inclient(i) \neq \undefined \qquad E.\outclient(i) \neq \undefined \qquad |E.\receivebuffer| > 0}{\clientr{\pullins}{\stateclient}\arro{\pulltran} \clientr{P}{\updatethree{E}{\known}{\reducestate{E.\known}{E.\receivebuffer}}{\receivebuffer}{\epsilon}{E.\pending}{\remove{E.\pending}{E.\receivebuffer}}}}
%	\mathaxiom{pull}
%		 {\begin{array}{l}
%			\addid{\iclientinst[\tpullins][@][@][@][@][{\agssegpair[x]}\cdot\receivebuffer]}\ \bigpar \ \isystemterm
%			\arro{\pulltran} 
%			\\[5pt]
%			\hspace{5.2cm}
%			\addid{\iclientinst[P]
%						 [\iapply{\known}{x}]
%						 [{\pending}\setminus x]
%						 [@][@]
%						 [{\receivebuffer}]}\ \bigpar \ \isystemterm
%	 	\end{array}
%		}
%		\\
%\henote{\mbox{el apply no est� en el paper, ellos usan reducestate}}
%\\
%
%%		\mathrule{read}{\readplus{r}{\curstate{\known}{\pending}{\pushbuffer}{\transactionbuffer}}}{\clientr{\readins{x}{r}}{\stateclient} \arro{\readtran{r}} \clientr{\update{P}{x}{v}}{\stateclient}}
%	\mathrule{read}
%		{\iread{r}{\curstate{\known}{\pending}{\pushbuffer}{\transactionbuffer}}}
%		{\begin{array}{l}
%		\addid{\iclientinst[\treadins{x}{r}]} \ \bigpar \ \isystemterm
%		\arro{\readtran{r}} 
%		\\[5pt]
%		\hspace{6.1cm}
%		\addid{\iclientinst[{P}\subst{x}{v}]}\ \bigpar \ \isystemterm
%		\end{array}
%		}
%\\
%%		\mathrule{pull}{E.\inclient(i) \neq \undefined \qquad E.\outclient(i) \neq \undefined \qquad |E.\receivebuffer| > 0}{\clientr{\pullins}{\stateclient}\arro{\pulltran} \clientr{P}{\updatethree{E}{\known}{\reducestate{E.\known}{E.\receivebuffer}}{\receivebuffer}{\epsilon}{E.\pending}{\remove{E.\pending}{E.\receivebuffer}}}}
%%	\mathrule{pull}
%%		{\appf{\inqueue}{\cid} \neq \undefined \qquad\qquad \appf{\outqueue}{\cid} \neq \undefined \qquad\qquad |\receivebuffer| > 0}
%%		{
%%			\addid{\iclientinst[\tpullins]}
%%			\arro{\pulltran} 
%%			\addid{\iclientinst[P]
%%						 [\reducestate{\known}{\receivebuffer}]
%%						 [\remove{\pending}{\receivebuffer}]
%%						 [@][@]
%%						 [{\epsilon}]
%%				}
%%		}
%%		\\
%%\\[15pt]
%%		\mathax{confirmed}{\clientr{\confirmedins{x}}{\stateclient} \arro{\confirmedtran}
% %\clientr{\update{P}{x \mapsto E.\pending = \epsilon \lor E.\pushbuffer = \epsilon \lor E.\transactionbuffer = \epsilon}{\textbf{true}}}{\stateclient}} 
%		\mathrule{confirm}
%		{ v = (\pending \cdot \pushbuffer \cdot\transactionbuffer == \epsilon) }
%		{\begin{array}{l}
%			\addid{\iclientinst[\tconfirmedins]} \bigpar \ \isystemterm
%			\arro{\confirmedtran} 
%			\\[5pt]
%			\hspace{6.1cm}
%			\addid{\iclientinst[P\subst{x}{v}]} \bigpar \ \isystemterm
%		\end{array}
%		}
%\\[25pt]
%\mathrule{while-true}
%	{\eval \cond {true}}
%	{
%		{\begin{array}{l}
%			\addid{\iclientinst[\pwhile{\cond}{P}[Q]]} \bigpar \ \isystemterm
%			\arro{\confirmedtran} 
%			\\[5pt]
%			\hspace{4.5cm}
%			\addid{\iclientinst[P;\pwhile{\cond}{P}[Q]]} \bigpar \ \isystemterm
%		\end{array}
%		}
%%	\tsystem{\tclienti{\pwhile{\cond}{\tprogram}[Q]}{\tknown}{\tpending}{\ttransactionbuffer}{\tsent}{\treceivebuffer}}
%%	\arro{\tau} 
%%	\tsystem{\tclienti{\tprogram;\pwhile{\cond}{\tprogram}[Q]}{\tknown}{\tpending}{\ttransactionbuffer}{\tsent}{\treceivebuffer+1}}
%	}
%
%\\[25pt]
%\mathrule{while-false}
%	{\eval \cond {false}}
%	{
%		{\begin{array}{l}
%			\addid{\iclientinst[\pwhile{\cond}{P}[Q]]} \bigpar \ \isystemterm
%			\arro{\confirmedtran} 
%			\\[5pt]
%			\hspace{7cm}
%			\addid{\iclientinst[Q]} \bigpar \ \isystemterm
%		\end{array}
%%	\tsystem{\tclienti{\pwhile{\cond}{\tprogram}[Q]}{\tknown}{\tpending}{\ttransactionbuffer}{\tsent}{\treceivebuffer}} 
%%	\arro{\tau} 
%%	\tsystem{\tclienti{Q}{\tknown}{\tpending}{\ttransactionbuffer}{\tsent}{\treceivebuffer+1}}
%	}}
%	\\
%
%
%
%
%%	\mathrule{receive}
%%		{\inclient(\cid) = \gs_0 \cdot \gs_t}
%%		{\begin{array}{l}
%%			{\addid{\iclientinst[@]}} \arro{\tau}
%%			\\
%%			\hspace{5cm} 
%%			{\addid{\iclientinst[@][@][@][@][@][\receivebuffer \cdot \gs_0.gssegment][@][\inclient{[\cid\mapsto\gs_t]}]}}
%%		\end{array}}
%
%\end{array}
% \]





% \[
% \begin{array}{l}
%    \hspace{-.3cm} \textsc{SERVER}\\
%    
%\mathrule{drop-conn}{b_i \in \inserver \qquad  b_i \in \outserver}{\server{\persistedstate}{\inserver}{\outserver} \auxarro{\dropconn{b_i}} \server{\persistedstate}{\inserver \setminus b_i}{\outserver \setminus b_i}} 
%\hfill
%\mathax{crash-and-recover}{\server{\persistedstate}{\inserver}{\outserver} \auxarro{\crashandrecover} \server{\persistedstate}{\undefined}{\undefined}}
%
%\\[25pt]
%
%\mathrule{batch}{rs=\receiveroundsname{\inserver} \qquad \agssegpair[@][{\amxrf[{'}]}] =\append{\emptygssegment}{rs} \qquad 
%  \astate[']=\iapply{\astate}{\adelta}}{\server{\agsprefpair }{\inserver}{\outserver} \arro{\tau} 
%  \server{\agsprefpair[{\astate[']}][{\amxrf[{'}]}] }{\clean{\inserver}}{\notify{dom(\outserver)}{\outserver}{gs}}}
%
%\\[25pt]
%
%\mathrule{accept-conn}{b_i \notin \inserver \qquad b_i \notin \outserver}{\server{\persistedstate}{\inserver}{\outserver} \auxarro{\acceptconn{b_i}} \server{\persistedstate}{\inserver}{\update{\outserver}{b_i}{\persistedstate}}}
%
%    \\[35pt]
%    \hspace{-.3cm} \textsc{COMMUNICATION}
%		
%\\
%    \mathrule{comm server-client}{\outserver(i)=\gs \cdot \gss}{\server{\state}{\inserver}{\outserver} \bigpar \clientr{P}{\stateclient} \arro{\tau} \server{\state}{\inserver}{\update{\outserver}{i}{\gss}} \bigpar 
%		\clientr{P}{\update{E}{\inclient}{E.\inclient \cdot \gs}}}
%		
%\\[25pt]
%
%	    \mathrule{comm client-server}{\outclient=\headerround \cdot \tailround}{\server{\state}{\inserver}{\outserver} \bigpar \clientr{P}{\stateclient} \arro{\tau} \server{\state}{\update{\inserver}{i}{\inserver(i) \cdot \headerround}}{\outserver} \bigpar 
%		\clientr{P}{\update{E}{\outclient}{\tailround}}}	
% \\
%
% \end{array}
% \]

\begin{figure}[tp]
{\small
 \[
 \begin{array}{l}
	\mathax{update}
		{\begin{array}{p{\linewidth}}
		$\addid{\iclientinst[\tupdateins]} \ \bigpar \ \isystemterm
		\arro{\updatetran{u}} $
		\\
		\hfill
		${\addid{\iclientinst[P][@][@][@][\iappend{\transactionbuffer}{u}]} \ \bigpar \   \isystemterm}
		$\end{array}}
\\
	\mathax{push}
		{\begin{array}{p{\linewidth}}
		$\addid{\iclientinst[\tpushins]} \ \bigpar \ \isystemterm
		\arro{\pushtran} $
		\\
		\hfill
		${\addid{\iclientinst[P][@][@][\ireduce{\pushbuffer\cdot \transactionbuffer}][\emptydelta][@][n+1]}\ \bigpar \ \isystemterm}
		$\end{array}}
\\
	 \mathrule{send}
	         {\pushbuffer \neq \emptydelta \qquad \cid\in\dom\inserver \qquad \around = \aroundtuple[\cid][\nround][\pushbuffer]}
 		{\begin{array}{p{\linewidth}}
		$\addid{\iclientinst} \ \bigpar \ \iserverins[\astate[']] \ \bigpar\ \iclient
		\arro{\tau} $
		\\
		\hfill
		$\addid{\iclientinst[@][@][\pending \cdot \around][\emptydelta]}
		\ \bigpar \ \iserverins[{\astate[']}][@][\inserver\upd{\cid}{\inserver(\cid)\cdot\around}]
		\ \bigpar\ \iclient
	 	$\end{array}
		}	
\\
	\mathrule{receive}
	         { \outserver(\cid) = \aseg\cdot\aseqseg}
		{\begin{array}{p{\linewidth}}
		$\addid{\iclientinst[@][@][@][@][@][@][@][\inclient]} \ \bigpar \ \iserverins[\astate[']] \ \bigpar\ \iclient
		\arro{\tau}$ 
		\\
		\hfill
		$\addid{\iclientinst[@][@][@][@][@][@][@][\inclient\cdot \aseg]} \ \bigpar \ \iserverins[{\astate[']}][@][@][\outserver\upd{\cid}\aseqseg] \ \bigpar\ \iclient
	 	$\end{array}
		}	
\\
	\mathrule{pull-1}
	{\aseqround['] = \filter{\amxrf[_k](\cid)} {\aseqround} \qquad \inclient = \agssegpair[{\adelta[_1]}][{\amxrf[_1]}]\ldots \agssegpair[{\adelta[_k]}][{\amxrf[_k]}]}
		 {\begin{array}{p{\linewidth}}
		 	$\addid{\iclientinst[\tpullins][@]
			[@]%[\aseqround\cdot{\aroundtuple}\cdot{\aseqround[']}]
			[@][@][@][n][\inclient]}
			    \ \bigpar \ \isystemterm
			\arro{\pulltran} 
			$\\
			$\hfill
			\addid{\iclientinst[P]
						 [\iapply{\known}{\ireduce{\adelta[_1]\cdots\adelta[_k]}}]
						 [\aseqround']
						 [@][@]
						 [{\receivebuffer}][n][\epsilon]}\ \bigpar \ \isystemterm
						 $
	 	\end{array}
		}
\\

	\mathrule{read}
		{\iread{r}{\iapply{\known}{{\igetdeltas\pending}\cdot{\pushbuffer}\cdot{\transactionbuffer}}}=v}
		{\begin{array}{p{\linewidth}}
		$\addid{\iclientinst[\treadins{x}{r}]} \ \bigpar \ \isystemterm
		\arro{\readtran{r}} $
		\\
		\hfill
		$\addid{\iclientinst[{P}\subst{x}{v}]}\ \bigpar \ \isystemterm
		$\end{array}
		}
\\
		\mathrule{confirm}
		{ v = (\pending  \cdot \pushbuffer \cdot\transactionbuffer == \epsilon) }
		{\begin{array}{p{\linewidth}}
			$\addid{\iclientinst[\tconfirmedins{x}]} \bigpar \ \isystemterm
			\arro{\confirmedtran} 
			$\\
			\hfill
			$\addid{\iclientinst[P\subst{x}{v}]} \bigpar \ \isystemterm
		$
		\end{array}
		}
		\\
\mathrule{batch}
	{
	\begin{array}{p{\linewidth}}	
	  $\agssegpair[@][{\amxrf[{'}]}] =\receiveroundsname{\inserver} \qquad\  %=\append{\emptygssegment}{rs} \qquad 
	  \astate[']=\iapply{\astate}{\adelta} \hfill
 	 \forall\cid.(\outserver['] (\cid)= \outserver(\cid)\cdot\agssegpair[@][{\amxrf[{'}]}] \, \land\,
 	 \inserver['](\cid) = \epsilon)$
\end{array}
  }
  {\iserverins\ \bigpar \ \iclient
  \tr{\tau} 
  {\iserverins[{\astate[']}][{\amxrf[{'}]}][{\inserver[']}][{\outserver[']}]}\ \bigpar \ \iclient}
\\
\mathrule{drop-conn}
	{\cid \in \inserver \qquad  \cid \in \outserver}
	{
	\begin{array}{p{\linewidth}}
	$\addid{\iclientinst[@][@][\aseqround][@][@][@][@][\inclient]}
		\ \bigpar \ \server{\astate}{\inserver}{\outserver} \ \bigpar \ \iclient
	\tr{\tau}$\\
	$\hfill
	\addid{\iclientinst[@][@][\aseqround][@][@][@][@][\epsilon]}
	\ \bigpar \server{\astate}{\inserver \setminus\cid}{\outserver \setminus\cid}\ \bigpar \ \iclient
	$\end{array}
	} 
\\
\mathrule{accept-conn}
{\cid\notin \inserver \qquad \cid \notin \outserver}
{\server{\astate}{\inserver}{\outserver} \ \bigpar\  \iclient 
\tr{\tau} 
\server{\astate}{\inserver\upd\cid\epsilon}{\update{\outserver}{\cid}{\agsprefpair}}\ \bigpar\  \iclient }
\\[25pt]
	\mathrule{pull-2}
	{
	\begin{array}{c}
	\inclient = \agssegpair[{\astate[''']}][{\amxrf[_0]}]\cdot\agssegpair[{\adelta[_1]}][{\amxrf[_1]}]\ldots \agssegpair[{\adelta[_k]}][{\amxrf[_k]}]
	\qquad \qquad\qquad
	\astate['']  = \iapply{\astate[''']}	{\ireduce{\emptydelta\cdot\adelta[_1]\cdots\adelta[_k]}} \\	
	\aseqround['] = \filter{\amxrf[_k](\cid)} \aseqround
	\end{array}
	 }
		 {\begin{array}{p{\linewidth}}
			$\addid{\iclientinst[\tpullins][@][\aseqround][@][@][@][@][\inclient]}
			     \ \bigpar \ \iserverins[{\astate[']}][{\amxrf}] \ \bigpar\ \iclient
 %[]\ \bigpar \ \isystemterm
			\arro{\pulltran} 
			$\\
			$\hfill
			\addid{\iclientinst[P]
						 [{\astate['']}]
						 [{\aseqround'}]
						 [@][@]
						 [{\receivebuffer}][@][\epsilon]}
		\ \bigpar \ \iserverins[{\astate[']}][{\amxrf}][\inserver\upd{\cid}{\inserver(\cid)\cdot\aseqround[']}]
		\ \bigpar\ \iclient
	 	$\end{array}
		}
\\
\mathax{crash}{\server{\astate}{\inserver}{\outserver}\ \bigpar \ \iclient
 \tr{\tau} \server{\astate}{\undefined}{\undefined}\ \bigpar \ \iclient}

\end{array}
 \]
 }
 \caption{Operational semantics of \igsp}
 \label{fig:semantics-igsp}
 \end{figure}

% \[
% \begin{array}{l}
%    \hspace{-.3cm} \textsc{SERVER}\\
%    
%
%\mathrule{batch}
%{\begin{array}{c}
%  \agssegpair[@][{\amxrf[{'}]}] =\receiveroundsname{\inserver} \qquad\qquad %=\append{\emptygssegment}{rs} \qquad 
%  \astate[']=\iapply{\astate}{\adelta} \\
%  \forall\cid.(\outserver['] (\cid)= \outserver(\cid)\cdot\agssegpair[@][{\amxrf[{'}]}] \quad \land\quad 
%  \inserver['](\cid) = \epsilon)
%\end{array}
%  }
%  {\iserverins\ \bigpar \ \iclient
%  \tr{\tau} 
%  {\iserverins[{\astate[']}][{\amxrf[{'}]}][{\inserver[']}][{\outserver[']}]}\ \bigpar \ \iclient}
%\\
%\mbox{\henote{este receiveRounds es distinto al tuyo, hace todo de una}}
%\\
%\mathrule{accept-conn}
%{\cid\notin \inserver \qquad \cid \notin \outserver}
%{\server{\astate}{\inserver}{\outserver} \ \bigpar\  \iclient 
%\tr{\tau} 
%\server{\astate}{\inserver\upd\cid\epsilon}{\update{\outserver}{\cid}{\agsprefpair}}\ \bigpar\  \iclient }
%\\
%\mathrule{drop-conn}
%	{\cid \in \inserver \qquad  \cid \in \outserver}
%	{
%	\begin{array}{l}
%	\addid{\iclientinst[@][@][\aseqround][@][@][@][@][\inclient]}
%		\ \bigpar \ \server{\astate}{\inserver}{\outserver} \ \bigpar \ \iclient
%	\tr{\tau} \hspace{4.5cm}\\
%	\hfill
%	\addid{\iclientinst[@][@][\aseqround][@][@][@][@][\epsilon]}
%	\ \bigpar \server{\astate}{\inserver \setminus\cid}{\outserver \setminus\cid}\ \bigpar \ \iclient
%	\end{array}
%	} 
%\\
%
%%{\cid \in \inserver \qquad  \cid \in \outserver}
%%{\server{\astate}{\inserver}{\outserver} \ \bigpar \ \iclient
%%\tr{\tau} 
%%\server{\astate}{\inserver \setminus\cid}{\outserver \setminus\cid}\ \bigpar \ \iclient} 
%%\\
%\mathax{crash}{\server{\astate}{\inserver}{\outserver}\ \bigpar \ \iclient
% \tr{\tau} \server{\astate}{\undefined}{\undefined}\ \bigpar \ \iclient}
%
%			     
%%{\cid \in \inserver \qquad  \cid \in \outserver}
%%{\server{\astate}{\inserver}{\outserver} \ \bigpar \ \iclient
%%\tr{\tau} 
%%\server{\astate}{\inserver \setminus\cid}{\outserver \setminus\cid}\ \bigpar \ \iclient} 
%%\\
%%\mathax{crash}{\server{\astate}{\inserver}{\outserver}\ \bigpar \ \iclient
% %\tr{\tau} \server{\astate}{\undefined}{\undefined}\ \bigpar \ \iclient}
%\\
%
%\\
%	\mathrule{pull-2}
%	{
%	\begin{array}{c}
%	\inclient = \agssegpair[{\astate[''']}][{\amxrf[_0]}]\cdot\agssegpair[{\adelta[_1]}][{\amxrf[_1]}]\ldots \agssegpair[{\adelta[_k]}][{\amxrf[_k]}]
%	\qquad 
%	\aseqround['] = \filter{\amxrf[_k](\cid)} \aseqround\\
%	\astate['']  = \iapply{\astate[''']}{\ireduce{\emptydelta\cdot\adelta[_1]\cdots\adelta[_k]}}
%	\end{array}
%	 }
%		 {\begin{array}{l}
%			\addid{\iclientinst[\tpullins][@][\aseqround][@][@][@][@][\inclient]}
%			     \ \bigpar \ \iserverins[{\astate[']}][{\amxrf}] \ \bigpar\ \iclient
% %[]\ \bigpar \ \isystemterm
%			\arro{\pulltran} \hspace{3cm}
%			\\
%			\hspace{1.5cm}
%			\hfill\addid{\iclientinst[P]
%						 [{\astate['']}]
%						 [{\aseqround'}]
%						 [@][@]
%						 [{\receivebuffer}]}
%		\ \bigpar \ \iserverins[{\astate[']}][{\amxrf}][\inserver\upd{\cid}{\inserver(\cid)\cdot\aseqround[']}]
%		\ \bigpar\ \iclient
%	 	\end{array}
%		}
%\\
%
%
%
%%\\
%%	\mathrule{pull-2}
%%	{\aseqround['] = filter\ (\geq \amxrf(\cid)) \ \aseqround}
%%		 {\begin{array}{l}
%%			\addid{\iclientinst[\tpullins][@][\aseqround][@][@][@][@][{\agssegpair[{\astate[']}]}\cdot\inclient]}
%%			     \ \bigpar \ \iserverins[{\astate['']}][{\amxrf[']}] \ \bigpar\ \iclient
%% %[]\ \bigpar \ \isystemterm
%%			\arro{\pulltran} 
%%			\\
%%			\hspace{1.5cm}
%%			\addid{\iclientinst[P]
%%						 [{\astate[']}]
%%						 [{\aseqround'}]
%%						 [@][@]
%%						 [{\receivebuffer}]}
%%		\ \bigpar \ \iserverins[{\astate['']}][{\amxrf[']}][\inserver\upd{\cid}{\inserver(\cid)\cdot\aseqround[']}]
%%		\ \bigpar\ \iclient
%%	 	\end{array}
%%		}
%\\
%%    \\[35pt]
%%    \hspace{-.3cm} \textsc{COMMUNICATION}
%%		
%%\\
%%    \mathrule{comm server-client}{\outserver(i)=\gs \cdot \gss}{\server{\state}{\inserver}{\outserver} \bigpar \clientr{P}{\stateclient} \arro{\tau} \server{\state}{\inserver}{\update{\outserver}{i}{\gss}} \bigpar 
%%		\clientr{P}{\update{E}{\inclient}{E.\inclient \cdot \gs}}}
%%		
%%\\[25pt]
%%
%%	    \mathrule{comm client-server}{\outclient=\headerround \cdot \tailround}{\server{\state}{\inserver}{\outserver} \bigpar \clientr{P}{\stateclient} \arro{\tau} \server{\state}{\update{\inserver}{i}{\inserver(i) \cdot \headerround}}{\outserver} \bigpar 
%%		\clientr{P}{\update{E}{\outclient}{\tailround}}}	
%% \\
%%
% \end{array}
% \]

Let $\inserver$ such that $\dom\inserver = \{\cid_0,\ldots,\cid_m\}$ and
       $\forall\cid_l\in\dom\inserver.\inserver(\cid_l) =  \aroundtuple[\cid_l][\nround^0_l][{\adelta^0_l}]\cdots
      \aroundtuple[\cid_l][\nround^{k_l}_l][{\adelta^{k_l}_l}]$. Then, 
$\receiveroundsname{\inserver} = \agssegpair[@][{\amxrf}]$ with
   \begin{itemize}
      \item $\adelta = \ireduce{\adelta^0_0\cdots\adelta^{k_0}_0\cdots\adelta^0_m\cdots\adelta^{k_m}_m}$
      \item $\dom{\amxrf} = \dom{\inserver}$
      \item $\forall\cid_l\in\dom\inserver.\amxrf(\cid_l) = n^{k_l}_l$
\end{itemize}

Let $\aseqround =  \aroundtuple[\cid][\nround_0][{\adelta_0}]\cdots
      \aroundtuple[\cid][\nround_k][{\adelta_k}]$ be a well-formed pending buffer. Then, 
      
      $\filter n \aseqround = \aroundtuple[\cid][\nround_j][{\adelta_j}]\cdots
      \aroundtuple[\cid][\nround_k][{\adelta_k}]$ if $\nround_{j-1}\leq n$ and $\nround_j>n$.


\henote{well-formedness: $i\in\dom{in}
 iff i\in\dom{out}$   y no conectado, no mensajes en las colas.
 todos los r estan crecientes y los f tambien}

\henote{
Rule $\textsc{(drop-conn)}$ removes from the server's queues the client called $b_i$. Rule $\textsc{(crash-and-recover)}$ leaves undefined the server's queues and preserves the server's persistent state.  Rule $\textsc{(accept-conn)}$ show how to add a new connecti(Read)send to its the persistent state. Last rule from \textbf{\textsc{(server)}} is Rule $\textsc{(batch)}$, the most interesting of this group. Its has three hypothesis, the first one, is responsible for receiving rounds from the queue of in-messages. Se\cond one, let $\emptygssegment$ be a empty segment, it gives back a delta object who represents the combination of numbers of rounds into a single object. Finally, this object is applied to the persistent state. As result, the persistent state is putted into the queues messages-out.
There are two \textbf{\textsc{(communications)}} rules. Rule $\textsc{(comm-server-client)}$ when the server has a message for client $i^{th}$, this is removed from the server's queue message-out and is putted into the queue message-in from client $i^{th}$. Rule $\textsc{(comm-client-server)}$ states when a round from client $i^{th}$ is left into server queue message-in.
Rule $\textsc{(Read)}$ gives the result of performing a lecture on messages queues from the client. Rule $\textsc{(update)}$ adds an update to the transaction buffer. Rule $\textsc{(push)}$ leaves into push buffer a delta object resulted of reducing the push buffer with transaction buffer. The transaction buffer is cleaned and the number of rounds sent is incremented by one. Rule $\textsc{(pull)}$ shows when a persisted state from a client is modified. The hypothesis are that a channel have been accepted, i.e., these must be defined for client $i^{th}$ besides the receive buffer should have an element at least. $\textsc{(confirmed)}$
computes the states of the internal queues,i.e., if these has any element. Rules $\textsc{(while-true)}$ and $\textsc{(while-false)}$ are standard. Rule $\textsc{(send)}$ creates a new round setting who is the client ($i^{th}$), how many rounds client ($i^{th}$) has sent and content from the push buffer. Finally rule $\textsc{(receive)}$ move out segments from the client's queue message-in to receive buffer.}
