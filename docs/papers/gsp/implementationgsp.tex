% !TEX root = main.tex
\section{Implementation of {\gsp}}
\label{sec:igsp}

The \tgspcalculus\ model describes an idealised system that abstracts  
away from several  implementation details, such as non-optimised 
representation of the state and unreliable communication. 
This section presents a formal model for the implementation   
proposed in~\cite{DBLP:conf/ecoop/BurckhardtLPF15}. 


\subsection{Syntax}
The implementation of \gsp\, 
 relies on a  compact representation for 
states and updates. Their precise definition highly depends
on the datatype of the values handled by the store,
%. \henote{agregar ejemplo?}. 
but they are 
%In order to keep the description of the model as general as possible, they have been
characterised  in terms of 
two abstract types:  $\statetype$ and $\deltatype$, which provides
 the following  operations in~\cite{DBLP:conf/ecoop/BurckhardtLPF15}:

{\small
\[
\begin{array}{lll}
%	\textbf{const} & 
	\emptydelta & : \deltatype 
	\\
%	\textbf{function} & 
	\iappendname & : \partialfunction{\deltatype \times \updatetype}{\deltatype} 
	\\
%	\textbf{function} & 
	\ireducename & : \partialfunction{\deltatype^*}{\deltatype} \\
\end{array}
\qquad\quad
\begin{array}{ll}
%	\textbf{const} & 
	\initialstate & : \statetype 
	\\
%	\textbf{function} & 
	\iapplyname & : \partialfunction{\statetype \times \deltatype^*}{\statetype} 
	\\
%	\textbf{function} & 
	\ireadname & : \partialfunction{\readtype \times \statetype}{\valuetype} 
	\\
\end{array}
\] }

Constants $\emptydelta$ and  $\initialstate$  denote the 
empty elements in their respective types. An object
$\adelta \in \deltatype$ describes the effects of a sequence of updates and 
is built  by either appending an update  to an existing delta ($\iappendname$) 
or  combining together several deltas  ($\ireducename$).
Operation $\ireadname$ is the 
interpretation function for  operations (\ie, the implementation counterpart of 
function $\rvalue{\_}{\_}$  used by the 
idealised model) and $\iapplyname$ corresponds to state transformations.
%
%We shall use  $\astate,\astate['],\ldots, \astate[{_1}],\ldots$ to denote values of  type $\statetype$ and 
%$\adelta,\adelta['],\ldots, \adelta[{_1}],\ldots$ for values of type $\pushbuffertype$.

Clients and the global store exchange $\adelta$ objects to communicate changes. As 
each single $\adelta$ may correspond to several update operations, clients
send each $\adelta$  accompanied by its own identifier and
 a sequence number $\irounds$. 
Precisely, clients send rounds, i.e.  triples $\around = \aroundtuple$. Differently,
the global store sends segments $\aseg = \agssegpair$, in which $\adelta$ is accompanied
by a function $\amxrf\in\idset\rightarrow \nat$. In this way, the global store 
confirms all changes from client $\cid$ until round $\amxrf(\cid)$. To 
deal with crashes and recovery, the server may send segments of the form
$\agsprefpair$, which communicates a complete state instead of a delta object. 

 \begin{definition}[GSP Language] 
The set of \igsp\ terms is given by the grammar in \figref{fig:syntax-igsp}.
\end{definition}
 \begin{figure}[t]
 \[
\begin{array}{l}
    \begin{array}{l@{\ }rcl}
	\rName{state}
	&
	\multicolumn 3 l {\astate,\astate['],\ldots, \astate[{_1}],\ldots \ \in\  \statetype}
	\\
    	\rName{delta} 
	&
	\multicolumn 3 l {\adelta,\adelta['],\ldots, \adelta[{_1}],\ldots \ \in\  \deltatype}
       \\
       \rName{round}
       &
       \around &::= & �\aroundtuple
        \\
       \rName{rnd Seq}
       &
       \aseqround &::= & \epsilon \ |\ \around\cdot\aseqround
	\\
	\rName{max rnd}
       &
       \multicolumn 3 l {\amxrf,\amxrf['],\ldots, \amxrf[{_1}],\ldots \,  \in \  \idset\rightarrow \nat}
        \\
       \rName{segment}
       &
       \aseg & ::= &\agssegpair  \; |\;  \agsprefpair 
%        \\
%       &
%       {\inclient} &::= &  \aseqseg%{{\agsseg}^* \cup \agspref}
%       \\
%       & 
%       \outclient &\in & {\aseqround}

\end{array}
\quad
    \begin{array}{l@{\ }rcl}
       \rName{sgmt seq}
       &
       \aseqseg &  \ ::=\  &\epsilon \ |\ \aseg\cdot\aseqseg
\\       \rName{in srv}
       &
       \inserver &\ \in\ & \partialfunction{\idset}\aseqround%\outclient
       \\
       \rName{out srv}
       & 
       \outserver\ &\ \in\ & \partialfunction{\idset}\aseqseg%\inclient
       \\
 	\rName{system} & \isystemterm & \ ::=\  & \iserver\ \bigpar\ \iclient \\
	\rName{server} & \iserver &\ ::=\  & \server{\astate}{\inserver}{\outserver} \\
	\rName{clients}& \iclient &\ ::=\  &\zero \;|\; 
	 						%\client{P}{<\astate,\pendingtype,\pushbuffertype,\transactionbuffertype,\receivebuffertype,n,\inqueue,\outqueue>} 
							\addid{\iclientsyntax}
							\;|\; \ 
							\iclient \bigpar \iclient  \\
      \end{array}
\end{array}
  \]
\caption{Syntax of the \igsp\ calculus}
\label{fig:syntax-igsp}
\end{figure}

As for \gsp, a system is composed by a global store $\iserver$ and possibly many clients $\iclient$. 
A global store is modelled by a tuple $\server{\astate}{\inserver}{\outserver}$ containing a state $\astate$, 
a function $\amxrf$ to keep track of  processed rounds and the communication buffers
$\inserver$ and $\outserver$. There are two dedicated buffers for 
 each client $\cid$: $\inserver(\cid)$ contains the  rounds received from $\cid$,  and $\outserver(\cid)$ 
  the  segments that have been sent to $\cid$. 
  
A client  is represented by a term of the form $\addid{\iclientinst[P][@][@][@][\transactionbuffer]}$. 
As for \gsp, $\cid$ is its identity and $P$ is its program. We remark that the language for programs remains unaltered.
The component $\transactionbuffer$ is analogous to $\ttransactionbuffer$ in the \gsp\ model, i.e., it keeps 
all local updates until the client performs $\pushcmd$. Differently, $\pushbuffer$ keeps all finished blocks that 
 have not been sent. The number $\irounds$ identifies the current round. Buffer $\pending$ keeps all
 sent rounds that have not been confirmed by the global store (similar to $\tpending$ in \gsp), while 
 $\inclient$ keeps all  received segments (analogous to $\treceivebuffer$).  

%In addition,  we will use partial functions  $\amxrf,\amxrf['],\ldots, \amxrf[{_1}],\ldots$ in  $\idset\rightarrow \nat$
%assigning clients with a natural number denoting \henote{the number of the last round exchanged with ...}
%
%%Let $\idset$ be a set of clients' name ranged over by $b$, $b_1$, $b_2$, $\ldots$; 
%
%The global state of the store is now represented by a pair $\gsprefixtype$ be a set over sets \statetype\ $\times$ $\idset$ $\rightarrow$ $\nat$, \gssegmenttype\ be a set operation over sets \deltatype $\times$ $\idset$ $\rightarrow$ $\nat$ and \roundtype\ the set over sets $\idset$ $\times$ $\nat$ $\times$ \deltatype; $\emptygssegment$\ is a fresh element of \gssegmenttype.
%%
%\henote{Explain gsspref... round}
%
%%Furthermore, we assume the following parametric functions defined over the abstract data types above borrowed from :
%
%
%%Let $\pendingtype$, $\sigma$ be sequence defined:
%
%%$\pendingtype$,$\sigma$ := $\epsilon$ \text{\textbar} [$\delta$]$\cdot$ $\pendingtype$
% 
%
%We assume the following countable sets of persisted state $\gsprefixtype$ ranged over by $\agspref$, $\agspref_1$, $\ldots$; 
%
%%\begin{figure}[t]
%%\[
%%   \begin{array}{l@{\quad}r@{\;::=\;}l}
%%	\rName{system} & \isystemterm &  \iserver\ \bigpar\ \iclient \\
%%	\rName{server} & \iserver & \server{\astate}{\inserver}{\outserver} \\
%%	\rName{clients}& \iclient & \zero \;|\; 
%%	 						%\client{P}{<\astate,\pendingtype,\pushbuffertype,\transactionbuffertype,\receivebuffertype,n,\inqueue,\outqueue>} 
%%							\addid{\iclientsyntax}
%%							\;|\; \ 
%%							\iclient \bigpar \iclient  \\
%%%			 (\textsc{program}) & P & \zero \;|\; \readins{x}{r} \;|\; \updateins{u} \;|\; 	\pushins \;|\; \pullins
%%     \end{array}
%%\]
%%\end{figure}
%%
%
% \begin{definition}[Implementation of GSP] 
%	The syntax of clients and server is given by the following grammar 
%   \end{definition}
% 
% \henote{
%The implementation of a system is a server and clients interacting concurrently. A server will be represented with a persisted state denoted by $\persistedstate$, input messages $\inserver$ and output messages $\outserver$.
%We will refer to clients, as tuple of program $P$ and set of states and sequences $E$. A program client stores a state ($E.\known$), a pending queue ($E.\pending$),i.e., updates sent to the server without confirmation of their reception, a push buffer ($E.\pushbuffer$) which holds updates that were pushed by the client but have not been sent to the server, a transaction buffer ($E.\transactionbuffer$) which holds updates for sending with another ones, a receive buffer ($E.\receivebuffertype$) for updates which were sent by the server, the number of round sent ($E.\irounds$), a input message queue ($E.\inclient$) and a output message queue ($E.\outclient$).
%}

We also impose the following well-formedness condition on systems. 

\begin{definition}[{\igsp} well-formedness]
\label{def:wf-igsp} A \igsp\ system
 $\isystemterm =  \iclient_0\bigpar \ldots\bigpar \iclient_m\bigpar\iserver$  with
 $\iserver=\server{\astate}{\inserver}{\outserver}$ and 
$\iclient_l = \addid{\iclientinst[P_l][\astate_l][{\pending}_l][\pushbuffer_l][\transactionbuffer_l][\receivebuffer_l][\irounds_l][\inclient_l][\outclient_l]}[{\cid_l}]$ 
for   $l\in \{0,\ldots,m\}$  is {\em well-formed} if the  following conditions hold 

\begin{enumerate}

  \item $\cid_l \neq \cid_{l'}$ for all $l\neq l'$;
  \item $\dom\inserver = \dom\outserver \subseteq  \{0,\ldots,m\}$;
  \item $\cid_l\not\in\dom\outserver$ implies $\inclient_l = \epsilon$;
  \item if $\cid_l\in\dom\outserver$ and $\inclient_l\cdot\outserver(\cid_l) \neq\epsilon$ then either
	\begin{enumerate}[$(i)$]
		\item ${\inclient}_l\cdot\outserver(\cid_l)= \agssegpairi[0] \cdots \agssegpairi[h]$; or
		\item  ${\inclient}_l\cdot\outserver(\cid_l)= \agssegpair[{\astate}][{\amxrf[_0]}]\cdot\agssegpairi[1]\cdots \agssegpairi[h]$.
	\end{enumerate}
	 $\amxrf_j(\cid_l) \leq \ \amxrf_k(\cid_l)$ for all  ${ j}<{k} �\in \{0 .. h\}$ and $\amxrf_h = \amxrf$
	
\item \label{wf-igsp-pending} $\pending_l = \aroundtuplei[0]\cdots\aroundtuplei[r]$,  $\irounds_j < \irounds_k$ for all  ${ j}<{k} \in \{0 .. r\}$ and either.
	\begin{enumerate}[$(i)$]
		\item $\pushbuffer_l =\emptydelta$, $\irounds_r \leq \irounds_l$ and $\amxrf(\cid_l) \leq \irounds_l$ or
		\item $\pushbuffer_l \neq \emptydelta$, $\irounds_r < \irounds_l$ and $\amxrf(\cid_l) < \irounds_l$;
	\end{enumerate}

\item  $l\in\{0,\ldots,m\}$ then either
  \begin{enumerate}[$(i)$]
    \item $\pending_l  = \epsilon$, $\amxrf(\cid_l) \leq \irounds_l$ and if $\cid_l \in\dom{\inserver}$ then $\inserver(\cid_l)=\epsilon$ ;
   % \item $\pending_l = \inserver(\cid_l)\cdot\pending_l'$; 
    \item $\pending_l = \inserver(\cid_l)$ %NECESITABAMOS PARA QUE SEA SINCRO QUE HAYA UN OUT.
    $= {\aroundtuple[\cid_l][{\irounds_{\it fst}}][\adelta_{\it fst}]} \cdot\pending_l'$ and  $ \irounds_{\it fst}> {\amxrf} (\cid_l)$.
  %  \item $\pending_l =\pending_l''\cdot{\aroundtuple[\cid_l][{\irounds_{lst}}][\adelta_{lst}]} \cdot \inserver(\cid_l)\cdot\pending_l'$  
   % 		and $\inclient_l\cdot\outserver (\cid_l) = \aseqseg\cdot\agssegpair[\delta'_{lst}][{\amxrf_{lst}}]$  with
%		 ${\amxrf_{lst}} (\cid_l) = \irounds_{lst}$. 
 \item $\pending_l =\pending_l''\cdot{\aroundtuple[\cid_l][{\irounds_{lst}}][\adelta_{lst}]} \cdot \inserver(\cid_l)$  
    		and $\inclient_l\cdot\outserver (\cid_l) = \agssegpairi[0] \cdots\agssegpair[\delta'_{lst}][{\amxrf_{lst}}]$  with
		 ${\amxrf_{lst}} (\cid_l) = \irounds_{lst}$. 
		 
\item $\inserver(\cid_l) = \epsilon$, $\pending_l =\pending_l''\cdot{\aroundtuple[\cid_l][{\irounds_{lst}}][\adelta_{lst}]}$  
    		and either $\inclient_l\cdot\outserver (\cid_l) = \agssegpair[{\astate}][{\amxrf[']}] \cdot\aseqseg$
		%\agssegpair[\delta'_{lst}][{\amxrf_{lst}}]$  
		with ${\amxrf[']} (\cid_l) \leq \irounds_{lst}$ or  $\inclient_l\cdot\outserver (\cid_l) = \epsilon$ and ${\amxrf} (\cid_l) \leq \irounds_{lst}$. 
		 
		 		 
\end{enumerate}
\end{enumerate}
\end{definition}

We require all clients to have different identifiers (1). Communication channels in the 
implementation are bidirectional, hence  $\cid_l\in\dom\inserver$ iff  $\cid_l\in\dom\outserver$ 
(2). Moreover, the input buffer of a  disconnected client is empty, \ie, $\cid_l\in\dom\outserver$ (3). 
Condition (4) states that $\agsprefpair[@][{\amxrf[']}]$ can appear only as the first message in the flow from 
the store and that the store confirms processed rounds in a non-decreasing order. Similarly, 
 clients send rounds with increasing round number (5). The last condition (6) states a 
 coherence requirement between  
 pending rounds and the confirmations sent by the store: stores confirms rounds within the range of 
 unconfirmed rounds. 





\subsection{Operational Semantics}

%For the operational semantics of \igsp\ we consider the set of labels extended as follows
%\[\mu := \ldots\ | \ \tau \]

  \begin{figure}[tp]
{\small
 \[
 \begin{array}{l}
	\mathax{i-update}
		{\begin{array}{p{\linewidth}}
		$\addid{\iclientinst[\tupdateins]} \ \bigpar \ \isystemterm
		\hfill \arro{\updatetran{\anupd}} \quad$
%		\\
%		\hfill
		${\addid{\iclientinst[P][@][@][@][\iappend{\transactionbuffer}{u}]} \ \bigpar \   \isystemterm}
		$\end{array}}
\\[20pt]
	\mathax{i-push}
		{\begin{array}{p{\linewidth}}
		$\addid{\iclientinst[\tpushins]} \ \bigpar \ \isystemterm
		\hfill\arro{\pushtran}\qquad $
%		\\
%		\hfill
		${\addid{\iclientinst[P][@][@][\ireduce{\pushbuffer\cdot \transactionbuffer}][\emptydelta][@][\irounds+1]}\ \bigpar \ \isystemterm}
		$\end{array}}
\\[20pt]
	 \mathrule{i-send}
	         {\pushbuffer \neq \emptydelta \qquad \cid\in\dom\inserver \qquad \around = \aroundtuple[\cid][\irounds][\pushbuffer]
	         \qquad
		\inclient \cdot\outserver(\cid) \neq \agssegpair[{\astate[']}][{\amxrf}]\cdot\aseqseg}
 		{\begin{array}{p{\linewidth}}
		$\addid{\iclientinst} \ \bigpar \ \iserverins[\astate[']] \ \bigpar\ \iclient
		\arro{\tau} $
		\\
		\hfill
		$\addid{\iclientinst[@][@][\pending \cdot \around][\emptydelta]}
		\ \bigpar \ \iserverins[{\astate[']}][@][\inserver\upd{\cid}{\inserver(\cid)\cdot\around}]
		\ \bigpar\ \iclient
	 	$\end{array}
		}	
\\[20pt]
	\mathrule{i-receive}
	         { \outserver(\cid) = \aseg\cdot\aseqseg}
		{\begin{array}{p{\linewidth}}
		$\addid{\iclientinst[@][@][@][@][@][@][@][\inclient]} \ \bigpar \ \iserverins[\astate[']] \ \bigpar\ \iclient
		\arro{\tau}$ 
		\\
		\hfill
		$\addid{\iclientinst[@][@][@][@][@][@][@][\inclient\cdot \aseg]} \ \bigpar \ \iserverins[{\astate[']}][@][@][\outserver\upd{\cid}\aseqseg] \ \bigpar\ \iclient
	 	$\end{array}
		}	
\\[20pt]
	\mathrule{i-pull$_1$}
	{\aseqround['] = \filter{\amxrf_k(\cid)} {\aseqround} \qquad \inclient = \agssegpair[{\adelta_1}][{\amxrf_1}]\ldots \agssegpair[{\adelta_k}][{\amxrf_k}]}
		 {\begin{array}{p{\linewidth}}
		 	$\addid{\iclientinst[\tpullins][@]
			[@]%[\aseqround\cdot{\aroundtuple}\cdot{\aseqround[']}]
			[@][@][@][@][\inclient]}
			    \ \bigpar \ \isystemterm
			\hfill\arro{\pulltran} \ 
			$
%			\\
			$
			\addid{\iclientinst[P]
						 [\iapply{\known}{\ireduce{\adelta_1\cdots\adelta_k}}]
						 [\aseqround']
						 [@][@]
						 [{\receivebuffer}][@][\epsilon]}\ \bigpar \ \isystemterm
						 $
	 	\end{array}
		}
\\[25pt]

	\mathrule{i-read}
		{\iread{r}{\iapply{\known}{{\igetdeltas\pending}\cdot{\pushbuffer}\cdot{\transactionbuffer}}}=v}
		{\begin{array}{p{\linewidth}}
		$\addid{\iclientinst[\treadins{x}{r}]} \ \bigpar \ \isystemterm
		\hfill\arro{\readtran{r}}\ $
%		\\
%		\hfill
		$\addid{\iclientinst[{P}\subst{x}{v}]}\ \bigpar \ \isystemterm
		$\end{array}
		}
\\[25pt]
\mathrule{i-confirm}
		{ v = (\pending  \cdot \pushbuffer \cdot\transactionbuffer == \epsilon) }
		{\begin{array}{p{\linewidth}}
			$\addid{\iclientinst[\tconfirmedins{x}]} \bigpar \ \isystemterm
			\hfill\arro{\confirmedtran}\ 
			$
%			\\
%			\hfill
			$\addid{\iclientinst[P\subst{x}{v}]} \bigpar \ \isystemterm
		$
		\end{array}
		}
\\[25pt]
\mathrule{i-batch}
	{
	\begin{array}{p{\linewidth}}	
	  $\agssegpair[@][{\amxrf[{'}]}] =\receiveroundsname{\inserver} \ \ \ \adelta\neq\emptydelta\ \ \ %=\append{\emptygssegment}{rs} \qquad 
	  \astate[']=\iapply{\astate}{\adelta} \ \ 
 	 \forall\cid.(\outserver['] (\cid)= \outserver(\cid)\cdot\agssegpair[@][{\amxrf}{[{\amxrf[{'}]}]}] \land
 	 \inserver['](\cid) = \epsilon)$
\end{array}
  }
  {\iserverins\ \bigpar \ \iclient
  \tr{\tau} 
  {\iserverins[{\astate[']}][{\amxrf}{[{\amxrf[{'}]}]}][{\inserver[']}][{\outserver[']}]}\ \bigpar \ \iclient}
\\[25pt]
\mathrule{i-drop-cxn}
	{\cid \in \inserver \qquad  \cid \in \outserver}
	{
	\begin{array}{p{\linewidth}}
	$\addid{\iclientinst[@][@][\aseqround][@][@][@][@][\inclient]}
		\ \bigpar \ \server{\astate}{\inserver}{\outserver} \ \bigpar \ \iclient
	\tr{\tau}$\\
	$\hfill
	\addid{\iclientinst[@][@][\aseqround][@][@][@][@][\epsilon]}
	\ \bigpar \server{\astate}{\inserver \setminus\cid}{\outserver \setminus\cid}\ \bigpar \ \iclient
	$\end{array}
	} 
\\[25pt]
\mathrule{i-accept-cxn}
{\cid\notin \inserver \qquad \cid \notin \outserver}
{\server{\astate}{\inserver}{\outserver} \ \bigpar\  \iclient_{\cid} \ \bigpar\  \iclient 
\tr{\tau} 
\server{\astate}{\inserver\upd\cid\epsilon}{\update{\outserver}{\cid}{\agsprefpair}}\ \bigpar\  \iclient_{\cid} \ \bigpar\  \iclient   }
\\[25pt]
\mathrule{i-pull$_2$}
	{
	\begin{array}{c}
	\inclient = \agssegpair[{\astate[''']}][{\amxrf_0}]\cdot\agssegpair[{\adelta_1}][{\amxrf_1}]\ldots \agssegpair[{\adelta_k}][{\amxrf_k}]
	\qquad \qquad\qquad
	\astate['']  = \iapply{\astate[''']}	{\ireduce{\emptydelta\cdot\adelta_1\cdots\adelta_k}} \\	
	\aseqround['] = \filter{\amxrf_k(\cid)} \aseqround
	\end{array}
	 }
		 {\begin{array}{p{\linewidth}}
			$\addid{\iclientinst[\tpullins][@][\aseqround][@][@][@][@][\inclient]}
			     \ \bigpar \ \iserverins[{\astate[']}][{\amxrf}] \ \bigpar\ \iclient
 %[]\ \bigpar \ \isystemterm
			\arro{\pulltran} 
			$\\
			$\hfill
			\addid{\iclientinst[P]
						 [{\astate['']}]
						 [{\aseqround'}]
						 [@][@]
						 [{\receivebuffer}][@][\epsilon]}
		\ \bigpar \ \iserverins[{\astate[']}][{\amxrf}][\inserver\upd{\cid}{\inserver(\cid)\cdot\aseqround[']}]
		\ \bigpar\ \iclient
	 	$\end{array}
		}

\end{array}
 \]
 }
 \caption{Operational semantics of \igsp}
 \label{fig:semantics-igsp}
 \end{figure}




As for the idealised model, the operational semantics  is given by a labelled transition system 
over well-formed terms, up-to  structural equivalence. We consider a new label $\tau$ without 
any client annotation for transitions associated with changes in the global store 
and communication failures.
%
%, which states that  $\bigpar$ is associative, 
%commutative and has $0$ as neutral element. 
The inference rules are in \figref{fig:semantics-igsp}.

Rule \ruleName{i-update}, which is analogous to rule \ruleName{update}, adds the operation $\anupd$ 
 to the temporary block $\transactionbuffer$. 
 The decoration $\vertice$ is irrelevant in this model, 
 hence we do not impose any freshness requirement.
 A client terminates a block by executing $\pushcmd$ \ruleName{i-push}. At this time, the block $\transactionbuffer$ 
 is appended to the already terminated blocks in $\pushbuffer$, which will be sent on the next round. 
 Additionally, the block counter $\irounds$ is incremented by $1$.
 By rule \ruleName{i-send}, a client sends changes to the global store. 
 This transition takes place whenever  the client is connected  (\ie, $\cid\in\dom\inserver$),
  there are finished blocks in $\pushbuffer$ (\ie, $\pushbuffer \neq \emptydelta$)  and
  there is no need to re-synchronise  
  (\ie, $\inclient \cdot\outserver(\cid) \neq \agssegpair[{\astate[']}][{\amxrf}]\cdot\aseqseg$)\footnote{For simplicity we check re-synchronisation by inspecting  communication buffers  instead of 
  explicitly adding  to the state the condition {\em channel established} used 
 by the implementation.}.
  %
 The 
 available blocks are sent within the same round $\around = \aroundtuple[@][@][\pushbuffer]$, which contains 
 the number  $\irounds$ corresponding to the last finished block. 
  The new round $\around$ is added to the corresponding input buffer in the store, \ie, $\inserver(\cid)$ 
  is updated to $\inserver(\cid)\cdot\around$ (where $\_\upd{\_}{\_}$ is the  update operator for functions).
  Additionally, $\around$ is added to the sequence of pending rounds $\pending$ and the 
  buffer $\pushbuffer$ is reset to $\emptydelta$.

  Symmetrically,  the client $\cid$  may receive an available segment 
   at any time  \ruleName{i-receive}. The new segment  $\aseg$ is removed from the buffer $\outserver(\cid)$
  on the global store and
    added to the input buffer on the client. As for the idealised model, all received changes are applied to 
   the local replica when $\cid$ performs $\pullcmd$. Rule \ruleName{i-pull$_1$} handles the case in which 
   the connection with the global store has not been previously reset. In such case, 
   all received segments are of the form $\agssegpair$. Therefore, the 
   changes $\adelta[_1]\cdots\adelta[_k]$ are applied to the local state $\astate$ and 
   all rounds  confirmed by the received segments are removed from  the  pending list $\pending$.
   By well-formedness (\defref{def:wf-igsp},\ref{wf-igsp-pending}), it suffices to consider  
    the confirmation $ \amxrf_k$, which is the greatest. %since   ensures  
    %$\amxrf_h(\cid) \leq \amxrf_j(\cid)$ for all ${h}\leq {j}$. 
    Hence, all rounds up-to  
    $\amxrf_k(\cid)$ are removed from $\pending$. This is done by the auxiliary function 
    $\filter {\_} {\_}$, defined as follows           
      \[
      \begin{array}{ll}
      \filter n \aseqround = \aroundtuple[\cid][\irounds_j][{\adelta_j}]\cdots
      \aroundtuple[\cid][\irounds_k][{\adelta_k}] \quad \mbox{if}\quad
      &\ \ 
      \aseqround =  \aroundtuple[\cid][\irounds_0][{\adelta_0}]\cdots\aroundtuple[\cid][\irounds_j][{\adelta_j}]\cdots
      \aroundtuple[\cid][\irounds_k][{\adelta_k}],\quad
      \\
      &
      \hfill\irounds_{j-1}\leq n \ \mbox{and}\ \irounds_j>n
      \end{array}
      \]
   
Rules \ruleName{i-read} and \ruleName{i-confirm} are analogous the ones in the \gsp\ calculus.   We use
$\igetdeltas{\_}$ as the function that projects a sequence of rounds into the sequence containing the 
corresponding $\adelta$s.
The global store changes its state as prescribed by rule  \ruleName{i-batch}: it collects 
all received rounds in $\inserver$ by using the auxiliary function $\receiveroundsname{\_}$, 
which builds a unique object $\adelta$ by appending  all rounds in the
input buffers, and 
a function $\amxrf$ that associates each client with the number of the last received round.  
Let $\inserver$ be such that $\dom\inserver = \{\cid_0,\ldots,\cid_m\}$ and
       $\forall\cid_l\in\dom\inserver.\inserver(\cid_l) = \pending_l\cdot
      \aroundtuple[\cid_l][\irounds^{k_l}_l][{\adelta^{k_l}_l}]$. 
      Then, 
$\receiveroundsname{\_}$ is defined as follows  
\[\begin{array}{ll}
   \receiveroundsname{\inserver} = \agssegpair[@][{\amxrf[']}] \quad \mbox{with}\quad\ &
 %  \begin{itemize}
      %\item
        \adelta = \ireduce{\igetdeltas{\inserver(\cid_0)} \cdots\igetdeltas{\inserver(\cid_m)}},
     % \item 
     \\
     &
     \dom{\amxrf[']} = \{\cid \ |\ \cid\in\dom{\inserver} \ \mbox{and}\ \inserver(\cid)\neq\epsilon\}\ \mbox{and}\  
      \\&\hfill \forall\cid_l\in\dom\inserver.\amxrf['](\cid_l) = \irounds^{k_l}_l
    \end{array}
 \] 

%       $\forall\cid_l\in\dom\inserver.\inserver(\cid_l) =  \aroundtuple[\cid_l][\irounds^0_l][{\adelta^0_l}]\cdots
%      \aroundtuple[\cid_l][\irounds^{k_l}_l][{\adelta^{k_l}_l}]$. 
%      Then, 
%$\receiveroundsname{\_}$ is defined such as 
%\[\begin{array}{ll}
%   \receiveroundsname{\inserver} = \agssegpair[@][{\amxrf[']}] \quad \mbox{with}\quad\ &
% %  \begin{itemize}
%      %\item
%        \adelta = \ireduce{\adelta^0_0\cdots\adelta^{k_0}_0\cdots\adelta^0_m\cdots\adelta^{k_m}_m},
%     % \item 
%     \\
%     &
%     \dom{\amxrf[']} = \{\cid \ |\ \cid\in\dom{\inserver} \ \mbox{and}\ \inserver(\cid)\neq\epsilon\}\ \mbox{and}\  
%      \\&\hfill \forall\cid_l\in\dom\inserver.\amxrf['](\cid_l) = \irounds^{k_l}_l
%    \end{array}
% \] 
%\end{itemize}
   The obtained $\adelta$ is applied the current state $\astate$ and 
   $\amxrf$ is updated with $\amxrf[']$. In addition, the new segment $\agssegpair[@][{\amxrf{[\amxrf[']]}}]$ is sent
   to  every connected client, i.e., it is added at the end of
    every buffer $\outserver(\cid)$. The input buffers  $\inserver(\cid_l)$ are
     emptied because  all received rounds have been processed.

The remaining rules  deal with connectivity issues: rule \ruleName{i-drop-cxn} models a client's disconnection: the buffers $\outserver(\cid)$ and $\inserver(\cid)$ are removed from the global store and also 
the input buffer of $\cid$ is set to $\epsilon$. When the client $\cid$ \mbox{(re-)}establishes its connection \ruleName{i-accept-cxn}, the store
creates the buffers for $\cid$ and sends a segment containing the current 
state of the store. Rule \ruleName{i-pull$_2$} is analogous to \ruleName{i-pull$_1$}, but handles the first
$\pullcmd$ after a reconnection. For this reason, the first received segment  
$\agssegpair[{\astate[''']}][{\amxrf[_0]}]$ contains a state instead of a delta object. 
The client uses this new state $\astate[''']$ instead of its local state to  resynchronise. The application of 
successive segments is analogous to rule \ruleName{i-pull$_1$}.

The proposed implementation allows for a server to crash, i.e., to close all communication buffers, but we
do not model explicitly this behaviour  because  it can be obtained by applying  rule 
\ruleName{i-drop-cxn} several times.
%
%\\[25pt]
%\mathax{i-crash}{\server{\astate}{\inserver}{\outserver}\ \bigpar \ \iclient
% \tr{\tau} \server{\astate}{\undefined}{\undefined}\ \bigpar \ \iclient}


    

% \[
% \begin{array}{l}
%    \hspace{-.3cm} \textsc{SERVER}\\
%    
%
%\mathrule{batch}
%{\begin{array}{c}
%  \agssegpair[@][{\amxrf[{'}]}] =\receiveroundsname{\inserver} \qquad\qquad %=\append{\emptygssegment}{rs} \qquad 
%  \astate[']=\iapply{\astate}{\adelta} \\
%  \forall\cid.(\outserver['] (\cid)= \outserver(\cid)\cdot\agssegpair[@][{\amxrf[{'}]}] \quad \land\quad 
%  \inserver['](\cid) = \epsilon)
%\end{array}
%  }
%  {\iserverins\ \bigpar \ \iclient
%  \tr{\tau} 
%  {\iserverins[{\astate[']}][{\amxrf[{'}]}][{\inserver[']}][{\outserver[']}]}\ \bigpar \ \iclient}
%\\
%\mbox{\henote{este receiveRounds es distinto al tuyo, hace todo de una}}
%\\
%\mathrule{accept-conn}
%{\cid\notin \inserver \qquad \cid \notin \outserver}
%{\server{\astate}{\inserver}{\outserver} \ \bigpar\  \iclient 
%\tr{\tau} 
%\server{\astate}{\inserver\upd\cid\epsilon}{\update{\outserver}{\cid}{\agsprefpair}}\ \bigpar\  \iclient }
%\\
%\mathrule{drop-conn}
%	{\cid \in \inserver \qquad  \cid \in \outserver}
%	{
%	\begin{array}{l}
%	\addid{\iclientinst[@][@][\aseqround][@][@][@][@][\inclient]}
%		\ \bigpar \ \server{\astate}{\inserver}{\outserver} \ \bigpar \ \iclient
%	\tr{\tau} \hspace{4.5cm}\\
%	\hfill
%	\addid{\iclientinst[@][@][\aseqround][@][@][@][@][\epsilon]}
%	\ \bigpar \server{\astate}{\inserver \setminus\cid}{\outserver \setminus\cid}\ \bigpar \ \iclient
%	\end{array}
%	} 
%\\
%
%%{\cid \in \inserver \qquad  \cid \in \outserver}
%%{\server{\astate}{\inserver}{\outserver} \ \bigpar \ \iclient
%%\tr{\tau} 
%%\server{\astate}{\inserver \setminus\cid}{\outserver \setminus\cid}\ \bigpar \ \iclient} 
%%\\
%\mathax{crash}{\server{\astate}{\inserver}{\outserver}\ \bigpar \ \iclient
% \tr{\tau} \server{\astate}{\undefined}{\undefined}\ \bigpar \ \iclient}
%
%			     
%%{\cid \in \inserver \qquad  \cid \in \outserver}
%%{\server{\astate}{\inserver}{\outserver} \ \bigpar \ \iclient
%%\tr{\tau} 
%%\server{\astate}{\inserver \setminus\cid}{\outserver \setminus\cid}\ \bigpar \ \iclient} 
%%\\
%%\mathax{crash}{\server{\astate}{\inserver}{\outserver}\ \bigpar \ \iclient
% %\tr{\tau} \server{\astate}{\undefined}{\undefined}\ \bigpar \ \iclient}
%\\
%
%\\
%	\mathrule{pull-2}
%	{
%	\begin{array}{c}
%	\inclient = \agssegpair[{\astate[''']}][{\amxrf[_0]}]\cdot\agssegpair[{\adelta[_1]}][{\amxrf[_1]}]\ldots \agssegpair[{\adelta[_k]}][{\amxrf[_k]}]
%	\qquad 
%	\aseqround['] = \filter{\amxrf[_k](\cid)} \aseqround\\
%	\astate['']  = \iapply{\astate[''']}{\ireduce{\emptydelta\cdot\adelta[_1]\cdots\adelta[_k]}}
%	\end{array}
%	 }
%		 {\begin{array}{l}
%			\addid{\iclientinst[\tpullins][@][\aseqround][@][@][@][@][\inclient]}
%			     \ \bigpar \ \iserverins[{\astate[']}][{\amxrf}] \ \bigpar\ \iclient
% %[]\ \bigpar \ \isystemterm
%			\arro{\pulltran} \hspace{3cm}
%			\\
%			\hspace{1.5cm}
%			\hfill\addid{\iclientinst[P]
%						 [{\astate['']}]
%						 [{\aseqround'}]
%						 [@][@]
%						 [{\receivebuffer}]}
%		\ \bigpar \ \iserverins[{\astate[']}][{\amxrf}][\inserver\upd{\cid}{\inserver(\cid)\cdot\aseqround[']}]
%		\ \bigpar\ \iclient
%	 	\end{array}
%		}
%\\
%
%
%
%%\\
%%	\mathrule{pull-2}
%%	{\aseqround['] = filter\ (\geq \amxrf(\cid)) \ \aseqround}
%%		 {\begin{array}{l}
%%			\addid{\iclientinst[\tpullins][@][\aseqround][@][@][@][@][{\agssegpair[{\astate[']}]}\cdot\inclient]}
%%			     \ \bigpar \ \iserverins[{\astate['']}][{\amxrf[']}] \ \bigpar\ \iclient
%% %[]\ \bigpar \ \isystemterm
%%			\arro{\pulltran} 
%%			\\
%%			\hspace{1.5cm}
%%			\addid{\iclientinst[P]
%%						 [{\astate[']}]
%%						 [{\aseqround'}]
%%						 [@][@]
%%						 [{\receivebuffer}]}
%%		\ \bigpar \ \iserverins[{\astate['']}][{\amxrf[']}][\inserver\upd{\cid}{\inserver(\cid)\cdot\aseqround[']}]
%%		\ \bigpar\ \iclient
%%	 	\end{array}
%%		}
%\\
%%    \\[35pt]
%%    \hspace{-.3cm} \textsc{COMMUNICATION}
%%		
%%\\
%%    \mathrule{comm server-client}{\outserver(i)=\gs \cdot \gss}{\server{\state}{\inserver}{\outserver} \bigpar \clientr{P}{\stateclient} \arro{\tau} \server{\state}{\inserver}{\update{\outserver}{i}{\gss}} \bigpar 
%%		\clientr{P}{\update{E}{\inclient}{E.\inclient \cdot \gs}}}
%%		
%%\\[25pt]
%%
%%	    \mathrule{comm client-server}{\outclient=\headerround \cdot \tailround}{\server{\state}{\inserver}{\outserver} \bigpar \clientr{P}{\stateclient} \arro{\tau} \server{\state}{\update{\inserver}{i}{\inserver(i) \cdot \headerround}}{\outserver} \bigpar 
%%		\clientr{P}{\update{E}{\outclient}{\tailround}}}	
%% \\
%%
% \end{array}
% \]



%\henote{
%Rule $\textsc{(drop-conn)}$ removes from the server's queues the client called $b_i$. Rule $\textsc{(crash-and-recover)}$ leaves undefined the server's queues and preserves the server's persistent state.  Rule $\textsc{(accept-conn)}$ show how to add a new connecti(Read)send to its the persistent state. Last rule from \textbf{\textsc{(server)}} is Rule $\textsc{(batch)}$, the most interesting of this group. Its has three hypothesis, the first one, is responsible for receiving rounds from the queue of in-messages. Se\cond one, let $\emptygssegment$ be a empty segment, it gives back a delta object who represents the combination of numbers of rounds into a single object. Finally, this object is applied to the persistent state. As result, the persistent state is putted into the queues messages-out.
%There are two \textbf{\textsc{(communications)}} rules. Rule $\textsc{(comm-server-client)}$ when the server has a message for client $i^{th}$, this is removed from the server's queue message-out and is putted into the queue message-in from client $i^{th}$. Rule $\textsc{(comm-client-server)}$ states when a round from client $i^{th}$ is left into server queue message-in.
%Rule $\textsc{(Read)}$ gives the result of performing a lecture on messages queues from the client. Rule $\textsc{(update)}$ adds an update to the transaction buffer. Rule $\textsc{(push)}$ leaves into push buffer a delta object resulted of reducing the push buffer with transaction buffer. The transaction buffer is cleaned and the number of rounds sent is incremented by one. Rule $\textsc{(pull)}$ shows when a persisted state from a client is modified. The hypothesis are that a channel have been accepted, i.e., these must be defined for client $i^{th}$ besides the receive buffer should have an element at least. $\textsc{(confirmed)}$
%computes the states of the internal queues,i.e., if these has any element. Rules $\textsc{(while-true)}$ and $\textsc{(while-false)}$ are standard. Rule $\textsc{(send)}$ creates a new round setting who is the client ($i^{th}$), how many rounds client ($i^{th}$) has sent and content from the push buffer. Finally rule $\textsc{(receive)}$ move out segments from the client's queue message-in to receive buffer.}
%

As for the \gsp, well-formedness is preserved by reductions.

\begin{lemma}\label{lemma:igsp-wf} Let  $\isystemterm$ be a well-formed \igsp\ system. If 
 $\isystemterm\arroi{\action}\isystemterm[']$, then $\isystemterm[']$ is well-formed.
\end{lemma}
