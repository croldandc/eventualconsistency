% !TEX root = main.tex

 % !TEX root = main.tex

\section{{\gsp} with atomic updates}
\label{sec:transactions}

In this section we study the extension of the \gsp\ model with atomic updates proposed in~\ref{}. 
The language of programs is extended in the following way: 
\[
 \rName{program} 
			 \qquad 
			 \tprogram \ ::=\  \ldots\ |\ \tsyncupdins 
\]

The intended meaning of the program  $\tsyncupdins$ is that it remains blocked until the 
update $\anupd$ is performed over the global store.  This is achieved by continuously pulling (i.e., 
a busy-waiting) until the updates are confirmed by the server. 

To formally define the semantics of atomic updates we consider the following 
runtime syntax.
\[
 \rName{program} 
			 \qquad 
			 \tprogram \ ::=\  \ldots\ |\  \ttrans\ |\ \waitcmd \ |\ \tguarded{} 
			 \]

We rely  on the following additional labels
			 
\[  \actbyc ::= \ldots \ |\ (\tupdlbl,\cid)\  |\ (\finishsynctran,\cid)
\]


We define the operator $[\_]$ over actions  $\lambda$ s.t. $[\updatetran{\anupd}] =\tupdlbl$ and it is the identity
over any other action.

\[
 \begin{array}{l}

\mathax{start-upd}
	{
	\begin{array}{l}
	\tsystem{\tclienti{\tsyncupdins}{\tknown}{\tpending}{\ttransactionbuffer}{\tsent}{\treceivebuffer}} 
%	\arroi{\syncupdtran{\tupdate}}
	\arroi{\tau}
	\hspace{6cm}\\\hfill
	\tsystem{\tclienti{\ttrans[\tupdins;\pushcmd;\waitcmd]}{\tknown}{\tpending}{\ttransactionbuffer}{\tsent}{\treceivebuffer}}
	\end{array}
	 }
 
 \\[15pt]

\mathrule{trans}
	{\tsystem{\tclienti{\tprogram}{\tknown}{\tpending}{\ttransactionbuffer}{\tsent}{\treceivebuffer}} 
	\arroi{\mu} 
	\tsystem{\tclienti{\tprogram'}{\tknown'}{\tpending'}{\ttransactionbuffer'}{\tsent'}{\treceivebuffer'}}}
	{\tsystem{\tclienti{[\tprogram];Q}{\tknown}{\tpending}{\ttransactionbuffer}{\tsent}{\treceivebuffer}}
	 \arroi{[\mu]} 
	 \tsystem{\tclienti{[\tprogram'];Q}{\tknown'}{\tpending'}{\ttransactionbuffer'}{\tsent'}{\treceivebuffer'}}}

 \\[25pt]

\mathax{end-upd}
	{\tsystem{\tclienti{[0];\tprogram}{\tknown}{\tpending}{\ttransactionbuffer}{\tsent}{\treceivebuffer}} 
	\arroi{\finishsynctran{}} 
	\tsystem{\tclienti{\tprogram}{\tknown}{\tpending}{\ttransactionbuffer}{\tsent}{\treceivebuffer}}}

\\[25pt]
\mathax{wait}
	{
	\begin{array}{l}
	\tsystem{\tclienti{\waitcmd}{\tknown}{\tpending}{\ttransactionbuffer}{\tsent}{\treceivebuffer}}
	\arro{\tau} 
	\\
	\hspace{2.2cm}
	\tsystem{\tclienti{\tconfirmedins{x}[{\tguarded[x][\pullcmd;\waitcmd][0]}]}{\tknown}{\tpending}{\ttransactionbuffer}{\tsent}{\treceivebuffer}}
	\end{array}
	}

\\[25pt]
\mathrule{guard-true}
	{\eval \cond {true}}
	{
	\tsystem{\tclienti{\tguarded[e][P][Q]}{\tknown}{\tpending}{\ttransactionbuffer}{\tsent}{\treceivebuffer}}
	\arro{\tau} 
	\tsystem{\tclienti{Q}{\tknown}{\tpending}{\ttransactionbuffer}{\tsent}{\treceivebuffer}}
	}

\\[25pt]
\mathrule{guard-false}
	{\eval \cond {false}}
	{
	\tsystem{\tclienti{\tguarded[e][P][Q]}{\tknown}{\tpending}{\ttransactionbuffer}{\tsent}{\treceivebuffer}}
	\arro{\tau} 
	\tsystem{\tclienti{P}{\tknown}{\tpending}{\ttransactionbuffer}{\tsent}{\treceivebuffer}}
	}
%	\\
%\henote{\mbox{Agregar reglas para if}}

%
%
% 
%\mathax{t-flush}{\tsystem{\tclienti{[\tflush];\tprogram}{\tknown}{\tpending}{\ttransactionbuffer}{\tsent}{\treceivebuffer}\ \bigpar\ C}{\queuemessage} \arroi{\tau} \tsystem{\tclienti{[\pushtran;\pwhile{!\confirmedtran()}{\pulltran};];\tprogram}{\tknown}{\tpending}{\ttransactionbuffer}{\tsent}{\treceivebuffer}\ \bigpar\ C}{\queuemessage}}
%\\
%\chnote{\mbox{En realidad el programa es: $\pushtran;let x = \confirmedtran();\pwhile{!x}{\pulltran};let x = \confirmedtran();$}}
%
% \\[15pt]

 

 \end{array}
 \]
 
%The nature of such cases is caused by having no transactions as the usual databases.

%The remaining of this section is proposing a environment that ensures these consistency guarantees. We start by refining the environment in GSP. We will distinguish a new relation:

%\[ 
%\begin{array}{r@{\ ::= \ }l}
%  \zeta & \mu \ | \ \startsynctran \ | \ \finishsynctran  \ | \  [\mu] \\
%\end{array}
%\]






%\[
% \begin{array}{l} \hspace{-.3cm} \textsc{PROGRAM}\\
%
%
%\mathax{t-sync-update}{\tsystem{\tclienti{\tsyncupd{u};\tprogram}{\tknown}{\tpending}{\ttransactionbuffer}{\tsent}{\treceivebuffer}\ \bigpar\ C}{\queuemessage} \arroi{\syncupdtran{\tupdate}} \tsystem{\tclienti{[\tupdins;\tflush];\tprogram}{\tknown}{\tpending}{\ttransactionbuffer}{\tsent}{\treceivebuffer}\ \bigpar\ C}{\queuemessage}}
% 
% \\[15pt]
%
%\mathrule{t-tran}{\tsystem{\tclienti{\tprogram}{\tknown}{\tpending}{\ttransactionbuffer}{\tsent}{\treceivebuffer}\ \bigpar\ C}{\queuemessage} \arroi{\mu} \tsystem{\tclienti{\tprogram'}{\tknown'}{\tpending'}{\ttransactionbuffer'}{\tsent'}{\treceivebuffer'}\ \bigpar\ C}{\queuemessage}}{\tsystem{\tclienti{[\tprogram];Q}{\tknown}{\tpending}{\ttransactionbuffer}{\tsent}{\treceivebuffer}\ \bigpar\ C}{\queuemessage} \arroi{[\mu]} \tsystem{\tclienti{[\tprogram'];Q}{\tknown'}{\tpending'}{\ttransactionbuffer'}{\tsent'}{\treceivebuffer'}\ \bigpar\ C}{\queuemessage}}
%
% \\[25pt]
%
% 
%\mathax{t-flush}{\tsystem{\tclienti{[\tflush];\tprogram}{\tknown}{\tpending}{\ttransactionbuffer}{\tsent}{\treceivebuffer}\ \bigpar\ C}{\queuemessage} \arroi{\tau} \tsystem{\tclienti{[\pushtran;\pwhile{!\confirmedtran()}{\pulltran};];\tprogram}{\tknown}{\tpending}{\ttransactionbuffer}{\tsent}{\treceivebuffer}\ \bigpar\ C}{\queuemessage}}
% 
%\\
%\chnote{\mbox{En realidad el programa es: $\pushtran;let x = \confirmedtran();\pwhile{!x}{\pulltran};let x = \confirmedtran();$}}
%
% \\[15pt]
%
%\mathax{t-end-sync}{\tsystem{\tclienti{[0];\tprogram}{\tknown}{\tpending}{\ttransactionbuffer}{\tsent}{\treceivebuffer}\ \bigpar\ C}{\queuemessage} \arroi{\finishsynctran(u^{[v]})} \tsystem{\tclienti{\tprogram}{\tknown}{\tpending}{\ttransactionbuffer}{\tsent}{\treceivebuffer}\ \bigpar\ C}{\queuemessage}}
% 
%
% \end{array}
% \]
% 


\subsection{Operational Semantic}

%\subsubsection{Notation}

%\begin{flushleft}
%\specfunction{$\downharpoonright$}{$\verticesets$}{$2^\verticesets$}{$2^\verticesets$} \\
%$\downharpoonright v^i \emptyset$ =  $\emptyset$ \\
%$\downharpoonright v^i (x^h \mapsto V)$ = x^h \mapsto V \cup \ v^i;  $\downharpoonright v^i V$ \\
%\end{flushleft}

%
%We assume the following countable sets of transactional vertices names $[\verticesets]$ ranged over by $[\vertice];[\vertice_0];[\vertice_1],\ldots$;
%
%The labeled transition system for the extension of GSP considers the following actions $\zeta$:	
%
%
%
%
%These labels allow system to perform an update synchronous. Label $\startsynctran$ stands for the begining of a transaction, and $\finishsynctran$ the end of it.

Since update operations are not instantaneous, now we introduce to the model the relations that take into account the 
beginning and finalisation of each write operation. In particular,

\begin{itemize}
   \item{\em Return Before} (\rb), which indicates the ordering of non-overlapping update operations. 
\end{itemize}

The extended environment will also contain two additional terms that are instrumental to the 
computation of $\rb$. 

\begin{itemize}
	\item $\tx$ that relates a vertex to set of vertices which have just not finished. Formally,  
	$\tx :\verticesets\ \mapsto 2^{\verticesets}$, i.e., it is a function from vertices to set of vertices.
	\item $\tc$ that denotes the set of closed transactions, i.e., $\tc\subseteq \verticesets$.
\end{itemize}

\[
    \begin{array}{l@{\quad}r@{\;::=\;}l}
			 (\textsc{environment}) & \environmentterm &  \environmenttran{\systemterm}{\op}{\so}{\vis}{\arb}{\rb}{\tx}{\tc} \\
			 (\textsc{transactions with overlapping}) & \tx &  \emptyset  \ | \ \verticesets\ \mapsto 2^{\vertice},\tx   \\
			 (\textsc{transactions closed}) & \tc &  \epsilon  \ | \ \vertice\ \cdot \tc 
	   \end{array}
\]


\paragraph{Notation} $\updopentx{\epsilon} = \epsilon$, and 
   $\updopentx{\vertice_0 \mapsto \{\instanceset\};tail} = \vertice_0 \mapsto \{\instanceset \cup \{\vertice\}\};\updopentx{tail}$.
		


		
 \[
 \begin{array}{l}  
\mathrule
	{e-start-sync}
	{\systemterm \arroi{\syncupdtran{\tupdate}} \systemterm' 
		\qquad [\vertice] \notin 
dom(\tx)}{\environmenttran{\systemterm}{\op}{\so}{\vis}{\arb}{\rb}{\tx}{\tc} \arroi{\startsynctran{\vertice}} \environmenttran{\systemterm'}{\op}{\so}{\vis}{\arb}{\rb}{\updopentx{\tx};\vertice \mapsto dom(\tc)}{\tc}}

 \\[35pt]
{
\mathrule{e-end-sync}{\systemterm \arroi{\finishsynctran{\tupdate}} \systemterm' \qquad [\vertice] \in 
dom(\tx)}{\environmenttran{\systemterm}{\op}{\so}{\vis}{\arb}{\rb}{\vertice \mapsto \instanceset,\tx}{\tc} \arroi{\finishsynctran{\vertice}} \environmenttran{\systemterm'}{\op}{\so}{\vis}{\arb}{\rb'}{\tx}{\tc \cdot\ \vertice}}}
\\
 \\[10pt]where \ 
\rb' = \rb \cup\ \{ (x,v) \ | \ \forall 0 \leq m,n < |\queuemessage|, \op(x) = \queuemessage[m] \ \land \ \op(v) = \queuemessage[n] \ \land \ m \leq\ n \ \land \ x\notin\tx(v) \}
 \end{array}
 \]




First, we prove a useful lemma: 

\begin{lemma}\label{lemma:empty_queue} 

$\forall C$ such that 
$\tclienti{Q}{\tknown}{\tpending}{\ttransactionbuffer}{\tsent}{\treceivebuffer} \ 
    \Arro^* \ \tclienti{\tsyncupd{u};\treadins{x}{r}}{\tknown}{\tpending}{\ttransactionbuffer}{\tsent}{\treceivebuffer} 
	  \Arro^* \ \tclienti{\treadins{x}{r}}{\tknown}{\tpending}{\ttransactionbuffer}{\tsent}{\treceivebuffer}$, then  
rvalue operation always reads above the store.

\end{lemma}

\begin{proof} Confirmed is evaluated to true when $\tpending$ and $\ttransactionbuffer$ are empty so that, after the loop, the read operation only return values which belong to the Store.
\end{proof}	


Single Order is a consistency guarantee that express a single order of operations observed by arbitration and visibility relation. The definition introduced by \cite{} requires 
$\ovis = \roarb$, where $F$ stands for the set of finished operations, i.e., when consider visibility restricted to the operations that belong to $\tc$. 
Since we defined $\arb$ and $\vis$ over restricted domains, the original formulation is not useful. Hence we will use an alternative characterisation, given by the 
following result.

\begin{lemma} $\ovis = \roarb$ iff 
\begin{enumerate}
   \item $\roarb;\ovis \subseteq \ovis$; and
   \item $\neg\ovis;\neg\roarb \subseteq \neg\ovis$.
\end{enumerate}
\end{lemma}

\begin{proof} $\Rightarrow$) 
\begin{enumerate}
\item
	\[ \begin{array}{l@{\ =\ }l@{\qquad}l}
		\roarb;\ovis &  \roarb;\roarb & \roarb =\ovis\\
		& \roarb & \roarb \mbox{is a partial order (i.e., transitive)}\\
		&  \ovis & \roarb =\ovis
   	\end{array}
	\]
\item Note that  $\roarb;\ovis =\ovis$ implies $\neg(\roarb;\ovis) =\neg\ovis$. By property of complement and composition, $\neg(\roarb;\ovis) =\neg\ovis;\neg\roarb =\neg\ovis$. 
\end{enumerate}

$\Leftarrow)$ We divided the proof in two implications
 \begin{enumerate}
    \item  $\roarb;\ovis \subseteq \ovis$ implies $\ovis\subseteq \roarb$. We proceed by contradiction. Assume $(a,b)\in\ovis$ but $(a,b)\not\in\roarb$.
    Since $\roarb$ is a total order, $(b,a)\in\roarb$. By assumption, $(b,a)\in\roarb$ and $(a,b)\in\ovis$ implies $(b,b)\in\ovis$, but this is in contradiction 
    with the fact that $\ovis$ is acyclic. 
        
    \item $\neg\ovis;\neg\roarb \subseteq \neg\ovis$ implies $\roarb\subseteq\ovis$. We proceed by contradiction. 
    Assume $(a,b)\in\roarb$ and $(a,b)\not\in\ovis$, and hence,  $(a,b)\in\neg\ovis$. Since,
    $\roarb$ is a strict total order $(b,a)\not\in\roarb$. Hence, $(b,a)\in\neg\roarb$. By assumption, $(a,b)\in\neg\ovis$ and $(b,a)\in\neg\roarb$ implies $(a,a)\in\neg\ovis$.
    Again, by assumption, 
 \end{enumerate}

\end{proof}

\begin{theorem}[\textsc{Single Order}]

Let $\textsc{\small{AR}}$ an arbitration relation and $\textsc{\small{VIS}}$ a visibility relation, if $\environmenttran{\systemterm}{\emptyset}{\emptyset}{\emptyset}{\emptyset}{\emptyset}{\undefined}{\emptyset} \arro{} ^*\ \environmenttran{\systemterm'}{\op}{\so}{\vis}{\arb}{\rb}{\tx}{\tc}$ then $\textsc{\small{AR}};\textsc{\small{VIS}} \subseteq \textsc{\small{VIS}}$ and  $\textsc{\small{AR}}^{-1};\neg\textsc{\small{VIS}} \subseteq \neg\textsc{\small{VIS}}$.


\end{theorem}
 

\begin{proof} 
Firstly, we will prove by induction on on the length of the derivation that $\textsc{\small{AR}};\textsc{\small{VIS}} \subseteq \textsc{\small{VIS}}$, i.e., for all $\vertice_a$,$\vertice_b$,$\vertice_c$ such that ($\vertice_a$,$\vertice_b$) $\in$ $\textsc{\small{AR}}$ and ($\vertice_b$,$\vertice_c$) $\in$ $\textsc{\small{VIS}}$ then ($\vertice_a$,$\vertice_c$) $\in$ $\textsc{\small{VIS}}$.

\begin{itemize}
   \item{\bf n=0}. It is easy to see because of both relations are empty, hence is trivial $\emptyset \subseteq \emptyset$.
   \item{\bf n=k+1}. Then $\environmenttran{\systemterm_0}{\emptyset}{\emptyset}{\emptyset}{\emptyset}{\emptyset}{\undefined}{\emptyset} \arro{} ^n\ \environmenttran{\systemterm}{\op}{\so}{\vis}{\arb}{\rb}{\tx}{\tc} \arroi{\alpha} \environmenttran{\systemterm'}{\op'}{\so'}{\vis'}{\arb'}{\rb'}{\tx'}{\tc'}$. We proceed by case analysis on the last transition:
	
	\begin{itemize}
        \item {\bf rule (\textsc{E-READ})}. $\textsc{\small{AR}}$ does not change, therefore, $\textsc{\small{AR}}'$ = $\textsc{\small{AR}}$.  Then, there exists two cases: \chnote{\mbox{la regla es S-READ, pero es igual a E-READ sin los ultimos terminos, charlar como escribir esto}}
				
				\begin{itemize}
					\item $\vertice_c \neq \vertice$, as $\vertice$ is an vertex associated to last computation step then ($\vertice_b$,$\vertice_c$) $\in$ $\textsc{\small{VIS}}$, therefore, these belong to $\textsc{\small{VIS}}'$. Then, by inductive hypothesis, ($\vertice_a$,$\vertice_c$) $\in$ $\textsc{\small{VIS}}$ so that as $\vertice_c \neq \vertice$ then ($\vertice_a$,$\vertice_c$) $\in$  $\textsc{\small{VIS}'}$.
					\item $\vertice_c$ = $\vertice$, as $\vertice_a$ and $\vertice_b$ are in $\textsc{\small{AR}}$ then there exist two updates into the Store associated to them. Let $\udec[u][\vertice_a]$, $\udec[u][\vertice_b]$ be these updates, then we know that $\udec[u][\vertice_a]$ = $\queuemessage$[m] and $\udec[u][\vertice_b]$ = $\queuemessage$[n] such that m < n. If ($\vertice_b$,$\vertice_c$) $\in$ $\textsc{\small{VIS}}$ then $\udec[u][\vertice_b]$ $\in \queuemessage[0..\tknown-1] \cdot \tpending \cdot [\ttransactionbuffer]$ by definition. Then, analyzing cases:
					\begin{itemize}
						\item $\udec[u][\vertice_b]$ $\in \queuemessage[0..\tknown-1]$. Then $\udec[u][\vertice_a]$ also is in $\queuemessage[0..\tknown-1]$ because of $\udec[u][\vertice_a]$ = $\queuemessage$[m] and $\udec[u][\vertice_b]$ = $\queuemessage$[n] and m < n.
						\item $\udec[u][\vertice_b]$ $\in \tpending$ or $\udec[u][\vertice_b]$ $\in [\ttransactionbuffer]$, however these do not be by $\lemref{lemma:empty_queue}$.
					\end{itemize}
					
				\end{itemize}
				
			\item The rest of the rules do not change $\textsc{\small{VIS}}$.

\end{itemize}

\end{itemize}

Now, we will prove by induction on on the length of the derivation that $\textsc{\small{AR}}^{-1};\neg\textsc{\small{VIS}} \subseteq \neg\textsc{\small{VIS}}$, i.e., for all $\vertice_a$,$\vertice_b$,$\vertice_c$ such that ($\vertice_b$,$\vertice_a$) $\in$ $\textsc{\small{AR}}$ and ($\vertice_a$,$\vertice_c$) $\notin$ $\textsc{\small{VIS}}$ then ($\vertice_b$,$\vertice_c$) $\notin$ $\textsc{\small{VIS}}$. This prove is analogue when $\vertice_c \neq \vertice$. Furthermore, if $\udec[u][\vertice_a]$ $\notin \queuemessage[0..\tknown-1]$ then $\udec[u][\vertice_b]$ $\notin \queuemessage[0..\tknown-1]$ because of $\udec[u][\vertice_b]$ is before to $\udec[u][\vertice_a]$ at Store, then 
\end{proof}
	
 \appendix
 
 \section{Functions for implementing datatypes}
 \subsection{Auxiliar Function}

Before describing the implementation of GSP, we will introduce the following auxiliars functions which will be used when we define the operational semantic.

\footnotesize
\ttfamily


\begin{flushleft}
\specfunction{append}{\gssegmenttype}{$\roundtype^*$}{\gssegmenttype} \\
\append{\gssegmentins{$\delta$}{\maxround}}{$\epsilon$} = \gssegmentins{$\delta$}{\maxround} \\
\append{\gssegmentins{$\delta$}{\maxround}}{$\headerround$:\tailround} = \append{\textless\reduce{$\delta$ $\cdot$ $\delta_0$ $\cdot$ $\epsilon$}, $\update{\maxround}{b_0}{n_0}$\textgreater}{\tailround} 
\end{flushleft}

\begin{flushleft}
\specfunction{apply}{\gsprefixtype}{\gssegmenttype}{\gsprefixtype} \\
\apply{$\gsprefixins{\state}{\maxround}$}{\gssegmentins{$\delta$}{$\maxround'$}} =  \gsprefixins{\applyplus{\state}{$\delta$ $\cdot$ $\epsilon$}}{\maxround[$\maxround'$]}  \\
\end{flushleft}

\begin{flushleft}
\specfunctiononeparameter{receivedrounds}{($\partialfunction{\idset}{\textless \dominserver \textgreater^*}$)}{$\roundtype^*$}\\
receivedrounds($\undefined$) = $\epsilon$ \\
receivedrounds($\partialfunction{b}{\textless n_0,\ \delta_0 \textgreater} \cdot\ f$) = \textless $b$, \ $n_0$, \ $\delta_0$\textgreater\ $\cdot$ receivedrounds($f$)
\end{flushleft}

\begin{flushleft}
\henote{arreglar las lineas comentadas}
\chnote{hecho}
\specfunctionforparameters{curstate}{\statetype}{$\roundtype^*$}{\deltatype}{\deltatype}{\statetype} \\
$curstate(\state, \pending, \pushbuffer, \transactionbuffer$) = $\applyplus{\state}{getdeltas(\pending) \cdot\pushbuffer\cdot \transactionbuffer}$
\end{flushleft}


\begin{flushleft}
\specfunctiononeparameter{getdeltas}{$\roundtype^*$}{$\deltatype^*$} \\
getdeltas($\epsilon$) = $\epsilon$ \\
getdeltas(\textless $n_0$,$\delta_0$ \textgreater $\cdot$ $\delta$) = $\delta_0$ $\cdot$ getdeltas($\delta$)\\
\end{flushleft}

\begin{flushleft}
\specfunctiononeparameter{remove}{$\deltatype^*$}{$\deltatype^*$} \\
remove(rs,$\epsilon$) = $\epsilon$ \\
remove(rs, xs) = $(foldr (\backslash x\ rec\ ys\to h \ (filter \ (/$=$x) ys))\ id) xs \ rs$\\
\end{flushleft}


\begin{flushleft}
\specfunctionthreeparameters{notify}{$\idset^*$}{($\partialfunction{\idset}{\gssegmenttype^* \cup \gsprefixtype}$)}{$\gssegmenttype$}{$(\partialfunction{\idset}{\gssegmenttype^* \cup \gsprefixtype})$}\\
notify($\{\cid\}$, $\outserver$, gs) = $\outserver[\cid \mapsto gss \cdot gs]$\\
notify($\cid \cdot bs$, $\outserver$, gs) = $\notify{bs}{\outserver}{gs}[\cid \mapsto gss \cdot gs]$ 
\end{flushleft}

\begin{flushleft}
\specfunction{\cleannamefun}{$\idset^*$}{$(\partialfunction{\idset}{\dominserver}$)}{$(\partialfunction{\idset}{\dominserver})$}\\
\cleannamefun($\{\cid\}$, $\outserver$, gs) = $\inserver[\cid \mapsto \emptyset]$\\
\cleannamefun($\cid \cdot bs$, $\outserver$, gs) = $\clean{bs}{\inserver[\cid \mapsto \emptyset]$}
\end{flushleft}


\normalfont
\normalsize

\paragraph{Notation.} Let $f$ and $g$ be a partial function, we define the update operator $\_[\_]$ such that
  $dom(f[g])  = dom(f) \cup dom(g)$ and 
 \[
   \begin{array}{l@{\ = \ \Bigg\{}l}
     f[g](x) & 
     \begin{array}{ll}
        f(x) & \mbox{if }\ x\not\in dom(g) \wedge x\in dom(f) \\
        g(x) & \mbox{if }\ x\in dom(g)\\
        \mathit{\undefined}  & \mbox{Otherwise} 
     \end{array}
   \end{array}
 \]    

 We write $[x_1 \mapsto y_1, \ldots, x_n \mapsto y_n]$ for the partial function $f$ such that $dom(f)=\{x_1,\ldots,x_n\}$ and $f(x_i)=y_i$; $A \setminus B$ to denote the usual difference of sets.



The functions \textbf{reduce} and \textbf{apply} are abstract and depend on the data model used. 

