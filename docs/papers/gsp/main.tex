% !TEX root = main.tex

\documentclass[envcountsect,runningheads,orivec]{llncs}

%%%%%%%%%%%%%%%%%%%%%%%%%%%%%%%%
% ENLARGED STYLE FOR SUBMISSION
%%%%%%%%%%%%%%%%%%%%%%%%%%%%%%%%
%\setlength{\textwidth}{15cm}

%\setlength{\textheight}{21cm}
%\addtolength{\oddsidemargin}{-1.25cm}pu
%\setlength{\evensidemargin}{\oddsidemargin}
%%%%%%%%%%%%%%%%%%%%%%%%%%%%%%%%

\usepackage[all]{xy}\CompileMatrices
\usepackage[english]{babel}
%\usepackage[mathcal]{euscript}
%\usepackage{latexsym}
\usepackage{amssymb}
%\usepackage{pslatex}

\usepackage{dsfont}
\usepackage{alltt}
\usepackage{bbm}
\usepackage{url}
\usepackage{subfigure}
%\usepackage{stmaryrd,amsmath,amsfonts,amstext,amssymb,fancybox}

\usepackage{bussproofs}
\usepackage{mathtools}
\usepackage{color} %para los comentarios
\usepackage[dvipsnames]{xcolor}

\usepackage{xargs}

\newcommand{\ancho}{ p{\textwidth} <}

%\newcommand{\powerset}{\mathcal{P}_f}
%\newcommand{\compatible}{compatible}
%\newcommand{\compatibility}{compatibility}

%\newcommand{\bn}{\mathop{\mathrm{bn}}}
%\newcommand{\initial}{\mathop{I}}
%\newcommand{\fn}{\mathop{\mathrm{fn}}}
%\newcommand{\struct}{\equiv}
%%\newcommand{\rec}[2]{\mathsf{rec}\;{#1}.{#2}}
%\newcommand{\sep}{\;\;\mid\;\;}
%\newcommand{\locpar}{|}
%\newcommand{\install}[1]{\mathsf{install}[{#1}]}
%\newcommand{\sedef}{\Rightarrow}
%\newcommand{\merg}[2]{\mathsf{merge}^{#1}\ {#2}}
%\newcommand{\invoke}[1]{\mathsf{invoke}\ {#1}}

%LTS
\newcommandx{\action}[1][1={}]{\lambda{#1}}
\newcommandx{\actbyc}[1][1={}]{\mu{#1}}

\newcommand{\tr}[1]{\xrightarrow[]{#1}}

\newcommandx{\arro}[2][2=\cid]{\xrightarrow[]{#1}_{#2}}
\newcommand{\arroi}[1]{\arro{#1}}
\newcommand{\arrobyclient}[2]{\arro{#1}[#2]}

\newcommand{\auxarro}[1]{\xmapsto{#1}}

%\newcommand{\typearro}[1]{\stackrel{#1}{\mapsto}}
%\newcommand{\labdef}[1]{\top #1}
%\newcommand{\session}{\rhd}
%\newcommand{\labinv}[1]{\bot #1}
%\newcommand{\labmerg}[2]{#1^{#2}}
\newcommand{\deduce}[2]{\frac{\displaystyle #1}{\displaystyle #2}}
\def \mathax #1#2{\begin{array}{l} {\mbox{\scriptsize {{\sc (#1)}}} } \\ #2
\end{array}}

\newcommand{\irule}[2]{\frac{\textstyle\rule[-1.3ex]{0cm}{3ex}#1}%
{\textstyle\rule[-.5ex]{0cm}{3ex}#2}}
%
\def \mathrule #1#2#3{\begin{array}{l}%
      {\mbox{\scriptsize {\ruleName{#1}}} }
    \\ \irule{#2}{#3}%
\end{array}}

%\def \mathrule #1#2#3{\begin{array}{l}
%        {\mbox{\scriptsize {\ruleName{#1}}} }
%        \\ \deduce{#2}{#3}
%\end{array}}

\def \mathrulean #1#2#3{\begin{array}{l}
        {\mbox{\scriptsize {\ruleName{#1}}} }
        \ \deduce{#2}{#3}
\end{array}}

\def \mathaxlean #1#2{\begin{array}{l}
        {\mbox{\scriptsize {\ruleName{#1}}} }
        \  {#2}
\end{array}}


\def \mathanrule #1#2{\begin{array}{l}
				\deduce{#1}{#2}
\end{array}}

\def \mathaxiom #1#2{\begin{array}{l}%
    {\mbox{\scriptsize ({\sc #1})} }%
    \\ \iaxiom{#2}%
    \end{array}}
\newcommand{\iaxiom}[1]{\textstyle\rule[-1.3ex]{0cm}{3ex}#1}
        
\newcommand{\ruleName}[1]{{\sc (#1)}}
      
\newcommand{\rName}[1]{\mbox{\scriptsize {\ruleName{#1}}} }


%\newcommand{\equal}[2]{#1 \stackrel{\cdot}{=} #2}
%\newcommand{\n}{\mathop{\mathrm{n}}}
%\newcommand{\muse}{$\mu\mathsf{se}$}
%\newcommand{\var}{{\mathcal A}}
%\newcommand{\varb}{{\mathcal B}}
%\newcommand{\subst}[2]{\{#1 / #2\}}
%\newcommand{\judge}[4]{#1\vdash #4:\{{#2}\,\nearrow\,{#3}\}}
%\newcommand{\judges}[3]{#1\vdash #3 :\{#2 \}}
%\newcommand{\compliance}{\thickapprox}

%\newcommand{\topfigrule}{\vskip2pt\noindent\rule{\textwidth}{1pt}}
%\newcommand{\olds}[1]{\oldstylenums{#1}}
%\newcommand{\oldsb}[1]{{\bfseries\olds{#1}}}
%\newif{\ifcomments}
%\commentsfalse \ifcomments
%\newcommand{\comment}[1]{\stepcounter{ncomm}%
%\vbox to0pt{\vss\llap{\tiny\oldsb{\arabic{ncomm}}}\vskip6pt}%
%\marginpar{\tiny\bf\raggedright%
%{\oldsb{\arabic{ncomm}}}.\hskip0.5em#1}}
%\newcounter{ncomm}
%\else
%\newcommand{\comment}[1]{}
%\fi
%\newcommand{\marginnote}[2]{\hrule\smallskip\textbf{#1}:{\sf #2}\smallskip\hrule}
%\newcommand{\rest}[1]{(\nu #1)}
%\newcommand{\pinull}{\mathbf{0}}
%\newcommand{\typenull}{\mathsf{0}}
%\newcommand{\subt}{\; \mbox{{{\tt <}} \hspace{-.29cm} \raisebox{.25ex}{\tt :}}\,}
%\newcommand{\bigfract}[2]{\frac{^{\textstyle #1}}{_{\textstyle #2}}}
%\newcommand{\lred}[1]{\stackrel{#1}{\longrightarrow}}
%
%\newcommand{\wsdl}{{\sc wsdl}}
%\newcommand{\lredm}[1]{\stackrel{#1}{\longmapsto}}
%
%
%\newcommand{\bla}{\fbox{bla, bla, bla...}}
%\newcommand{\replace}[2]{{\fbox{#2}}\marginpar{$\star\star$}}
%\newcommand{\replace}[2]{{#2}}


%
% MACRO FOR COMMENTS
%
%\newcommand{\nota}[1]{\noindent \fbox{ \parbox{\textwidth}{#1} }  }
\newcommand{\nota}[1]{}

%
% ABBREVIATIONS AND SYMBOLS
%
\newcommand{\eg}{e.g.}
\newcommand{\ie}{i.e.}
\newcommand{\wrt}{w.r.t.}
\newcommand{\tdot}{..}

\newcommand{\dom}[1]{\textit{dom}(#1)}

\newcommand{\igsp}{\textcolor{red}{\sc igsp}}

\newcommand{\gsp}{\textcolor{red}{\sc gsp}}
\newcommand{\tgspcalculus}{{\sc gsp}}
\newcommand{\gspcalculus}{\textit{gsp-calculus}}

% SEQUENCES

\newcommand{\emptysequence}{\textcolor{OliveGreen}{\epsilon}}


%%%%SYNTAX GSP

\newcommandx{\Absclient}[1][1={}]{\textcolor{OliveGreen}{C{#1}}}
\newcommand{\queuemessage}{\textcolor{OliveGreen}{S}}
\newcommandx{\systemterm}[1][1={}]{\textcolor{OliveGreen}{N{#1}}}

\newcommand{\Absserver}{\textcolor{OliveGreen}{C}}


%sets 
\newcommand{\idset}{\textcolor{blue}{\mathcal{I}}}

\newcommand{\verticesets}{\textcolor{blue}{\mathbb{V}}}
\newcommand{\updatesets}{\textcolor{blue}{\mathcal{U}}}
\newcommand{\readset}{\textcolor{blue}{\mathcal{R}}}
\newcommand{\varset}{\textcolor{blue}{\mathcal{X}}}
\newcommand{\valueset}{\textcolor{blue}{\mathcal{V}}}
\newcommand{\opset}{\textcolor{blue}{\mathcal{O} }}
\newcommandx{\cid}[1][1={}]{\textcolor{blue}{{\sf i#1}}}
\newcommand{\cidj}{\textcolor{blue}{{\sf j}}}

\newcommand{\relrestriccion}[1]{\downarrow_{#1}}

\newcommand{\bigpar}{|\!|}
\newcommandx{\tsystem}[2][2=\systemterm]{#1\ \bigpar\ #2}

% Clients configuration
\newcommandx{\tclient}[8][7=\cid,8=\trounds,usedefault=@]{\langle #1 , #4, #5, #3,  #2, #6, #8\, \rangle_{#7}}
\newcommandx{\tclienti}[8][7=\cid,8=\trounds,usedefault=@]{{\tclient{#1}{#2}{#3}{#4}{#5}{#6}[#7][{#8}]}}

\newcommandx{\anabstcli}[7][1=\cid,2=\tprogram,3=\tknown,4=\tpending,5=\ttransactionbuffer,6=\tsent,7=\treceivebuffer,usedefault=@]
  {\tclient {#2_{#1}}{#3_{#1}}{#4_{#1}}{#5_{#1}}{#6_{#1}}{#7_{#1}}[#1]}
\newcommandx{\anabstcliprime}[7][1=\cid,2=\tprogram',3=\tknown',4=\tpending',5=\ttransactionbuffer',6=\tsent',7=\treceivebuffer']
  {\tclient {#2_{#1}}{#3_{#1}}{#4_{#1}}{#5_{#1}}{#6_{#1}}{#7_{#1}}[#1]}

\newcommand{\tknown}{\textcolor{OliveGreen}{k}}

\newcommandx{\anupd}[1][1={}]{\textcolor{OliveGreen}{{u#1}}}
\newcommandx{\aread}[1][1={}]{\textcolor{OliveGreen}{{r#1}}}

\newcommandx{\anupdseq}[1][1={}]{\textcolor{OliveGreen}{\mathsf{u#1}}}

\newcommandx{\ttransactionbuffer}[1][1={}]{\anupdseq[#1]}
\newcommandx{\ublock}[2][1=\ttransactionbuffer,2=\cid]{\textcolor{Gray}{[#1]}}


\newcommandx{\blockseq}[1][1={}]{\textcolor{OliveGreen}{\mathsf{b#1}}}

\newcommandx{\tpending}[1][1={}]{\textcolor{OliveGreen}{\blockseq[_P]}}
\newcommand{\tsent}{\blockseq[_S]}
\newcommand{\tsenthead}[1][]{\textcolor{Gray}{[\ttransactionbuffer[']#1]}}

\newcommand{\treceivebuffer}{\textcolor{OliveGreen}{j}}

\newcommand{\trounds}{\textcolor{OliveGreen}{n}}

%%%%Syntax of processes
\newcommandx{\updcmd}[1][1=\anupd]{{\tt update}(#1)}
\newcommandx{\readcmd}[1][1= r]{{\tt read}(#1)}
\newcommand{\pullcmd}{{\tt pull}}
\newcommand{\pushcmd}{{\tt push}}
\newcommand{\confcmd}{{\tt confirmed}}



\newcommand{\tupdate}{\udec}

\newcommand{\tupdins}{\updcmd}
\newcommand{\tupdateins}{\tupdins;P}

\newcommandx{\treadins}[3][3=P]{{\tt let}\ #1 = {\tt read}(#2)\ {\tt in}\ #3}

%\newcommand{\treadins}[2]{{\tt let}\ #1 = {\tt read}(#2)\ {\tt in}\ P}
\newcommand{\tpushins}{{\tt push};P}
\newcommandx{\tconfirmedins}[2][2=P]{{\tt let}\ #1 = {\tt confirmed} \ {\tt in}\ #2}
\newcommand{\tpullins}{{\tt pull};P}

\newcommand{\cond}{\textcolor{OliveGreen}{{e}}}
\newcommand{\eval}[2]{#1\downarrow #2}

%\newcommandx{\pwhile}[3][3=P]{{\tt while} (#1)\ {\tt do}\ #2 \ {\tt od}; #3}
\newcommandx{\pwhile}[3][3=P]{{\tt while}\ #1\  \{ #2\};  #3}

\newcommandx{\pifte}[4][1=\cond,2=P,3=P,4=P]{{\tt if} (#1)\ {\tt then}\ #2 \ {\tt else}\ #3 \ {\tt fi}; #4}

\newcommand{\treceivetran}{{\tt receive}}

\newcommand{\plet}[3]{{\tt let}\ #1 = #2 \ {\tt in}\ #3}


\newcommand{\environmentterm}{\mathbb{A}}

\newcommand{\nat}{\textcolor{blue}{\mathbbm{N}}}

%Operational Semantics

\newcommand{\subst}[2]{\{#2/{#1}\}}

%Labels

\newcommandx{\udec}[2][1=u,2=\vertice,usedefault=@]{\textcolor{OliveGreen}{#1\!^{\scalebox{.8}{$#2$}}}}
\newcommandx{\vertice}[2][1=v, 2={},usedefault=@]{\textcolor{OliveGreen}{\mathbbmss{#1}{#2}}}
\newcommand{\readtran}[1]{\textcolor{OliveGreen}{\textit{rd}(#1)}}
\newcommand{\readtranaux}[1]{\textcolor{OliveGreen}{\textit{rd}(#1)}}
%\newcommand{\receive}{\textcolor{red}{\textit{receive}}}
\newcommand{\pulltran}{\textcolor{OliveGreen}{\textit{pull}}}

\newcommand{\confirmedtran}{\textcolor{OliveGreen}{\textit{cfm}}}
\newcommand{\pushtran}{\textcolor{OliveGreen}{\textit{push}}}
\newcommandx{\updatetran}[2][2=\vertice]{\textcolor{OliveGreen}{\textit{wr}(\udec[{#1}][#2])}}
\newcommand{\updatetranaux}[1]{\textcolor{OliveGreen}{\textit{wr}(#1)}}

%\newcommand{\processtran}{\textit{process}}



%
% SERVER AND CLIENT
%

%IMPLEMENTATION DATA TYPES
\newcommand{\statetype}{{State}}
\newcommand{\deltatype}{{Delta}}


%\newcommand{\initialstate}{\textcolor{Orange}{\textit{initialstate}}}
\newcommand{\initialstate}{\textcolor{Orange}{\emptyset}}

%\newcommand{\emptydelta}{\textcolor{Orange}{\textit{emptydelta}}}
\newcommand{\emptydelta}{\textcolor{Orange}{\delta_\emptyset}}

\newcommand{\ireadname}{\textcolor{orange}{\textit{read}}}
\newcommand{\iread}[2]{\textcolor{orange}{\ireadname(#1,#2)}}

\newcommand{\iapplyname}{\textcolor{orange}{\textit{apply}}}
\newcommand{\iapply}[2]{\textcolor{orange}{\iapplyname(#1,#2)}}

\newcommand{\ireducename}{\textcolor{orange}{\textit{reduce}}}
\newcommand{\ireduce}[1]{\textcolor{orange}{\ireducename(#1)}}

\newcommand{\iappendname}{\textcolor{orange}{\textit{append}}}
\newcommand{\iappend}[2]{\textcolor{orange}{\iappendname(#1,#2)}}


\newcommand{\iremovename}{\textcolor{red}{\textit{remove}}}
\newcommand{\iremove}[2]{\textcolor{red}{\iremovename(#1,#2)}}

%%%%GS SOMETHING
\newcommandx{\astate}[2][1={},2=s,usedefault=@]{\textcolor{Brown}{\mathtt{#2#1}}}
\newcommandx{\adelta}[2][1={},2=\delta,usedefault=@]{\textcolor{Brown}{\mathtt{#2#1}}}
%\newcommandx{\adeltaseq}[2][1={},2=d,usedefault=@]{\textcolor{Brown}{\mathbbm{#2#1}}}
\newcommand{\nodelta}{\emptysequence}
\newcommandx{\amxrf}[2][1={},2=f,usedefault=@]{\textcolor{Brown}{\mathtt{#2#1}}}


\newcommandx{\adeltaorstate}[2][1={},2=x,usedefault=@]{\textcolor{Brown}{\mathtt{#2#1}}}

\newcommandx{\agspref}[2][1={},2=gsp,usedefault=@]{\textcolor{Brown}{\mathtt{#2#1}}}
\newcommandx{\agsprefpair}[2][1=\astate,2=\amxrf,usedefault=@]{
		\textcolor{Brown}{{\langle#1,#2\rangle}}}


\newcommandx{\agsseg}[2][1={},2=gss,usedefault=@]{\textcolor{Brown}{\mathtt{#2#1}}}
\newcommandx{\agssegpair}[2][1=\adelta,2=\amxrf,usedefault=@]{
		\textcolor{Brown}{{\langle#1,#2\rangle}}}

\newcommandx{\agssegpairi}[1][1=0]{\agssegpair[\adelta_{#1}][\amxrf_{#1}]}

\newcommandx{\aseg}[2][1={},2=seg,usedefault=@]{\textcolor{Brown}{\mathtt{#2#1}}}
\newcommandx{\aseqseg}[2][1={},2=seg,usedefault=@]{\textcolor{Brown}{\mathbbm{#2#1}}}

\newcommandx{\around}[2][1={},2=r,usedefault=@]{\textcolor{Brown}{\mathtt{#2#1}}}
\newcommandx{\aseqround}[2][1={},2=r,usedefault=@]{\textcolor{Brown}{\mathbbm{#2#1}}}
\newcommandx{\aroundtuple}[3][1=\cid,2=n,3=\adelta,usedefault=@]{
		\textcolor{Brown}{{\langle#1,#2,{#3}\rangle}}}

\newcommandx{\aroundtuplei}[1][1=i]{\aroundtuple[\cid_{#1}][n_{#1}][\adelta_{#1}]}

\newcommand{\igetdeltas}[1]{\textcolor{orange}{\textit{getdeltas}(#1)}}


%\newcommand{\pendingtype}{\rho}
\newcommandx{\pendingtype}[1][1={},usedefault=@]{\textcolor{Brown}{\aseqround{#1}}}

%

%\newcommand{\server}[3]{\langle #1 , #2, #3\rangle}
\newcommandx{\server}[4][4=\amxrf]{\langle #1, #4,  #2, #3\rangle}

%\newcommand{\persistedstate}{\textit{ps}}
\newcommand{\persistedstate}{\agsseg}

%\newcommand{\transactionbuffertype}{\delta}
\newcommand{\transactionbuffertype}{\adelta}

%\newcommand{\pushbuffertype}{\delta}
\newcommand{\pushbuffertype}{\adelta}

%\newcommand{\receivebuffertype}{\gssegmenttype \cup \gsprefixtype}
\newcommand{\receivebuffertype}{\aseqseg}

\newcommandx{\inqueue}[1][1={}]{\textcolor{Brown}{\mathtt{in#1}}}
\newcommandx{\outqueue}[1][1={}]{\textcolor{Brown}{\mathtt{out#1}}}

\newcommandx{\inserver}[1][1={}]{\inqueue[_s{#1}]}
\newcommandx{\outserver}[1][1={}]{\outqueue[_s{#1}]}
%\newcommand{\inclient}{\inqueue[_c]}
\newcommand{\inclient}{\inqueue[_c]}

\newcommand{\outclient}{\outqueue[_c]}


\newcommand{\outclientlist}{\textit{\headerround} \cdot \textit{\tailround}}

\newcommand{\nroundtype}{\mathbb{N}}




\newcommandx{\isystemterm}[1][1={}]{\textcolor{Brown}{\mathtt{N{#1}}}}
\newcommandx{\iserver}[1][1={}]{\textcolor{Brown}{\mathtt{S{#1}}}}
\newcommandx{\iclient}[1][1={}]{\textcolor{Brown}{\mathtt{C{#1}}}}

\newcommandx{\clientr}[3][3=\cid]{\client{#1}{#2}_{#3}}
\newcommand{\client}[2]{\langle#1 , #2\rangle}


\newcommandx{\iclientsyntax}[9]
	[1=P,2=\astate,3={\pendingtype},4=\pushbuffertype,5=\transactionbuffertype,
			6=\receivebuffertype,7=n,8=\aseqseg,9=\outclient,usedefault=@]
	{
   	\langle #1,#2,#5,#4,#7,#3,#8\rangle
	%al cambiar esta, hay que cambiar la de abajo.
	}

\newcommandx{\iclientinst}[9]
	[1=P,2=\astate,3={\pending},4=\pushbuffer,5=\transactionbuffer,
			6=\receivebuffer,7=n,8=\inclient,9=\outclient,usedefault=@]
	{
   	\langle #1,#2,#5,#4,#7,#3,#8\rangle
%   	\langle #1,#2,#3,#4,#5,#6,#7,#8,#9\rangle
	}
%\newcommandx{\iclientinst}[9]
%	[1=\_,2=\_,3=\_,4=\_,5=\_,
%			6=\_,7=\_,8=\_,9=\_,usedefault=@]
%	{
%   	\langle #1,#2,#3,#4,#5,#6,#7,#8,#9\rangle
%	}
\newcommandx{\iclientinstJ}{
	\iclientinst[P_l][\astate_l][{\pending}_l][{\pushbuffer}_l][{\transactionbuffer}_l][\receivebuffer_l][n_l][\inclient_l][\outclient_l]}

\newcommandx{\iserverins}[4][1=\astate,2=\amxrf,3=\inserver, 4= \outserver,usedefault=@] {\langle #1,#2,#3,#4\rangle}

\newcommandx{\addid}[2][2=\cid]{#1_{#2}}

%\newcommand{\clienti}[9]{\langle#1 , #2, #3, #4, #5, #6, #7, #8\rangle_#9}

%\newcommand{\known}{\textit{k}}
\newcommand{\known}{\astate}

\newcommand{\pending}{\aseqround}
\newcommand{\transactionbuffer}{\textcolor{Brown}{\adelta_{\sf T}}}
\newcommand{\pushbuffer}{\textcolor{Brown}{\adelta_{\sf P}}}
\newcommand{\receivebuffer}{\aseqseg}


\newcommand{\state}{\textit{ps}}
\newcommand{\stateclient}{E}


%\newcommand{\updatetype}{\textit{Update}}
\newcommand{\updatetype}{\updatesets}
\newcommand{\readtype}{\readset}
%\newcommand{\readtype}{\textit{Read}}
%\newcommand{\valuetype}{\textit{Value}}
\newcommand{\valuetype}{\valueset}



\newcommand{\knowntype}{\statetype}

\newcommand{\rvaluename}{\textcolor{Orange}{\textit{rvalue}}}

\newcommand{\rvalue}[2]{\textcolor{Orange}{\rvaluename(#1, #2)}}
\newcommand{\maxround}{\textit{mr}}
\newcommand{\append}[2]{append(#1, #2)}
\newcommand{\apply}[2]{apply(#1, #2)}
\newcommand{\remove}[2]{remove(#1, #2)}

\newcommand{\gs}{\textit{gs}}
\newcommand{\gss}{\textit{gss}}
\newcommand{\gssegment}{\textless	\delta,\maxround \textgreater}
\newcommand{\emptygssegment}{\theta}
\newcommand{\gsprefix}{\textless	\state,\maxround \textgreater}
\newcommand{\round}[3]{\textless	#1, #2, #3 \textgreater}

\newcommand{\specfunction}[4]{#1 :: #2 $\times$ #3 $\rightarrow$ #4}
\newcommand{\specOperator}[4]{#1 :: #2 $\times$ #3  $\times$ #4}

\newcommand{\specfunctiononeparameter}[3]{#1 :: #2 $\rightarrow$ #3}
\newcommand{\specfunctionforparameters}[6]{#1 :: #2 $\times$ #3 $\times$ #4 $\times$ #5 $\rightarrow$ #6}
\newcommand{\specfunctionthreeparameters}[5]{#1 :: #2 $\times$ #3 $\times$ #4 $\rightarrow$ #5}

\newcommand{\cleannamefun}{\textbf{\textit{clean}}}


\newcommand{\gssegmenttype}{\textit{GSSegment}}
\newcommand{\gsprefixtype}{\textit{GSPrefix}}
\newcommand{\roundtype}{\textit{Round}}
\newcommand{\headerround}{\textless	i, n_0, \delta_0\textgreater}

\newcommand{\tailround}{\textit{rs}}

\newcommand{\partialfunction}[2]{#1 \rightarrow #2}
\newcommand{\appf}[2]{#1(#2)}

\newcommand{\reduce}[1]{\textbf{\textit{reduce}}(#1)}
\newcommand{\applyplus}[2]{\textbf{\textit{apply}}(#1, #2)}

\newcommand{\readplus}[2]{\readp{#1}{#2} = \textit{v}}

\newcommand{\readp}[2]{\textbf{\textit{read}}(#1, #2)}

\newcommand{\appendplus}[2]{\textbf{\textit{append}}(#1, #2)}

\newcommand{\reducestate}[2]{\textbf{\textit{reducestate}}(#1, #2)}


\newcommand{\gsprefixins}[2]{\textless #1, #2\textgreater}
\newcommand{\gssegmentins}[2]{\textless #1, #2\textgreater}


\newcommand{\updatevtran}[2]{\textit{update}(#1^#2)}
\newcommand{\syncupdtran}[1]{\textit{sync\_update}(#1)}

\newcommand{\updatebyclient}[1]{\textit{update}(u)_#1}
\newcommand{\readbyclient}[1]{\textit{read}(r)_#1}

\newcommand{\sendtran}{\textit{send}}
\newcommand{\dropconn}[1]{\dagger(#1)}
\newcommand{\dropconnectionclient}{\dagger}
\newcommand{\tprogram}{P}
\newcommand{\tflush}{\textit{flush()}}
\newcommand{\tsyncupd}[1]{\textit{sync\_update(#1)}}

\newcommand{\acceptconn}[1]{\oplus(#1)}
\newcommand{\crashandrecover}{\ddagger}
\newcommand{\undefined}{\perp}
\newcommand{\receiveroundsname}[1]{\textit{rnds}(#1)}
\newcommand{\notify}[3]{\textit{notify}(#1, #2, #3)}
\newcommand{\clean}[2]{\textit{\cleannamefun}(#1, #2)}
\newcommand{\sclean}[1]{\textit{\cleannamefun}(#1)}
\newcommand{\filter}[2]{\textit{filter}(#1,#2)}


\newcommand{\asoc}[2]{#1 \mapsto #2}
\newcommand{\upd}[2]{[\asoc{#1}{#2}]}

\newcommand{\updatein}[2]{#1^#2}
\newcommand{\update}[3]{#1\upd{#2}{#3}}
\newcommand{\updatethree}[7]{#1[#2 \mapsto #3; #4 \mapsto #5; #6 \mapsto #7]}
\newcommand{\updatetwo}[5]{#1[#2 \mapsto #3; #4 \mapsto #5]}
\newcommand{\updatefour}[9]{#1[#2 \mapsto #3; #4 \mapsto #5; #6 \mapsto #7; #8 \mapsto #9]}


\newcommandx{\updev}[2][1=u,2=v,usedefault=@]{\textcolor{blue}{#1^{#2}}}


\newcommand{\readins}[2]{\treadins}





\newcommand{\updateins}[1]{\textit{update}(#1);P}
\newcommand{\pushins}{\textit{push}();P}
\newcommand{\pullins}{\textit{pull}();P}
\newcommand{\curstate}[4]{\textit{curstate}(#1, #2, #3, #4)}
\newcommand{\nround}{\textit{n}}
\newcommand{\confirmedins}[1]{\textit{let}\ #1 = \textit{confirmed}();P}
\newcommand{\domround}{\mathcal{N} \times \mathbb{N} \times \deltatype}
\newcommand{\dominserver}{\mathbb{N} \times \deltatype}




\newcommand{\tuple}[1]{\vec{#1}}
\newcommand{\zero} {0}
\newcommand{\outp}[2]{\overline{#1}\langle #2 \rangle}
\newcommand{\inp}[2]{#1(#2)}
\newcommand{\ifte}[3]{{\bf if}\ #1\ {\bf then}\ #2\ {\bf else}\ #3}
\newcommand{\rec}[2]{{\bf rec}_{#1}\ #2}

\newcommand{\corrinst}[2]{#1\triangleright[#2]}

\newcommand{\service}[4]{#1_{#2}\{#3\ ,\ #4\}}



\newcommand{\denote}[1]{\llbracket #1\rrbracket}

 \newcommand{\unexcplbl}[1]{[#1]}
 \newcommand{\unexcp}{\circledast 	}

 \newcommand{\mayhandle}[2]{#1\downarrow_{#2}}
 \newcommand{\maynothandle}[2]{#1\not\downarrow_{#2}}

\newcommand{\Arro}{\Rightarrow}



\newcommand{\condition}[2]{\neg(\eval{#1})=#2}
\newcommandx{\updopentx}[2][1=\vertice]{{\nabla}_{#1}(#2)}
\newcommandx{\instanceset}{V}

%CONSISTENCY GRAMMAR

\newcommandx{\environment}[6][6=\sse]{\langle #1 , #2, #6,  #3, #4, #5\rangle}
\newcommand{\environmenttran}[8]{\{#1 , #2, #3, #4, #5, #6, #7, #8\}}

\newcommand{\vis}{\textsc{\scriptsize{VIS}}}
\newcommand{\op}{\textsc{\scriptsize{OP}}}
\newcommand{\arb}{\textsc{\scriptsize{AR}}}
\newcommand{\so}{\textsc{\scriptsize{SO}}}
\newcommand{\sse}{\textsc{\scriptsize{SS}}}


\newcommand{\soby}[2]{#1\triangleright_i{#2}}
\newcommand{\updateinqueuemessage}[2]{\queuemessage[#1]_#2}
\newcommand{\rb}{\textsc{\scriptsize{RB}}}
\newcommand{\tx}{\textsc{\scriptsize{TO}}}
\newcommand{\tc}{\textsc{\scriptsize{TC}}}



%%%%%%%%%%%%%%%%%%%%%%%%%%%%%%%%%%%%%%%%%%%%%%%%%%%%
% COMMENTS
%%%%%%%%%%%%%%%%%%%%%%%%%%%%%%%%%%%%%%%%%%%%%%%%%%%%
\newcommand{\cf}[2]{
    \fontsize{#1}{#1}{\selectfont{#2}}
  }

\newcommand{\hernan}[1]{{\marginpar{\cf{6}{{HM: #1}}}}}
%\newcommand{\emi}[1]{{\marginpar{\cf{6}{{#1}}}}}
%  \newcommand{\cf}[2]{
%    \fontsize{#1}{#1}{\selectfont{#2}}
%  }
%\newcommand{\emic}[2]{\shadowbox{\fbox{\parbox{.8\textwidth}{\begin{description}\item[\cf{8}{\sc\bf #2}]\cf{8}{#1}\end{description}}}}}

%%%%%%%%%%%%%%%%%%%%%%%%%%%%%%%%%%%%%%%%%%%%%%%%%%%%
% Para Borrar
%%%%%%%%%%%%%%%%%%%%%%%%%%%%%%%%%%%%%%%%%%%%%%%%%%%%
\newcommand{\ignorar}[1]{}


%%%%%%%%%%%%%%%%%%%%%%%%%%%%%%%%%%%%%%%%%%%%%%%%%%%%
% Type system
%%%%%%%%%%%%%%%%%%%%%%%%%%%%%%%%%%%%%%%%%%%%%%%%%%%%

\newcommand{\type}{\mathcal{T}}
\newcommand{\patternmtch}{\owedge}
\newcommand{\tjudge}[4]{#1 \vdash #2 : {#3} ,#4}


%%%%%%%%%%%%%%%%%%%%%%%%%%%%%%%%%%%%%%%%%%c%%%%%%%%%%
% Cross-ref
%%%%%%%%%%%%%%%%%%%%%%%%%%%%%%%%%%%%%%%%%%%%%%%%%%%%
\newcommand{\equref}[1]{Eq.~\eqref{#1}}
\newcommand{\thmref}[1]{Thm.~\ref{#1}}
\newcommand{\lemref}[1]{Lem.~\ref{#1}}
\newcommand{\figref}[1]{Fig.~\ref{#1}}
\newcommand{\defref}[1]{Def.~\ref{#1}}
\newcommand{\secref}[1]{\S~\ref{#1}}
\newcommand{\exref}[1]{Ex.~\ref{#1}}
\newcommand{\propref}[1]{Prop.~\ref{#1}}


%%%%%%%%%%%%%%%%%%%%%%%%%%%%%%%%%%%%%%%%%%%%%%%%%%%%
% comments
%%%%%%%%%%%%%%%%%%%%%%%%%%%%%%%%%%%%%%%%%%%%%%%%%%%%


\newcommand{\changed}[2]{\textcolor{blue}{#2}}
\newcommand{\henote}[1]{\textcolor{red}{[H: #1]}}
\newcommand{\chnote}[1]{\textcolor{green}{[C: #1]}}

\newcommand{\flatten}[1]{\textcolor{Orange}{\textit{flatten}(#1)}}
\newcommand{\length}[1]{|#1|}

%%%%%
% Section 3
%%%%
\newcommand{\paralelo}[4]{\parallel_{#1\in #2} {#3} \parallel \ #4}

\newcommandx{\abstsyst}[4][1=l,2=L,3=\Absclient_j, 4=\queuemessage,usedefault=@]{\paralelo{#1}{#2}{#3}{#4}}

\newcommandx{\abstcliJ}[2][1=l,2=n,usedefault=@]{\tclienti{\tprogram_{#1}}{\tknown_{#1}}{\tpending_{#1}}{\ttransactionbuffer_{#1}}{\tsent_{#1}}{\treceivebuffer_{#1}}[{\cid_{#1}}][{#2}_{#1}]}


\newcommandx{\concsyst}[4][1=l,2=L,3={\iclient_l},4=\iserver,usedefault=@]{\paralelo{#1}{#2}{#3}{#4}}



%labels inferencia triangulitos
\newcommand{\triangemptydelta}{$\triangleleft\emptydelta$}
\newcommand{\triangappend}{$\triangleleft\iappendname$}
\newcommand{\triangreduce}{$\triangleleft$-\ireducename}
\newcommand{\triangremove}{$\triangleleft$-\iremovename}

\newcommand{\trianginitialstate}{$\triangleleft-\initialstate$}
\newcommand{\triangapply}{$\triangleleft$-\iapplyname}
\newcommand{\triangread}{$\triangleleft$-\ireadname}

\newcommand{\triangreduceemp}{$\triangleleft$-\ireducename-$\emptydelta$}
\newcommand{\triangapplyemp}{$\triangleleft$-\iapplyname-$\emptysequence$}


\newcommand{\implements}[2]{{#1}\ \mathit{ implements}	\ {#2}}
%volver

%SECTION 4
\newcommand{\rest}[1]{\!\!\downarrow_{#1}}
\newcommand{\restRel}[3]{#1\rest{#2\times#3}}
\newcommand{\restUR}[1]{\restRel{#1}{\mathbb{U}}{\mathbb{R}}}
\newcommand{\restUO}[1]{\restRel{#1}{\mathbb{U}}{\mathbb{O}}}
\newcommand{\restRO}[1]{\restRel{#1}{\mathbb{R}}{\mathbb{O}}}
\newcommand{\restUU}[1]{\restRel{#1}{\mathbb{U}}{\mathbb{U}}}
\newcommand{\restOU}[1]{\restRel{#1}{\mathbb{O}}{\mathbb{U}}}


%Section 5
\newcommandx{\syncupdcmd}[1][1=\anupd]{{\tt syncUpd}(#1)}



\newcommand{\waitcmd}{{\tt wait}}

\newcommandx{\tsyncupdins}[2][1=\anupd,2=\tprogram,usedefault=@]{\syncupdcmd[#1];#2}
\newcommandx{\ttrans}[2][1=\tprogram,2=\tprogram,usedefault=@]{[#1];#2}

\newcommandx{\tguarded}[3][1=e,2=\tprogram,3=\tprogram,usedefault=@]{#1\triangleright(#2);#3}



\newcommand{\startsynctran}{\dagger}
\newcommand{\finishsynctran}{\dagger}


\newcommandx{\tupdlbl}[2][1=\anupd,2=\vertice,usedefault=@]{*\textit{wr}(\udec[{#1}][#2])}


\newcommand{\ovis}{\mathsf{\scriptsize{VIS}}}
\newcommand{\oarb}{\mathsf{\scriptsize{AR}}}

\newcommandx{\roarb}[1][1=F]{\oarb\rest{#1}}


\newcommand{\refprop}[1]{\defref{def:implementation}, Prop. \ref{#1}}

\newcommand{\irounds}{\textcolor{Brown}{n}}





\bibliographystyle{plain}

\title{GSP-Calculus}

\author{Hern\'an Melgratti\inst{1,2} \and Christian Rold\'an\inst{1} 
        }

\institute{
  Departamento de Computaci\'on, FCEyN, Universidad de Buenos Aires.
 %\email{hmelgra@dc.uba.ar} 
\\
\and CONICET.}

%\titlerunning{Correlation sets}

%\authorrunning{R. Bruni, H. Melgratti, U. Montanari}


%%%%%%%%%%%%%%%%%%%%%%%%%%%%%%%%%%%%%%%%%%%%%%%%%%%%%%%%%%%%%%%%%
% DOCUMENT
%%%%%%%%%%%%%%%%%%%%%%%%%%%%%%%%%%%%%%%%%%%%%%%%%%%%%%%%%%%%%%%%%

\begin{document}

\maketitle

\begin{abstract}
\end{abstract}

% !TEX root = main.tex


\section{Global Sequence Protocol Calculus}
\label{sec:gsp}

\henote{check store-store-etc.}


\subsection{Syntax}

Clients interact with a store by performing operations in $\updatesets\cup\readset$,
where the elements in $\updatesets$ denote update operations and those in $\readset$ 
stand for read operations. 
None operation can simultaneously read and update a store, therefore we assume
 $\updatesets \cap \readset= \emptyset$.
 We write 
 $\anupd, \anupd['], \anupd[''],\ldots$ for  updates in $\updatesets$  and 
$\aread, \aread['], \aread[''],\ldots$ for reads in $\readset$. 

The  state of a store is 
represented by sequences of update operations. 
For technical 
convenience (particularly in \secref{sec:properties-gsp}), we find useful to 
distinguish  different executions of the same operation. Formally, stores 
 associate each update with a fresh event identifier.
We assume a set $\verticesets$ of event identifiers  $\vertice$, $\vertice[@][_0]$, \ldots, $\vertice[@][']$,$\ldots$ 
and write $\udec$ for the update  $\anupd$ associated with the event $\vertice$.
We sometimes omit the decoration when it is not relevant.  \henote{Is this true?}

We  use $\anupdseq$ to denote sequences of decorated updates and
$\ublock$ for an atomic block of updates. 
(We reserve the term 
transaction for a different notion addressed in \secref{sec:transactions}). 
We write  $\blockseq$  for sequences of blocks and use  $\emptysequence$ for the empty sequence.
%  defined by the following grammar
% \[\tpending,{\tpending[']} ::= \emptysequence \ |\  [\alpha] \cdot \tpending\]
% \changed{}{A sequence $[\alpha] \cdot \tpending]$ stands for ....}, and we 
% use $\emptysequence$ to denote the empty sequence, \ie, the sequence of length 0.
We use usual operations on sequences such as $\blockseq{[i]}$ to denote  
 the $i$-th  element of $\blockseq$, $\blockseq{[i..j]}$
for the sequence of elements from position $i$ to $j$, $|\blockseq|$ for the length and 
$\blockseq\setminus\blockseq[']$ for the relative complement. We will use
 $\flatten{\blockseq}$ to obtain a plain sequence of updates by 
forgetting  the structure of constituent blocks, if any.
%is such that $\flatten{\emptysequence} = \emptysequence$, and 
%    $\flatten{\ublock\cdot\blockseq} = \ttransactionbuffer\cdot\flatten\blockseq$.

We rely on countable sets $\varset$ of program variables $x,x',\ldots$ 
and $\idset$ of client identifiers $\cid,\cid',\ldots,\cid_1,\ldots$.



 
 \begin{figure}
 \[
\begin{array}{l}
    \begin{array}{l@{\ }rcl}
		 \rName{naturals}
		 &
		 \multicolumn 3 l {\treceivebuffer, \tknown, \trounds \in \nat}
		 \\
    		 \rName{update} 
		 &
		 \updatesets
		 &{ =  }&
		 \{ \anupd, \anupd['], \ldots,  \anupd[_0],\ldots\}
		 \\
   		 \rName{read} 
		 &
		 \readset
		 &{ =  }&
		 \{ \aread, \aread['], \ldots,  \aread[_0],\ldots\} 
		  \\
   		 \rName{event} 
		 &
		 \verticesets
		 &{ =  }&
		 \{\vertice, \vertice[@]['], \ldots, \vertice[@][_0],\ldots\} 
		 \\
		 \rName{var}
		 &
		  \varset &{ =  }&
		 \{x,x',\ldots,x_0,\ldots\}
		 \\
		 \rName{ids}
		 &
		  \idset &{ =  }&
		 \{\cid,\cidj,\cid['],\ldots,\cid[_0],\ldots\}
\end{array}
\qquad
    \begin{array}{l@{\ }r@{}l}
    \\
		 \rName{upd seq}
		 &
		\anupdseq 
		 \ ::=\ 
			&
		 \emptysequence \ |\ \udec\cdot\anupdseq 
		 \\
		 \rName{block seq}	
		 &	 
		 \blockseq
		 \ ::=\ 
		&
		 \emptysequence\ |\ \ublock\cdot\blockseq	 
			\\
		\rName{system} 
		& \systemterm\ ::= \ 
		& \queuemessage \, \bigpar\, \Absclient
		 \\     	  \rName{store} 
			 & \queuemessage \ ::=\ 
			 &  \blockseq 
			 \\
      		\rName{client} 
			& \Absclient \ ::=\ 
			& 0 \;|\;�\tclient{\tprogram}{\tknown}{\blockseq}{\ttransactionbuffer}{\blockseq}{\treceivebuffer} \;|\;
       				  \Absclient\, \bigpar\, \Absclient %\hfill \text{with}\  n \in \nat
      \end{array}
      \\[40pt]
      \begin{array}{ll}
     		 \rName{program} 
			 \ \tprogram \ ::=\ &
			  \tupdateins \;|\; \treadins{x}{\aread} \;|\; \tpullins \;|\;
			   \\
			  & \tpushins \;|\;
			 \tconfirmedins{x} 
			%\;|\; \pwhile{\cond}{\tprogram}  \;|\;
			%\\
			%&
			%& \pifte
     \end{array}
\end{array}
  \]
\caption{Syntax of the \gsp\ calculus}
\label{fig:syntax-gsp}
\end{figure}

 \begin{definition}[GSP Language] 
The set of \tgspcalculus\ terms is given by the grammar in \figref{fig:syntax-gsp}
\end{definition}

A \gsp\ system $\systemterm$ consists of  a store and zero or more  clients.
The state of a global store $\queuemessage$ is represented by a sequence of  blocks.
The term $\tclienti{\tprogram}{\tknown}{\tpending}{\ttransactionbuffer}{\tsent}{\treceivebuffer}[\cid][n]$ 
stands for a client identified by $\cid$ and is engaged on the execution of the 
program $P$. 
The remaining elements 
are used to describe the state of the local replica: $\ttransactionbuffer$ contains the  updates 
that have been made locally and are  part of an unfinished block; $\tsent$ models the 
communication buffer, which keeps all blocks sent by the client but not received by the 
global store; 
$\tpending$ is the pending buffer, which contains all blocks completed locally but 
unconfirmed by the global store.
For simplicity, we do not explicitly replicate the portion of the global store that 
each client knows; we use instead a natural number $\tknown$ corresponding to 
 the length of the prefix of the global store known by the client. Specifically, the replicated 
 part of the global store known by $\cid$ is the 
sequence $\queuemessage[0..\tknown-1]$. Similarly, the number $\treceivebuffer$ identifies 
the new updates known by the client that needs to be added to the local replica, i.e., 
the client knows the new updates contained in the segment $\queuemessage[\tknown..\tknown+\treceivebuffer-1]$.
Value $\trounds$ counts the number of update operations executed by the client. 


A program $P$ is built as a sequence of  operations that handle the  
interaction with 
the store: $\readcmd,  \updcmd, \pullcmd,  \pushcmd, \confcmd$
 (we postpone their description until \secref{sec:gsp-sos}). 
 %Conditional and loop statements contain expressions 
%$\cond, \cond', \ldots$ written in some language that we left unspecified. We only assume   
%such language to be equipped with a valuation function  that associates each expression $\cond$  
%with a value $v$ in some domain $\valueset$, written $\eval{\cond}{v}$.  
A program $\plet{x}{\ldots}{P}$ introduces a bound variable whose  scope is $P$. The definition of 
free variables of a program is  standard. We say that a process $P$ is {\em closed} when it does not contain  
 free variables. 

We keep the language for programs simple. We remark that this 
choice do not affect the results presented in this paper. Actually, we could have just characterised 
 the behaviour of programs in terms of a labelled transition system, 
but we prefer to have a syntax through the presentation. 

%\henote{agregar algun comentario sobre otras estructuras de control}



\begin{definition}[Well-formedness]
A \gsp\ system $\systemterm =  \Absclient_0\bigpar \ldots\bigpar \Absclient_m\bigpar\queuemessage$  where 
$\Absclient_l = \tclienti{\tprogram_l}{\tknown_l}{\tpending_l}{\ttransactionbuffer_l}{\tsent_l}{\treceivebuffer_l}[\cid_l][\trounds_l]$ 
for all  $l\in \{0,\ldots,m\}$  is {\em well-formed} if the  following conditions hold 
\begin{enumerate}
  \item $\cid_{l} \neq \cid_{l'}$ for all $l\neq l'$;
  \item $k_l+j_l\leq \length \queuemessage$ for all $l$;
  \item $\tpending_l = \ublock[{\anupdseq[_1]}]\cdots\ublock[{\anupdseq[_p]}]\cdot\tsent_l$ and 
  for all ${\sf 1} \leq x < y \leq{\sf p}$ there exists $x',y'$ s.t. $\queuemessage[x'] = \ublock[{\anupdseq}_x]$, $\queuemessage[y'] =\ublock[{\anupdseq}_y]$ and 
  $\tknown_l\leq x' < y'$; and
%  \item if $\ublock \in {\tpending_l}\cdot {\tsent_l}$ then $\cid = \cid_l$, \ie,  blocks in the local queues of 
%  a client $l$ are  all decorated with the identifier $\cid_l$.
%  \item $\cid \in {\cid_0,\ldots, \cid_n} $ for all $\ublock \in S$, \henote{Pensar bien, tal vez no es necesaria}
  \item $\anupdseq = \flatten{\queuemessage\cdot\tsent_0\cdots\tsent_m\cdot\ttransactionbuffer_0\cdots\ttransactionbuffer_m}$,
  if $\anupdseq{[x]}=\udec$, $\anupdseq{[y]}=\udec[@][{\vertice[@][']}]$ and $x \neq y$ then $\vertice \neq \vertice[@][']$.
\end{enumerate}
\end{definition}
Condition (1) ensures that all   clients have different identifiers. Condition (2) states that each client can see at most every messages in the store, while 
(3) ensures that all unconfirmed blocks in $\tpending_l$ are  either in the unseen part of the store  ($\ublock[{\anupdseq[_1]}]\cdots\ublock[{\anupdseq[_p]}]$) or in  the 
   communication buffer $\tsent_l$. Morevoer,  messages are kept in the relative order in which they have been generated. Finally, (4) ensures
   that an event  is associated with a unique update operation. 
    
   

Hereafter, we assume any system to be well-formed.

%\changed{}{
%\begin{example} Show an example, for instance a server with two clients with different information.
%\end{example}
%}

\subsection{Operational Semantics}
\label{sec:gsp-sos}

The operational semantics of \tgspcalculus\ is given by a labeled transition system 
over well-formed terms, quotiented by the structural equivalence $\equiv$ defined as the least equivalence 
such that  $\bigpar$ is associative, commutative and has $0$ as neutral element.
Transitions are be labelled by pairs $(\mu,\cid)$ where $\mu$ is the action performed 
by the client $\cid$. The set of actions is given by the following grammar:
\[ 
\begin{array}{r@{\ ::= \ }l}
  \action & \tau \ |\  \readtran{r} \ | \  \updatetran{u} \ | \ \pulltran \ | \ \pushtran \ | \ \confirmedtran \\
  \actbyc &  (\lambda,\cid)
\end{array}
\]
As usual,  $\tau$ stands for an internal, unobservable action, while the remaining actions allow 
a client to update, read from and synchronise with a global store. 
In what follows we will write labelled transitions $\arro{\mu}$ instead of $\tr{(\mu,\cid)}$. 


\begin{figure*}[t]
{\small \[
 \begin{array}{l}
\mathrule{update}
         { \vertice \ \mbox{fresh}
         }
 	{
	\tsystem{\tclienti{\tupdateins}{\tknown}{\tpending}{\ttransactionbuffer}{\tsent}{\treceivebuffer}} 
	\ \ \,\arro{\updatetran{u}} \ \ \,
	\tsystem{\tclienti{\tprogram}{\tknown}{\tpending}{\ttransactionbuffer \cdot \tupdate}{\tsent}{\treceivebuffer}}
	}

\\[20pt]
\mathax{push}
	{
	\tsystem{\tclienti{\tpushins}{\tknown}{\tpending}{\ttransactionbuffer}{\tsent}{\treceivebuffer} }
	\ \ \, \arro{\pushtran}\ \ \,
	\tsystem{\tclienti{\tprogram}{\tknown}{\tpending \cdot \ublock}{\emptysequence}{\tsent \cdot \ublock}{\treceivebuffer}[@][{\trounds+1}]}
	}

\\[12pt]
\mathax{send}
	{
	\tsystem{\tclient{\tprogram}{\tknown}{\tpending}{\ttransactionbuffer}{\tsenthead \cdot \tsent}{\treceivebuffer}} 
	\ \arro{\tau}\ \tsystem{\tclient{\tprogram}{\tknown}{\tpending}{\ttransactionbuffer}{\tsent}{\treceivebuffer}}[\queuemessage \cdot \tsenthead]
	}

\\[10pt]
\mathrule{receive}
	{\tknown + \treceivebuffer< \text{\textbar} S \text{\textbar} } 
	{
	\tsystem{\tclient{\tprogram}{\tknown}{\tpending}{\ttransactionbuffer}{\tsent}{\treceivebuffer}} 
	\ \ \,\arro{\tau} \ \ \,
	\tsystem{\tclienti{\tprogram}{\tknown}{\tpending}{\ttransactionbuffer}{\tsent}{\treceivebuffer+1}}
	}
\\[18pt]
\mathax{pull}
	{
	\tsystem{\tclienti{\tpullins}{\tknown}{\tpending}{\ttransactionbuffer}{\tsent}{\treceivebuffer}} 
	\ \ \, \arro{\pulltran}\ \ \,
	\tsystem{\tclienti{\tprogram}{\tknown+\treceivebuffer}{\tpending \setminus \queuemessage[\tknown .. \tknown + \treceivebuffer {\ -1}]}
		{\ttransactionbuffer}{\tsent}{0}}
	}
\\[12pt]
\mathrule{read}
  	{
	\rvalue{r}{\flatten {\queuemessage[0..\tknown-1] \cdot \tpending} \cdot \ttransactionbuffer} = v
	}	
	{
	\tsystem{\tclienti{\treadins{x}{r}}{\tknown}{\tpending}{\ttransactionbuffer}{\tsent}{\treceivebuffer}}
	\  \, \arro{\readtran{\aread}} \ 
	\tsystem{\tclienti{{\tprogram}\subst{x}{v}}{\tknown}{\tpending}{\ttransactionbuffer}{\tsent}{\treceivebuffer}}
	}


\\[18pt]
\mathrule{confirm}
 %	{\eval{(\tpending \cdot  \ttransactionbuffer == \emptysequence)}  v}
	{v = {(\tpending \cdot  \ttransactionbuffer == \emptysequence)}}
	{
	\tsystem{\tclienti{\tconfirmedins{x}}{\tknown}{\tpending}{\ttransactionbuffer}{\tsent}{\treceivebuffer}} 
	\arro{\confirmedtran} 
	\tsystem{\tclienti{{\tprogram}\subst{x}{v}}
		{\tknown}{\tpending}{\ttransactionbuffer}{\tsent}{\treceivebuffer}}
	}
	
%\\[25pt]
%\mathrule{while-true}
%	{\eval \cond {true}}
%	{
%	\tsystem{\tclienti{\pwhile{\cond}{\tprogram}[Q]}{\tknown}{\tpending}{\ttransactionbuffer}{\tsent}{\treceivebuffer}}
%	\arro{\tau} 
%	\tsystem{\tclienti{\tprogram;\pwhile{\cond}{\tprogram}[Q]}{\tknown}{\tpending}{\ttransactionbuffer}{\tsent}{\treceivebuffer+1}}
%	}
%
%\\[25pt]
%\mathrule{while-false}
%	{\eval \cond {false}}
%	{
%	\tsystem{\tclienti{\pwhile{\cond}{\tprogram}[Q]}{\tknown}{\tpending}{\ttransactionbuffer}{\tsent}{\treceivebuffer}} 
%	\arro{\tau} 
%	\tsystem{\tclienti{Q}{\tknown}{\tpending}{\ttransactionbuffer}{\tsent}{\treceivebuffer+1}}
%	}
%	\\
%\henote{\mbox{Agregar reglas para if}}
 \end{array}
\]}
\caption{Operational semantics for \tgspcalculus}
\label{fig:OS-tgsp}
\end{figure*}


We now comment on the inference rules in \figref{fig:OS-tgsp}, which define the operational semantics of \tgspcalculus.
When a client performs an update (rule \textsc{update}),  the changes have only local effects:
the sequence of  local updates $\ttransactionbuffer$ is extended with the 
update $\anupd$ decorated with a globally fresh identifier $\vertice$. We remark that 
decorations are used for technical reasons but they are 
 operationally irrelevant (see \secref{sec:properties-gsp}).

A client propagates its local changes to the global store by executing $\pushcmd$ (rule \textsc{push}):
all local changes in $\ttransactionbuffer$ will be transmitted as a block $\ublock$, \ie, as an atomic unit.
 Nevertheless, these changes are not  made immediately available 
 at the global store because of the asynchronous communication model. 
 In fact, the new block  $\ublock$ is added to the communication buffer $\tsent$, which 
contains all sent blocks that still have not reached the global store.  Also,  
 $\ublock$ is  added  to the  sequence of pending messages $\tpending$ to 
  be used  until the block is finally added to the store to evaluate locally in the subsequent read. 

Rule \textsc{send}  stands for  a  block  that finally reaches the global store. Conversely, 
rule \textsc{receive} models the reception of
 a new update. The received update is not immediately incorporated to the local replica. Actually, 
 each client  explicitly refreshes its local view of the global store by executing  {\pullcmd} (rule \textsc{pull}).  
 At this time, the  previously received updates $\treceivebuffer$ are incorporated to the 
 local copy (\ie, $\tknown$ is changed to $\tknown+\treceivebuffer$). Additionally, 
 all pending updates  in the new fragment $\queuemessage[k..k+\treceivebuffer-1]$ 
 are remove from 
 %All pending blocks in 
 $\tpending$.
 % that are in the new fragment 
 %$\queuemessage[k..k+\treceivebuffer-1]$ are removed from  $\tpending$. 
% 


The semantics of operations is  defined abstractly by the interpretation function
 $\rvaluename:\readset\times\updatesets^*\rightarrow  \valueset$, 
\ie,  a function  that takes a read operation and a sequence of updates and returns 
a  value in some domain $\valueset$.  
A read operation $\aread$ is evaluated over the local state of the client (rule \textsc{read}), 
i.e., 
the known prefix of the global store $\queuemessage[0..\tknown-1]$ and the local updates in 
 $\tpending$ and $\ttransactionbuffer$. The  value $v$ is bound to the 
 variable $x$, and hence all free occurrences of $x$ in the 
 continuation $P$  are substituted by $v$.
 %
 A client may perform $\confcmd$  to check whether its executed updates have been already 
 applied to the global store: this operation  
  returns true only when the local buffers  $\tpending$ and $\ttransactionbuffer$ are both empty.
 %
%Remaining rules are standard.
%$\textsc{(t-while-trye)}$ and $\textsc{(t-while-false)}$ are standard. Rule $\textsc{(t-receive)}$ states that a counter $\treceivebuffer$ is incremented (by 1). Finally, the last one rule, $\textsc{(t-process)}$, moves updates from sent queue to the message queue.  

%In the next sections  will  make use of following operations on sequences: 
%\begin{itemize}
%    \item Relative complement:  
%    %Let $\tpending$ and $\ttransactionbuffer$ be sequence, we write $\setminus$ as the 
%    %relative complement of $\tpending$ in $\ttransactionbuffer$, \ie,
%     $\blockseq \setminus {\blockseq[']} = [\ublock | u_i \in \blockseq \wedge\ u_i\notin  {\blockseq[']} ]$. 
%     \henote{fix}
%    \item Flatten: $\flatten{\emptysequence} = \emptysequence$, and 
%    $\flatten{\ublock\cdot\blockseq} = \ttransactionbuffer\cdot\flatten\blockseq$.
%    \item subsequence: ${\ttransactionbuffer}[j.. k]$ with $j,k\in\nat$.
%    \item length: $\length\_$
%\end{itemize}


We remark that the operational semantics preserves well-formedness.

\begin{lemma} If $\systemterm$ is well-formed and $\systemterm\arroi{\action}\systemterm[']$, then $\systemterm[']$ is well-formed.
\end{lemma}


\section{Implementation of GSP}

The \tgspcalculus\ model formalised in the previous section corresponds to an idealised system that
abstracts away from important implementation details, mainly, the  problem of maintaining a constantly growing 
sequence of updates to represent states.  A realistic implementation for such model has been 
proposed in\henote{~\cite{}}, in which states and updates have a compact representation in 
terms of two different objects: a {\em state} and {\em delta}. Their precise definition highly depends
on the datatype of the values kept in the store. \henote{agregar ejemplo?}. 
In order to keep the description of the model as general as possible, they have been
characterised \henote{in~\cite{}} with 
two abstract types, namely $\statetype$ and $\deltatype$, which are equipped with the following  operations.
\[
\begin{array}{lll}
	\textbf{const} & \initialstate & : \statetype \\
	\textbf{function} & \ireadname & : \partialfunction{\readtype \times \statetype}{\valuetype} \\
	\textbf{function} & \iapplyname & : \partialfunction{\statetype \times \deltatype^*}{\statetype} \\
	\textbf{const} & \emptydelta & : \deltatype \\
	\textbf{function} & \iappendname & : \partialfunction{\deltatype \times \updatetype}{\deltatype} \\
	\textbf{function} & \ireducename & : \partialfunction{\deltatype^*}{\deltatype} \\
\end{array}
\] 

Operations $\initialstate$ and $\emptydelta$ are the constructors for the respective types. Function $\ireadname$ is the 
interpretation function for read operations (\ie, the implementation counterpart of the operation $\rvalue{\_}{\_}$  in the 
idealised model) while $\iapplyname$ is for state transformations. The remaining operations $\iappendname$ and
$\ireducename$ account for compacting the description of several updates. 
%
We shall use  $\astate,\astate['],\ldots, \astate[{_1}],\ldots$ to denote values of  type $\statetype$ and 
$\adelta,\adelta['],\ldots, \adelta[{_1}],\ldots$ for values of type $\pushbuffertype$.

In addition,  we will use partial functions  $\amxrf,\amxrf['],\ldots, \amxrf[{_1}],\ldots$ in  $\idset\rightarrow \nat$
assigning clients with a natural number denoting \henote{the number of the last round exchanged with ...}

%Let $\idset$ be a set of clients' name ranged over by $b$, $b_1$, $b_2$, $\ldots$; 

The global state of the store is now represented by a pair $\gsprefixtype$ be a set over sets \statetype\ $\times$ $\idset$ $\rightarrow$ $\mathbb{N}$, \gssegmenttype\ be a set operation over sets \deltatype $\times$ $\idset$ $\rightarrow$ $\mathbb{N}$ and \roundtype\ the set over sets $\idset$ $\times$ $\mathbb{N}$ $\times$ \deltatype; $\emptygssegment$\ is a fresh element of \gssegmenttype.
%
\henote{Explain gsspref... round}

For convenience, notational conventions are in~\figref{}
\begin{figure}
  \[
   \begin{array}{lrll}
%       &\astate,\astate['],\ldots, \astate[{_1}],\ldots &{\ \in\ }& \statetype
%       \\
%       &\adelta,\adelta['],\ldots, \adelta[{_1}],\ldots &{\ \in\ }& \deltatype
%       \\
%       &,\amxrf['],\ldots, \amxrf[{_1}],\ldots &{\ \in\ }& \idset\rightarrow \nat 
       & \cid & \in & \idset
       \\
       & n & \in & \nat
       \\
       &\astate & \in & \statetype
       \\
       &\adelta & \in & \deltatype \cup \{\nodelta\}
       \\
       &\pendingtype & \in & \deltatype^* \quad \mbox{\henote{antes $\rho$ y $\sigma$}}\\
       &\amxrf & \in & \idset\rightarrow \nat
       \\
%       \rName{GSprefix}
%       &
%       \agspref & ::= & \agsprefpair
%       \\
%       \rName{GSSegment}
%       &
%       \agsseg & ::= & \agssegpair
%       \\
%       \rName{Segment}
%       &
%       \aseg & ::= & \agspref \; |\;  \agsseg
%        \\
       \rName{Segment}
       &
       \aseg & ::= & \agsprefpair \; |\;  \agssegpair
        \\
       \rName{Round}
       &
       \around &::= & \aroundtuple
        \\
       \rName{Round Seq}
       &
       \aseqround &\in & \around^*
        \\
       \rName{Seg Seq}
       &
       \aseqseg &\in & \aseg^*
        \\
       &
       {\inclient} &::= &  \aseqseg%{{\agsseg}^* \cup \agspref}
       \\
       & 
       \outclient &\in & {\aseqround}
      \\
       &
       {\inserver} & \in  & \partialfunction{\idset}\aseqround%\outclient
       \\
       & 
       \outserver &\in & \partialfunction{\idset}\aseqseg%\inclient
   \end{array}
  \]
  \caption{Syntax}
  \label{fig:syntax-implementation}
\end{figure}

%Furthermore, we assume the following parametric functions defined over the abstract data types above borrowed from :


%Let $\pendingtype$, $\sigma$ be sequence defined:

%$\pendingtype$,$\sigma$ := $\epsilon$ \text{\textbar} [$\delta$]$\cdot$ $\pendingtype$
 

We assume the following countable sets of persisted state $\gsprefixtype$ ranged over by $\agspref$, $\agspref_1$, $\ldots$; 

Let \\



\subsection{Syntax}
 \begin{definition}[Implementation of GSP] 
	The syntax of clients and server is given by the following grammar 
  \[
    \begin{array}{l@{\quad}r@{\;::=\;}l}
	\rName{system} & \isystemterm &  \iserver\ \bigpar\ \iclient \\
	\rName{server} & \iserver & \server{\persistedstate}{\inqueue}{\outqueue} \\
	\rName{clients}& \iclient & \zero \;|\; 
	 						%\client{P}{<\astate,\pendingtype,\pushbuffertype,\transactionbuffertype,\receivebuffertype,n,\inqueue,\outqueue>} 
							\addid{\iclientsyntax}
							\;|\; \ 
							\iclient \bigpar \iclient  \\
%			 (\textsc{program}) & P & \zero \;|\; \readins{x}{r} \;|\; \updateins{u} \;|\; 	\pushins \;|\; \pullins
     \end{array}
  \]
 \end{definition}
 
The implementation of a system is a server and clients interacting concurrently. A server will be represented with a persisted state denoted by $\persistedstate$, input messages $\inserver$ and output messages $\outserver$.
We will refer to clients, as tuple of program $P$ and set of states and sequences $E$. A program client stores a state ($E.\known$), a pending queue ($E.\pending$),i.e., updates sent to the server without confirmation of their reception, a push buffer ($E.\pushbuffer$) which holds updates that were pushed by the client but have not been sent to the server, a transaction buffer ($E.\transactionbuffer$) which holds updates for sending with another ones, a receive buffer ($E.\receivebuffertype$) for updates which were sent by the server, the number of round sent ($E.\nround$), a input message queue ($E.\inclient$) and a output message queue ($E.\outclient$).


\subsection{Operational Semantics}

The operational semantics of \gspcalculus\ is defined by a labeled transition system 
over well-formed terms, up-to the structural congruence.

The labeled transition system considers the following actions:
\[ 
\begin{array}{r@{\ ::= \ }l}
  \alpha & \tau \ | \ \readtran{r} \ | \  \updatetran{u} \ | \ \pulltran \ | \ \pushtran \ | \ \confirmedtran 
  \\
\end{array}
\]



 \[
 \begin{array}{l}
   \hspace{-.3cm} \textsc{CLIENTS}\\
%		\mathax{update}{\clientr{\updateins{u}}{\stateclient} \arro{\updatetran{u}} \clientr{P}{\update{E}{\transactionbuffer}{\appendplus{E.\transactionbuffer}{u}}}} \\[15pt]
%		
	\mathax{update}
		{\begin{array}{l}
		\addid{\iclientinst[\tupdateins]} \ \bigpar \ \isystemterm
		\arro{\updatetran{u}} 
		\\[5pt]
		\hspace{5.5cm}
		{\addid{\iclientinst[P][@][@][@][\iappend{\transactionbuffer}{u}]} \ \bigpar \   \isystemterm}
		\end{array}}
\\
% 		\mathax{push}{\clientr{\pushins}{\stateclient} \arro{\pushtran} \clientr{P}{\updatethree{E}{\pushbuffer}{\reduce{E.\pushbuffer\cdot E.\transactionbuffer}}{\transactionbuffer}{\epsilon}{\nround}{E.\nround+1}  }} \\[25pt]
	\mathax{push}
		{\begin{array}{l}
		\addid{\iclientinst[\tpushins]} \ \bigpar \ \isystemterm
		\arro{\pushtran} 
		\\[5pt]
		\hspace{5cm}
		{\addid{\iclientinst[P][@][@][\ireduce{\pushbuffer\cdot \transactionbuffer}][\epsilon][@][n+1]}\ \bigpar \ \isystemterm}
		\end{array}}
%\updatethree{E}{\pushbuffer}{\appendplus{\pushbuffer}{\transactionbuffer}}{\transactionbuffer}{\epsilon}{\nround}{\nround+1}
\\
%	 {\clientr{P}{\stateclient} \arro{\sendtran} \clientr{P}{\updatethree{E}{\pending}{E.\pending \cdot E.\pushbuffer}{\pushbuffer}{\epsilon}{\outclient(i)}{\outclient(i) \cdot \round{i}{E.\nround}{E.\pushbuffer}}}}

	 \mathrule{send}
	         {\pushbuffer \neq \epsilon}
 		{\begin{array}{l}
		\addid{\iclientinst} \ \bigpar \ \isystemterm
		\arro{\tau} 
		\\[5pt]
		\hspace{5.3cm}
		\addid{\iclientinst[@][@][\pending \cdot \pushbuffer][\epsilon][@][@][@][@][\outclient \cdot {\aroundtuple[\cid][\nround][\pushbuffer]}]}\ \bigpar \ \isystemterm
	 	\end{array}
		}	
\\
%		\mathrule{receive}{E.\inclient = \gs_0 \cdot \gs_t}{\clientr{P}{\stateclient} \arro{\receive} \clientr{P}{\updatetwo{E}{\receivebuffer}{E.\receivebuffer \cdot \gs_0)}{\inclient}{\gs_t)}}}\\[25pt]
	\mathax{receive}
 		{\begin{array}{l}
		\addid{\iclientinst[@][@][@][@][@][@][@][\aseg \cdot \inclient]} \ \bigpar \ \isystemterm
		\arro{\tau} 
		\\[5pt]
		\hspace{6.4cm}
		\addid{\iclientinst[@][@][@][@][@][\receivebuffer \cdot \aseg]}\ \bigpar \ \isystemterm
	 	\end{array}
		}	
\\

%		\mathrule{pull}{E.\inclient(i) \neq \undefined \qquad E.\outclient(i) \neq \undefined \qquad |E.\receivebuffer| > 0}{\clientr{\pullins}{\stateclient}\arro{\pulltran} \clientr{P}{\updatethree{E}{\known}{\reducestate{E.\known}{E.\receivebuffer}}{\receivebuffer}{\epsilon}{E.\pending}{\remove{E.\pending}{E.\receivebuffer}}}}
	\mathaxiom{pull}
		 {\begin{array}{l}
			\addid{\iclientinst[\tpullins][@][@][@][@][{\agssegpair[x]}\cdot\receivebuffer]}\ \bigpar \ \isystemterm
			\arro{\pulltran} 
			\\[5pt]
			\hspace{5.2cm}
			\addid{\iclientinst[P]
						 [\iapply{\known}{x}]
						 [{\pending}\setminus x]
						 [@][@]
						 [{\receivebuffer}]}\ \bigpar \ \isystemterm
	 	\end{array}
		}
		\\
\henote{\mbox{el apply no est� en el paper, ellos usan reducestate}}
\\

%		\mathrule{read}{\readplus{r}{\curstate{\known}{\pending}{\pushbuffer}{\transactionbuffer}}}{\clientr{\readins{x}{r}}{\stateclient} \arro{\readtran{r}} \clientr{\update{P}{x}{v}}{\stateclient}}
	\mathrule{read}
		{\iread{r}{\curstate{\known}{\pending}{\pushbuffer}{\transactionbuffer}}}
		{\begin{array}{l}
		\addid{\iclientinst[\treadins{x}{r}]} \ \bigpar \ \isystemterm
		\arro{\readtran{r}} 
		\\[5pt]
		\hspace{6.1cm}
		\addid{\iclientinst[{P}\subst{x}{v}]}\ \bigpar \ \isystemterm
		\end{array}
		}
\\

\\[15pt]

%		\mathrule{pull}{E.\inclient(i) \neq \undefined \qquad E.\outclient(i) \neq \undefined \qquad |E.\receivebuffer| > 0}{\clientr{\pullins}{\stateclient}\arro{\pulltran} \clientr{P}{\updatethree{E}{\known}{\reducestate{E.\known}{E.\receivebuffer}}{\receivebuffer}{\epsilon}{E.\pending}{\remove{E.\pending}{E.\receivebuffer}}}}
%	\mathrule{pull}
%		{\appf{\inqueue}{\cid} \neq \undefined \qquad\qquad \appf{\outqueue}{\cid} \neq \undefined \qquad\qquad |\receivebuffer| > 0}
%		{
%			\addid{\iclientinst[\tpullins]}
%			\arro{\pulltran} 
%			\addid{\iclientinst[P]
%						 [\reducestate{\known}{\receivebuffer}]
%						 [\remove{\pending}{\receivebuffer}]
%						 [@][@]
%						 [{\epsilon}]
%				}
%		}
%		\\
%\\[15pt]
%		\mathax{confirmed}{\clientr{\confirmedins{x}}{\stateclient} \arro{\confirmedtran}
 %\clientr{\update{P}{x \mapsto E.\pending = \epsilon \lor E.\pushbuffer = \epsilon \lor E.\transactionbuffer = \epsilon}{\textbf{true}}}{\stateclient}} 
		\mathrule{confirm}
		{ v = (\pending \cdot \pushbuffer \cdot\transactionbuffer == \epsilon) }
		{
			\addid{\iclientinst[\tpullins]}
			\arro{\pulltran} 
			\addid{\iclientinst[P\subst{x}{v}]		
				}
		}
\\[20pt]

\mathrule{while-true}{\condition{\cond}{true}}{\clientr{\pwhile{\cond}{\tprogram}}{\stateclient} \arro{\tau} \clientr{\tprogram;\pwhile{\cond}{\tprogram}}{\stateclient}}

\\[25pt]

\mathrule{while-false}{\condition{\cond}{true}}{\clientr{\pwhile{\cond}{\tprogram;Q}}{\stateclient} \arro{\tau} \clientr{Q}{\stateclient}}




%	\mathrule{receive}
%		{\inclient(\cid) = \gs_0 \cdot \gs_t}
%		{\begin{array}{l}
%			{\addid{\iclientinst[@]}} \arro{\tau}
%			\\
%			\hspace{5cm} 
%			{\addid{\iclientinst[@][@][@][@][@][\receivebuffer \cdot \gs_0.gssegment][@][\inclient{[\cid\mapsto\gs_t]}]}}
%		\end{array}}

\end{array}
 \]


 \[
 \begin{array}{l}
    \hspace{-.3cm} \textsc{SERVER}\\
    
\mathrule{drop-conn}{b_i \in \inserver \qquad  b_i \in \outserver}{\server{\persistedstate}{\inserver}{\outserver} \auxarro{\dropconn{b_i}} \server{\persistedstate}{\inserver \setminus b_i}{\outserver \setminus b_i}} 
\hfill
\mathax{crash-and-recover}{\server{\persistedstate}{\inserver}{\outserver} \auxarro{\crashandrecover} \server{\persistedstate}{\undefined}{\undefined}}

\\[25pt]

\mathrule{batch}{rs=\receiveroundsname{\inserver} \qquad \agssegpair[@][{\amxrf[{'}]}] =\append{\emptygssegment}{rs} \qquad 
  \astate[']=\iapply{\astate}{\adelta}}{\server{\agsprefpair }{\inserver}{\outserver} \arro{\tau} 
  \server{\agsprefpair[{\astate[']}][{\amxrf[{'}]}] }{\clean{\inserver}}{\notify{dom(\outserver)}{\outserver}{gs}}}

\\[25pt]

\mathrule{accept-conn}{b_i \notin \inserver \qquad b_i \notin \outserver}{\server{\persistedstate}{\inserver}{\outserver} \auxarro{\acceptconn{b_i}} \server{\persistedstate}{\inserver}{\update{\outserver}{b_i}{\persistedstate}}}

    \\[35pt]
    \hspace{-.3cm} \textsc{COMMUNICATION}
		
\\
    \mathrule{comm server-client}{\outserver(i)=\gs \cdot \gss}{\server{\state}{\inserver}{\outserver} \bigpar \clientr{P}{\stateclient} \arro{\tau} \server{\state}{\inserver}{\update{\outserver}{i}{\gss}} \bigpar 
		\clientr{P}{\update{E}{\inclient}{E.\inclient \cdot \gs}}}
		
\\[25pt]

	    \mathrule{comm client-server}{\outclient=\headerround \cdot \tailround}{\server{\state}{\inserver}{\outserver} \bigpar \clientr{P}{\stateclient} \arro{\tau} \server{\state}{\update{\inserver}{i}{\inserver(i) \cdot \headerround}}{\outserver} \bigpar 
		\clientr{P}{\update{E}{\outclient}{\tailround}}}	
 \\

 \end{array}
 \]


Rule $\textsc{(drop-conn)}$ removes from the server's queues the client called $b_i$. Rule $\textsc{(crash-and-recover)}$ leaves undefined the server's queues and preserves the server's persistent state.  Rule $\textsc{(accept-conn)}$ show how to add a new connecti(Read)send to its the persistent state. Last rule from \textbf{\textsc{(server)}} is Rule $\textsc{(batch)}$, the most interesting of this group. Its has three hypothesis, the first one, is responsible for receiving rounds from the queue of in-messages. Se\cond one, let $\emptygssegment$ be a empty segment, it gives back a delta object who represents the combination of numbers of rounds into a single object. Finally, this object is applied to the persistent state. As result, the persistent state is putted into the queues messages-out.

There are two \textbf{\textsc{(communications)}} rules. Rule $\textsc{(comm-server-client)}$ when the server has a message for client $i^{th}$, this is removed from the server's queue message-out and is putted into the queue message-in from client $i^{th}$. Rule $\textsc{(comm-client-server)}$ states when a round from client $i^{th}$ is left into server queue message-in.

Rule $\textsc{(Read)}$ gives the result of performing a lecture on messages queues from the client. Rule $\textsc{(update)}$ adds an update to the transaction buffer. Rule $\textsc{(push)}$ leaves into push buffer a delta object resulted of reducing the push buffer with transaction buffer. The transaction buffer is cleaned and the number of rounds sent is incremented by one. Rule $\textsc{(pull)}$ shows when a persisted state from a client is modified. The hypothesis are that a channel have been accepted, i.e., these must be defined for client $i^{th}$ besides the receive buffer should have an element at least. $\textsc{(confirmed)}$
computes the states of the internal queues,i.e., if these has any element. Rules $\textsc{(while-true)}$ and $\textsc{(while-false)}$ are standard. Rule $\textsc{(send)}$ creates a new round setting who is the client ($i^{th}$), how many rounds client ($i^{th}$) has sent and content from the push buffer. Finally rule $\textsc{(receive)}$ move out segments from the client's queue message-in to receive buffer.

\section{Equivalence}

We have introduced an abstract GSP protocol and a robust streaming server-client implementation of GSP. There exists a relation represented by $\triangleleft$ [~ref:] which relates state and delta objects to the update sequences. 
\begin{itemize}
	\item On $\mathit{Delta} \times \mathit{Update}^*$, let $\triangleleft$ be the smallest relation such that (1) $\emptydelta \triangleleft []$, and
(2) $d \triangleleft a$ implies $append(d,u) \triangleleft a \cdot u$ for all updates $u$, and (3) $d_1 \triangleleft a_1 \ldots d_n \triangleleft a_n$ implies $reduce(d_1 \ldots d_n) \triangleleft a_1 \cdots a_n$ and (4) $d_1 \triangleleft a_1 \ldots d_n \triangleleft a_n$ implies $\remove{d_1 \ldots d_{n}}{d_n} \triangleleft a_1 \cdots a_{n-1}$. \marginpar{El (4) lo agregamos nosotros. ¿Se aclara? ¿C\'omo?}
  
\item On $State \times Update^*$, let $\triangleleft$ be the smallest relation such that (1) $\initialstate \triangleleft []$, and
(2) $s \triangleleft a \land \ d_1 \triangleleft a_1 \land \ \ldots \ \land \ d_n \triangleleft a_n$ implies $apply(s,d_1 \ldots d_n) \triangleleft a \cdot a_1 \cdots a_n$.
\end{itemize}

Next theorem states one of most important result of the paper, saying that abstract GSP protocol and the implementation of GSP are weak bisimulation equivalent (or weakly bisimilar). 

\begin{theorem}
Let $A$ be a system from abstract GSP Protocol, defined as  then both system are weakly bisimilar, written $A \approx C$, if $(A,C)  \in \ \mathcal{R}$, where $\mathcal{R}$ be a binary relation defined as $\{ (\parallel_{i \ \in\ I_{\{0 \ldots n\}}} A_i \parallel \ \queuemessage_A),(\parallel_{i \ \in\ I_{\{0 \ldots n\}}} C_i \parallel \ \queuemessage_C) \ | \ \forall i \ \in\ \mathbb{N}, \forall r: \readtype, A_i = \ \tclient{\tprogram_i}{\tknown_i}{\tpending_i}{\ttransactionbuffer_i}{\tsent_i}{\treceivebuffer_i} \land C_i = \client{P_i}{\stateclient_i}$ and the conjunction of the following properties: %\land \ \rvalue{r}{\flatten(\queuemessage_{A}[0..\tknown-1] \cdot \tpending) \cdot \ttransactionbuffer} = \readp{r}{\curstate{E_{i}.\known}{E_{i}.\pending}{E_{i}.\pushbuffer}{E_{i}.\transactionbuffer}} \land \ \queuemessage_C.\state \triangleleft \queuemessage_A \land \ E_{i}.\transactionbuffer = [\ttransactionbuffer_i] \land \ E_{i}.\receivebuffer[0].delta \triangleleft \queuemessage_A[\tknown_i] \ldots\ E_{i}.\receivebuffer[j - 1].delta \triangleleft \queuemessage_A[\tknown_i + \treceivebuffer_i] - 1 \ \land \ E_{i}.known \triangleleft \queuemessage_A[0 \ldots \tknown_i - 1] \ \land \  E_{i}.\pending \cdot E_{i}.\pushbuffer \triangleleft \tpending_i \ \land \ \forall l, 1 \leq l < |\inclient|, E_{i}.\inclient[l] \triangleleft \queuemessage_A[\tknown + \treceivebuffer - 1 +l] \ \land \ \tknown + \treceivebuffer + 1 \leq |\queuemessage_A| \Leftrightarrow |E_{i}.\inclient| + |\queuemessage_C.\outserver| > 0 \ \land \ \forall \beta \ \in \ \tsent_i, \beta \ \in \ E_{i}.\outclient \ \cup \ \queuemessage_C.\inserver
\marginpar{Ac\'á deber\'ia decir que R cumple las propiedades de weak bisimulation. C\'omo lo escribo mejor?}
\begin{enumerate}
	\item $\rvalue{r}{\flatten(\queuemessage_{A}[0..\tknown-1] \cdot \tpending) \cdot \ttransactionbuffer} = \readp{r}{\curstate{E_{i}.\known}{E_{i}.\pending}{E_{i}.\pushbuffer}{E_{i}.\transactionbuffer}}$
	\item $\queuemessage_C.\state \triangleleft \queuemessage_A$
	\item $\ E_{i}.\transactionbuffer \triangleleft [\ttransactionbuffer_i]$
	\item $E_{i}.\receivebuffer[0].delta \triangleleft \queuemessage_A[\tknown_i] \ldots\ E_{i}.\receivebuffer[j - 1].delta \triangleleft \queuemessage_A[\tknown_i + \treceivebuffer_i - 1]$
	\item $E_{i}.known \triangleleft \queuemessage_A[0 \ldots \tknown_i - 1]$
	\item $E_{i}.\pending \cdot E_{i}.\pushbuffer \triangleleft \tpending_i$
	\item For all natural $l$, such that $1 \leq l < |\inclient|$ then $E_{i}.\inclient[l] \triangleleft \queuemessage_A[\tknown + \treceivebuffer - 1 +l]$
	\item $\tknown + \treceivebuffer + 1 \leq |\queuemessage_A| \Leftrightarrow |E_{i}.\inclient| + |\queuemessage_C.\outserver| > 0$
	\item For all update sequence $\beta \ \in \ \tsent_i$ then $\beta \ \in \ E_{i}.\outclient \ \cup \ \queuemessage_C.\inserver$
		\item For all natural $m$, such that $0 \leq m < |\outserver(i)| - 1$ then $\outserver(i)[m] \triangleleft \queuemessage_A[\tknown + \treceivebuffer + |\inclient| + m]$

		

\end{enumerate}
\end{theorem}



\begin{proof}

The proof follows by induction on the length of the derivation $\arro{} ^*$. We use $\tilde{A}$ and $\tilde{C}$ to denote clients, except client $i^{th}$, interacting concurrently, i.e., $\parallel_{j \ \in\ I_{\{0 \ldots n\} - i}}$ $A_j$ or $C_j$ respectively.

\begin{itemize}
   \item{\bf n=0}. Then $A$ = $\tsystem{\tclienti{0}{\emptyset}{\emptyset}{\emptyset}{\emptyset}{0}\ \bigpar\ \tilde{A}}{\queuemessage_A}$ and $C$ = $\clientr{E}{\initialstate}$. In particular,

			\begin{enumerate}
				\item The operation $rvalue$ in the abstract language is performing on empty sequences, hence, the result is undefined. Moreover the parametric function $read$ tries to get a result on a $\initialstate$, then the result returned is undefined.
				\item It is easy to see, due to property (1) of $\triangleleft$ between States and Updates.
				\item It is guaranteed by property (1) of $\triangleleft$ between Deltas and Updates.
				\item Analogous to property 2.
				\item Analogous to property 3.
				\item Precondition of implication is $false$.
				\item Precondition of both implications are $false$. 
				\item $\emptyset \in \ \emptyset$.
				
			\end{enumerate}
	
   \item{\bf n=k+1}. $\forall (A,C) \ \in \ \mathcal{R}$
	
			
			\begin{itemize}
				\item {\bf rule (\textsc{t-read})}. If $A \arroi{\readtran{r}} A'$, then by rule (\textsc{\footnotesize{t-read}}), $A$ must be the following term: $ \tsystem{\tclienti{\treadins{x}{r}}{\tknown}{\tpending}{\ttransactionbuffer}{\tsent}{\treceivebuffer}\ \bigpar\ \tilde{A}}{\queuemessage_A}$, therefore, the client $i^{th}$ is only one who changes. So that, $A'$ will be $\tsystem{\tclienti{\update{\tprogram}{x}{v}}{\tknown}{\tpending}{\ttransactionbuffer}{\tsent}{\treceivebuffer}\bigpar\ \tilde{A}}{\queuemessage_A}$. As $(A,C)$ belongs $\mathcal{R}$ then $C$ must be $\clientr{\readins{x}{r}}{\stateclient} \bigpar\ \tilde{C} \ \bigpar\ \ \queuemessage_C$, and the read transition is the only one that $C$ could perform.
				Then, looking at the rule, $C' = \clientr{\update{P}{x}{v}}{\stateclient}\ \bigpar\ \tilde{C} \ \bigpar\ \ \queuemessage_C$. We prove that $(A',C') \ \in \ \mathcal{R}$ if ten properties introduced are keeping.				
					\begin{enumerate}
						\item It is easy to see, since $P$ is the same in both clients and the value got back by each read operations is equals because of $(A,C) \ \in \ \mathcal{R}$ then property 1 is worth. 
						\item to 9. Do not change.
						

					\end{enumerate}
	
			\item {\bf rule (\textsc{t-update})}. If $A \arroi{\updatetran{u}} A'$, then by rule (\textsc{\footnotesize{t-update}}), $A$ must be the following term: $ \tsystem{\tclienti{\tupdateins}{\tknown}{\tpending}{\ttransactionbuffer}{\tsent}{\treceivebuffer}\ \bigpar\ \tilde{A}}{\queuemessage_A}$, therefore, the client $i^{th}$ is only one who changes. So that, $A'$ will be $\tsystem{\tclienti{\tprogram}{\tknown}{\tpending}{\ttransactionbuffer \cdot \tupdate}{\tsent}{\treceivebuffer}\bigpar\ \tilde{A}}{\queuemessage_A}$. We know that $(A,C)$ belongs $\mathcal{R}$, then $C$ must be $\clientr{\updateins{u}}{\stateclient} \bigpar\ \tilde{C} \ \bigpar\ \ \queuemessage_C$, and the update transition is the only one that $C$ could perform. Then, $C' = \clientr{P}{\update{E}{\transactionbuffer}{\appendplus{E.\transactionbuffer}{u}}}\ \bigpar\ \tilde{C} \ \bigpar\ \ \queuemessage_C$. We prove that $(A',C') \ \in \ \mathcal{R}$ if ten properties introduced are keeping.					
						\begin{enumerate}
							\item States $\rvalue{r}{\flatten(\queuemessage_{A}[0..\tknown_i-1] \cdot \tpending_i) \cdot \ttransactionbuffer_i} = \readp{r}{\applyplus{E_{i}.\known}{E_{i}.\pending \cdot E_{i}.\pushbuffer \cdot E_{i}.\transactionbuffer}}$ however after applying rules, the transaction buffers change. Then, performing a read action in abstract GSP, $\rvalue{r}{\flatten(\queuemessage_{A}[0..\tknown_i-1] \cdot \tpending_i) \cdot \ttransactionbuffer_i \cdot \tupdate}$ and the implementation, $\readp{r}{\applyplus{E_{i}.\known}{E_{i}.\pending \cdot E_{i}.\pushbuffer \cdot \appendplus{E.\transactionbuffer}{u}}}$. We rename $\ttransactionbuffer \cdot \tupdate$ by $\ttransactionbuffer$' and  $\appendplus{E.\transactionbuffer}{u}$ by $E_{i}.\transactionbuffer$'. If we show that $\forall u: Update$ such that $\appendplus{E.\transactionbuffer}{u} \triangleleft \ttransactionbuffer \cdot \tupdate$, we will prove that $(A',C') \ \in \ \mathcal{R}$. By property 3, we know that $\ E_{i}.\transactionbuffer \triangleleft [\ttransactionbuffer_i]$, using the property of relation between Delta and Update, $\appendplus{E.\transactionbuffer}{u} \triangleleft \ttransactionbuffer \cdot \tupdate$.
							\item to 9. Do not change.
						
						\end{enumerate}
	
\item {\bf rule (\textsc{t-push})}. If $A \arroi{\pushtran} A'$, then by rule (\textsc{\footnotesize{t-push}}), $A$ must be the following term: $ \tsystem{\tclienti{\tpushins}{\tknown}{\tpending}{\ttransactionbuffer}{\tsent}{\treceivebuffer}\ \bigpar\ \tilde{A}}{\queuemessage_A}$, therefore, the client $i^{th}$ is only one who changes. So that, $A'$ will be $\tsystem{\tclienti{\tprogram}{\tknown}{\tpending \cdot [\ttransactionbuffer]}{\ttransactionbuffer \cdot \tupdate}{\tsent \cdot [\ttransactionbuffer]}{\treceivebuffer}\bigpar\ \tilde{A}}{\queuemessage_A}$. We know that $(A,C)$ belongs $\mathcal{R}$, then $C$ must be $\clientr{\pushins}{\stateclient} \bigpar\ \tilde{C} \ \bigpar\ \ \queuemessage_C$, and the push transition is the only one that $C$ could perform. Then, after applying the rule, $C' = \clientr{P}{\updatethree{E}{\pushbuffer}{\reduce{E.\pushbuffer\cdot E.\transactionbuffer}}{\transactionbuffer}{\epsilon}{\nround}{E.\nround+1}}\ \bigpar\ \tilde{C} \ \bigpar\ \ \queuemessage_C$. We prove that $(A',C') \ \in \ \mathcal{R}$ if ten properties introduced are keeping.					
						\begin{enumerate}
							\item We have to prove that after push transitions, the operations $rvalue$ and $read$ get back the same value. Then $\rvalue{r}{\flatten(\queuemessage_{A}[0..\tknown_i-1] \cdot \tpending_i) \cdot \ttransactionbuffer_i} = \rvalue{r}{\flatten(\queuemessage_{A}[0..\tknown_i-1] \cdot (\tpending_i \cdot [\ttransactionbuffer_i])) \cdot \epsilon}$ by Lemma XX besides $\readp{r}{\curstate{E_{i}.\known}{E_{i}.\pending}{E_{i}.\pushbuffer}{E_{i}.\transactionbuffer}}$ = $
						\readp{r}{\curstate{E_{i}.\known}{E_{i}.\pending}{ \reduce{E.\pushbuffer\cdot E.\transactionbuffer}}{\epsilon}}$ by Lemma YY.
							\item It does not change.
							\item $E_{i}.\transactionbuffer = \epsilon$ and $\ttransactionbuffer_i = \epsilon$. Using the property of the relation $\triangleleft$, $\epsilon \triangleleft \epsilon$ 
							\item It does not change.
							\item It does not change.
							\item States $E_{i}.\pending \cdot E_{i}.\pushbuffer \triangleleft \tpending_i$ besides by 3. $\ E_{i}.\transactionbuffer \triangleleft [\ttransactionbuffer_i]$. Finally, applying the property of the relation $\triangleleft$, $\reduce{E_i.\pending \cdot E_i.\pushbuffer \cdot E_i.\transactionbuffer} \triangleleft \tpending_i \cdot [\ttransactionbuffer_i]$.
						  \item to 9. do not change.
						\end{enumerate}
						\item {\bf rule (\textsc{t-pull})}. If $A \arroi{\pulltran} A'$, then by rule (\textsc{\footnotesize{t-pull}}), $A$ must be the following term: $ \tsystem{\tclienti{\tpullins}{\tknown}{\tpending}{\ttransactionbuffer}{\tsent}{\treceivebuffer}\ \bigpar\ \tilde{A}}{\queuemessage_A}$, therefore, the client $i^{th}$ is only one who changes. So that, $A'$ will be $\tclienti{\tprogram}{\tknown+\treceivebuffer}{\tpending \setminus \queuemessage[\tknown .. \tknown + \treceivebuffer]}{\ttransactionbuffer}{\tsent}{0} \bigpar \ \tilde{A} \bigpar\ \queuemessage_A$. Due to $(A,C)$ belongs $\mathcal{R}$, then $C$ must be $\clientr{\pullins}{\stateclient} \bigpar\ \tilde{C} \ \bigpar\ \ \queuemessage_C$, and the pull transition is the only one that $C$ could perform. Then, after applying the rule, $C' = \clientr{P}{\updatethree{E}{\known}{\reducestate{E.\known}{E.\receivebuffer}}{\receivebuffer}{\epsilon}{E.\pending}{\remove{E.\pending}{E.\receivebuffer}}} \bigpar\ \tilde{C} \ \bigpar\ \ \queuemessage_C$. We prove that $(A',C') \ \in \ \mathcal{R}$ if ten properties introduced are keeping.	
						
						\begin{enumerate}
							\item We have to prove that after push transitions, the operations $rvalue$ and $read$ get back the same value. Then 
							$\rvalue{r}{\flatten{\queuemessage_{A}[0..\tknown_i+\treceivebuffer-1] \cdot (\tpending \setminus \queuemessage[\tknown .. \tknown + \treceivebuffer])} \cdot \ttransactionbuffer_i} = \readp{r}{\curstate{\reducestate{E.\known}{E.\receivebuffer}}{\remove{E.\pending}{E.\receivebuffer}}{E.\pushbuffer}{E.\transactionbuffer}}$ = $\readp{r}{\applyplus{\reducestate{E.\known}{E.\receivebuffer}}{\remove{E.\pending}{E.\receivebuffer} \cdot E.\pushbuffer \cdot E.\transactionbuffer}}$. On the one hand, property 5 states that $E_{i}.known \triangleleft \queuemessage_A[0 \ldots \tknown_i - 1]$, on the other hand by property 4, $E_{i}.\receivebuffer[0].delta \triangleleft \queuemessage_A[\tknown_i] \ldots\ E_{i}.\receivebuffer[j - 1].delta \triangleleft \queuemessage_A[\tknown_i + \treceivebuffer_i -1]$, after, using the property of the relation $\triangleleft$ between state and updates, we know that $\applyplus{E_{i}.\known}{E_{i}.\receivebuffer[0].delta \cdots \ E_{i}.\receivebuffer[\treceivebuffer_i-1].delta} \triangleleft \queuemessage_A[0 \ldots \tknown_i - 1] \cdot  \cdots  \cdot \ \queuemessage_A[\tknown_i + \treceivebuffer_i -1]$. As you note, the left side is reducestate's definition, then $\reducestate{E.\known}{E.\receivebuffer} u\triangleleft \queuemessage_A[0 \ldots \tknown_i + \treceivebuffer_i -1]$. Next, property by 6., we know that $E_{i}.\pending \cdot E_{i}.\pushbuffer \triangleleft \tpending_i$, if we apply the extra axiom of $\triangleleft$, we will get that $\remove{E_i.\pending \cdot E_i.\pushbuffer}{E_i.\receivebuffer[0].delta} \triangleleft \tpending \setminus \queuemessage[\tknown]$, so that, $\remove{E_i.\pending \cdot E_i.\pushbuffer}{E_i.\receivebuffer} \triangleleft \tpending \setminus \queuemessage[\tknown .. \tknown + \treceivebuffer]$. Therefore it is guaranteed.
							\item It does not change.
							\item It does not change.
							\item $E_{i}.\receivebuffer$ is $\epsilon$ and $\treceivebuffer$ is 0, therefore it property is guaranteed.
							\item It does not change.
							\item We know that $E_{i}.\pending \cdot E_{i}.\pushbuffer \triangleleft \tpending_i$, if we apply the extra axiom of $\triangleleft$, we will get that $\remove{E_i.\pending \cdot E_i.\pushbuffer}{E_i.\receivebuffer[0].delta} \triangleleft \tpending \setminus \queuemessage[\tknown]$, so that, $\remove{E_i.\pending \cdot E_i.\pushbuffer}{E_i.\receivebuffer} \triangleleft \tpending \setminus \queuemessage[\tknown .. \tknown + \treceivebuffer]$.
							\item to 9. do not change.
						\end{enumerate}
					\item {\bf rule (\textsc{t-confirmed})} If $A \arroi{\confirmedtran} A'$, then by rule (\textsc{\footnotesize{t-confirmed}}), $A$ must be the following term: $ \tsystem{\tclienti{\tconfirmedins{x}}{\tknown}{\tpending}{\ttransactionbuffer}{\tsent}{\treceivebuffer}\ \bigpar\ \tilde{A}}{\queuemessage_A}$, therefore, the client $i^{th}$ is only one who changes. So that, $A'$ will be $\tsystem{\tclienti{\update{\tprogram}{x}{\tpending \neq \emptysequence \vee\ \ttransactionbuffer \neq \epsilon}}{\tknown}{\tpending}{\ttransactionbuffer}{\tsent}{\treceivebuffer}\bigpar\ \tilde{A}}{\queuemessage_A}$. As $(A,C)$ belongs $\mathcal{R}$ then $C$ must be $\clientr{\confirmedins{x}}{\stateclient} \bigpar\ \tilde{C} \ \bigpar\ \ \queuemessage_C$, and the read transition is the only one that $C$ could perform.
				Then, looking at the rule, $C' = \clientr{\update{P}{x \mapsto E.\pending = \epsilon \lor E.\pushbuffer = \epsilon \lor E.\transactionbuffer = \epsilon}{\textbf{true}}}{\stateclient} \bigpar\ \tilde{C} \ \bigpar\ \ \queuemessage_C$. We prove that $(A',C') \ \in \ \mathcal{R}$ if ten properties introduced are keeping.				
					\begin{enumerate}
						\item Both clients has the same program $P$. By 6. we know that the value of $E.\pending \cdot E.\pushbuffer$ is related to $\tpending$, thereby, when E.$\pending$ or E.$\pushbuffer$ have any elements then $\tpending$ also has it. Finally, $\ E_{i}.\transactionbuffer$ has elements iff $\ttransactionbuffer_i$ also has by property 3.
						\item to 9. Do not change.
					\end{enumerate}
					\item {\bf rule (\textsc{t-receive})} If $A \arroi{\tau} A'$, then by internal rule (\textsc{\footnotesize{t-receive}}), $A$ must be the following term: $\tsystem{\tclient{\tprogram}{\tknown}{\tpending}{\ttransactionbuffer}{\tsent}{\treceivebuffer}\ \bigpar\ \tilde{A}}{\queuemessage_A}$, therefore, the client $i^{th}$ is only one who changes. So that, $A'$ will be $\tsystem{\tclient{\tprogram}{\tknown}{\tpending}{\ttransactionbuffer}{\tsent}{\treceivebuffer + 1}\bigpar\ C}{\queuemessage}$. As	$(A,C)$ belongs $\mathcal{R}$ then $C$ must be $\clientr{\tprogram}{\stateclient} \bigpar\ \tilde{C} \ \bigpar\ \ \queuemessage_C$, such that after internal transitions $\tau$, it will become $C'$. By Hypothesis of (\textsc{\footnotesize{t-receive}}, we know that $\tknown + \treceivebuffer + 1 \leq \text{\textbar} S_A \text{\textbar}$, then, by Property 8., $|E_{i}.\inclient| + |\queuemessage_C.\outserver| > 0$. We will consider two cases:
						
						\begin{itemize}
							\item $|E_{i}.\inclient| > 0$, then $|E_{i}.\inclient|$ has a $GsSegment$ at least, i.e., $E.\inclient = \gs_0 \cdot \gs_t$. If $C$ took the internal rule (\textsc{\footnotesize{RECEIVE}}), we will get the following $C'$ termn, $\clientr{P}{\updatetwo{E}{\receivebuffer}{E.\receivebuffer \cdot \gs_0.gssegment)}{\inclient}{\gs_t)}}$. Now, we will check if $(A',C')$ belongs to $\mathcal{R}$.
							
							\begin{enumerate}
								\item The inputs term in read operation have not changed.
								\item It does not change.
								\item It does not change.
								\item Term $\treceivebuffer$ was incremented by 1. Then, we should prove that $E_{i}.\receivebuffer[0].delta \triangleleft \queuemessage_A[\tknown_i] \ldots\ E_{i}.\receivebuffer[(j+1) - 1].delta \triangleleft \queuemessage_A[\tknown_i + (\treceivebuffer_i+1) - 1]$ is guaranteed. By Property 4., we only should prove $E_{i}.\receivebuffer[j].delta \triangleleft \queuemessage_A[\tknown_i + \treceivebuffer_i]$ allow $(A',C')$ to belong to $\mathcal{R}$. It worths by Property 7., in particular with $l$ = 1.  
								\item It does not change.
								\item It does not change.
							  \item It is easy to see such that $|\inclient'| < |\inclient|$, then we use Property 7.
								\item to 9. They do not change.
							\end{enumerate}
						\end{itemize}
					\item $|\queuemessage_C.\outserver| > 0$, then $|\queuemessage_C.\outserver|$ has a $GsSegment$ at least, i.e., $\queuemessage_C.\outserver(i)$ = $\gs \cdot \gss$. Then, by internal rule (\textsc{\footnotesize{comm-server-client}} we get $\update{\outserver}{i}{\gss}$ and $\update{E}{\inclient}{E.\inclient \cup \{\gs\}}$. Then, by internal rule (\textsc{\footnotesize{recieve}}), we get a term who has $|E_{i}.\inclient| > 0$. The rest of the prove is equivalent to above case.
						
\item {\bf rule (\textsc{t-process})} If $A \arroi{\tau} A'$, then by rule (\textsc{\footnotesize{t-process}}), $A$ must be the following term: $\tsystem{\tclient{\tprogram}{\tknown}{\tpending}{\ttransactionbuffer}{[\tsenthead] \cdot \tsent}{\treceivebuffer}\ \bigpar\ \tilde{A}}{\queuemessage_A}$, therefore, the client $i^{th}$ is only one who changes. So that, $A'$ will be $\tsystem{\tclient{\tprogram}{\tknown}{\tpending}{\ttransactionbuffer}{\tsent}{\treceivebuffer}\bigpar\ C}{\queuemessage \cdot \tsenthead}$. Property 9 states that for all $\beta$ such that $\beta \ \in \ \tsent_i$ then $\beta \ \in \ E_{i}.\outclient \ \cup \ \queuemessage_C.\inserver$. Analyzing cases:
					
					
					
					\begin{itemize}
						\item If $\beta \ \in\ E_{i}.\outclient$ then, by rule (\textsc{\footnotesize{comm-client-server}}), $C \arroi{\tau} C'$, so that, $C'$ = $\clientr{P}{\update{E}{\outclient}{\tailround}} \bigpar\ \tilde{C} \ \bigpar\ \ \server{\state}{\update{\inserver}{i}{\headerround}}{\outserver}_C$ and $\delta_0 \triangleleft \beta$. Performing a new internal action by rule (\textsc{\footnotesize{batch}}),i.e., $C' \arroi{\tau} C''$, then $C''$ = $\server{\apply{\persistedstate}{d}}{\inserver}{\notify{dom(\outserver)}{\outserver}{gs}}  \bigpar\ \tilde{C} \bigpar \clientr{P}{\stateclient}$. We have to prove that $(A',C'')$ belongs $\mathcal{R}$.
						
						\begin{enumerate}
							\item Theirs terms do not change.
							\item Our hypothesis is $\queuemessage_C.\state \triangleleft \queuemessage_A$ besides $\delta_0 \triangleleft \beta$, then using the property of the relation $\triangleleft$ between state and updates, we will get that $apply(\queuemessage_C.\state,\delta_0) \triangleleft \queuemessage_A \cdot \beta$.
							\item It does not change.
							\item It does not change.
							\item It does not change.
							\item It does not change.
							\item It does not change.
							\item $\queuemessage_A$  has incremented by one, so that, by Hypothesis, $\tknown + \treceivebuffer + 1 \leq |\queuemessage_A| < |\queuemessage_A| + 1$, besides, $\Leftrightarrow |E_{i}.\inclient| + |\queuemessage_C.\outserver| > 0$ because of neither $E_{i}.\inclient$ nor $\queuemessage_C.\outserver$ have changed.
\item By Hypothesis, for all element in $\tsent$, those belongs to $E_{i}.\outclient \ \cup \ \queuemessage_C.\inserver$, in particular, $\tsent$ without a element keeps these guarantee.
						\end{enumerate}
\item If $\beta \ \in\ E_{i}.\inserver$ then, applying rule (\textsc{\footnotesize{batch}}), we can see that we are at above case.
					\end{itemize}
					
\end{itemize}

\end{itemize}

We have prove that if $A$ perform an action, $C$ also can perform an action, the terms after both transitions are in $\mathcal{R}$. Now, we will prove the opposite side. When $C$ perform an action, then A can perform an action and the news terms are in $\mathcal{R}$.


\begin{itemize}
   \item{\bf n=0}. Then $C$ = $\clientr{E}{\initialstate}$ and $A$ = $\tsystem{\tclienti{0}{\emptyset}{\emptyset}{\emptyset}{\emptyset}{0}\ \bigpar\ \tilde{A}}{\queuemessage_A}$. It is analogous to the base case previously proved.
			
   \item{\bf n=k+1}. $\forall (C,A) \ \in \ \mathcal{R}$
	
			
			\begin{itemize}
				\item {\bf rule (\textsc{comm-server-client})}. If $C \arroi{\tau} C'$, then by rule (\textsc{\footnotesize{comm-server-client}}), $C'$ must be the following term: $\server{\state}{\inserver}{\update{\outserver}{i}{\gss}} \bigpar 
		\clientr{P}{\update{E}{\inclient}{E.\inclient \cup \{\gs\}}}$, therefore, the client $i^{th}$ at C, is the only one who changed. We will prove that $(C',A) \ \in \ \mathcal{R}$ if ten properties introduced are keeping.				
					\begin{enumerate}
						\item to 6. Their terms do not changed.
						\setcounter{enumi}{6}	
							\item $l'$ = $(|E_{i}.\inclient| + 1)$, so that, we have to prove that $E_{i}.\inclient[l'] \triangleleft \queuemessage_A[\tknown + \treceivebuffer - 1 + l']$, however, it easy to see because by property 10. when $m$ = 0 then $\outserver(i)[0] \triangleleft \queuemessage_A[\tknown + \treceivebuffer + |\inclient|]$.
						\item It is easy to see that $(|E_{i}.\inclient| + 1) + (|\queuemessage_C.\outserver|-1) > 0$.
						\item It does not change.
						\item It is easy to see, because, Property 10 is guaranteed with $0 \leq l$ therefore $1 \leq l$. 
					\end{enumerate}
					
					
				\item {\bf rule (\textsc{comm-client-server})}. If $C \arroi{\tau} C'$, then by rule (\textsc{\footnotesize{comm-client-server}}), $C'$ must be the following term: $\server{ps'}{\inserver}{\notify{dom(\outserver)}{\outserver}{gs}}$, therefore, the client $i^{th}$ is the only one who has changed. We will prove that $(C',A) \ \in \ \mathcal{R}$ if the ten properties introduced are keeping.				
					\begin{enumerate}
						\item to 8. Their terms do not changed.
						\setcounter{enumi}{8}	
						\item It is easy to see, because, we have exchanged a sequence from $\outclient$ to $\inserver$. 
						\item It does not change.
						\end{enumerate}
				


				
						\item {\bf rule (\textsc{batch})}. If $C \arroi{\tau} C'$, then by rule (\textsc{\footnotesize{batch}}), $C'$ must be the following term: $\server{\state}{\inserver}{\update{\outserver}{i}{\gss}} \bigpar 
		\clientr{P}{\update{E}{\inclient}{E.\inclient \cup \{\gs\}}}$, therefore, the client $i^{th}$ at C, is the only one who changed. Hence	$(C,A)$ belongs to $\mathcal{R}$ then $A$ after of internal transition $\tau$ is $\tsystem{\tclient{\tprogram}{\tknown}{\tpending}{\ttransactionbuffer}{\tsent}{\treceivebuffer}\bigpar\ S_C}{\queuemessage \cdot \tsenthead}$. We will prove that $(C',A') \ \in \ \mathcal{R}$ if the ten properties introduced are keeping.	
				\begin{enumerate}
						\item Their terms do not change.
						\item Property 9 states that for all $\beta$ such that $\beta \ \in \ \tsent_i$ then $\beta \ \in \ E_{i}.\outclient \ \cup \ \queuemessage_C.\inserver$. In particular, we consider $\beta \ \in \ \queuemessage_C.\inserver$, then there exists $\delta_o$ such that $\delta_0 \triangleleft \beta$, besides by Property 2., $\queuemessage_C.\state \triangleleft \queuemessage_A$, so that using the property of the relation $\triangleleft$ between state and updates, we will get that $apply(\queuemessage_C.\state,\delta_0) \triangleleft \queuemessage_A \cdot \beta$.
							\item It does not change.
							\item It does not change.
							\item It does not change.
							\item It does not change.
							\item It does not change.
							\item $\queuemessage_C.\outserver$ is updated with the new persisted state however its size does not change. By Hypothesis 9, $\tknown + \treceivebuffer + 1 \leq |\queuemessage_A| < |\queuemessage_A| + 1$.
							\item It does not change.
							\item It does not change.

						\end{enumerate}
		

				
			\end{itemize}
		\end{itemize}
		
\end{proof}


\section{Consistency Guarantees}
\label{sec:properties-gsp}

We shall introduce a series of store-level consistency guarantees and then we shall show which are captured by application written in GSP. We start identifying three kinds of relations between actions of update and read:

 \paragraph{Session Order} relates whatever pair of actions from the same client, indicating the program order. It is a total order on actions. 

 \paragraph{Visibility} relates Updates with Reads. It is used to indicate is if an action of Update is visible for an action of Read.

 \paragraph{Arbitration} relates Updates with Updates. It is used to resolve update conflicts. It is a total order on actions of update.




We extend the GSP language with a new term which capture the relations amount operation in our system. 
 
\[
    \begin{array}{l@{\quad}r@{\;::=\;}l}
			 (\textsc{environment}) & \environmentterm &  \environment{\systemterm}{\op}{\so}{\vis}{\arb} \\
	    \end{array}
  \]
	
Let $\environmentterm$, a new term, where $\systemterm$ represents our system introduced in Definition 1.1, $\op$ is a mapping of vertices to actions, $\so$ is a session order relation defined from vertices to relations of vertices $\verticesets$ $\times$ ($\verticesets$ $\times$ $\verticesets$) and $\vis$,$\arb$ are visibility and arbitration relation.


 \paragraph{Notation.} Given a session order relation $\so$ from client $i$ and a vertex $v$, we shall write $\soby{\so}{v}$  meaning that $\soby{(\mathcal{V}, \mathcal{R})}{v} = (\mathcal{V} \ \cup \ \{v\}, \mathcal{R}\ \cup \ \{(x,v) / x \in \verticesets\})$. We shall refer to an update action on the queue message as $\updateinqueuemessage{n}{i}$.The arbitration relation $\arb$ is defined as $\{ (v,w) / \{v \mapsto \updateinqueuemessage{m}{h}\} \in \op \land \ \{w \mapsto \updateinqueuemessage{n}{i} \} \in \op \land \ m < n \}$. A transition $\arroi{\alpha}$ denotes the fact that action $\alpha$ is perfomed by client $i$.

 
The following operational semantic allow to understand how working the environment when the actions are executed.

 \[
 \begin{array}{l}
		
\mathrule{e-read}{\systemterm \arroi{\readtran{r}} \systemterm' \qquad v \notin 
dom(\op) \qquad \ \vis' = \vis \ \cup \ \{ (x,v) / \{x \mapsto \updatebyclient{h}\} \in \op \ \land \  u^x \in \ \queuemessage[0..\tknown-1] \cdot \tpending \cdot [\ttransactionbuffer] \}}{\environment{\systemterm}{\op}{\so}{\vis}{\arb} \arroi{\readtran{r}} \environment{\systemterm'}{\op \ \cup \ \{v \mapsto \readtran{r}\} }{\soby{\so}{v}}{\vis'}{\arb}}

 \\[35pt]

		\mathrule{e-update}{\systemterm \arroi{\updatevtran{u}{v}} \systemterm' \qquad v \notin 
dom(\op)}{\environment{\systemterm}{\op}{\so}{\vis}{\arb} \arroi{\updatevtran{u}{v}} \environment{\systemterm'}{\op \ \cup \ \{v \mapsto \updatetran{u} \}}{\soby{\so}{v}}{\vis}{\arb}}
 
 \end{array}
 \]

\subsection{Ordering Guarantees}

We now prove what ordering guarantees are assured by GSP language and what do not. 


First, we prove a useful lemma: 
\begin{lemma}\label{lemma:update-ever-belong} Let u an update action, $\queuemessage$ a message queue, $\tpending_i$ and $\ttransactionbuffer_i$ a pending queue and transaction queue from the client i, if $\environment{\systemterm}{\emptyset}{\emptyset}{\emptyset}{\emptyset} \arro{} ^*\ \environment{\systemterm}{\op}{\so}{\vis}{\arb}$ then $\{x \mapsto \updatebyclient{i} \}\ \in \ \op\ \Rightarrow\ u^x \in\  \queuemessage[0..\tknown_i-1] \cdot \tpending_i \cdot [\ttransactionbuffer_i]$

\end{lemma}

\begin{proof} The proof follows by induction on the length of the derivation $\arro{} ^*$.
\begin{itemize}
   \item{\bf n=0}. Then $\op$ is $\emptyset$, so that antecedent is false, then the preoposition is true.
   \item{\bf n=k+1}. Then $\environment{\systemterm}{\emptyset}{\emptyset}{\emptyset}{\emptyset} \arro{} ^n\ \environment{\systemterm}{\op}{\so}{\vis}{\arb} \arroi{} \{x \mapsto \updatebyclient{i} \}\ \in \ \op\ \Rightarrow\ u^x \in\  \queuemessage[0..\tknown_i-1] \cdot \tpending_i \cdot [\ttransactionbuffer_i]$. We proceed by 
case analysis on the last transition:
	
	\begin{itemize}
        \item {\bf rule (\textsc{E-READ})}. As $x$ is an update operation then it must not be $v$, so that $\{x \mapsto \updatebyclient{i} \} \in\ \op$, then by inductive hypothesis $u^x \in\  \queuemessage[0..\tknown_i-1] \cdot \tpending_i \cdot [\ttransactionbuffer_i]$. When $\systemterm \arroi{\readtran{r}} \systemterm'$, $\tknown_i$, $\tpending_i$,$\ttransactionbuffer_i$ do not change.
				\item {\bf rule (\textsc{E-UPDATE})}. There are two possibilities:
				
				
					\begin{itemize}
						\item {\bf $x \neq v$}. Then we can use inductive hypothesis, so that it is easy to see that if $u^x \in\  \queuemessage[0..\tknown_i-1] \cdot \tpending_i \cdot [\ttransactionbuffer_i]$ then $u^x \in\  \queuemessage[0..\tknown_i-1] \cdot \tpending_i \cdot [\ttransactionbuffer_i] \cdot u_{t}^v$ too.
						\item {\bf $x =\ v$}. When $\systemterm \arroi{\updatevtran{u}{v}} \systemterm'$, $u^v \in \ttransactionbuffer'_i$ (with $\ttransactionbuffer'_i$ transaction queue in $\systemterm'$) because $\ttransactionbuffer'_i = \ttransactionbuffer_i \cdot u^v$. It is immediate to note that $u^x \in\  \queuemessage[0..\tknown'_i-1] \cdot \tpending'_i \cdot [\ttransactionbuffer'_i]$.
					\end{itemize}
				\item {\bf rule (\textsc{E-PUSH})}. As $\op$ do not change, then by inductive hypothesis, $u^x \in\  \queuemessage[0..\tknown_i-1] \cdot \tpending_i \cdot [\ttransactionbuffer_i] \equiv\ u^x \in\  \queuemessage[0..\tknown_i-1] \cdot (\tpending_i \cdot [\ttransactionbuffer_i]) \cdot \epsilon$. When $\systemterm \arroi{\pushtran} \systemterm'$, $\tknown_i$' = $\tknown_i$, $\tpending_i$' = $\tpending_i \cdot [\ttransactionbuffer_i]$ and $\ttransactionbuffer_i$' = $\epsilon$.
\item {\bf rule (\textsc{E-PULL})}. As $\op$ do not change, then by inductive hypothesis, $u^x \in\  \queuemessage[0..\tknown_i-1] \cdot \tpending_i \cdot [\ttransactionbuffer_i]$. We should prove that it is equivalent to $u^x \in\  \queuemessage[0..\tknown_i - 1 + \treceivebuffer_i] \cdot \tpending_i \setminus \queuemessage[\tknown_i .. \tknown_i + \treceivebuffer_i] \cdot [\ttransactionbuffer_i]$. There are two interesting cases to consider:
		\begin{itemize}
			\item If $u^x \in\ \tpending_i\  \land\ u^x \notin\ \queuemessage[\tknown_i .. \tknown_i + \treceivebuffer_i]$, then $u^x \in \ \tpending_i$'.
			\item If $u^x \in\ \tpending_i\  \land\ u^x \in\ \queuemessage[\tknown_i .. \tknown_i + \treceivebuffer_i]$, then $u^x \notin \ \tpending_i$' but $u^x \in\ \queuemessage[\tknown_i .. \tknown_i + \treceivebuffer_i]$. 
		\end{itemize}
\end{itemize}
\end{itemize}

The proof for the remaining cases is by inductive hypothesis because $\op$, $\tknown_i$, $\tpending_i$ and $\ttransactionbuffer_i$ do not change.

\end{proof}

\begin{theorem}[\textsc{Read My Writes}]\label{theorem:read-my-writes}

Let $\textsc{\small{SO}}_R$ the se\cond component of the relation $\textsc{\small{SO}}$ and $\textsc{\small{VIS}}$ a visibility relation, if $\environment{\systemterm_0}{\emptyset}{\emptyset}{\emptyset}{\emptyset} \arro{} ^*\ \environment{\systemterm}{\op}{\so}{\vis}{\arb}$ then $\textsc{\small{SO}}_R \cap \ (\mathbb{U}\ \times \ \mathbb{R})  \subseteq \ \textsc{\small{VIS}}$

\end{theorem}


\begin{proof} The proof follows by induction on the length of the derivation $\arro{} ^*$.
\begin{itemize}
   \item{\bf n=0}. In particular $\op$ and $\vis$ are $\emptyset$, so that $\emptyset \subseteq \emptyset$.
   \item{\bf n=k+1}. Then $\environment{\systemterm_0}{\emptyset}{\emptyset}{\emptyset}{\emptyset} \arro{} ^n\ \environment{\systemterm}{\op}{\so}{\vis}{\arb} \arroi{\alpha} \environment{\systemterm'}{\op'}{\so'}{\vis'}{\arb}$. We proceed by 
definition:
	
	\begin{itemize}
        \item $\textsc{\small{SO}}_R$' = $\textsc{\small{SO}}_R \cup\ \{(w,v) / \{w \mapsto \readtran{r}\} \lor\ \{w \mapsto \ \updatebyclient{j}\} \}$. Applying the intersection $(\mathbb{U}\ \times \ \mathbb{R})$, we shall obtain $\textsc{\small{SO}}_R \ \cup\ \{(w,v) / \{w \mapsto \updatebyclient{j}\} \}$.				
				\item $\vis' = \vis \ \cup \ \{ (x,v) / \{x \mapsto \updatebyclient{h}\} \in \op \ \land \  u^x \in \ \queuemessage[0..\tknown-1] \cdot \tpending \cdot [\ttransactionbuffer] \}$.
\end{itemize}
By inductive hypothesis, $\textsc{\small{SO}}_R \ \cup\ (\mathbb{U}\ \times \ \mathbb{R}) \subseteq \vis$. We only have to prove that $\{(w,v) / \{w \mapsto \updatebyclient{j}\} \in \op  \ \} \subseteq \{ (x,v) / \{x \mapsto \updatebyclient{h}\} \in \op \ \land \  u^x \in \ \queuemessage[0..\tknown-1] \cdot \tpending \cdot [\ttransactionbuffer] \}$. When $j = h$ and $w = x$, we can use Lemma~\ref{lemma:update-ever-belong}. So that, we have proved that $\textsc{\small{SO}}_R$' $\cap \ (\mathbb{U}\ \times \ \mathbb{R})  \subseteq \ \textsc{\small{VIS'}}$.
\end{itemize}
\end{proof}



\begin{theorem}[\textsc{Monotonic Read}]
\label{theorem:monotonic-read}
Let $\textsc{\small{SO}}_R$ the se\cond component of the relation $\textsc{\small{SO}}$ and $\textsc{\small{VIS}}$ a visibility relation, if $\environment{\systemterm_0}{\emptyset}{\emptyset}{\emptyset}{\emptyset} \arro{} ^*\ \environment{\systemterm}{\op}{\so}{\vis}{\arb}$ then $(\textsc{\small{VIS}};\textsc{\small{SO}}_R) \cap \ (\mathbb{U}\ \times \ \mathbb{R})  \subseteq \ \textsc{\small{VIS}}$

\end{theorem}


\begin{proof} The proof follows by induction on the length of the derivation $\arro{} ^*$.
\begin{itemize}
   \item{\bf n=0}. In particular $\textsc{\small{SO}}_R$ and $\textsc{\small{VIS}}$ are $\emptyset$, so that $\emptyset \subseteq \emptyset$.
   \item{\bf n=k+1}. Then $\environment{\systemterm_0}{\emptyset}{\emptyset}{\emptyset}{\emptyset} \arro{} ^n\ \environment{\systemterm}{\op}{\so}{\vis}{\arb} \arroi{\alpha} \environment{\systemterm'}{\op'}{\so'}{\vis'}{\arb}$. Let $R$ be a composition of relations. We shall say that if $(x,y) \in \ R $ iff $\ \exists y \in\ \verticesets\ $ such that $(x,y) \in\ \textsc{\small{VIS}}' \land \ (y,z) \in\  \textsc{\small{SO}}_R$'. We have to prove that $(x,z) \in \ \textsc{\small{VIS}}$'. We proceed by 
case analysis on the last transition:
	
	\begin{itemize}
        \item {\bf rule (\textsc{E-READ})}. We know that $v$ is fresh, therefore, $v$ have not be neither $x$ nor $y$ because there exists $z$ such that, $z$ happens after from $x$ and $y$. There are only two possibilities:
					
					\begin{itemize}
						\item $z = v$. As $(y,z) \in \ \textsc{\small{SO}}_R$, then $y$ and $z$ are from the same client, called $i$. We are only interested in relations of $update \times\ read$. We know that $v$ is associated to an read action besides $x$ have to be an update action. As $(x,y) \in \ \textsc{\small{VIS}}$' then $y$ is an read action,i.e., $\{y \mapsto \readbyclient{i} \}\ \in \ \op$.By Lemma xx, there exists an update $u$ such that $u^x \in\ \ \queuemessage[0..\tknown_i-1] \cdot \tpending_i \cdot [\ttransactionbuffer_i]$. Substituting $w$ by $v$ in $\textsc{\small{VIS}}$', we prove that $(x,z) \in \textsc{\small{VIS}}$'.
						item $z \neq v$. This case follows immediately by inductive hypothesis.
					\end{itemize}
					
			  \item{\bf rule (\textsc{E-UPDATE})}. Visibility relation does not change,i.e.,$\textsc{\small{VIS}}$ = $\textsc{\small{VIS}}$'. Let $v$ be a vertex associated an update action, then $\textsc{\small{SO}}_R$' = $\textsc{\small{SO}}_R \cup\ \{(w,v) / \{w \mapsto \readtran{r}\} \lor\ \{w \mapsto \ \updatebyclient{j}\} \}$. As we are only interested in relations of $update \times\ read$, then $(\textsc{\small{VIS}}';\textsc{\small{SO}'}_R) \cap \ (\mathbb{U}\ \times \ \mathbb{R}) \equiv \ (\textsc{\small{VIS}};\textsc{\small{SO}}_R) \cap \ (\mathbb{U}\ \times \ \mathbb{R})$. By inductive hypothesis we can prove that $(\textsc{\small{VIS}}';\textsc{\small{SO}}_R)' \cap \ (\mathbb{U}\ \times \ \mathbb{R}) \ \subseteq \ \textsc{\small{VIS}} \subseteq \ \textsc{\small{VIS}}$'.				
				The proof for the remaining cases follow are not interesting because the relations does not change.
\end{itemize}
\end{itemize}
\end{proof}



\begin{theorem}[\textsc{No Circular Causality}]

Let $\textsc{\small{SO}}_R$ the se\cond component of the relation $\textsc{\small{SO}}$ and $\textsc{\small{VIS}}$ a visibility relation, if $\environment{\systemterm_0}{\emptyset}{\emptyset}{\emptyset}{\emptyset} \arro{} ^*\ \environment{\systemterm}{\op}{\so}{\vis}{\arb}$ then $\textbf{acyclic}(\textsc{\small{SO}}_R \ \cup \ \textsc{\small{VIS}} ) ^+$.

\end{theorem}

\begin{proof}

Since $\textsc{\small{VIS}}$ is acyclic and $\textsc{\small{SO}}_R \cap \ (\mathbb{U}\ \times \ \mathbb{R})  \subseteq \ \textsc{\small{VIS}}$ by Theorem~\ref{lemma:update-ever-belong}, we have to prove that $\textsc{\small{VIS}} ^+$ is acyclic. In particular, the transitive closure of an acyclic graph is the reachability relation of the directed acyclic graph and a strict partial order.

\end{proof}





\begin{theorem}[\textsc{Causal Visibility}]

Let $\textsc{\small{SO}}_R$ the se\cond component of the relation $\textsc{\small{SO}}$ and $\textsc{\small{VIS}}$ a visibility relation, $(\textsc{\small{SO}}_R \ \cup \ \textsc{\small{VIS}} ) ^{+} \cap \ (\mathbb{U}\ \times \ \mathbb{R}) \subseteq \textsc{\small{VIS}}.$ 

\end{theorem}
 

\begin{proof} The proof follows by induction on the number of union sets between $\textsc{\small{SO}}_R$ and $\textsc{\small{VIS}}$. Then, $\bigcup_{n=1}^{\infty} (\textsc{\small{SO}}_R \ \cup \ \textsc{\small{VIS}} ) ^{n} \cap \ (\mathbb{U}\ \times \ \mathbb{R}) \subseteq \textsc{\small{VIS}}.$
\begin{itemize}
   \item{\bf n=0}. This means that $\textsc{\small{SO}}_R$ and $\textsc{\small{VIS}}$ are $\emptyset$, so that $\emptyset \subseteq \emptyset$.
   \item{\bf n=k+1}. Suppose that we have proved that the number of union sets $< k+1$. Now, we have to prove that: $\forall (a,b) \mid (a,b \in \verticesets \Rightarrow\ (a,b)\ \in\ (\textsc{\small{SO}}_R \ \cup \ \textsc{\small{VIS}} ) ^{k+1} \cap \ (\mathbb{U}\ \times \ \mathbb{R})) \Rightarrow\ (a,b) \in \textsc{\small{VIS}}$.
	
Assume $(a, x_1),(x_1, x_2),\ldots(x_{k{-}1}, x_k),(x_{k}, b)$ are relations from $(\textsc{\small{SO}}_R \ \cup \ \textsc{\small{VIS}} ) ^{k+1}$. Then $(a, x_1),(x_1, x_2),\ldots(x_{k{-}1}, x_k)$ are relations from $(\textsc{\small{SO}}_R \ \cup \ \textsc{\small{VIS}} ) ^{k}$. By the induction hypothesis, $(a, x_k) \in\ (\textsc{\small{SO}}_R \ \cup \ \textsc{\small{VIS}} ) ^{k}$, and we also have $(x_k, b) \in\ (\textsc{\small{SO}}_R \ \cup \ \textsc{\small{VIS}} )$. Thus by the definition of $(\textsc{\small{SO}}_R \ \cup \ \textsc{\small{VIS}} ) ^{k+1}$, $(a, b) \in
(\textsc{\small{SO}}_R \ \cup \ \textsc{\small{VIS}} ) ^{k+1}$.
Conversely, assume $(a,b) \in\ (\textsc{\small{SO}}_R \ \cup \ \textsc{\small{VIS}} ) ^{k+1}$ = $(\textsc{\small{SO}}_R \ \cup \ \textsc{\small{VIS}} ) ^{k} \circ (\textsc{\small{SO}}_R \ \cup \ \textsc{\small{VIS}})$. Then there is a vertex $c \in \verticesets$ such
that $(a,c) \in\ (\textsc{\small{SO}}_R \ \cup \ \textsc{\small{VIS}} ) ^{k}$ and $(c, b) \in\ (\textsc{\small{SO}}_R \ \cup \ \textsc{\small{VIS}} )$.

We are only interested when $a$ is an update action and $b$ a read action. It is because of the intersection with $(Update \times Read)$. We have two possible cases:

\begin{itemize}
	\item c is a read action. It means that $(c, b)$ only can be in $\textsc{\small{SO}}_R$ because $\textsc{\small{VIS}}$ requires that $c$ will be an update action. In particular, if  $(c, b) \in\ \textsc{\small{SO}}_R$, they belong to the same client. 
	
	\begin{itemize}
		\item if $(a,c) \in\ \textsc{\small{VIS}}$ and $(c,b) \in\ \textsc{\small{SO}}_R$, by Theorem~\ref{theorem:monotonic-read}, $(a,b) \in\ \textsc{\small{VIS}}$.
		
		\item if $(a,c) \in\ \textsc{\small{SO}}_R$ then by Theorem~\ref{theorem:read-my-writes}, $(a,c) \in\ \textsc{\small{VIS}}$- Then, it is analogous to the previous case.
		
	\end{itemize}
	\item c is an update action. It means that $(a, c)$ only can be in $\textsc{\small{SO}}_R$ because $\textsc{\small{VIS}}$ requires that $c$ will be an read action. In particular, if  $(a,c) \in\ \textsc{\small{SO}}_R$, they belong to the same client.
	
	\begin{itemize}
		\item if $(a,c) \in\ \textsc{\small{SO}}_R$ and $(c,b) \in \textsc{\small{VIS}}$, by Theorem Monotonic Writes (Falta probar!), $(a,b) \in\ \textsc{\small{VIS}}$.
		\item if $(a,c) \in\ \textsc{\small{VIS}}$ ... VER.
	\end{itemize}
\end{itemize}


	
	
\end{itemize}
\end{proof}


\begin{theorem}[\textsc{Causal Arbitration}]

Let $\textsc{\small{SO}}_R$ the se\cond component of the relation $\textsc{\small{SO}}$, $\textsc{\small{VIS}}$ a visibility relation and $\textsc{\small{AR}}$ an arbitration relation then $(\mathbb{U}\ \times \ \mathbb{U}) \cap \ (\textsc{\small{SO}}_R \ \cup \ \textsc{\small{VIS}} ) ^{+} - \textsc{\small{SO}}_R \subseteq \textsc{\small{AR}}.$ 

\end{theorem}
 

\begin{proof} 
The proof follows by induction on the number of union sets between $\textsc{\small{SO}}_R$ and $\textsc{\small{VIS}}$. Then, $(\mathbb{U}\ \times \ \mathbb{U}) \cap \ \bigcup_{n=1}^{\infty} (\textsc{\small{SO}}_R \ \cup \ \textsc{\small{VIS}} ) ^{n} - \textsc{\small{SO}}_R \subseteq \textsc{\small{AR}}.$


\begin{itemize}
   \item{\bf n=0}. This means that $\textsc{\small{SO}}_R$, $\textsc{\small{VIS}}$ and  $\textsc{\small{AR}}$ are $\emptyset$, so that $\emptyset \subseteq \emptyset$.
   \item{\bf n=k+1}. Suppose that we have proved that the number of union sets $< k+1$. Now, we have to prove that: $\forall (a,b) \mid (a,b \in \verticesets \Rightarrow\ (a,b)\ \in\ (\mathbb{U}\ \times \ \mathbb{U}) \cap \ \bigcup_{n=1}^{\infty} (\textsc{\small{SO}}_R \ \cup \ \textsc{\small{VIS}} ) ^{k+1} - \textsc{\small{SO}}_R \Rightarrow\ (a,b) \in \textsc{\small{AR}}$. 

Assume $(a,b) \in\ (\textsc{\small{SO}}_R \ \cup \ \textsc{\small{VIS}} ) ^{k+1} $ = $(\textsc{\small{SO}}_R \ \cup \ \textsc{\small{VIS}} ) ^{k} \circ (\textsc{\small{SO}}_R \ \cup \ \textsc{\small{VIS}})$. Then there is a vertex $c \in \verticesets$ such
that $(a,c) \in\ (\textsc{\small{SO}}_R \ \cup \ \textsc{\small{VIS}} ) ^{k}$ and $(c, b) \in\ (\textsc{\small{SO}}_R \ \cup \ \textsc{\small{VIS}} )$.

We are only interested when $a$ and $b$ are an update actions. It is because of the intersection with $(Update \times Update)$. We have two possible cases:
\begin{itemize}
	\item c is a read action. It means that $(c,b)$ only can be in $\textsc{\small{SO}}_R$ because $\textsc{\small{VIS}}$ requires that $c$ will be an update action. In particular, if  $(c,b) \in\ \textsc{\small{SO}}_R$, they belong to the same client. COMPLETAR
	
	\item c is an update action. COMPLETAR
\end{itemize}
\end{itemize}

\end{proof}


\begin{example}[Consistent Prefix] 
\label{consistent-prefix}Consider the following system built-up from two 
different client:

\[
\begin{array}{l}
E = \environment{\tclient{\textit{update(aList.add('a'));push();}}{0}{\epsilon}{\epsilon}{\epsilon}{0}[1]\ \bigpar\
\\
\tclient{\textit{update(aList.add('b'));push();let v = read(x);}}{0}{\epsilon}{\epsilon}{\epsilon}{0}[2]  \bigpar\ \epsilon}{\emptyset}{\emptyset}{\emptyset}{\emptyset}
\end{array}
\]
\end{example}
		
Environment has a system $\systemterm$ with two clients and a message queue $S$ without updates. Relation sets $\op$,$\so$,$\vis$,$\arb$ are empty, i.e.,there were not an execution in $\systemterm$ captured by the relation sets. The update action add a string to an object called $aList$ which has not elements. 
In this state, $C_1$ and $C_2$ can perform their update actions. E may non-deterministically choose to do $update(aList.add('a')$ or $update(aList.add('b')$. If the first communication takes place over $update(aList.add('a')$, then the system evolves as follows:

\[
\begin{array}{l}
E \arrobyclient{\updatevtran{aList.add('a')}{{v_0}}}{1} \environment{\tclient{push();}{0}{\epsilon}{[aList.add('a')]}{\epsilon}{0}[1]\ \bigpar\
\\
\tclient{\textit{update(aList.add('b'));push();let x = read(aList);}}{0}{\epsilon}{\epsilon}{\epsilon}{0}[2]  \bigpar\ \epsilon} 
{\\ \emptyset}{\{( \{ v_0 \} , \emptyset ) \}}{\emptyset}{\emptyset} 
\end{array}
\]

The action $\updatetran{aList.add('a')}$ of the client $1$ will be identified by vertex $v_0$ by rule $\textsc{\small{E-UPDATE}}$. In particular, the update action will be left into the transactional queue of Client $1$ besides the new vertex will be added to vertices in $\so$. Now, Client 1 performs $\pushtran$: 

\[
\begin{array}{l}
E \arrobyclient{\pushtran}{1} \environment{\tclient{0}{0}{[aList.add('a')]}{\epsilon}{[aList.add('a')]}{0}[1]\ \bigpar\
\\
\tclient{\textit{update(aList.add('b'));push();let x = read(aList);}}{0}{\epsilon}{\epsilon}{\epsilon}{0}[2]  \bigpar\ \epsilon} 
{\\ \emptyset}{\{( \{ v_0 \} , \emptyset ) \}}{\emptyset}{\emptyset} 
\end{array}
\]
		
When it happens, $aList.add('a')$ is moved to the sent queue and pending queue. At this moment, Client 2 can perform an update action however Client 1 will realize an internal action which is given by $\textsc{\small{E-PROCESS}}$.  			

\[
\begin{array}{l}
E \arrobyclient{\tau}{1} \environment{\tclient{0}{0}{[aList.add('a')]}{\epsilon}{\epsilon}{0}[1]\ \bigpar\ \\
\tclient{\textit{update(aList.add('b'));push();let x = read(aList);}}{0}{\epsilon}{\epsilon}{\epsilon}{0}[2]  \bigpar\ \\ 
aList.add('a')} 
{\emptyset}{\{( \{ v_0 \} , \emptyset ) \}}{\emptyset}{\emptyset} 
\end{array}
\]

Analogously, Client 2 realize its actions leaving to environment evolves as below:

\[
\begin{array}{l}
E \arrobyclient{\updatetran{aList.add('b')}}{2} \environment{\tclient{0}{0}{[aList.add('a')]}{\epsilon}{\epsilon}{0}[1]\ \bigpar\
\\
\tclient{\textit{push();let x = read(aList);}}{0}{\epsilon}{[aList.add('b')]}{\epsilon}{0}[2]  \bigpar\ 
\\ aList.add('a')} {\emptyset}{\{( \{ v_0,v_1 \} , \{\ (v_0,v_1) \} ) \}}{\emptyset}{\emptyset} 
\\
\\
\arrobyclient{\pushtran}{2} \environment{\tclient{0}{0}{[aList.add('a')]}{\epsilon}{\epsilon}{0}[1]\ \bigpar\
\\
\tclient{\textit{let x = read(aList);}}{0}{[aList.add('b')]}{\epsilon}{[aList.add('b')]}{0}[2]  \bigpar\ 
\\ aList.add('a')}{\emptyset}{\{( \{ v_0,v_1 \} , \{\ (v_0,v_1) \} ) \}}{\emptyset}{\emptyset} 
\\
\\
\arrobyclient{\tau}{2} \environment{\tclient{0}{0}{[aList.add('a')]}{\epsilon}{\epsilon}{0}[1]\ \bigpar\
\\
\tclient{\textit{let x = read(aList);}}{0}{[aList.add('b')]}{\epsilon}{\epsilon}{0}[2]  \bigpar\ 
\\aList.add('a') \cdot [aList.add('b')]} {\emptyset}{\{( \{ v_0,v_1 \} , \{\ (v_0,v_1) \} ) \}}{\emptyset}{\{ (v_0,v_1) \}} 
\\
\\
\arrobyclient{\readtran{aList}}{2} \environment{\tclient{0}{0}{[aList.add('a')]}{\epsilon}{\epsilon}{0}[1]\ \bigpar\
\\
\tclient{\update{0}{x}{['b']}}{0}{[aList.add('b')]}{\epsilon}{\epsilon}{0}[2]  \bigpar\ aList.add('a') \cdot [aList.add('b')]} 
\\
{\emptyset}{\emptyset}{\emptyset}{\emptyset} 
\end{array}
\]
		
Note that $\arb$ is modified because the message queue has two elements. The value returned will be the list with a only element, ['b'], because of the internal action associated to the rule $\textsc{\small{e-receive}}$ did not perform it. Then, if the rule $\textsc{\small{e-receive}}$ were performed, the value of the list would be ['a','b']. It is because the content of the pending queue is removed when the server left their message. 

The guarantee Consistent Prefix, which rules is $(\textsc{\small{ar}};\textsc{\small{vis}}) \subseteq \ \textsc{\small{ar}};\textsc{\small{VIS}}$, states that if we see an result from a client in a particular order, we will never see this result in a different order.


 % !TEX root = main.tex

\section{{\gsp} with atomic updates}
\label{sec:transactions}

In this section we study the extension of the \gsp\ model with atomic updates proposed in~\ref{}. 
The language of programs is extended in the following way: 
\[
 \rName{program} 
			 \qquad 
			 \tprogram \ ::=\  \ldots\ |\ \tsyncupdins 
\]

The intended meaning of the program  $\tsyncupdins$ is that it remains blocked until the 
update $\anupd$ is performed over the global store.  This is achieved by continuously pulling (i.e., 
a busy-waiting) until the updates are confirmed by the server. 

To formally define the semantics of atomic updates we consider the following 
runtime syntax.
\[
 \rName{program} 
			 \qquad 
			 \tprogram \ ::=\  \ldots\ |\  \ttrans\ |\ \waitcmd \ |\ \tguarded{} 
			 \]

We rely  on the following additional labels
			 
\[  \actbyc ::= \ldots \ |\ (\tupdlbl,\cid)\  |\ (\finishsynctran,\cid)
\]


We define the operator $[\_]$ over actions  $\lambda$ s.t. $[\updatetran{\anupd}] =\tupdlbl$ and it is the identity
over any other action.

\[
 \begin{array}{l}

\mathax{start-upd}
	{
	\begin{array}{l}
	\tsystem{\tclienti{\tsyncupdins}{\tknown}{\tpending}{\ttransactionbuffer}{\tsent}{\treceivebuffer}} 
%	\arroi{\syncupdtran{\tupdate}}
	\arroi{\tau}
	\hspace{6cm}\\\hfill
	\tsystem{\tclienti{\ttrans[\tupdins;\pushcmd;\waitcmd]}{\tknown}{\tpending}{\ttransactionbuffer}{\tsent}{\treceivebuffer}}
	\end{array}
	 }
 
 \\[15pt]

\mathrule{trans}
	{\tsystem{\tclienti{\tprogram}{\tknown}{\tpending}{\ttransactionbuffer}{\tsent}{\treceivebuffer}} 
	\arroi{\mu} 
	\tsystem{\tclienti{\tprogram'}{\tknown'}{\tpending'}{\ttransactionbuffer'}{\tsent'}{\treceivebuffer'}}}
	{\tsystem{\tclienti{[\tprogram];Q}{\tknown}{\tpending}{\ttransactionbuffer}{\tsent}{\treceivebuffer}}
	 \arroi{[\mu]} 
	 \tsystem{\tclienti{[\tprogram'];Q}{\tknown'}{\tpending'}{\ttransactionbuffer'}{\tsent'}{\treceivebuffer'}}}

 \\[25pt]

\mathax{end-upd}
	{\tsystem{\tclienti{[0];\tprogram}{\tknown}{\tpending}{\ttransactionbuffer}{\tsent}{\treceivebuffer}} 
	\arroi{\finishsynctran{}} 
	\tsystem{\tclienti{\tprogram}{\tknown}{\tpending}{\ttransactionbuffer}{\tsent}{\treceivebuffer}}}

\\[25pt]
\mathax{wait}
	{
	\begin{array}{l}
	\tsystem{\tclienti{\waitcmd}{\tknown}{\tpending}{\ttransactionbuffer}{\tsent}{\treceivebuffer}}
	\arro{\tau} 
	\\
	\hspace{2.2cm}
	\tsystem{\tclienti{\tconfirmedins{x}[{\tguarded[x][\pullcmd;\waitcmd][0]}]}{\tknown}{\tpending}{\ttransactionbuffer}{\tsent}{\treceivebuffer}}
	\end{array}
	}

\\[25pt]
\mathrule{guard-true}
	{\eval \cond {true}}
	{
	\tsystem{\tclienti{\tguarded[e][P][Q]}{\tknown}{\tpending}{\ttransactionbuffer}{\tsent}{\treceivebuffer}}
	\arro{\tau} 
	\tsystem{\tclienti{Q}{\tknown}{\tpending}{\ttransactionbuffer}{\tsent}{\treceivebuffer}}
	}

\\[25pt]
\mathrule{guard-false}
	{\eval \cond {false}}
	{
	\tsystem{\tclienti{\tguarded[e][P][Q]}{\tknown}{\tpending}{\ttransactionbuffer}{\tsent}{\treceivebuffer}}
	\arro{\tau} 
	\tsystem{\tclienti{P}{\tknown}{\tpending}{\ttransactionbuffer}{\tsent}{\treceivebuffer}}
	}
%	\\
%\henote{\mbox{Agregar reglas para if}}

%
%
% 
%\mathax{t-flush}{\tsystem{\tclienti{[\tflush];\tprogram}{\tknown}{\tpending}{\ttransactionbuffer}{\tsent}{\treceivebuffer}\ \bigpar\ C}{\queuemessage} \arroi{\tau} \tsystem{\tclienti{[\pushtran;\pwhile{!\confirmedtran()}{\pulltran};];\tprogram}{\tknown}{\tpending}{\ttransactionbuffer}{\tsent}{\treceivebuffer}\ \bigpar\ C}{\queuemessage}}
%\\
%\chnote{\mbox{En realidad el programa es: $\pushtran;let x = \confirmedtran();\pwhile{!x}{\pulltran};let x = \confirmedtran();$}}
%
% \\[15pt]

 

 \end{array}
 \]
 
%The nature of such cases is caused by having no transactions as the usual databases.

%The remaining of this section is proposing a environment that ensures these consistency guarantees. We start by refining the environment in GSP. We will distinguish a new relation:

%\[ 
%\begin{array}{r@{\ ::= \ }l}
%  \zeta & \mu \ | \ \startsynctran \ | \ \finishsynctran  \ | \  [\mu] \\
%\end{array}
%\]






%\[
% \begin{array}{l} \hspace{-.3cm} \textsc{PROGRAM}\\
%
%
%\mathax{t-sync-update}{\tsystem{\tclienti{\tsyncupd{u};\tprogram}{\tknown}{\tpending}{\ttransactionbuffer}{\tsent}{\treceivebuffer}\ \bigpar\ C}{\queuemessage} \arroi{\syncupdtran{\tupdate}} \tsystem{\tclienti{[\tupdins;\tflush];\tprogram}{\tknown}{\tpending}{\ttransactionbuffer}{\tsent}{\treceivebuffer}\ \bigpar\ C}{\queuemessage}}
% 
% \\[15pt]
%
%\mathrule{t-tran}{\tsystem{\tclienti{\tprogram}{\tknown}{\tpending}{\ttransactionbuffer}{\tsent}{\treceivebuffer}\ \bigpar\ C}{\queuemessage} \arroi{\mu} \tsystem{\tclienti{\tprogram'}{\tknown'}{\tpending'}{\ttransactionbuffer'}{\tsent'}{\treceivebuffer'}\ \bigpar\ C}{\queuemessage}}{\tsystem{\tclienti{[\tprogram];Q}{\tknown}{\tpending}{\ttransactionbuffer}{\tsent}{\treceivebuffer}\ \bigpar\ C}{\queuemessage} \arroi{[\mu]} \tsystem{\tclienti{[\tprogram'];Q}{\tknown'}{\tpending'}{\ttransactionbuffer'}{\tsent'}{\treceivebuffer'}\ \bigpar\ C}{\queuemessage}}
%
% \\[25pt]
%
% 
%\mathax{t-flush}{\tsystem{\tclienti{[\tflush];\tprogram}{\tknown}{\tpending}{\ttransactionbuffer}{\tsent}{\treceivebuffer}\ \bigpar\ C}{\queuemessage} \arroi{\tau} \tsystem{\tclienti{[\pushtran;\pwhile{!\confirmedtran()}{\pulltran};];\tprogram}{\tknown}{\tpending}{\ttransactionbuffer}{\tsent}{\treceivebuffer}\ \bigpar\ C}{\queuemessage}}
% 
%\\
%\chnote{\mbox{En realidad el programa es: $\pushtran;let x = \confirmedtran();\pwhile{!x}{\pulltran};let x = \confirmedtran();$}}
%
% \\[15pt]
%
%\mathax{t-end-sync}{\tsystem{\tclienti{[0];\tprogram}{\tknown}{\tpending}{\ttransactionbuffer}{\tsent}{\treceivebuffer}\ \bigpar\ C}{\queuemessage} \arroi{\finishsynctran(u^{[v]})} \tsystem{\tclienti{\tprogram}{\tknown}{\tpending}{\ttransactionbuffer}{\tsent}{\treceivebuffer}\ \bigpar\ C}{\queuemessage}}
% 
%
% \end{array}
% \]
% 


\subsection{Operational Semantic}

%\subsubsection{Notation}

%\begin{flushleft}
%\specfunction{$\downharpoonright$}{$\verticesets$}{$2^\verticesets$}{$2^\verticesets$} \\
%$\downharpoonright v^i \emptyset$ =  $\emptyset$ \\
%$\downharpoonright v^i (x^h \mapsto V)$ = x^h \mapsto V \cup \ v^i;  $\downharpoonright v^i V$ \\
%\end{flushleft}

%
%We assume the following countable sets of transactional vertices names $[\verticesets]$ ranged over by $[\vertice];[\vertice_0];[\vertice_1],\ldots$;
%
%The labeled transition system for the extension of GSP considers the following actions $\zeta$:	
%
%
%
%
%These labels allow system to perform an update synchronous. Label $\startsynctran$ stands for the begining of a transaction, and $\finishsynctran$ the end of it.

Since update operations are not instantaneous, now we introduce to the model the relations that take into account the 
beginning and finalisation of each write operation. In particular,

\begin{itemize}
   \item{\em Return Before} (\rb), which indicates the ordering of non-overlapping update operations. 
\end{itemize}

The extended environment will also contain two additional terms that are instrumental to the 
computation of $\rb$. 

\begin{itemize}
	\item $\tx$ that relates a vertex to set of vertices which have just not finished. Formally,  
	$\tx :\verticesets\ \mapsto 2^{\verticesets}$, i.e., it is a function from vertices to set of vertices.
	\item $\tc$ that denotes the set of closed transactions, i.e., $\tc\subseteq \verticesets$.
\end{itemize}

\[
    \begin{array}{l@{\quad}r@{\;::=\;}l}
			 (\textsc{environment}) & \environmentterm &  \environmenttran{\systemterm}{\op}{\so}{\vis}{\arb}{\rb}{\tx}{\tc} \\
			 (\textsc{transactions with overlapping}) & \tx &  \emptyset  \ | \ \verticesets\ \mapsto 2^{\vertice},\tx   \\
			 (\textsc{transactions closed}) & \tc &  \epsilon  \ | \ \vertice\ \cdot \tc 
	   \end{array}
\]


\paragraph{Notation} $\updopentx{\epsilon} = \epsilon$, and 
   $\updopentx{\vertice_0 \mapsto \{\instanceset\};tail} = \vertice_0 \mapsto \{\instanceset \cup \{\vertice\}\};\updopentx{tail}$.
		


		
 \[
 \begin{array}{l}  
\mathrule
	{e-start-sync}
	{\systemterm \arroi{\syncupdtran{\tupdate}} \systemterm' 
		\qquad [\vertice] \notin 
dom(\tx)}{\environmenttran{\systemterm}{\op}{\so}{\vis}{\arb}{\rb}{\tx}{\tc} \arroi{\startsynctran{\vertice}} \environmenttran{\systemterm'}{\op}{\so}{\vis}{\arb}{\rb}{\updopentx{\tx};\vertice \mapsto dom(\tc)}{\tc}}

 \\[35pt]
{
\mathrule{e-end-sync}{\systemterm \arroi{\finishsynctran{\tupdate}} \systemterm' \qquad [\vertice] \in 
dom(\tx)}{\environmenttran{\systemterm}{\op}{\so}{\vis}{\arb}{\rb}{\vertice \mapsto \instanceset,\tx}{\tc} \arroi{\finishsynctran{\vertice}} \environmenttran{\systemterm'}{\op}{\so}{\vis}{\arb}{\rb'}{\tx}{\tc \cdot\ \vertice}}}
\\
 \\[10pt]where \ 
\rb' = \rb \cup\ \{ (x,v) \ | \ \forall 0 \leq m,n < |\queuemessage|, \op(x) = \queuemessage[m] \ \land \ \op(v) = \queuemessage[n] \ \land \ m \leq\ n \ \land \ x\notin\tx(v) \}
 \end{array}
 \]




First, we prove a useful lemma: 

\begin{lemma}\label{lemma:empty_queue} 

$\forall C$ such that 
$\tclienti{Q}{\tknown}{\tpending}{\ttransactionbuffer}{\tsent}{\treceivebuffer} \ 
    \Arro^* \ \tclienti{\tsyncupd{u};\treadins{x}{r}}{\tknown}{\tpending}{\ttransactionbuffer}{\tsent}{\treceivebuffer} 
	  \Arro^* \ \tclienti{\treadins{x}{r}}{\tknown}{\tpending}{\ttransactionbuffer}{\tsent}{\treceivebuffer}$, then  
rvalue operation always reads above the store.

\end{lemma}

\begin{proof} Confirmed is evaluated to true when $\tpending$ and $\ttransactionbuffer$ are empty so that, after the loop, the read operation only return values which belong to the Store.
\end{proof}	


Single Order is a consistency guarantee that express a single order of operations observed by arbitration and visibility relation. The definition introduced by \cite{} requires 
$\ovis = \roarb$, where $F$ stands for the set of finished operations, i.e., when consider visibility restricted to the operations that belong to $\tc$. 
Since we defined $\arb$ and $\vis$ over restricted domains, the original formulation is not useful. Hence we will use an alternative characterisation, given by the 
following result.

\begin{lemma} $\ovis = \roarb$ iff 
\begin{enumerate}
   \item $\roarb;\ovis \subseteq \ovis$; and
   \item $\neg\ovis;\neg\roarb \subseteq \neg\ovis$.
\end{enumerate}
\end{lemma}

\begin{proof} $\Rightarrow$) 
\begin{enumerate}
\item
	\[ \begin{array}{l@{\ =\ }l@{\qquad}l}
		\roarb;\ovis &  \roarb;\roarb & \roarb =\ovis\\
		& \roarb & \roarb \mbox{is a partial order (i.e., transitive)}\\
		&  \ovis & \roarb =\ovis
   	\end{array}
	\]
\item Note that  $\roarb;\ovis =\ovis$ implies $\neg(\roarb;\ovis) =\neg\ovis$. By property of complement and composition, $\neg(\roarb;\ovis) =\neg\ovis;\neg\roarb =\neg\ovis$. 
\end{enumerate}

$\Leftarrow)$ We divided the proof in two implications
 \begin{enumerate}
    \item  $\roarb;\ovis \subseteq \ovis$ implies $\ovis\subseteq \roarb$. We proceed by contradiction. Assume $(a,b)\in\ovis$ but $(a,b)\not\in\roarb$.
    Since $\roarb$ is a total order, $(b,a)\in\roarb$. By assumption, $(b,a)\in\roarb$ and $(a,b)\in\ovis$ implies $(b,b)\in\ovis$, but this is in contradiction 
    with the fact that $\ovis$ is acyclic. 
        
    \item $\neg\ovis;\neg\roarb \subseteq \neg\ovis$ implies $\roarb\subseteq\ovis$. We proceed by contradiction. 
    Assume $(a,b)\in\roarb$ and $(a,b)\not\in\ovis$, and hence,  $(a,b)\in\neg\ovis$. Since,
    $\roarb$ is a strict total order $(b,a)\not\in\roarb$. Hence, $(b,a)\in\neg\roarb$. By assumption, $(a,b)\in\neg\ovis$ and $(b,a)\in\neg\roarb$ implies $(a,a)\in\neg\ovis$.
    Again, by assumption, 
 \end{enumerate}

\end{proof}

\begin{theorem}[\textsc{Single Order}]

Let $\textsc{\small{AR}}$ an arbitration relation and $\textsc{\small{VIS}}$ a visibility relation, if $\environmenttran{\systemterm}{\emptyset}{\emptyset}{\emptyset}{\emptyset}{\emptyset}{\undefined}{\emptyset} \arro{} ^*\ \environmenttran{\systemterm'}{\op}{\so}{\vis}{\arb}{\rb}{\tx}{\tc}$ then $\textsc{\small{AR}};\textsc{\small{VIS}} \subseteq \textsc{\small{VIS}}$ and  $\textsc{\small{AR}}^{-1};\neg\textsc{\small{VIS}} \subseteq \neg\textsc{\small{VIS}}$.


\end{theorem}
 

\begin{proof} 
Firstly, we will prove by induction on on the length of the derivation that $\textsc{\small{AR}};\textsc{\small{VIS}} \subseteq \textsc{\small{VIS}}$, i.e., for all $\vertice_a$,$\vertice_b$,$\vertice_c$ such that ($\vertice_a$,$\vertice_b$) $\in$ $\textsc{\small{AR}}$ and ($\vertice_b$,$\vertice_c$) $\in$ $\textsc{\small{VIS}}$ then ($\vertice_a$,$\vertice_c$) $\in$ $\textsc{\small{VIS}}$.

\begin{itemize}
   \item{\bf n=0}. It is easy to see because of both relations are empty, hence is trivial $\emptyset \subseteq \emptyset$.
   \item{\bf n=k+1}. Then $\environmenttran{\systemterm_0}{\emptyset}{\emptyset}{\emptyset}{\emptyset}{\emptyset}{\undefined}{\emptyset} \arro{} ^n\ \environmenttran{\systemterm}{\op}{\so}{\vis}{\arb}{\rb}{\tx}{\tc} \arroi{\alpha} \environmenttran{\systemterm'}{\op'}{\so'}{\vis'}{\arb'}{\rb'}{\tx'}{\tc'}$. We proceed by case analysis on the last transition:
	
	\begin{itemize}
        \item {\bf rule (\textsc{E-READ})}. $\textsc{\small{AR}}$ does not change, therefore, $\textsc{\small{AR}}'$ = $\textsc{\small{AR}}$.  Then, there exists two cases: \chnote{\mbox{la regla es S-READ, pero es igual a E-READ sin los ultimos terminos, charlar como escribir esto}}
				
				\begin{itemize}
					\item $\vertice_c \neq \vertice$, as $\vertice$ is an vertex associated to last computation step then ($\vertice_b$,$\vertice_c$) $\in$ $\textsc{\small{VIS}}$, therefore, these belong to $\textsc{\small{VIS}}'$. Then, by inductive hypothesis, ($\vertice_a$,$\vertice_c$) $\in$ $\textsc{\small{VIS}}$ so that as $\vertice_c \neq \vertice$ then ($\vertice_a$,$\vertice_c$) $\in$  $\textsc{\small{VIS}'}$.
					\item $\vertice_c$ = $\vertice$, as $\vertice_a$ and $\vertice_b$ are in $\textsc{\small{AR}}$ then there exist two updates into the Store associated to them. Let $\udec[u][\vertice_a]$, $\udec[u][\vertice_b]$ be these updates, then we know that $\udec[u][\vertice_a]$ = $\queuemessage$[m] and $\udec[u][\vertice_b]$ = $\queuemessage$[n] such that m < n. If ($\vertice_b$,$\vertice_c$) $\in$ $\textsc{\small{VIS}}$ then $\udec[u][\vertice_b]$ $\in \queuemessage[0..\tknown-1] \cdot \tpending \cdot [\ttransactionbuffer]$ by definition. Then, analyzing cases:
					\begin{itemize}
						\item $\udec[u][\vertice_b]$ $\in \queuemessage[0..\tknown-1]$. Then $\udec[u][\vertice_a]$ also is in $\queuemessage[0..\tknown-1]$ because of $\udec[u][\vertice_a]$ = $\queuemessage$[m] and $\udec[u][\vertice_b]$ = $\queuemessage$[n] and m < n.
						\item $\udec[u][\vertice_b]$ $\in \tpending$ or $\udec[u][\vertice_b]$ $\in [\ttransactionbuffer]$, however these do not be by $\lemref{lemma:empty_queue}$.
					\end{itemize}
					
				\end{itemize}
				
			\item The rest of the rules do not change $\textsc{\small{VIS}}$.

\end{itemize}

\end{itemize}

Now, we will prove by induction on on the length of the derivation that $\textsc{\small{AR}}^{-1};\neg\textsc{\small{VIS}} \subseteq \neg\textsc{\small{VIS}}$, i.e., for all $\vertice_a$,$\vertice_b$,$\vertice_c$ such that ($\vertice_b$,$\vertice_a$) $\in$ $\textsc{\small{AR}}$ and ($\vertice_a$,$\vertice_c$) $\notin$ $\textsc{\small{VIS}}$ then ($\vertice_b$,$\vertice_c$) $\notin$ $\textsc{\small{VIS}}$. This prove is analogue when $\vertice_c \neq \vertice$. Furthermore, if $\udec[u][\vertice_a]$ $\notin \queuemessage[0..\tknown-1]$ then $\udec[u][\vertice_b]$ $\notin \queuemessage[0..\tknown-1]$ because of $\udec[u][\vertice_b]$ is before to $\udec[u][\vertice_a]$ at Store, then 
\end{proof}
	
 \appendix
 
 \section{Functions for implementing datatypes}
 \subsection{Auxiliar Function}

Before describing the implementation of GSP, we will introduce the following auxiliars functions which will be used when we define the operational semantic.

\footnotesize
\ttfamily


\begin{flushleft}
\specfunction{append}{\gssegmenttype}{$\roundtype^*$}{\gssegmenttype} \\
\append{\gssegmentins{$\delta$}{\maxround}}{$\epsilon$} = \gssegmentins{$\delta$}{\maxround} \\
\append{\gssegmentins{$\delta$}{\maxround}}{$\headerround$:\tailround} = \append{\textless\reduce{$\delta$ $\cdot$ $\delta_0$ $\cdot$ $\epsilon$}, $\update{\maxround}{b_0}{n_0}$\textgreater}{\tailround} 
\end{flushleft}

\begin{flushleft}
\specfunction{apply}{\gsprefixtype}{\gssegmenttype}{\gsprefixtype} \\
\apply{$\gsprefixins{\state}{\maxround}$}{\gssegmentins{$\delta$}{$\maxround'$}} =  \gsprefixins{\applyplus{\state}{$\delta$ $\cdot$ $\epsilon$}}{\maxround[$\maxround'$]}  \\
\end{flushleft}

\begin{flushleft}
\specfunctiononeparameter{receivedrounds}{($\partialfunction{\idset}{\textless \dominserver \textgreater^*}$)}{$\roundtype^*$}\\
receivedrounds($\undefined$) = $\epsilon$ \\
receivedrounds($\partialfunction{b}{\textless n_0,\ \delta_0 \textgreater} \cdot\ f$) = \textless $b$, \ $n_0$, \ $\delta_0$\textgreater\ $\cdot$ receivedrounds($f$)
\end{flushleft}

\begin{flushleft}
\henote{arreglar las lineas comentadas}
\chnote{hecho}
\specfunctionforparameters{curstate}{\statetype}{$\roundtype^*$}{\deltatype}{\deltatype}{\statetype} \\
$curstate(\state, \pending, \pushbuffer, \transactionbuffer$) = $\applyplus{\state}{getdeltas(\pending) \cdot\pushbuffer\cdot \transactionbuffer}$
\end{flushleft}


\begin{flushleft}
\specfunctiononeparameter{getdeltas}{$\roundtype^*$}{$\deltatype^*$} \\
getdeltas($\epsilon$) = $\epsilon$ \\
getdeltas(\textless $n_0$,$\delta_0$ \textgreater $\cdot$ $\delta$) = $\delta_0$ $\cdot$ getdeltas($\delta$)\\
\end{flushleft}

\begin{flushleft}
\specfunctiononeparameter{remove}{$\deltatype^*$}{$\deltatype^*$} \\
remove(rs,$\epsilon$) = $\epsilon$ \\
remove(rs, xs) = $(foldr (\backslash x\ rec\ ys\to h \ (filter \ (/$=$x) ys))\ id) xs \ rs$\\
\end{flushleft}


\begin{flushleft}
\specfunctionthreeparameters{notify}{$\idset^*$}{($\partialfunction{\idset}{\gssegmenttype^* \cup \gsprefixtype}$)}{$\gssegmenttype$}{$(\partialfunction{\idset}{\gssegmenttype^* \cup \gsprefixtype})$}\\
notify($\{\cid\}$, $\outserver$, gs) = $\outserver[\cid \mapsto gss \cdot gs]$\\
notify($\cid \cdot bs$, $\outserver$, gs) = $\notify{bs}{\outserver}{gs}[\cid \mapsto gss \cdot gs]$ 
\end{flushleft}

\begin{flushleft}
\specfunction{\cleannamefun}{$\idset^*$}{$(\partialfunction{\idset}{\dominserver}$)}{$(\partialfunction{\idset}{\dominserver})$}\\
\cleannamefun($\{\cid\}$, $\outserver$, gs) = $\inserver[\cid \mapsto \emptyset]$\\
\cleannamefun($\cid \cdot bs$, $\outserver$, gs) = $\clean{bs}{\inserver[\cid \mapsto \emptyset]$}
\end{flushleft}


\normalfont
\normalsize

\paragraph{Notation.} Let $f$ and $g$ be a partial function, we define the update operator $\_[\_]$ such that
  $dom(f[g])  = dom(f) \cup dom(g)$ and 
 \[
   \begin{array}{l@{\ = \ \Bigg\{}l}
     f[g](x) & 
     \begin{array}{ll}
        f(x) & \mbox{if }\ x\not\in dom(g) \wedge x\in dom(f) \\
        g(x) & \mbox{if }\ x\in dom(g)\\
        \mathit{\undefined}  & \mbox{Otherwise} 
     \end{array}
   \end{array}
 \]    

 We write $[x_1 \mapsto y_1, \ldots, x_n \mapsto y_n]$ for the partial function $f$ such that $dom(f)=\{x_1,\ldots,x_n\}$ and $f(x_i)=y_i$; $A \setminus B$ to denote the usual difference of sets.



The functions \textbf{reduce} and \textbf{apply} are abstract and depend on the data model used. 


\end{document}
