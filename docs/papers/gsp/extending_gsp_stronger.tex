\section{Extending GSP to Stronger Model}
\label{sec:transactions}

As illustrated in above examples, there are situations in which a environment does not ensures consistency guarantees as Consistent Prefix or Single Order. The nature of such cases is caused by having no transactions as the usual databases.

The remaining of this section is proposing a environment that ensures these consistency guarantees. We start by refining the environment in GSP. We will distinguish a new relation:

\paragraph{Return Before} relates updates actions. It indicates the ordering of non-overlapping operations. 

Furthermore, we will add two additional terms; $\tx$ that relates a vertex to set of vertex which have just not finished and $\tc$ that denotes transaction closed.


\[
    \begin{array}{l@{\quad}r@{\;::=\;}l}
			 (\textsc{environment}) & \environmentterm &  \environmenttran{\systemterm}{\op}{\so}{\vis}{\arb}{\rb}{\tx}{\tc} \\
			 (\textsc{transactions with overlapping}) & \tx &  \emptyset  \ | \ \verticesets\ \mapsto 2^{\vertice},\tx   \\
			 (\textsc{transactions closed}) & \tc &  \epsilon  \ | \ \vertice\ \cdot \tc 
	   \end{array}
\]


\subsection{Operational Semantic}

%\subsubsection{Notation}

%\begin{flushleft}
%\specfunction{$\downharpoonright$}{$\verticesets$}{$2^\verticesets$}{$2^\verticesets$} \\
%$\downharpoonright v^i \emptyset$ =  $\emptyset$ \\
%$\downharpoonright v^i (x^h \mapsto V)$ = x^h \mapsto V \cup \ v^i;  $\downharpoonright v^i V$ \\
%\end{flushleft}


We assume the following countable sets of transactional vertices names $[\verticesets]$ ranged over by $[\vertice];[\vertice_0];[\vertice_1],\ldots$;

The labeled transition system for the extension of GSP considers the following actions $\zeta$:	

\[ 
\begin{array}{r@{\ ::= \ }l}
  \zeta & \mu \ | \ \startsynctran \ | \ \finishsynctran  \ | \  [\mu] \\
\end{array}
\]

These labels allow system to perform an update synchronous. Label $\startsynctran$ stands for the begining of a transaction, and $\finishsynctran$ the end of it.

\paragraph{Notation} $\updopentx{\epsilon} = \epsilon$, and 
   $\updopentx{\vertice_0 \mapsto \{\instanceset\};tail} = \vertice_0 \mapsto \{\instanceset \cup \{\vertice\}\};\updopentx{tail}$.
		
		
 \[
 \begin{array}{l}    \hspace{-.3cm} \textsc{ENVIRONMENT}\\

		
\mathrule{e-start-sync}{\systemterm \arroi{\syncupdtran{\tupdate}} \systemterm' \qquad [\vertice] \notin 
dom(\tx)}{\environmenttran{\systemterm}{\op}{\so}{\vis}{\arb}{\rb}{\tx}{\tc} \arroi{\startsynctran{\vertice}} \environmenttran{\systemterm'}{\op}{\so}{\vis}{\arb}{\rb}{\updopentx{\tx};\vertice \mapsto dom(\tc)}{\tc}}

 \\[35pt]
{
\mathrule{e-end-sync}{\systemterm \arroi{\finishsynctran{\tupdate}} \systemterm' \qquad [\vertice] \in 
dom(\tx)}{\environmenttran{\systemterm}{\op}{\so}{\vis}{\arb}{\rb}{\vertice \mapsto \instanceset,\tx}{\tc} \arroi{\finishsynctran{\vertice}} \environmenttran{\systemterm'}{\op}{\so}{\vis}{\arb}{\rb'}{\tx}{\tc \cdot\ \vertice}}}
\\
 \\[10pt]where \ 
\rb' = \rb \cup\ \{ (x,v) \ | \ \forall 0 \leq m,n < |\queuemessage|, \op(x) = \queuemessage[m] \ \land \ \op(v) = \queuemessage[n] \ \land \ m \leq\ n \ \land \ x\notin\tx(v) \}
 \end{array}
 \]


\[
 \begin{array}{l} \hspace{-.3cm} \textsc{PROGRAM}\\


\mathax{t-sync-update}{\tsystem{\tclienti{\tsyncupd{u};\tprogram}{\tknown}{\tpending}{\ttransactionbuffer}{\tsent}{\treceivebuffer}\ \bigpar\ C}{\queuemessage} \arroi{\syncupdtran{\tupdate}} \tsystem{\tclienti{[\tupdins;\tflush];\tprogram}{\tknown}{\tpending}{\ttransactionbuffer}{\tsent}{\treceivebuffer}\ \bigpar\ C}{\queuemessage}}
 
 \\[15pt]

\mathrule{t-tran}{\tsystem{\tclienti{\tprogram}{\tknown}{\tpending}{\ttransactionbuffer}{\tsent}{\treceivebuffer}\ \bigpar\ C}{\queuemessage} \arroi{\mu} \tsystem{\tclienti{\tprogram'}{\tknown'}{\tpending'}{\ttransactionbuffer'}{\tsent'}{\treceivebuffer'}\ \bigpar\ C}{\queuemessage}}{\tsystem{\tclienti{[\tprogram];Q}{\tknown}{\tpending}{\ttransactionbuffer}{\tsent}{\treceivebuffer}\ \bigpar\ C}{\queuemessage} \arroi{[\mu]} \tsystem{\tclienti{[\tprogram'];Q}{\tknown'}{\tpending'}{\ttransactionbuffer'}{\tsent'}{\treceivebuffer'}\ \bigpar\ C}{\queuemessage}}

 \\[25pt]

 
\mathax{t-flush}{\tsystem{\tclienti{[\tflush];\tprogram}{\tknown}{\tpending}{\ttransactionbuffer}{\tsent}{\treceivebuffer}\ \bigpar\ C}{\queuemessage} \arroi{\tau} \tsystem{\tclienti{[\pushtran;\pwhile{!\confirmedtran()}{\pulltran};];\tprogram}{\tknown}{\tpending}{\ttransactionbuffer}{\tsent}{\treceivebuffer}\ \bigpar\ C}{\queuemessage}}
 
\\
\chnote{\mbox{En realidad el programa es: $\pushtran;let x = \confirmedtran();\pwhile{!x}{\pulltran};let x = \confirmedtran();$}}

 \\[15pt]

\mathax{t-end-sync}{\tsystem{\tclienti{[0];\tprogram}{\tknown}{\tpending}{\ttransactionbuffer}{\tsent}{\treceivebuffer}\ \bigpar\ C}{\queuemessage} \arroi{\finishsynctran(u^{[v]})} \tsystem{\tclienti{\tprogram}{\tknown}{\tpending}{\ttransactionbuffer}{\tsent}{\treceivebuffer}\ \bigpar\ C}{\queuemessage}}
 

 \end{array}
 \]
 

First, we prove a useful lemma: 

\begin{lemma}\label{lemma:empty_queue} 

$\forall C$ such that 
$\tclienti{Q}{\tknown}{\tpending}{\ttransactionbuffer}{\tsent}{\treceivebuffer} \ 
    \Arro^* \ \tclienti{\tsyncupd{u};\treadins{x}{r}}{\tknown}{\tpending}{\ttransactionbuffer}{\tsent}{\treceivebuffer} 
	  \Arro^* \ \tclienti{\treadins{x}{r}}{\tknown}{\tpending}{\ttransactionbuffer}{\tsent}{\treceivebuffer}$, then  
rvalue operation always reads above the store.

\end{lemma}

\begin{proof} Confirmed is evaluated to true when $\tpending$ and $\ttransactionbuffer$ are empty so that, after the loop, the read operation only return values which belong to the Store.
\end{proof}	


Single Order is a consistency guarantee that express a single order of operations observed by arbitration and visibility relation. The definition introduced by [] requires that arbitration and visibility are the same except by operations which have not finished, i.e., it considers only operations that belong to $\tc$. However, we do not model if an update is visible to another update since the effects of an update only are seen by a read operations. Furthermore, there not exists conflicts between read operation and update operation, the Store keep update actions. Therefore, we can not compare these sets. 

\begin{theorem}[\textsc{Single Order}]

Let $\textsc{\small{AR}}$ an arbitration relation and $\textsc{\small{VIS}}$ a visibility relation, if $\environmenttran{\systemterm}{\emptyset}{\emptyset}{\emptyset}{\emptyset}{\emptyset}{\undefined}{\emptyset} \arro{} ^*\ \environmenttran{\systemterm'}{\op}{\so}{\vis}{\arb}{\rb}{\tx}{\tc}$ then $\textsc{\small{AR}};\textsc{\small{VIS}} \subseteq \textsc{\small{VIS}}$ and  $\textsc{\small{AR}}^{-1};\neg\textsc{\small{VIS}} \subseteq \neg\textsc{\small{VIS}}$.


\end{theorem}
 

\begin{proof} 
Firstly, we will prove by induction on on the length of the derivation that $\textsc{\small{AR}};\textsc{\small{VIS}} \subseteq \textsc{\small{VIS}}$, i.e., for all $\vertice_a$,$\vertice_b$,$\vertice_c$ such that ($\vertice_a$,$\vertice_b$) $\in$ $\textsc{\small{AR}}$ and ($\vertice_b$,$\vertice_c$) $\in$ $\textsc{\small{VIS}}$ then ($\vertice_a$,$\vertice_c$) $\in$ $\textsc{\small{VIS}}$.

\begin{itemize}
   \item{\bf n=0}. It is easy to see because of both relations are empty, hence is trivial $\emptyset \subseteq \emptyset$.
   \item{\bf n=k+1}. Then $\environmenttran{\systemterm_0}{\emptyset}{\emptyset}{\emptyset}{\emptyset}{\emptyset}{\undefined}{\emptyset} \arro{} ^n\ \environmenttran{\systemterm}{\op}{\so}{\vis}{\arb}{\rb}{\tx}{\tc} \arroi{\alpha} \environmenttran{\systemterm'}{\op'}{\so'}{\vis'}{\arb'}{\rb'}{\tx'}{\tc'}$. We proceed by case analysis on the last transition:
	
	\begin{itemize}
        \item {\bf rule (\textsc{E-READ})}. $\textsc{\small{AR}}$ does not change, therefore, $\textsc{\small{AR}}'$ = $\textsc{\small{AR}}$.  Then, there exists two cases: \chnote{\mbox{la regla es S-READ, pero es igual a E-READ sin los ultimos terminos, charlar como escribir esto}}
				
				\begin{itemize}
					\item $\vertice_c \neq \vertice$, as $\vertice$ is an vertex associated to last computation step then ($\vertice_b$,$\vertice_c$) $\in$ $\textsc{\small{VIS}}$, therefore, these belong to $\textsc{\small{VIS}}'$. Then, by inductive hypothesis, ($\vertice_a$,$\vertice_c$) $\in$ $\textsc{\small{VIS}}$ so that as $\vertice_c \neq \vertice$ then ($\vertice_a$,$\vertice_c$) $\in$  $\textsc{\small{VIS}'}$.
					\item $\vertice_c$ = $\vertice$, as $\vertice_a$ and $\vertice_b$ are in $\textsc{\small{AR}}$ then there exist two updates into the Store associated to them. Let $\udec[u][\vertice_a]$, $\udec[u][\vertice_b]$ be these updates, then we know that $\udec[u][\vertice_a]$ = $\queuemessage$[m] and $\udec[u][\vertice_b]$ = $\queuemessage$[n] such that m < n. If ($\vertice_b$,$\vertice_c$) $\in$ $\textsc{\small{VIS}}$ then $\udec[u][\vertice_b]$ $\in \queuemessage[0..\tknown-1] \cdot \tpending \cdot [\ttransactionbuffer]$ by definition. Then, analyzing cases:
					\begin{itemize}
						\item $\udec[u][\vertice_b]$ $\in \queuemessage[0..\tknown-1]$. Then $\udec[u][\vertice_a]$ also is in $\queuemessage[0..\tknown-1]$ because of $\udec[u][\vertice_a]$ = $\queuemessage$[m] and $\udec[u][\vertice_b]$ = $\queuemessage$[n] and m < n.
						\item $\udec[u][\vertice_b]$ $\in \tpending$ or $\udec[u][\vertice_b]$ $\in [\ttransactionbuffer]$, however these do not be by $\lemref{lemma:empty_queue}$.
					\end{itemize}
					
				\end{itemize}
				
			\item The rest of the rules do not change $\textsc{\small{VIS}}$.

\end{itemize}

\end{itemize}

Now, we will prove by induction on on the length of the derivation that $\textsc{\small{AR}}^{-1};\neg\textsc{\small{VIS}} \subseteq \neg\textsc{\small{VIS}}$, i.e., for all $\vertice_a$,$\vertice_b$,$\vertice_c$ such that ($\vertice_b$,$\vertice_a$) $\in$ $\textsc{\small{AR}}$ and ($\vertice_a$,$\vertice_c$) $\notin$ $\textsc{\small{VIS}}$ then ($\vertice_b$,$\vertice_c$) $\notin$ $\textsc{\small{VIS}}$. This prove is analogue when $\vertice_c \neq \vertice$. Furthermore, if $\udec[u][\vertice_a]$ $\notin \queuemessage[0..\tknown-1]$ then $\udec[u][\vertice_b]$ $\notin \queuemessage[0..\tknown-1]$ because of $\udec[u][\vertice_b]$ is before to $\udec[u][\vertice_a]$ at Store, then 
\end{proof}
	