% !TEX root = main.tex

\section{{\gsp} with atomic updates}
\label{sec:transactions}

In this section we study an extension of the \gsp\ model with atomic updates proposed in~\cite{DBLP:conf/ecoop/BurckhardtLPF15}. 
We extend the language of programs as follows: 
\[
 \rName{program} 
			 \qquad 
			 \tprogram \ ::=\  \ldots\ |\ \tsyncupdins 
\]

The execution of a program  $\tsyncupdins$ remains blocked until the 
update $\anupd$ is performed over the global store.  This is achieved by continuously pulling (i.e., 
a busy-waiting) until the updates are confirmed by global store. In order to provide 
the formal semantics of the language, we consider the following 
runtime syntax for programs.
\[
 \rName{run-time-program} 
			 \qquad 
			 \tprogram \ ::=\  \ldots\  |\ \waitcmd;P \ |\ \tguarded{} 
			 \]

%We rely  on the following additional labels
			 
%\[  \actbyc ::= \ldots \ |\ (\tupdlbl,\cid)\  |\ (\finishsynctran,\cid)
%\]

The operational semantics for the new primitives is given by the rules in 
\figref{fig:OS-tgsp}.  Rule \ruleName{sync-upd} rewrites each synchronous update as
the sequence consisting of an asynchronous update followed by \pullcmd\ and  \waitcmd. 
Processes \waitcmd\  continuously checks whether  local changes have been confirmed by 
the global store. As described by rule \ruleName{wait}, it is implemented as a busy-waiting
loop that first checks the local buffers by executing \confcmd\ and then
performs the conditional jump $\tguarded[x][\pullcmd;\waitcmd][P]$.  
If the condition $x$ is true, then it follows as $P$ otherwise it continues as 
$\pullcmd;\waitcmd$, as described by rules \ruleName{guard-true} and \ruleName{guard-false}.
  
%
%
%We define the operator $[\_]$ over actions  $\lambda$ s.t. $[\updatetran{\anupd}] =\tupdlbl$ and it is the identity
%over any other action.
%
\begin{figure}[t]
\[
 \begin{array}{l}

\mathax{sync-upd}
	{
	\begin{array}{p{\linewidth}}
	$\tsystem{\tclienti{\tsyncupdins}{\tknown}{\tpending}{\ttransactionbuffer}{\tsent}{\treceivebuffer}} 
%	\arroi{\syncupdtran{\tupdate}}
	\quad \arroi{\tau}
	$\\
	$\hfill
	\tsystem{\tclienti{\tupdins;\pushcmd;\waitcmd;P}{\tknown}{\tpending}{\ttransactionbuffer}{\tsent}{\treceivebuffer}}
	$\end{array}
	 }
 %\mathrule{trans}
%	{\tsystem{\tclienti{\tprogram}{\tknown}{\tpending}{\ttransactionbuffer}{\tsent}{\treceivebuffer}} 
%	\quad\arroi{\mu} \quad
%	\tsystem{\tclienti{\tprogram'}{\tknown'}{\tpending'}{\ttransactionbuffer'}{\tsent'}{\treceivebuffer'}}}
%	{\tsystem{\tclienti{[\tprogram];Q}{\tknown}{\tpending}{\ttransactionbuffer}{\tsent}{\treceivebuffer}}
%	 \quad\arroi{[\mu]} \quad
%	 \tsystem{\tclienti{[\tprogram'];Q}{\tknown'}{\tpending'}{\ttransactionbuffer'}{\tsent'}{\treceivebuffer'}}}
%
% \\[25pt]
%
%\mathax{end-upd}
%	{\tsystem{\tclienti{[0];\tprogram}{\tknown}{\tpending}{\ttransactionbuffer}{\tsent}{\treceivebuffer}} 
%	\quad\arroi{\finishsynctran{}}\quad
%	\tsystem{\tclienti{\tprogram}{\tknown}{\tpending}{\ttransactionbuffer}{\tsent}{\treceivebuffer}}}

\\[15pt]
\mathax{wait}
	{
	\begin{array}{p{\linewidth}}
	$\tsystem{\tclienti{\waitcmd;P}{\tknown}{\tpending}{\ttransactionbuffer}{\tsent}{\treceivebuffer}}
	\quad\arro{\tau}$
	\\
	$\hfill
	\tsystem{\tclienti{\tconfirmedins{x}[{\tguarded[x][\pullcmd;\waitcmd;P][P]}]}{\tknown}{\tpending}{\ttransactionbuffer}{\tsent}{\treceivebuffer}}
	$\end{array}
	}

\\[5pt]
\mathrule{guard-true}
	{\eval \cond {\it true}}
	{
	\tsystem{\tclienti{\tguarded[e][P][Q]}{\tknown}{\tpending}{\ttransactionbuffer}{\tsent}{\treceivebuffer}}
	\quad\arro{\tau}\quad 
	\tsystem{\tclienti{Q}{\tknown}{\tpending}{\ttransactionbuffer}{\tsent}{\treceivebuffer}}
	}

\\[15pt]
\mathrule{guard-false}
	{\eval \cond {\it false}}
	{
	\tsystem{\tclienti{\tguarded[e][P][Q]}{\tknown}{\tpending}{\ttransactionbuffer}{\tsent}{\treceivebuffer}}
	\quad\arro{\tau}\quad 
	\tsystem{\tclienti{P}{\tknown}{\tpending}{\ttransactionbuffer}{\tsent}{\treceivebuffer}}
	}
%	\\
%\henote{\mbox{Agregar reglas para if}}

%
%
% 
%\mathax{t-flush}{\tsystem{\tclienti{[\tflush];\tprogram}{\tknown}{\tpending}{\ttransactionbuffer}{\tsent}{\treceivebuffer}\ \bigpar\ C}{\queuemessage} \arroi{\tau} \tsystem{\tclienti{[\pushtran;\pwhile{!\confirmedtran()}{\pulltran};];\tprogram}{\tknown}{\tpending}{\ttransactionbuffer}{\tsent}{\treceivebuffer}\ \bigpar\ C}{\queuemessage}}
%\\
%\chnote{\mbox{En realidad el programa es: $\pushtran;let x = \confirmedtran();\pwhile{!x}{\pulltran};let x = \confirmedtran();$}}
%
% \\[15pt]
 \end{array}
 \]
\caption{Semantics of \gsp\ with atomic updates}
\label{fig:sos-atomic-upd}
\end{figure}

 
%The nature of such cases is caused by having no transactions as the usual databases.

%The remaining of this section is proposing a environment that ensures these consistency guarantees. We start by refining the environment in GSP. We will distinguish a new relation:

%\[ 
%\begin{array}{r@{\ ::= \ }l}
%  \zeta & \mu \ | \ \startsynctran \ | \ \finishsynctran  \ | \  [\mu] \\
%\end{array}
%\]






%\[
% \begin{array}{l} \hspace{-.3cm} \textsc{PROGRAM}\\
%
%
%\mathax{t-sync-update}{\tsystem{\tclienti{\tsyncupd{u};\tprogram}{\tknown}{\tpending}{\ttransactionbuffer}{\tsent}{\treceivebuffer}\ \bigpar\ C}{\queuemessage} \arroi{\syncupdtran{\tupdate}} \tsystem{\tclienti{[\tupdins;\tflush];\tprogram}{\tknown}{\tpending}{\ttransactionbuffer}{\tsent}{\treceivebuffer}\ \bigpar\ C}{\queuemessage}}
% 
% \\[15pt]
%
%\mathrule{t-tran}{\tsystem{\tclienti{\tprogram}{\tknown}{\tpending}{\ttransactionbuffer}{\tsent}{\treceivebuffer}\ \bigpar\ C}{\queuemessage} \arroi{\mu} \tsystem{\tclienti{\tprogram'}{\tknown'}{\tpending'}{\ttransactionbuffer'}{\tsent'}{\treceivebuffer'}\ \bigpar\ C}{\queuemessage}}{\tsystem{\tclienti{[\tprogram];Q}{\tknown}{\tpending}{\ttransactionbuffer}{\tsent}{\treceivebuffer}\ \bigpar\ C}{\queuemessage} \arroi{[\mu]} \tsystem{\tclienti{[\tprogram'];Q}{\tknown'}{\tpending'}{\ttransactionbuffer'}{\tsent'}{\treceivebuffer'}\ \bigpar\ C}{\queuemessage}}
%
% \\[25pt]
%
% 
%\mathax{t-flush}{\tsystem{\tclienti{[\tflush];\tprogram}{\tknown}{\tpending}{\ttransactionbuffer}{\tsent}{\treceivebuffer}\ \bigpar\ C}{\queuemessage} \arroi{\tau} \tsystem{\tclienti{[\pushtran;\pwhile{!\confirmedtran()}{\pulltran};];\tprogram}{\tknown}{\tpending}{\ttransactionbuffer}{\tsent}{\treceivebuffer}\ \bigpar\ C}{\queuemessage}}
% 
%\\
%\chnote{\mbox{En realidad el programa es: $\pushtran;let x = \confirmedtran();\pwhile{!x}{\pulltran};let x = \confirmedtran();$}}
%
% \\[15pt]
%
%\mathax{t-end-sync}{\tsystem{\tclienti{[0];\tprogram}{\tknown}{\tpending}{\ttransactionbuffer}{\tsent}{\treceivebuffer}\ \bigpar\ C}{\queuemessage} \arroi{\finishsynctran(u^{[v]})} \tsystem{\tclienti{\tprogram}{\tknown}{\tpending}{\ttransactionbuffer}{\tsent}{\treceivebuffer}\ \bigpar\ C}{\queuemessage}}
% 
%
% \end{array}
% \]
% 


%\subsection{Operational Semantic}

%\subsubsection{Notation}

%\begin{flushleft}
%\specfunction{$\downharpoonright$}{$\verticesets$}{$2^\verticesets$}{$2^\verticesets$} \\
%$\downharpoonright v^i \emptyset$ =  $\emptyset$ \\
%$\downharpoonright v^i (x^h \mapsto V)$ = x^h \mapsto V \cup \ v^i;  $\downharpoonright v^i V$ \\
%\end{flushleft}

%
%We assume the following countable sets of transactional vertices names $[\verticesets]$ ranged over by $[\vertice];[\vertice_0];[\vertice_1],\ldots$;
%
%The labeled transition system for the extension of GSP considers the following actions $\zeta$:	
%
%
%
%
%These labels allow system to perform an update synchronous. Label $\startsynctran$ stands for the begining of a transaction, and $\finishsynctran$ the end of it.

%Since update operations are not instantaneous, now we introduce to the model the relations that take into account the 
%beginning and finalisation of each write operation. In particular,

%\begin{itemize}
%  \item{\em Return Before} (\rb), which indicates the ordering of non-overlapping update operations. 
%\end{itemize}

%The extended environment will also contain two additional terms that are instrumental to the 
%computation of $\rb$. 

%\begin{itemize}
%	\item $\tx$ that relates a vertex to set of vertices which have just not finished. Formally,  
%	$\tx :\verticesets\ \mapsto 2^{\verticesets}$, i.e., it is a function from vertices to set of vertices.
%	\item $\tc$ that denotes the set of closed transactions, i.e., $\tc\subseteq \verticesets$.
%\end{itemize}

%\[
 %   \begin{array}{l@{\quad}r@{\;::=\;}l}
%			 (\textsc{environment}) & \environmentterm &  \environmenttran{\systemterm}{\op}{\so}{\vis}{\arb}{\rb}{\tx}{\tc} \\
%			 (\textsc{transactions with overlapping}) & \tx &  \emptyset  \ | \ \verticesets\ \mapsto 2^{\vertice},\tx   \\
%			 (\textsc{transactions closed}) & \tc &  \epsilon  \ | \ \vertice\ \cdot \tc 
%	   \end{array}
%\]


%\paragraph{Notation} $\updopentx{\epsilon} = \epsilon$, and 
 %  $\updopentx{\vertice_0 \mapsto \{\instanceset\};tail} = \vertice_0 \mapsto \{\instanceset \cup \{\vertice\}\};\updopentx{tail}$.
		


		
 %\[
 %\begin{array}{l}  
%\mathrule
%	{e-start-sync}
%	{\systemterm \arroi{\syncupdtran{\tupdate}} \systemterm' 
%		\qquad [\vertice] \notin 
%dom(\tx)}{\environmenttran{\systemterm}{\op}{\so}{\vis}{\arb}{\rb}{\tx}{\tc} \arroi{\startsynctran{\vertice}} \environmenttran{\systemterm'}{\op}{\so}{\vis}{\arb}{\rb}{\updopentx{\tx};\vertice \mapsto dom(\tc)}{\tc}}

% \\[35pt]
%{
%\mathrule{e-end-sync}{\systemterm \arroi{\finishsynctran{\tupdate}} \systemterm' \qquad [\vertice] \in 
%dom(\tx)}{\environmenttran{\systemterm}{\op}{\so}{\vis}{\arb}{\rb}{\vertice \mapsto \instanceset,\tx}{\tc} \arroi{\finishsynctran{\vertice}} \environmenttran{\systemterm'}{\op}{\so}{\vis}{\arb}{\rb'}{\tx}{\tc \cdot\ \vertice}}}
%\\
 %\\[10pt]where \ 
%\rb' = \rb \cup\ \{ (x,v) \ | \ \forall 0 \leq m,n < |\queuemessage|, \op(x) = \queuemessage[m] \ \land \ \op(v) = \queuemessage[n] \ \land \ m \leq\ n \ \land \ x\notin\tx(v) \}
 %\end{array}
 %\]





Single Order is a consistency guarantee that express a single order of operations observed by arbitration and visibility relation. The definition introduced by \cite{} requires 
$\ovis = \roarb$, where $F$ stands for the set of finished operations, i.e., when consider visibility restricted to the operations that belong to $\tc$. 
Since we defined $\arb$ and $\vis$ over restricted domains, the original formulation is not useful. Hence we will use an alternative characterisation, given by the 
following result.

\begin{theorem}[\textsc{Single Order}]
\label{thm:single-order}
Let $\textsc{\small{AR}}$ an arbitration relation and $\textsc{\small{VIS}}$ a visibility relation, if $\environment{\systemterm}{\emptyset}{\emptyset}{\emptyset}{\emptyset}[\emptyset] \arro{} ^*\ \environment{\systemterm}{\op}{\so}{\vis}{\arb} $ then $\textsc{\small{AR}};\textsc{\small{VIS}} \subseteq \textsc{\small{VIS}}$ and  $\textsc{\small{AR}}^{-1};\neg\textsc{\small{VIS}} \subseteq \neg\textsc{\small{VIS}}$.


\end{theorem}
 

