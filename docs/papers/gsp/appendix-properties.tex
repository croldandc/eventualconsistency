% !TEX root = main.tex
\section{Proof of results in~\secref{sec:properties-gsp}}

We start by proving some auxiliary results that 
will be useful when proving the main results in this section.

The following results state the 
\begin{lemma}
\label{lemma:update-ever-belong} 
Let $\systemterm$ a \gsp\ system. %For all  $\systemterm[']$ and $u\in\updatesets$, 
If 
$\environment{\systemterm}{\emptyset}{\emptyset}{\emptyset}{\emptyset}[\emptyset] \tr{} ^*\ \environment{\systemterm[']}{\op}{\so}{\vis}{\arb}$
then for all $\vertice\in\dom\op$ if $\op(\vertice)\in\updatesets$ {and $\vertice\in\sse(\cid)$} then
\[\systemterm' = \anabstcli \ \bigpar\ \queuemessage\ \bigpar\ \Absclient\]
% u an update action, $\queuemessage$ a message queue, $\tpending_i$ and
 % $\ttransactionbuffer_i$ a pending queue and transaction queue from the client i, if $\environment{\systemterm}{\emptyset}{\emptyset}{\emptyset}{\emptyset} \arro{} ^*\ \environment{\systemterm}{\op}{\so}{\vis}{\arb}$ then $\{x \mapsto \updatebyclient{i} \}\ \in \ \op\ \Rightarrow\ 
and $\udec \in\  \queuemessage[0..\tknown_{\cid}-1] \cdot \tpending_{\cid} \cdot [\ttransactionbuffer_{\cid}]$.
\end{lemma}

\begin{proof} The proof follows by induction on the length of the derivation $\tr{} ^n$.
\begin{itemize}
   \item{\bf n=0}. Then $\op$ is $\emptyset$, and hence the thesis trivially holds. 
      \item{\bf n=k+1}. Then 
      \[\environment{\systemterm}{\emptyset}{\emptyset}{\emptyset}{\emptyset}[\emptyset] 
         \tr{} ^k\ 
         \environment{\systemterm['']}{\op'}{\so'}{\vis'}{\arb'}[\sse'] 
         \arro{\lambda}[\cidj]
 	\environment{\systemterm[']}{\op}{\so}{\vis}{\arb}
	 \]
	 
	By inductive hypothesis, if $\vertice\in\dom{\op'}$ s.t. $\op'(\vertice) = u$ then
	\[\systemterm'' = \anabstcliprime \ \bigpar\ \queuemessage'\ \bigpar\ \Absclient'\]
       and $\udec\in\  \queuemessage[0..\tknown'_{\cid}-1] \cdot \tpending'_{\cid} \cdot [\ttransactionbuffer'_{\cid}]$. We proceed by 
case analysis on the inference rule applied for the last transition:
	
	\begin{itemize}
        
        \item rule \ruleName{a-read}. Then,  $\systemterm''{\arro{\readtranaux{r^{\vertice[w]}}}[\cidj]}\systemterm'$ and     
        $\op = \op'\upd{\vertice[w]}r$ with $r\in\readset$.  
        If $\vertice[v]\in\dom\op$ and $\op(\vertice[v])\in\updatesets$, then
        $\vertice\in\dom{\op'}$ because the new event $\vertice[w]$ is associated with the read operation $r$. 
        Moreover,  $\systemterm''{\arro{\readtran{\udec[r]}}[\cidj]}\systemterm'$ implies 
        $\queuemessage = \queuemessage'$ and for all ${\cid}$,  ${\tknown'}_{\cid} = \tknown_{\cid}$ and 
        $\tpending'_{\cid} = \tpending_{\cid}$. Therefore, 
        $\queuemessage[0..\tknown_{\cid}-1] = \queuemessage'[0..\tknown_{\cid}-1]$ for all $\cid$.
                Hence, the case follows by inductive hypothesis.
        
%        ate operation then it must not be $v$, so that $\{x \mapsto \updatebyclient{i} \} \in\ \op$, then by inductive hypothesis $u^x \in\  \queuemessage[0..\tknown_i-1] \cdot \tpending_i \cdot [\ttransactionbuffer_i]$. When $\systemterm \arroi{\readtran{r}} \systemterm'$, $\tknown_i$, $\tpending_i$,$\ttransactionbuffer_i$ do not change.
	
	
	\item rule \ruleName{a-update}. Then, $\systemterm''{\arro{\updatetranaux{\anupd^{\vertice[w]}}}[\cidj]}\systemterm'$
	and     
        $\op = \op'\upd{\vertice[w]}\anupd$ with $\anupd\in\updatesets$. There are two possibilities:
		\begin{itemize}
			\item {\bf $\vertice \neq \vertice[w] $}. Then, the case follows  analogously to the previous case by using inductive hypothesis. 
						%Then we can use inductive hypothesis, so that it is easy to see that if $u^x \in\  \queuemessage[0..\tknown_i-1] \cdot \tpending_i \cdot [\ttransactionbuffer_i]$ then $u^x \in\  \queuemessage[0..\tknown_i-1] \cdot \tpending_i \cdot [\ttransactionbuffer_i] \cdot u_{t}^v$ too.
			
			\item  {\bf $\vertice = \vertice[w] $}. Note that $\systemterm''{\arro{\updatetranaux{\anupd^{\vertice[w]}}}[\cidj]}\systemterm'$ 
			implies  $\ttransactionbuffer_{\cidj} = \ttransactionbuffer'_{\cidj}\cdot \anupd^{\vertice[w]}$. Then,
			take $\cid = \cidj$ and conclude that  
			$\anupd^{\vertice[w]}\in\  \queuemessage[0..\tknown_{\cidj}-1] \cdot \tpending_{\cidj} \cdot [\ttransactionbuffer'_{\cidj}\cdot\anupd^{\vertice[w]}]$.
		\end{itemize}
        
        \item rule�\ruleName{a-arb}. Then, $\queuemessage \neq \queuemessage'$. By rule inspection, the only possibility is 
        $\systemterm''\arro{\tau}[\cidj]\systemterm'$ by using rule \ruleName{send}. The proof follows from inductive hypothesis because $\tknown_{\cid}=\tknown'_{\cid}$, $\tpending'_{\cid}=\tpending_{\cid}$ and $\ttransactionbuffer' _{\cid}=\ttransactionbuffer_{\cid}$ for all $\cid$ and $\op'=\op$.
        \item rule \ruleName{a-int}. We note that $\op' = \op$ and $\queuemessage = \queuemessage'$. Then, we proceed by case analysis on 
        the reduction $\systemterm''\arro{\tau}[{\cidj}]\systemterm'$
            \begin{itemize}
		\item rule \ruleName {push}.  Then,  $\tknown_{\cidj} = \tknown_{\cidj}'$, $\tpending_{\cidj} = \tpending_{\cidj}' \cdot [\ttransactionbuffer_{\cidj}']$  
		and $\ttransactionbuffer_{\cidj} = \epsilon$ while  $\tknown_{\cid} = \tknown_{\cid}'$, $\tpending_{\cid} = \tpending_{\cid}$ and  
		$\ttransactionbuffer_{\cid}=\ttransactionbuffer_{\cid}'$
		for all $\cid \neq \cidj$.
		By inductive hypothesis, there exists $\cid$ s.t.
		 $\udec \in\  \queuemessage[0..\tknown'_{\cid}-1] \cdot \tpending'_{\cid} \cdot [\ttransactionbuffer'_{\cid}]$.
		 %
		 When $\cid = \cidj$, note that $\udec \in\  \queuemessage[0..\tknown'_{\cidj}-1] \cdot \tpending'_{\cidj} \cdot [\ttransactionbuffer'_{\cidj}]$
		 implies $\udec\in\  \queuemessage[0..\tknown'_{\cidj}-1] \cdot (\tpending'_{\cidj}\cdot [\ttransactionbuffer'_{\cidj}]) \cdot [\epsilon] =
		  \queuemessage[0..\tknown_{\cidj}-1] \cdot \tpending_{\cidj} \cdot [\ttransactionbuffer_{\cidj}]$.
		 Otherwise ($\cid \neq \cidj$), the case follows from the inductive hypothesis.
		 \item  rule \ruleName {pull}. Then,  $\tknown_{\cidj} = \tknown_{\cidj}'+\treceivebuffer_{\cidj}'$, 
		 $\tpending_{\cidj} = \tpending_{\cidj}' \setminus \queuemessage[\tknown_{\cidj}' .. \tknown_{\cidj}' + 
		 \treceivebuffer_{\cidj}'-1]$  
		and $\ttransactionbuffer_{\cidj} = \ttransactionbuffer_{\cidj}'$ 
		while  $\tknown_{\cid} = \tknown_{\cid}'$, $\tpending_{\cid} = \tpending_{\cid}$ and  
		$\ttransactionbuffer_{\cid}=\ttransactionbuffer_{\cid}'$
		for all $\cid \neq \cidj$.
		%
		By inductive hypothesis, there exists $\cid$ s.t.
		 $\udec \in\  \queuemessage[0..\tknown'_{\cid}-1] \cdot \tpending'_{\cid} \cdot [\ttransactionbuffer'_{\cid}]$.
		 %
		 When $\cid = \cidj$, we show that 
		 $\udec \in\  \queuemessage[0..\tknown'_{\cidj}-1] \cdot \tpending'_{\cidj} \cdot [\ttransactionbuffer'_{\cidj}]$ implies
		 $\udec \in\  \queuemessage[0..\tknown_{\cidj}-1] \cdot \tpending_{\cidj} \cdot [\ttransactionbuffer_{\cidj}] =
		 \queuemessage[0..\tknown'_{\cidj}+ \treceivebuffer'_{\cidj}-1] \cdot \tpending_{\cidj} 
		 	\setminus (\queuemessage[\tknown'_{\cidj}..\tknown'_{\cidj}+ \treceivebuffer'_{\cidj}-1]) \cdot [\ttransactionbuffer_{\cid}]$.
%		 We should prove that it is equivalent to $\anupd^{\vertice[w]} \in\  \queuemessage[0..\tknown_{\cidj} - 1 + 
%		 \treceivebuffer_{\cidj}] \cdot \tpending_i \setminus \queuemessage[\tknown_i .. \tknown_i + 
%		 \treceivebuffer_i] \cdot [\ttransactionbuffer_i]$. 
		 In fact, there are two interesting cases:
		\begin{itemize}
			\item If $\udec \in\ \tpending_{\cidj}'\  \land\ \udec \notin\ \queuemessage[\tknown'_{\cidj}..\tknown'_{\cidj}+ \treceivebuffer'_{\cidj}-1]$, 
			then $\udec \in \ \tpending_{\cidj}$.
			\item If $\udec \in\ \tpending'_{\cidj}\  \land\ \udec \in\ \queuemessage[\tknown'_{\cidj}..\tknown'_{\cidj}+ \treceivebuffer'_{\cidj}-1]$, 
				then $\udec \notin \ \tpending_{\cidj}$ but 
				$\udec \in\ \queuemessage[0 .. \tknown'_{\cidj}+ \treceivebuffer'_{\cidj}-1]=\queuemessage[0 .. \tknown_{\cidj} -1] $. 
		\end{itemize}
		The remaining cases ($\udec \in \queuemessage[0..\tknown'_{\cidj}-1]$ or $\udec\in\ttransactionbuffer'_{\cidj}$) are straightforward.
		
		\item rules \ruleName{receive}, \ruleName{confirm}:%, \ruleName{while-true}, \ruleName{while-false}, \henote{reglas para if}
		The proof follows from inductive hypothesis because $\tknown_{\cid}=\tknown'_{\cid}$, $\tpending_{\cid}=\tpending'_{\cid}$ 
		and $\ttransactionbuffer _{\cid}=\ttransactionbuffer' _{\cid}$ for all $\cid$ and $\op'=\op$.

	\end{itemize}
	
\end{itemize}
\end{itemize}
\qed
\end{proof}


\begin{lemma}
\label{lemma:local-updates-preserver-order} 
Let $\systemterm$ a \gsp\ system. If 
$\environment{\systemterm}{\emptyset}{\emptyset}{\emptyset}{\emptyset}[\emptyset] \tr{} ^*\ \environment{\systemterm[']}{\op}{\so}{\vis}{\arb}$
then for all $(\vertice,\vertice[w])\in\so$ if $\op(\vertice)\in\updatesets$,
$\op(\vertice[w])\in\updatesets$ and  $\anupd_0^{\vertice[w]}\in\queuemessage[0..k]$
then there exists $\anupd_1$ s.t. $\anupd_1^{\vertice}\in\queuemessage[0..k-1]$
\end{lemma}



\begin{proof} The proof follows by induction on the length of the derivation $\tr{} ^n$.
	
\begin{itemize}
   \item{\bf n=0}. Then $\so$ is $\emptyset$, and hence the thesis trivially holds. 
      \item{\bf n=k+1}. Then 
      \[\environment{\systemterm}{\emptyset}{\emptyset}{\emptyset}{\emptyset}[\emptyset] 
         \tr{} ^k\ 
         \environment{\systemterm['']}{\op'}{\so'}{\vis'}{\arb'}[\sse'] 
         \arro{\lambda}[\cidj]
 	\environment{\systemterm[']}{\op}{\so}{\vis}{\arb}
	 \]
	 
	By inductive hypothesis, if $(\vertice,\vertice[w])\in\so'$ s.t. $\op'(\vertice) \in\updatesets$ and $\op'(\vertice[w])\in\updatesets$ then
	\[\systemterm'' = \anabstcliprime \ \bigpar\ \queuemessage'\ \bigpar\ \Absclient'\]
	and $\anupd_0^{\vertice[w]}\in\queuemessage'[0..k_{\cid}^{'}]$
	then there exists $\anupd_1$ s.t. $\anupd_1^{\vertice}\in\queuemessage'[0..k_{\cid}^{'}-1]$. We proceed by 
case analysis on the inference rule applied for the last transition:
	
	\begin{itemize}
        
        \item rule \ruleName{a-read}. Then,  $\systemterm''{\arro{\readtranaux{r^{\vertice[v']}}}[\cidj]}\systemterm'$ and     
        $\op = \op'\upd{\vertice[v']}r$ with $r\in\readset$.  
        If $(\vertice,\vertice[w])\in\so$ s.t. $\op(\vertice) \in\updatesets$ and $\op(\vertice[w])\in\updatesets$ , then
        $(\vertice,\vertice[w])\in\so'$ s.t. $\op'(\vertice) \in\updatesets$ and $\op'(\vertice[w])\in\updatesets$ because the new event $\vertice[v']$ is associated with the read operation $r$. 
        Moreover,  $\systemterm''{\arro{\readtran{\udec[r]}}[\cidj]}\systemterm'$ implies 
        $\queuemessage = \queuemessage'$ and for all ${\cid}$,  ${\tknown'}_{\cid} = \tknown_{\cid}$. Therefore, 
        $\queuemessage[0..\tknown_{\cid}-1] = \queuemessage'[0..\tknown_{\cid}-1]$.
                Hence, the case follows by inductive hypothesis.
       
       \item rule \ruleName{a-update}. Then, $\systemterm''{\arro{\updatetranaux{\anupd^{\vertice[v']}}}[\cidj]}\systemterm'$
	and     
        $\op = \op'\upd{\vertice[v']}\anupd$ with $\anupd\in\updatesets$. Since $\vertice[v']$ is the last event then $\vertice[v] \neq \vertice[v']$. There are two possibilities:
		\begin{itemize}
			\item {\bf $\vertice[w] \neq \vertice[v'] $}. Then, the case follows  analogously to the previous case by using inductive hypothesis. 
						%Then we can use inductive hypothesis, so that it is easy to see that if $u^x \in\  \queuemessage[0..\tknown_i-1] \cdot \tpending_i \cdot [\ttransactionbuffer_i]$ then $u^x \in\  \queuemessage[0..\tknown_i-1] \cdot \tpending_i \cdot [\ttransactionbuffer_i] \cdot u_{t}^v$ too.
			
			\item  {\bf $\vertice[w] = \vertice[v'] $}. Note that $\systemterm''{\arro{\updatetranaux{\anupd^{\vertice[v']}}}[\cidj]}\systemterm'$ 
			implies  $\ttransactionbuffer_{\cidj} = \ttransactionbuffer'_{\cidj}\cdot \anupd^{\vertice[v']}$. Then, for all client $\cid$,
			$\anupd^{\vertice[v']}\notin\  \queuemessage[0..\tknown_{\cidj}-1]$, hence, the thesis trivially holds. 
		\end{itemize}
       
        \item rule�\ruleName{a-arb}. Then, $\queuemessage \neq \queuemessage'$. By rule inspection, the only possibility is 
        $\systemterm''\arro{\tau}[\cidj]\systemterm'$ by using rule \ruleName{send}. The proof follows from inductive hypothesis because $\tknown_{\cid}'=\tknown_{\cid}$	and $\so'=\so$ and $\op'=\op$.

       
        \item rule \ruleName{a-int}. We note that $\so' = \so$, $\op' = \op$ and $\queuemessage = \queuemessage'$. Then, we proceed by case analysis on 
        the reduction $\systemterm''\arro{\tau}[{\cidj}]\systemterm'$
            \begin{itemize}
		 \item  rule \ruleName {pull}. \chnote{esto me quedo incompleto} Then,  $\tknown_{\cidj} = \tknown_{\cidj}'+\treceivebuffer_{\cidj}'$, while  $\tknown_{\cid} = \tknown_{\cid}'$
		for all $\cid \neq \cidj$.
		By inductive hypothesis, there exists $\cid$ s.t.
		if $\anupd_0^{\vertice[w]}\in\queuemessage'[0..k_{\cid}^{'}]$
	then there exists $\anupd_1$ s.t. $\anupd_1^{\vertice}\in\queuemessage'[0..k_{\cid}^{'}-1]$.
	
	
		 When $\cid = \cidj$, we show that 
		 $\anupd_0^{\vertice[w]} \in\  \queuemessage[0..\tknown'_{\cidj}]$ then there exists $\anupd_1$ s.t. $\anupd_1^{\vertice}\in\queuemessage'[0..k_{\cid}^{'}-1]$ %implies
		 %if $\anupd_0^{\vertice[w]} \in\  \queuemessage[0..\tknown_{\cidj}]$ =
		 %$\queuemessage[0..\tknown'_{\cidj}+ \treceivebuffer'_{\cidj}]$ then, there exists $\anupd_1$ s.t. $\anupd_1^{\vertice}\in\queuemessage[0..k_{\cidj}-1]$ = $\queuemessage'[0..k_{\cidj}^{'}+\treceivebuffer'_{\cidj}-1]$ 
		%\begin{itemize}
		%	\item If $\udec \in\ \tpending_{\cidj}'\  \land\ \udec \notin\ \queuemessage[\tknown'_{\cidj}..\tknown'_{\cidj}+ \treceivebuffer'_{\cidj}-1]$, 
		%	then $\udec \in \ \tpending_{\cidj}$.
		%	\item If $\udec \in\ \tpending'_{\cidj}\  \land\ \udec \in\ \queuemessage[\tknown'_{\cidj}..\tknown'_{\cidj}+ \treceivebuffer'_{\cidj}-1]$, 
		%		then $\udec \notin \ \tpending_{\cidj}$ but 
		%		$\udec \in\ \queuemessage[0 .. \tknown'_{\cidj}+ \treceivebuffer'_{\cidj}-1]=\queuemessage[0 .. \tknown_{\cidj} -1] $. 
		%\end{itemize}
		%The remaining cases ($\udec \in \queuemessage[0..\tknown'_{\cidj}-1]$ or $\udec\in\ttransactionbuffer'_{\cidj}$) are straightforward.
		
		\item rules \ruleName{push},\ruleName{receive}, \ruleName{confirm}, \ruleName{while-true}, \ruleName{while-false}, 
		The proof follows from inductive hypothesis because $\tknown_{\cid}'=\tknown_{\cid}$, $\so'=\so$ and $\op'=\op$.

	\end{itemize}
	
         
	\end{itemize}

	\end{itemize}
\qed

\end{proof}

\begin{lemma}
\label{lemma:update-ever-belong-different-client} 
Let $\systemterm$ a \gsp\ system. If 
$\environment{\systemterm}{\emptyset}{\emptyset}{\emptyset}{\emptyset}[\emptyset] \tr{} ^*\ \environment{\systemterm[']}{\op}{\so}{\vis}{\arb}$
then for all $(\vertice[w],\vertice)\in\vis$ if $\vertice[w]\in\sse(\cid)$, 
$\vertice\in\sse(\cidj)$ and $\cid\neq\cidj$ then
 there exists $\anupd[^{\vertice[w]}]$ s.t. $\anupd[^{\vertice[w]}] \in\  \queuemessage[0..\tknown_{\cidj}-1]$.
\end{lemma}



\begin{proof}It follows by induction on the length of
the derivation  $\tr{} ^*$.
\begin{itemize}
   \item{\bf n=0}. It holds trivially, because $\vis=\emptyset$.
      \item{\bf n=k+1}. Then $\environment{\systemterm}{\emptyset}{\emptyset}{\emptyset}{\emptyset}[\emptyset]
    \tr{} ^k\ \environment{\systemterm''}{\op'}{\so'}{\vis'}{\arb'}[\sse'] 
    \arroi{\lambda} \environment{\systemterm'}{\op}{\so}{\vis}{\arb}[\sse]$. 
    By inductive hypothesis, for all $(\vertice[w],\vertice)\in\vis'$ if $\vertice[w]\in\sse'(\cid)$, 
$\vertice\in\sse'(\cidj)$ and $\cid\neq\cidj$ then
 there exists $\anupd[^{\vertice[w]}]$ s.t. $\anupd[^{\vertice[w]}] \in\  \queuemessage'[0..\tknown'_{\cidj}-1]$.
    We proceed by 
    case analysis on the last applied rule:
	
    \begin{itemize}
	 \item rule \ruleName{a-read}. Then,
	  \[\vis= \vis' \cup  \vis_{\mbox{\sf new}} \ \mbox{with} \ 
	  \vis_{\mbox{\sf new}} = (\{\vertice[w] \ |\  \udec[@][{\vertice[w]}] \in \queuemessage'[0..\tknown'_{\,\cidj}-1] \cdot \tpending'_{\cidj} \cdot \ttransactionbuffer'_{\,\cidj} \}\times\{\vertice\})\]
	  %
	  Therefore, 
	 %
	 if $(\vertice[w],\vertice)\in\vis'$,	follow immediately by inductive hypothesis after noticing that $\tknown_{\cidj} = \tknown'_{\cidj}$, $\queuemessage = \queuemessage'$.
	 It remains to prove that $(\vertice[w],\vertice)\in \vis_{\mbox{\sf new}}$. Since, $\vertice[w]\in\sse(\cid)$, 
$\vertice\in\sse(\cidj)$ and $\cid\neq\cidj$ then $\anupd[^{\vertice[w]}] \notin \tpending'_{\cidj} \cdot \ttransactionbuffer'_{\,\cidj}$.
	Then, $\anupd[^{\vertice[w]}] \in \queuemessage[0..\tknown_{\,\cidj}-1]$.
	
	\item rules \ruleName{a-upd},\ruleName{a-arb} and \ruleName{a-int} follow immediately by inductive hypothesis
	after noticing that  $\vis = \vis'$.
	\end{itemize}

\end{itemize}\qed
\end{proof}
\begin{lemma}
\label{lem:server-reflect-arb}
 Let $\systemterm$ a \gsp\ system. If  
 $\environment{\systemterm}{\emptyset}{\emptyset}{\emptyset}{\emptyset}[\emptyset] \tr{} ^*\ \environment{\systemterm'}{\op}{\so}{\vis}{\arb}$
 then 
for all $(\vertice[w],\vertice)\in\arb$, if $\anupd_0^{\vertice}\in\queuemessage[0..m]$ then there exists  $\anupd_1$ s.t 
$\anupd_1^{\vertice[w]}\in\queuemessage[0..m-1]$.
\end{lemma}

\begin{proof} By induction on the length of the derivation $\tr{} ^*$.
\begin{itemize}
   \item{\bf n=0}. It holds trivially, because $\arb=\emptyset$.
      \item{\bf n=k+1}. Then $\environment{\systemterm}{\emptyset}{\emptyset}{\emptyset}{\emptyset}[\emptyset]
    \tr{} ^k\ \environment{\systemterm''}{\op'}{\so'}{\vis'}{\arb'}[\sse'] 
    \arroi{\lambda} \environment{\systemterm'}{\op}{\so}{\vis}{\arb}[\sse]$. 
    By inductive hypothesis, for all $(\vertice[w],\vertice)\in\arb'$, if $\anupd_0^{\vertice}\in\queuemessage'[0..m]$ then there exists  $\anupd_1$ s.t 
$\anupd_1^{\vertice[w]}\in\queuemessage'[0..m-1]$.
    We proceed by 
    case analysis on the last applied rule:
	
    \begin{itemize}
	 \item rule \ruleName{a-update}. Then,
	         \[\arb = \arb' \cup \arb_{\mbox{\sf new}} \ \mbox{with} \ \arb_{\mbox{\sf new}}= (\{ \vertice[w] \ |\ \udec[@][{\vertice[w]}]\in \queuemessage' \}\times\{\vertice\})\]
        
        if $(\vertice[w],\vertice)\in\arb'$,	follow immediately by inductive hypothesis  after noticing that $\tknown_{\cidj} = \tknown'_{\cidj}$ and $\queuemessage = \queuemessage'$.
        It remains to prove that $(\vertice[w],\vertice)\in\arb_{\mbox{\sf new}}$.
        Since $\vertice$ is fresh, $\anupd_0^{\vertice}\notin\queuemessage[0..m]$. Hence, it holds.
        
          \item rule \ruleName{a-arb}. Then, $\queuemessage = \queuemessage'\cdot\ublock[{\udec[{\anupd[_0]}][{\vertice[@][_0]}]}\cdots{\udec[{\anupd[_n]}][{\vertice[@][_n]}]}]$, $\so = \so'$, 
        $\arb = \arb' \cup \arb_{\mbox{\sf new}}$ with 
         $\arb_{\mbox{\sf new}}= \{ (\vertice, \vertice[w])\ \ |\  \  \vertice[w]\in\dom{\op}, \ \op(\vertice[w])\in\updatesets, \
		     \forall\anupd_0. \anupd_0^{\vertice[w]}\not\in\queuemessage'\cdot\ublock[{\udec[{\anupd[_0]}][{\vertice[@][_0]}]}\cdots{\udec[{\anupd[_n]}][{\vertice[@][_n]}]}]\} 
		     \cup\  \{(\vertice[@][_i],\vertice[@][_j]) \ |\ \vertice[@][_i],\vertice[@][_j]\in \{\vertice[@][_0],\ldots,\vertice[@][_n]\}, i<j\}
		     $
       
       By inductive hypothesis, if $(\vertice[w],\vertice)\in\arb'$ then it holds.
       It remains to prove that for all $(\vertice[w],\vertice)\in\arb_{\mbox{\sf new}}$ if $\anupd_0^{\vertice}\in\queuemessage[0..m]$ then there exists  $\anupd_1$ s.t 
$\anupd_1^{\vertice[w]}\in\queuemessage[0..m-1]$.
       There are two cases:
       \begin{itemize}
       \item $(\vertice[w],\vertice[v]) \in \ (\{\vertice[w][_0],\ldots,\vertice[w][_n]\}\times 
		\{ \vertice\ \ |\ \  \vertice\in\mathbb{U}, \
		     \forall\anupd. \anupd^{\vertice}\not\in\queuemessage'\cdot\anupdseq\})$ with $\anupdseq=\ublock[{\udec[{\anupd[_0]}][{\vertice[w][_0]}]}\cdots{\udec[{\anupd[_n]}][{\vertice[w][_n]}]}]$,then,  there is not $\anupd_0$ s.t.
       $ \anupd_0^{\vertice[y]}\not\in\queuemessage''$. 
       Therefore, it holds.
       \item $(\vertice,\vertice[y]) \in \{(\vertice[@][_i],\vertice[@][_j]) \ |\ \vertice[@][_i],\vertice[@][_j]\in \{\vertice[@][_0],\ldots,\vertice[@][_n]\}, i<j\} $ follow immediately by \lemref{lemma:upd-in-buffers-updinSO}. \chnote{esto no s� si es suficiente justificacion. El hecho de estar en el session order significa q hay un orden entre los vertices que estaban en los buffer}
\end{itemize}
        
	\item rules \ruleName{a-read} and \ruleName{a-int} follow immediately by inductive hypothesis
	after noticing that  $\arb = \arb'$.
	\end{itemize}

\end{itemize}\qed
\end{proof}

\begin{lemma}
\label{lemma:ar-stable-updates} 
Let $\systemterm$ a \gsp\ system. If
$\environment{\queuemessage\bigpar\Absclient}{\emptyset}{\emptyset}{\emptyset}{\emptyset}[\emptyset] 
\tr{} ^*\ \environment{\queuemessage'\bigpar\Absclient'}{\op}{\so}{\vis}{\arb}$ then
 if
$(\vertice,\vertice[w])\in\arb$, then
% $\anupd^{\vertice[w]} \in \queuemessage'$ implies 
there exists $\anupd$ s.t. 
$\anupd^{\vertice} \in \queuemessage'$.
\end{lemma}

\begin{proof} By straightforward induction on the  length of the derivation. \qed
\end{proof}




\begin{lemma}
\label{lemma:sovis-in-vis} 
Let $\systemterm$ a \gsp\ system. For all  $\systemterm[']$, if 
\[\environment{\systemterm}{\emptyset}{\emptyset}{\emptyset}{\emptyset}[\emptyset] \tr{} ^*\ \environment{\systemterm[']}{\op}{\so}{\vis}{\arb}\]
then $\restUR{(\so;\vis)}\subseteq \vis$.
\end{lemma}

\begin{proof} It follows by induction on the length of
the derivation  $\tr{} ^*$.
% that for all $(\vertice,\vertice[x])\in\so$ if $(\vertice[x],\vertice[w])\in\vis$ then  $(\vertice,\vertice[w])\in\vis$.
\begin{itemize}
   \item{\bf n=0}. It holds trivially, because $\op=\vis=\emptyset$.
      \item{\bf n=k+1}. Then $\environment{\systemterm}{\emptyset}{\emptyset}{\emptyset}{\emptyset}[\emptyset]
    \tr{} ^k\ \environment{\systemterm''}{\op'}{\so'}{\vis'}{\arb'}[\sse'] 
    \arroi{\lambda} \environment{\systemterm'}{\op}{\so}{\vis}{\arb}[\sse]$. 
    By inductive hypothesis, $\restUR{(\so';\vis')}  \subseteq \ \vis'$.
    We proceed by 
    case analysis on the last applied rule:
	
    \begin{itemize}
	\item rule \ruleName{a-update}. Then, $\vis=\vis'$ and $\so = \so'\cup (\sse(\cid)\times\{\vertice\})$ with $\op(\vertice)\in\updatesets$. Therefore,
	 \[\restUR{(\so;\vis)} = \restUR{((\so'\cup (\sse(\cid)\times\{\vertice\}));\vis')} \] 
	 Note that $(\sse(\cid)\times\{\vertice\});\vis' = \emptyset$ because $\vertice$ is fresh and does not appears in $\vis'$. Then, by 
	 using inductive hypothesis,
	 \[\restUR{(\so;\vis)} = \restUR{(\so';\vis')} \subseteq\vis' = \vis \] 
	 
	 \item rule \ruleName{a-read}. Then, $\so = \so'\cup  \so_{\mbox{\sf new}}$ with  
	  $\so_{\mbox{\sf new}} = (\sse'(\cid)\times\{\vertice\})$, 
	  %
	 \[\vis= \vis' \cup  \vis_{\mbox{\sf new}} \ \mbox{with} \ 
	  \vis_{\mbox{\sf new}} = (\{\vertice[w] \ |\  \udec[@][{\vertice[w]}] \in \queuemessage''[0..\tknown''_{\,\cid}-1] \cdot \tpending''_{\cid} \cdot \ttransactionbuffer''_{\,\cid} \}\times\{\vertice\})\]
	  %
	  and $\op(\vertice)\in\readset$. Therefore, 
	 %
	 \[\restUR{(\so;\vis)} = \restUR{(( \so'\cup  \so_{\mbox{\sf new}}); (\vis'\cup	\vis_{\mbox{\sf new}}))}\]
	Note that $ \so_{\mbox{\sf new}}; (\vis\cup	\vis_{\mbox{\sf new}})) = \emptyset$ because the second component 
	of each pair in  $ \so_{\mbox{\sf new}}$ is the fresh vertex $\vertice$. Hence,
	%
		 \[\restUR{(\so;\vis)} = \restUR{(\so'; (\vis'\cup	\vis_{\mbox{\sf new}}))} = \restUR{(\so';\vis')} \cup\restUR{(\so';\vis_{\mbox{\sf new}})}  \]
	By inductive hypothesis,  $\restUR{(\so';\vis')}\subseteq \vis'\subseteq\vis$. 
	It remains to prove that $\restUR{(\so';\vis_{\mbox{\sf new}})}\subseteq\vis$.
	Then,
		    for all $(\vertice[x],\vertice[y])\in\restUR{(\so';\vis_{\mbox{\sf new}})}$, 
		    $\op(\vertice[x])\in\updatesets$ and $\vertice[y] = \vertice$. Moreover, 
	     there exists $\vertice[z]$ s.t. $(\vertice[x],\vertice[z])\in\so'$ and $(\vertice[z],\vertice)\in\vis_{\mbox{\sf new}}$.
	    Since $(\vertice[z],\vertice[w])\in\vis_{\mbox{\sf new}}$, $\op(\vertice[z])\in\updatesets$. Moreover, 
	    $(\vertice[x],\vertice[z])\in\so'$ implies  $\{\vertice,\vertice[x]\} \in \sse(\cidj)$ for 
	    some $\cidj$. There are two cases, 
	    \begin{itemize}
	    	\item $\cidj = \cid$. Then, it follows from \lemref{lemma:update-ever-belong} that
		$\anupd^{\vertice[x]} \in\  \queuemessage[0..\tknown_{\cid}-1] \cdot \tpending_{\cid} \cdot [\ttransactionbuffer_{\cid}]$.
		Hence $(\vertice[x],\vertice)\in\restUR{\vis_{\mbox{\sf new}}} \subseteq \vis$.
		\item $\cidj \neq \cid$. Then, $\anupd^{\vertice[x]} \in\  \queuemessage[0..\tknown_{\cid}-2]$ by
		\lemref{lemma:local-updates-preserver-order}. Therefore  $(\vertice[x],\vertice)\in\restUR{\vis_{\mbox{\sf new}}} \subseteq \vis$.
	    \end{itemize}
	   	
	
	
	
	\item rules \ruleName{a-arb} and \ruleName{a-int} follow immediately by inductive hypothesis
	after noticing that  $\vis = \vis'$ and $\so = \so'$.
	\end{itemize}

\end{itemize}\qed
\end{proof}


\begin{lemma} 
\label{lem:so-transitive}
Let $\systemterm$ a \gsp\ system. For all  $\systemterm[']$, if 
\[\environment{\systemterm}{\emptyset}{\emptyset}{\emptyset}{\emptyset}[\emptyset] \tr{} ^*\ \environment{\systemterm[']}{\op}{\so}{\vis}{\arb}\]
then $\so$ is transitive.
\end{lemma}


\begin{proof}
It follows by induction on the length of
the derivation  $\tr{} ^*$.
\begin{itemize}
   \item{\bf n=0}. It holds trivially, because $\so=\emptyset$.
      \item{\bf n=k+1}. Then $\environment{\systemterm}{\emptyset}{\emptyset}{\emptyset}{\emptyset}[\emptyset]
    \tr{} ^k\ \environment{\systemterm''}{\op'}{\so'}{\vis'}{\arb'}[\sse'] 
    \arroi{\lambda} \environment{\systemterm'}{\op}{\so}{\vis}{\arb}[\sse]$. 
    By inductive hypothesis, $\so'$ is transitive.
    We proceed by 
    case analysis on the last applied rule:
	
    \begin{itemize}
	\item rule \ruleName{a-update}. Then, $\so = \so'\cup (\sse'(\cid)\times\{\vertice\})$ for all $\cid$. Therefore, we have to prove that
	for all $(\vertice[x],\vertice[y])\in\so$ and $(\vertice[y],\vertice[z])\in\so$ then $(\vertice[x],\vertice[z])\in\so$. Since we are analysing the last rule, then
	$(\vertice[x],\vertice[y])\in\so'$. There are two cases:
	\begin{itemize}
	\item $\vertice \neq \vertice[z]$. Then, it follow immediately by inductive hypothesis.
	\item $\vertice = \vertice[z]$. Since $(\vertice[x],\vertice[y])\in\so'$ then $\vertice[x]\in \sse'(\cid)$. Therefore,  $(\vertice[x],\vertice[z])\in\so$.
	\end{itemize} 
	 \item rule \ruleName{a-read}. The rule follows analogously to the previous rule.
	
	\item rules \ruleName{a-arb} and \ruleName{a-int} follow immediately by inductive hypothesis
	after noticing that  $\so = \so'$.
	
	\chnote{esta prueba no me salio sino es a la "algo3"}
\end{itemize}

\end{itemize}\qed
\end{proof}

\begin{lemma}
\label{lem:soUvis-acyclic}
  If $\environment{\systemterm}{\emptyset}{\emptyset}{\emptyset}{\emptyset}[\emptyset] \tr{} ^*\ \environment{\systemterm'}{\op}{\so}{\vis}{\arb}$ 
then $(\so \cup \vis )$ is acyclic.
\end{lemma}

\begin{proof}
The proof follows induction on the length of the derivation  $\environment{\systemterm}{\emptyset}{\emptyset}{\emptyset}{\emptyset}[\emptyset] \tr{} ^n\ \environment{\systemterm'}{\op}{\so}{\vis}{\arb}$. Inductive step follows by case analysis on the last applied rule. Rules that change the
relations $\so$ and $\vis$ are  \ruleName{a-update} and  \ruleName{a-read}. The fact that $(\so \cup \vis )$ follows from the fact that the 
reduction step only add pairs $(\vertice,\vertice[w])$ with $\vertice[w]$. Then, no cycles can be introduced.
\henote{low priority: escribir mejor}\qed
\end{proof}


\begin{lemma}
\label{lemma:upd-in-buffers-updinSO} 
Let $\systemterm$ a \gsp\ system. 
If 
$\environment{\systemterm}{\emptyset}{\emptyset}{\emptyset}{\emptyset}[\emptyset] \tr{} ^*\ \environment{\systemterm[']}{\op}{\so}{\vis}{\arb}$
then for all $m < n < |\flatten{\tpending_{\cid}} \cdot \ttransactionbuffer_{\cid}|$ if $\anupd[^{\vertice[w]}_0]= \flatten{\tpending_{\cid}} \cdot \ttransactionbuffer_{\cid}[m]$ {and $\anupd[^{\vertice}_1]=\flatten{\tpending_{\cid}} \cdot \ttransactionbuffer_{\cid}[n]$} then
 $(\vertice[w],\vertice)\in\so$.
\end{lemma}
\begin{proof} It follows by induction on the length of
the derivation  $\tr{} ^*$.
% that for all $(\vertice,\vertice[x])\in\so$ if $(\vertice[x],\vertice[w])\in\vis$ then  $(\vertice,\vertice[w])\in\vis$.
\begin{itemize}
   \item{\bf n=0}. It holds, because $\tpending = \epsilon$ and $[\ttransactionbuffer] = \epsilon$ and $\op=\emptyset$.
      \item{\bf n=k+1}. Then $\environment{\systemterm}{\emptyset}{\emptyset}{\emptyset}{\emptyset}[\emptyset]
    \tr{} ^k\ \environment{\systemterm''}{\op'}{\so'}{\vis'}{\arb'}[\sse'] 
    \arroi{\lambda} \environment{\systemterm'}{\op}{\so}{\vis}{\arb}[\sse]$. 
    By inductive hypothesis, for all $m < n < |\flatten{\tpending_{\cid}} \cdot \ttransactionbuffer_{\cid}|$ if $\anupd[^{\vertice[w]}_0]=\flatten{\tpending_{\cid}'} \cdot \ttransactionbuffer_{\cid}'[m]$ {and $\anupd[^{\vertice[w']}_1]=\flatten{\tpending_{\cid}'} \cdot \ttransactionbuffer_{\cid}'[n]$} and $m < n$ then
 $(\vertice[w],\vertice[w'])\in\so'$.
    We proceed by 
    case analysis on the last applied rule:
	
    \begin{itemize}
	\item rule \ruleName{a-update}. Then, $\vis=\vis'$ and $\so = \so'\cup (\sse(\cid)\times\{\vertice\})$ with $\op(\vertice)\in\updatesets$. There are two cases, 
	    \begin{itemize}
	    	\item $\vertice[w'] = \vertice$. Since, $\systemterm'' \arroi{\updatetran{u}} \systemterm'$, by rule \textsc{update}, $\anupd[_{\vertice}] \in \flatten{\tpending} \cdot \ttransactionbuffer$, hence  $\ttransactionbuffer = \ttransactionbuffer' \cdot \anupd[_{\vertice}]$.
		\item $\vertice[w'] \neq \vertice$. Follow immediately by inductive hypothesis.
	    \end{itemize}
	 %
	  \item rule \ruleName{a-read}. Then, $\so = \so'\cup  \so_{\mbox{\sf new}}$ with  
	  $\so_{\mbox{\sf new}} = (\sse'(\cid)\times\{\vertice\})$, $\tpending = \tpending'$ and $\ttransactionbuffer = \ttransactionbuffer'$
	  %
	Since, $\op' = \op\upd \vertice r$ then $(\vertice[x],\vertice[y])\in\so'$. 
	\item rules \ruleName{a-int}, \ruleName{a-arb} follow immediately by inductive hypothesis
	after noticing that  $\so = \so'$.
	\end{itemize}
\chnote{dentro de las internas, el moviemiento que se genera por la push me parece trivial e inmediato, por eso no lo separ�}
\end{itemize}\qed
\end{proof}


\begin{lemma}
\label{lemma:arSO-in-arUso} 
Let $\systemterm$ a \gsp\ system. For all  $\systemterm[']$, if 
\[\environment{\systemterm}{\emptyset}{\emptyset}{\emptyset}{\emptyset}[\emptyset] \tr{} ^*\ \environment{\systemterm[']}{\op}{\so}{\vis}{\arb}\]
then $\arb;\restUU{\so}\subseteq \arb\cup\restUU{\so}$.
\end{lemma}

\begin{proof} We proceed by induction on the length of the derivation 
$\environment{\systemterm}{\emptyset}{\emptyset}{\emptyset}{\emptyset}[\emptyset] \tr{} ^n\ \environment{\systemterm[']}{\op}{\so}{\vis}{\arb}$.

\begin{itemize}
	    \item {\bf n=0}. Trivially, because $\arb=\so=\emptyset$.
    \item{\bf n=k+1}. Then 
      \[\environment{\systemterm}{\emptyset}{\emptyset}{\emptyset}{\emptyset}[\emptyset] 
         \tr{} ^k\ 
         \environment{\systemterm['']}{\op'}{\so'}{\vis'}{\arb'}[\sse'] 
         \arro{\lambda}[\cid]
 	\environment{\systemterm[']}{\op}{\so}{\vis}{\arb}
	 \]
	 
	By inductive hypothesis, $\arb';\restUU{\so'}\subseteq \arb'\cup\restUU{\so'}$. We proceed by 
case analysis on the rule applied for the last transition:
	
	\begin{itemize}
        
        \item rule \ruleName{a-read}. Then, $\arb=\arb'$  and $\so = \so' \cup (\sse(\cid)\times\{\vertice\})$ and $\op(\vertice)\in\readset$. 
        Hence, $\restUU{\so} =\restUU{\so'}$. Consequently,  
        \[\arb;\restUU{\so}=\arb';\restUU{\so'}\subseteq \arb'\cup\restUU{\so'}
        \]
        holds by inductive hypothesis.
        
        
        \item rule \ruleName{a-update}. $\so = \so'\cup (\sse(\cid)\times\{\vertice\})$, 
        $\arb = \arb' \cup \arb_{\mbox{\sf new}}$ with  $\arb_{\mbox{\sf new}}= (\{ \vertice[w] \ |\ \udec[@][{\vertice[w]}]\in \queuemessage' \}\times\{\vertice\})$
        and $\op(\vertice)\in\updatesets$.  Then

        \[\arb;\restUU{\so} = \arb';\restUU{\so}\  \cup\ \arb_{\mbox{\sf new}};\restUU{\so} 
        \]
        
        Since $\vertice$ is fresh, $\arb_{\mbox{\sf new}};\restUU{\so} =\emptyset$. Hence, 

        \[\arb;\restUU{\so} =  \arb';\restUU{\so} = \arb';\restUU{\so'}\cup\  \arb';\restUU{(\sse(\cid)\times\{\vertice\})}\]

       By inductive hypothesis, $\arb';\restUU{\so'}\ \subseteq\ {\arb'} \cup \restUU{\so'}\ \subseteq\ \arb�\cup \restUU{\so}$.
       It remains to prove that $\arb';\restUU{(\sse(\cid)\times\{\vertice\})}\ \subseteq\ \arb \cup \restUU{\so}$.
       If $(\vertice[x],\vertice[y]) \in \arb'$, then there exists $\anupd_0$ s.t.  ${\anupd_0}^{\vertice[x]}\in S'$ by
       \lemref{lemma:ar-stable-updates}.
       Therefore,
       $(\vertice[x],\vertice) \in \arb_{\mbox{\sf new}}$ and, therefore, $(\vertice[x],\vertice) \in \arb\ \subseteq\ \arb�\cup \restUU{\so}$.
       
       \item rule \ruleName{a-arb}. Then, $\queuemessage = \queuemessage'\cdot\ublock[{\udec[{\anupd[_0]}][{\vertice[@][_0]}]}\cdots{\udec[{\anupd[_n]}][{\vertice[@][_n]}]}]$, $\so = \so'$, and $\arb = \arb' \cup \arb_{\mbox{\sf new}}$ with 
         \[\arb_{\mbox{\sf new}}= \{(\vertice[@][_i],\vertice[@][_j]) \ |\ i,j\in \{0,\ldots,n\}, i<j\} \ \cup
         \ (\{\vertice[@][_0],\ldots,\vertice[@][_n]\}\times 
		\{ \vertice[w]\ \ |\ \  \vertice[w]\in\mathbb{U}, \
		     \forall\anupd. \anupd^{\vertice[w]}\not\in\queuemessage\cdot\anupdseq\})\]
        
        Then,         
        \[\arb;\restUU{\so} = \arb';\restUU{\so'}\  \cup\ \arb_{\mbox{\sf new}};\restUU{\so'} 
        \]
        
       By inductive hypothesis, $\arb';\restUU{\so'}\ \subseteq\ {\arb'} \cup \restUU{\so'}\ \subseteq\ \arb�\cup \restUU{\so}$.
       It remains to prove that $\arb_{\mbox{\sf new}};\restUU{\so'} \subseteq\ \arb \cup \restUU{\so}$. 
       For all $\vertice[y]$ s.t. $(\vertice,\vertice[y]) \in \arb_{\mbox{\sf new}}$ there are two cases:
       \begin{itemize}
       \item $(\vertice,\vertice[y]) \in \{(\vertice[@][_i],\vertice[@][_j]) \ |\ \vertice[@][_i],\vertice[@][_j]\in \{\vertice[@][_0],\ldots,\vertice[@][_n]\}, i<j\} $ follow immediately by \lemref{lemma:upd-in-buffers-updinSO}. \chnote{revisar si esto es suficiente, el hecho de usar el lema. Mismo caso que prueba anterior.}
       \item $(\vertice,\vertice[y]) \in \ (\{\vertice[@][_0],\ldots,\vertice[@][_n]\}\times 
		\{ \vertice[w]\ \ |\ \  \vertice[w]\in\mathbb{U}, \
		     \forall\anupd. \anupd^{\vertice[w]}\not\in\queuemessage\cdot\anupdseq\})
 $, with $\anupdseq=\ublock[{\udec[{\anupd[_0]}][{\vertice[@][_0]}]}\cdots{\udec[{\anupd[_n]}][{\vertice[@][_n]}]}]$,
then,  there is not $\anupd_0$ s.t.
       $ \anupd_0^{\vertice[y]}\not\in\queuemessage'$.  
       Moreover, for all $(\vertice[y],\vertice[z])\in \restUU{\so'}$ s.t. there is not $\anupd_0$ s.t.
       $ \anupd_0^{\vertice[y]}\not\in\queuemessage'$ implies that there is not $ \anupd_1^{\vertice[z]}\not\in\queuemessage'$
       because of (the contrapositive of) \lemref{lemma:local-updates-preserver-order}.
       Therefore, $(\vertice,\vertice[y]) \in \arb_{\mbox{\sf new}} \subseteq \arb \cup \restUU{\so}$.

        \end{itemize}

    \end{itemize}
	
    \end{itemize}
       \qed

\end{proof}



\subsection{Proof of Theorem ~\ref{theorem:properties}}

\begin{proof} [Read My Writes (\ref{theorem:read-my-writes})] The proof follows by induction on the length of the derivation $\tr{} ^*$.
\begin{itemize}
   \item{\bf n=0}. It holds trivially, because $\op=\vis=\emptyset$%, so that $\emptyset \subseteq \emptyset$.
   \item{\bf n=k+1}. Then $\environment{\systemterm}{\emptyset}{\emptyset}{\emptyset}{\emptyset}[\emptyset]
    \tr{} ^k\ \environment{\systemterm''}{\op'}{\so'}{\vis'}{\arb'}[\sse'] 
    \arroi{\lambda} \environment{\systemterm'}{\op}{\so}{\vis}{\arb}[\sse]$. 
    By inductive hypothesis, $\restUR{\so'}  \subseteq \ \vis'$.
    We proceed by 
    case analysis on the last applied rule:
	
    \begin{itemize}
	\item rule \ruleName{a-update}. Then, $\vis=\vis'$ and $\so = \so'\cup (\sse(\cid)\times\{\vertice\})$ with $\op(\vertice)\in\updatesets$. Therefore,
	 $\restUR{\so} = \restUR{\so'}\subseteq\vis' = \vis$.
	 
	 \item rule \ruleName{a-read}. Then, $\so = \so'\cup (\sse'(\cid)\times\{\vertice\})$ and 
	 \[\vis= \vis' 
	 \cup (\{\vertice[w] \ |\  \udec[@][{\vertice[w]}] \in \queuemessage''[0..\tknown''_{\,\cid}-1] \cdot \tpending''_{\cid} \cdot \ttransactionbuffer''_{\,\cid} \}\times\{\vertice\})\]
	  with $\op(\vertice)\in\readset$. Therefore, 
	 \[\restUR{\so} = \restUR{\so'}\cup \{(\vertice[w],\vertice)\ |\ \vertice[w] \in \sse'(\cid) \ \land \op'(\vertice[w])\in\updatesets\}\]
	
        % $\textsc{\small{SO}}_R$' = $\textsc{\small{SO}}_R \cup\ \{(w,v) / \{w \mapsto \readtran{r}\} \lor\ \{w \mapsto \ \updatebyclient{j}\} \}$. Applying the intersection $(\mathbb{U}\ \times \ \mathbb{R})$, we shall obtain $\textsc{\small{SO}}_R \ \cup\ \{(w,v) / \{w \mapsto \updatebyclient{j}\} \}$.				
%				\item $\vis' = \vis \ \cup \ \{ (x,v) / \{x \mapsto \updatebyclient{h}\} \in \op \ \land \  u^x \in \ \queuemessage[0..\tknown-1] \cdot \tpending \cdot [\ttransactionbuffer] \}$. 
	By inductive hypothesis, $\restUR{\so'} \subseteq \vis'\subseteq \vis$. It remains to prove that 
	\[
	 \{(\vertice[w],\vertice)\ |\ \vertice[w] \in \sse'(\cid) \ \land \op'(\vertice[w])\in\updatesets\} \subseteq \vis
	\]
	Actually, we show that 
	\[
	\{(\vertice[w],\vertice)\ |\ \vertice[w] \in \sse'(\cid) \ \land \op'(\vertice[w])\in\updatesets\} 
	\subseteq
	(\{\vertice[w] \ |\  \udec[@][{\vertice[w]}] \in \queuemessage''[0..\tknown''_{\,\cid}-1] \cdot \tpending''_{\cid} \cdot \ttransactionbuffer''_{\,\cid} \}\times\{\vertice\})
	\]
	
	%We only have to prove that $\{(w,v) / \{w \mapsto \updatebyclient{j}\} \in \op  \ \} \subseteq \{ (x,v) / \{x \mapsto \updatebyclient{h}\} \in \op \ \land \  u^x \in \ \queuemessage[0..\tknown-1] \cdot \tpending \cdot [\ttransactionbuffer] \}$. 
	By using \lemref{lemma:update-ever-belong}, which allows us to conclude that $\vertice[w] \in \sse'(\cid) \ \land \op'(\vertice[w])\in\updatesets$ 
	implies
	$\vertice[w] \in \queuemessage''[0..\tknown''_{\,\cid}-1] \cdot \tpending''_{\cid} \cdot \ttransactionbuffer''_{\,\cid}$
	
	\item rules \ruleName{a-arb} and \ruleName{a-int} follow immediately by inductive hypothesis
	after noticing that  $\vis = \vis'$ and $\so = \so'$.
	\end{itemize}
\end{itemize}\qed
\end{proof}



\begin{proof} [Monotonic Read (\ref{theorem:monotonic-read})] The proof follows by induction on the length of the derivation $\tr{} ^*$.
\begin{itemize}
   \item{\bf n=0}. It holds trivially, because $\op=\vis=\emptyset$.
   \item{\bf n=k+1}. 
   Then $\environment{\systemterm}{\emptyset}{\emptyset}{\emptyset}{\emptyset}[\emptyset]
    \tr{} ^k\ \environment{\systemterm''}{\op'}{\so'}{\vis'}{\arb'}[\sse'] 
    \arroi{\lambda} \environment{\systemterm'}{\op}{\so}{\vis}{\arb}[\sse]$. 
    By inductive hypothesis, $\restUR{(\vis';\so')}  \subseteq \ \vis'$.
    
    We proceed by 
    case analysis on the last applied rule.
    
%%    to show that
%%    for all $\vertice,\vertice[w],\vertice[x]$, if $(\vertice,\vertice[w])\in\vis$ and $(\vertice[w],\vertice[x])\in\so$  then $(\vertice,\vertice[w])\in\vis$.
%%    
%%    
%%        Moreover $(\vertice,\vertice[w]) \in \restUR{\vis;\so} $ iff $\ \exists \vertice[w] \in\ \verticesets\ $ 
%%    such that $(\vertice,\vertice[w]) \in\ \vis \land \ (\vertice[w],\vertice[x]) \in\  \so$. We have to prove that $(\vertice,\vertice[x]) \in \ \textsc{\small{VIS}}$'. We proceed by 
%%case analysis on the last transition:
%
%   Then $\environment{\systemterm_0}{\emptyset}{\emptyset}{\emptyset}{\emptyset} \arro{} ^n\ \environment{\systemterm}{\op}{\so}{\vis}{\arb} \arroi{\alpha} \environment{\systemterm'}{\op'}{\so'}{\vis'}{\arb}$. Let $R$ be a composition of relations. 

%case analysis on the last transition:
	
	\begin{itemize}
        \item rule \ruleName {a-read}. 
        Then, $\so = \so'\cup (\sse'(\cid)\times\{\vertice\})$ and 
	 \[\vis= \vis' 
	 \cup \vis_{\mbox{\sf new}} \ \mbox{with} \ 
	  \vis_{\mbox{\sf new}} = (\{\vertice[w] \ |\  \udec[@][{\vertice[w]}] \in \queuemessage''[0..\tknown''_{\,\cid}-1] \cdot \tpending''_{\cid} \cdot \ttransactionbuffer''_{\,\cid} \}\times\{\vertice\})\]
	  with $\op(\vertice)\in\readset$. Therefore, 
	  
	  \[\vis;\so = \vis';\so \cup\  \vis_{\mbox{\sf new}};\so\] 
	  
	  Since $\vertice$ is fresh, $\vertice\notin\sse'(\cid)$ and there is no $\vertice[@][']$ s.t. $(\vertice, \vertice[@]['])\in \so'$.
	  Hence, there is no  $\vertice[@][']$ s.t. $(\vertice, \vertice[@]['])\in \so$. Therefore,  $\vis_{\mbox{\sf new}};\so = \emptyset$.
	  It remains to prove that $\restUR{(\vis';\so)} \subseteq \vis$. 
	 \[\vis';\so = \vis';\so'\cup \vis';(\sse'(\cid)\times\{\vertice\})\]
	
	By inductive hypothesis,  $\vis';\so'\subseteq\vis'$. By definition,  $\vis'\subseteq\vis$. Hence, $\vis';\so'\subseteq\vis$ holds. 
	It remains to prove that 
        	\[
	 \restUR{(\vis';(\sse'(\cid)\times\{\vertice\}))} \subseteq \vis
	\]

        Take  $(\vertice[x],\vertice[y]) \in \vis'$ and $(\vertice[y],\vertice) \in (\sse'(\cid)\times\{\vertice\})$. Then, $\vertice[y]\in\sse'(\cid)$, 
        $\op'{(\vertice[y])}\in\readset$ and $\op'({\vertice[x]})\in\updatesets$. There are two cases: 
         \begin{itemize}
         	\item  $\vertice[x]\in\sse'(\cid)$. By \lemref{lemma:update-ever-belong}, there exists
		$\udec[@][{\vertice[x]}] \in\  \queuemessage''[0..\tknown''_{\cid}-1] \cdot \tpending''_{\cid} \cdot [\ttransactionbuffer''_{\cid}]$. 
		Consequently, $(\vertice[x],\vertice) \in\vis$.
		\item  $\vertice[x]\notin\sse'(\cid)$. By \lemref{lemma:update-ever-belong-different-client},
		there exists an update $\anupd$ 
		such that $\anupd^{\vertice[x]} \in\ \ \queuemessage[0..\tknown_{\cid}-1]$. Therefore
		$\udec[@][{\vertice[x]}] \in\  \queuemessage''[0..\tknown''_{\cid}-1] \cdot \tpending''_{\cid} \cdot [\ttransactionbuffer''_{\cid}]$. 
		Consequently, $(\vertice[x],\vertice) \in\vis$.

	\end{itemize}
        
%        By \lemref{}, there exists an update $\anupd$ 
%	such that $\anupd^{\vertice[x]} \in\ \ \queuemessage[0..\tknown_{\cidj}-1]$. Therefore
%						  
%						 Substituting $w$ by $v$ in $\textsc{\small{VIS}}$', we prove that $(\vertice,\vertice[x]) \in \textsc{\small{VIS}}$'.
%						\item $\vertice[x] \neq v$. This case follows immediately by inductive hypothesis.
%
%             % $\textsc{\small{SO}}_R$' = $\textsc{\small{SO}}_R \cup\ \{(w,v) / \{w \mapsto \readtran{r}\} \lor\ \{w \mapsto \ \updatebyclient{j}\} \}$. Applying the intersection $(\mathbb{U}\ \times \ \mathbb{R})$, we shall obtain $\textsc{\small{SO}}_R \ \cup\ \{(w,v) / \{w \mapsto \updatebyclient{j}\} \}$.				
%%				\item $\vis' = \vis \ \cup \ \{ (x,v) / \{x \mapsto \updatebyclient{h}\} \in \op \ \land \  u^x \in \ \queuemessage[0..\tknown-1] \cdot \tpending \cdot [\ttransactionbuffer] \}$. 
% 
%        
%        $\ \exists \vertice[w] \in\ \verticesets\ $ such that $(\vertice,\vertice[w]) \in\ \textsc{\small{VIS}}' \land \ (\vertice[w],\vertice[x]) \in\  \textsc{\small{SO}}_R$'. We have to prove that $(\vertice,\vertice[x]) \in \ \textsc{\small{VIS}}$'. We proceed by 
%
%        We know that $v$ is fresh, therefore, $v$ have not be neither $\vertice$ nor $\vertice[w]$ because there exists $\vertice[x]$ such that, $\vertice[x]$ happens after from $\vertice$ and $\vertice[w]$. There are only two possibilities:
%					
%					\begin{itemize}
%						\item $\vertice[x] = v$. As $(\vertice[w],\vertice[x]) \in \ \textsc{\small{SO}}_R$, then $\vertice[w]$ and $\vertice[x]$ 
%						are from the same client, called $i$. We are only interested in relations of $update \times\ read$. We know that
%						 $v$ is associated to an read action besides $\vertice$ have to be an update action. 					\end{itemize}
%					
			  \item rule \ruleName{a-update}. 
			  Then $\so = \so'\cup (\sse'(\cid)\times\{\vertice\})$, $\vis= \vis'$  and $\op(\vertice)\in\updatesets$.
			  By definition of $\so$
			  \[\restUR{(\vis;\so)} = \restUR{(\vis;(\so'\cup(\sse'(\cid)\times\{\vertice\})))}\] 
			  Since $\op(\vertice)\in\updatesets$, 
			  \[\restUR{(\vis;\so)} = \restUR{(\vis;\so')}\]
			  By using inductive hypothesis, we conclude  
			  \[\restUR{(\vis;\so)} = \restUR{(\vis;\so')} \subseteq \vis\]
%			  			   Visibility relation does not change,i.e.,$\textsc{\small{VIS}}$ = $\textsc{\small{VIS}}$'. Let $v$ be a vertex associated an update action, then $\textsc{\small{SO}}_R$' = $\textsc{\small{SO}}_R \cup\ \{(w,v) / \{w \mapsto \readtran{r}\} \lor\ \{w \mapsto \ \updatebyclient{j}\} \}$. As we are only interested in relations of $update \times\ read$, then $(\textsc{\small{VIS}}';\textsc{\small{SO}'}_R) \cap \ (\mathbb{U}\ \times \ \mathbb{R}) \equiv \ (\textsc{\small{VIS}};\textsc{\small{SO}}_R) \cap \ (\mathbb{U}\ \times \ \mathbb{R})$. By inductive hypothesis we can prove that $(\textsc{\small{VIS}}';\textsc{\small{SO}}_R)' \cap \ (\mathbb{U}\ \times \ \mathbb{R}) \ \subseteq \ \textsc{\small{VIS}} \subseteq \ \textsc{\small{VIS}}$'.				
%				The proof for the remaining cases follow are not interesting because the relations does not change.
	\item rules \ruleName{a-arb} and \ruleName{a-int} follow immediately by inductive hypothesis
	after noticing that  $\vis = \vis'$ and $\so = \so'$.
	
\end{itemize}

\end{itemize}\qed
\end{proof}


\begin{proof}[No Circular Causality (\ref{thm:no-circular-causality})]

By \lemref{lem:soUvis-acyclic},
 $(\so \cup \vis )$ is acyclic. Then, the proof follows from the fact that
  the transitive closure of an 
acyclic relation is a strict partial order.
\qed
\end{proof}




 
 
\begin{proof}[Causal Visibility (\ref{thm:causal-visibility})]
	We first show by induction on $n$   that 
	   %\[\restUO{(\so \cup \vis ) ^{+}}  =  \[\bigcup_{n=1}^{\infty}
	   \[ \restUO{(\so \cup \vis )^{n}}\ \  \subseteq\ \  \so  \cup \vis\]

	\begin{itemize}
	    \item {\bf n=1}. Trivially, because ${(\so \cup \vis )} \subseteq  (\so  \cup \vis)$.
	    \item {\bf n=k+1}. By definition of composition, 
	    ${(\so \cup \vis ) ^{n} }  = {(\so \cup \vis ) ^{k};(\so \cup \vis )}$. Moreover, 
	    $(\so \cup \vis ) ^{k}$ can be partitioned in two sets: $\restUO{(\so \cup \vis ) ^{k}}$ and $\restRO{(\so \cup \vis)^k}$.
	    Therefore, 
	    \[(\so \cup \vis ) ^{k};(\so \cup \vis ) = (\restUO{(\so \cup \vis ) ^{k}} \cup\ \restRO{(\so \cup \vis)^k});(\so \cup \vis )\]
	    
	    Note that 	    
	    $\restUR{(\restRO{(\so \cup \vis)^k};(\so\cup\vis))} = \emptyset$. Consequently,
	    \[\restUR{(\so \cup \vis ) ^{n}} = \restUR{(\restUO{(\so \cup \vis)^k};(\so\cup\vis))}\]
	  
	    By inductive hypothesis, $\restUO{(\so \cup \vis ) ^{k}}\ \ \subseteq\ \ (\so\cup\vis)$
	    Hence, 
	    \[\restUR{(\so \cup \vis ) ^{n}}\ \ \subseteq\ \ \restUR{((\so \cup \vis);(\so\cup\vis))}\]
	    
	    By using distributivity of operations over sets, 
	    \[\restUR{(\so \cup \vis ) ^{n}}\subseteq \restUR{(\so;\so)} \cup \restUR{(\so;\vis)} \cup \restUR{(\vis;\so)} \cup \restUR{(\vis;\vis)}\]
	    
	    Note that $\vis;\vis = \emptyset$ because $(\vertice,\vertice[w])\in\vis$ implies $\op(\vertice)\in\updatesets$ and 
	    $\op(\vertice[w])\in\readset$. Moreover $\so;\so = \so$ because of \lemref{lem:so-transitive}. Then,
	    
            \[\restUR{(\so \cup \vis ) ^{n}}\ \ \subseteq\ \ \restUR{\so} \cup \restUR{(\so;\vis)} \cup \restUR{(\vis;\so)}\]

	    It is immediate that $\restUR{\so} \subseteq \so\cup\vis$. Moreover, 
	    $\restUR{(\so;\vis)} \subseteq \vis$ by \lemref{lemma:sovis-in-vis} and
	     $\restUR{(\vis;\so)} \subseteq \vis$ 
	    by \thmref{theorem:monotonic-read}.
	    Hence, $\restUR{(\so \cup \vis ) ^{n}}\subseteq \so\cup\vis$.
	\end{itemize}
	
	The proof follows by noting that
	  $\restUO{(\so \cup \vis )^{n}}\ \
	\subseteq\ \ \so\cup\vis$ for all $n$ implies
	  $\restUO{(\so \cup \vis )^{+}}\ \
	\subseteq\ \ \so\cup\vis$. Moreover, 
	\[\restUR{(\restUO{(\so \cup \vis ) ^{+} })}\ \ \subseteq\ \ \restUR{(\so\cup\vis)}\]
	Note that $\restUR{(\restUO{(\so \cup \vis ) ^{+} })} = \restUR{{(\so \cup \vis ) ^{+} }}$.
	 Consequently,
		\[\restUR{{(\so \cup \vis ) ^{+} }}\ \ \subseteq\ \ \restUR{\so}\cup\ \restUR{\vis}\]
	The proof is completed by noting that $\restUR{\so}\ \ \subseteq\ \ \vis$ by \thmref{theorem:read-my-writes} and 
	$\restUR{\vis} \subseteq \vis$.
\qed
\end{proof}


%\begin{proof} 
%  	The proof follows by induction on the number of union sets between $\textsc{\small{SO}}_R$ and $\textsc{\small{VIS}}$. Then, $\bigcup_{n=1}^{\infty} (\textsc{\small{SO}}_R \ \cup \ \textsc{\small{VIS}} ) ^{n} \cap \ (\mathbb{U}\ \times \ \mathbb{R}) \subseteq \textsc{\small{VIS}}.$
%\begin{itemize}
%   \item{\bf n=0}. This means that $\textsc{\small{SO}}_R$ and $\textsc{\small{VIS}}$ are $\emptyset$, so that $\emptyset \subseteq \emptyset$.
%   \item{\bf n=k+1}. Suppose that we have proved that the number of union sets $< k+1$. Now, we have to prove that: $\forall (a,b) \mid (a,b \in \verticesets \Rightarrow\ (a,b)\ \in\ (\textsc{\small{SO}}_R \ \cup \ \textsc{\small{VIS}} ) ^{k+1} \cap \ (\mathbb{U}\ \times \ \mathbb{R})) \Rightarrow\ (a,b) \in \textsc{\small{VIS}}$.
%	
%Assume $(a, x_1),(x_1, x_2),\ldots(x_{k{-}1}, x_k),(x_{k}, b)$ are relations from $(\textsc{\small{SO}}_R \ \cup \ \textsc{\small{VIS}} ) ^{k+1}$. Then $(a, x_1),(x_1, x_2),\ldots(x_{k{-}1}, x_k)$ are relations from $(\textsc{\small{SO}}_R \ \cup \ \textsc{\small{VIS}} ) ^{k}$. By the induction hypothesis, $(a, x_k) \in\ (\textsc{\small{SO}}_R \ \cup \ \textsc{\small{VIS}} ) ^{k}$, and we also have $(x_k, b) \in\ (\textsc{\small{SO}}_R \ \cup \ \textsc{\small{VIS}} )$. Thus by the definition of $(\textsc{\small{SO}}_R \ \cup \ \textsc{\small{VIS}} ) ^{k+1}$, $(a, b) \in
%(\textsc{\small{SO}}_R \ \cup \ \textsc{\small{VIS}} ) ^{k+1}$.
%Conversely, assume $(a,b) \in\ (\textsc{\small{SO}}_R \ \cup \ \textsc{\small{VIS}} ) ^{k+1}$ = $(\textsc{\small{SO}}_R \ \cup \ \textsc{\small{VIS}} ) ^{k} \circ (\textsc{\small{SO}}_R \ \cup \ \textsc{\small{VIS}})$. Then there is a vertex $c \in \verticesets$ such
%that $(a,c) \in\ (\textsc{\small{SO}}_R \ \cup \ \textsc{\small{VIS}} ) ^{k}$ and $(c, b) \in\ (\textsc{\small{SO}}_R \ \cup \ \textsc{\small{VIS}} )$.
%
%We are only interested when $a$ is an update action and $b$ a read action. It is because of the intersection with $(Update \times Read)$. We have two possible cases:
%
%\begin{itemize}
%	\item c is a read action. It means that $(c, b)$ only can be in $\textsc{\small{SO}}_R$ because $\textsc{\small{VIS}}$ requires that $c$ will be an update action. In particular, if  $(c, b) \in\ \textsc{\small{SO}}_R$, they belong to the same client. 
%	
%	\begin{itemize}
%		\item if $(a,c) \in\ \textsc{\small{VIS}}$ and $(c,b) \in\ \textsc{\small{SO}}_R$, by Theorem~\ref{theorem:monotonic-read}, $(a,b) \in\ \textsc{\small{VIS}}$.
%		
%		\item if $(a,c) \in\ \textsc{\small{SO}}_R$ then by Theorem~\ref{theorem:read-my-writes}, $(a,c) \in\ \textsc{\small{VIS}}$- Then, it is analogous to the previous case.
%		
%	\end{itemize}
%	\item c is an update action. It means that $(a, c)$ only can be in $\textsc{\small{SO}}_R$ because $\textsc{\small{VIS}}$ requires that $c$ will be an read action. In particular, if  $(a,c) \in\ \textsc{\small{SO}}_R$, they belong to the same client.
%	
%	\begin{itemize}
%		\item if $(a,c) \in\ \textsc{\small{SO}}_R$ and $(c,b) \in \textsc{\small{VIS}}$, by Theorem Monotonic Writes (Falta probar!), $(a,b) \in\ \textsc{\small{VIS}}$.
%		\item if $(a,c) \in\ \textsc{\small{VIS}}$ ... VER.
%	\end{itemize}
%\end{itemize}
%
%
%	
%	
%\end{itemize}
%\end{proof}


\begin{proof} [Causal Arbitration (\ref{thm:causal-arbitration})]
We prove by induction on $n$ that 
$\restUU{(\so \cup \vis) ^{n}}\  \subseteq\ \arb \cup \restUU{\so}$.

\begin{itemize}
	\item{\bf n=1}. Then, $\restUU{(\so \cup \vis)} = \restUU{\so}\cup \restUU{\vis}$. 
	Since $\restUU{\vis}= \emptyset$, we have
	 \[\restUU{(\so \cup \vis)} =\restUU{\so} \subseteq \arb \cup \restUU{\so}\]
	
	
	\item{\bf n=k+1}.Then
	\[\restUU{((\so \cup \vis) ^{k}; (\so \cup \vis))} = \restUU{((\so \cup \vis) ^{k}; \so)}\]
	because $\restUU{((\so \cup \vis) ^{k};  \vis)} =\emptyset$.
	By using inductive hypothesis, 
		\[\restUU{((\so \cup \vis) ^{k}; \so)}\  \subseteq\ \restUU{((\arb \cup \restUU{\so}); \so)}\]
		\[\restUU{((\so \cup \vis) ^{k}; \so)}\  \subseteq\ \restUU{(\arb;\so)} \cup \restUU{(\so;\so)}\]
   
      By \lemref{lemma:arSO-in-arUso}, $\restUU{(\arb;\so)} = {\arb;\restUU{\so}} \subseteq \arb \cup \restUU{\so}$. Moreover, 
       $\restUU{(\so;\so)} \subseteq \restUU{\so}$ because of \lemref{lem:so-transitive}
	
\end{itemize}
%\[ = (\restUU{(\so \cup \vis) ^{k}};\restUU{ \so} ) \setminus \restUU{\so} \cup \restUR{(\so \cup \vis) ^{k}};\restRU{ \so} ) \setminus \restUU{\so}\]


%	We consider  partitions for $((\so \cup \vis) ^{k}: \restUR{ ((\so \cup \vis) ^{k}}$ and 
%	 \[\restUU{((\so \cup \vis) ^{k}; (\so \cup \vis) \setminus \so)} = \restUU{((\so \cup \vis) ^{k}; \so \setminus \so)}\]
Note that $\restUU{(\so \cup \vis) ^{n}}\  \subseteq\ \arb \cup \restUU{\so}$  implies 
$\restUU{((\so \cup \vis) ^{n} \setminus \so)}\  \subseteq\ \arb$. Moreover, since
$\restUU{((\so \cup \vis) ^{n} \setminus \so)}\  \subseteq\ \arb$ holds for any $n$, we have 
 $\restUU{((\so \cup \vis) ^{+} \setminus \so)}\  \subseteq\ \arb$.
\qed
\end{proof}






%\begin{theorem}[\textsc{Causal Arbitration}]
%
%Let $\textsc{\small{SO}}_R$ the se\cond component of the relation $\textsc{\small{SO}}$, $\textsc{\small{VIS}}$ a visibility relation and $\textsc{\small{AR}}$ an arbitration relation then $(\mathbb{U}\ \times \ \mathbb{U}) \cap \ (\textsc{\small{SO}}_R \ \cup \ \textsc{\small{VIS}} ) ^{+} - \textsc{\small{SO}}_R \subseteq \textsc{\small{AR}}.$ 
%
%\end{theorem}
% 
%
%\begin{proof} 
%The proof follows by induction on the number of union sets between $\textsc{\small{SO}}_R$ and $\textsc{\small{VIS}}$. Then, $(\mathbb{U}\ \times \ \mathbb{U}) \cap \ \bigcup_{n=1}^{\infty} (\textsc{\small{SO}}_R \ \cup \ \textsc{\small{VIS}} ) ^{n} - \textsc{\small{SO}}_R \subseteq \textsc{\small{AR}}.$
%
%
%\begin{itemize}
%   \item{\bf n=0}. This means that $\textsc{\small{SO}}_R$, $\textsc{\small{VIS}}$ and  $\textsc{\small{AR}}$ are $\emptyset$, so that $\emptyset \subseteq \emptyset$.
%   \item{\bf n=k+1}. Suppose that we have proved that the number of union sets $< k+1$. Now, we have to prove that: $\forall (a,b) \mid (a,b \in \verticesets \Rightarrow\ (a,b)\ \in\ (\mathbb{U}\ \times \ \mathbb{U}) \cap \ \bigcup_{n=1}^{\infty} (\textsc{\small{SO}}_R \ \cup \ \textsc{\small{VIS}} ) ^{k+1} - \textsc{\small{SO}}_R \Rightarrow\ (a,b) \in \textsc{\small{AR}}$. 
%
%Assume $(a,b) \in\ (\textsc{\small{SO}}_R \ \cup \ \textsc{\small{VIS}} ) ^{k+1} $ = $(\textsc{\small{SO}}_R \ \cup \ \textsc{\small{VIS}} ) ^{k} \circ (\textsc{\small{SO}}_R \ \cup \ \textsc{\small{VIS}})$. Then there is a vertex $c \in \verticesets$ such
%that $(a,c) \in\ (\textsc{\small{SO}}_R \ \cup \ \textsc{\small{VIS}} ) ^{k}$ and $(c, b) \in\ (\textsc{\small{SO}}_R \ \cup \ \textsc{\small{VIS}} )$.
%
%We are only interested when $a$ and $b$ are an update actions. It is because of the intersection with $(Update \times Update)$. We have two possible cases:
%\begin{itemize}
%	\item c is a read action. It means that $(c,b)$ only can be in $\textsc{\small{SO}}_R$ because $\textsc{\small{VIS}}$ requires that $c$ will be an update action. In particular, if  $(c,b) \in\ \textsc{\small{SO}}_R$, they belong to the same client. COMPLETAR
%	
%	\item c is an update action. COMPLETAR
%\end{itemize}
%\end{itemize}
%\qed
%\end{proof}


\begin{proof} [Consistent Prefix (\ref{thm:consistent-prefix})]
We proceed by induction on the length of the derivation 
$\environment{\systemterm}{\emptyset}{\emptyset}{\emptyset}{\emptyset}[\emptyset] \tr{} ^n\ \environment{\systemterm[']}{\op}{\so}{\vis}{\arb}$.

\begin{itemize}
	    \item {\bf n=0}. Trivially, because $\arb=\vis=\emptyset$.
    \item{\bf n=k+1}. Then 
      \[\environment{\systemterm}{\emptyset}{\emptyset}{\emptyset}{\emptyset}[\emptyset] 
         \tr{} ^k\ 
         \environment{\systemterm['']}{\op'}{\so'}{\vis'}{\arb'}[\sse'] 
         \arro{\lambda}[\cid]
 	\environment{\systemterm[']}{\op}{\so}{\vis}{\arb}
	 \]
	 
	By inductive hypothesis, $\arb';(\vis'\setminus\sse')\  \subseteq\ \vis'$. We proceed by 
case analysis on the inference rule applied for the last transition:

	\begin{itemize}
      
        \item rule \ruleName{a-read}. Then, $\arb=\arb'$, $\so = \so' \cup (\sse'(\cid)\times\{\vertice\})$, $\op(\vertice)\in\readset$,
        $\sse = \sse'\upd \cid {\sse'(\cid)\cup\{\vertice\}}$, and
        		 $\vis = \vis' \cup \vis_{\mbox{\sf new}}$ with $\vis_{\mbox{\sf new}} =   (\{\vertice[w] \ |\  \udec[@][{\vertice[w]}] \in \queuemessage[0..\tknown_{\,\cid}-1] \cdot \tpending_{\cid} \cdot \ttransactionbuffer_{\,\cid} \}\times\{\vertice\})$.
        Hence, 
        \[\arb;(\vis\setminus\sse) = \arb';(\vis'\setminus\sse) \cup \arb';(\vis_{\mbox{\sf new}}\setminus\sse)\]
        Note that $(\vis'\setminus\sse) =  (\vis'\setminus\sse')$ because $\vertice$ is fresh, and hence it does not appear in $\vis'$. 
        By inductive hypothesis,    $\arb';(\vis'\setminus\sse) \subseteq \vis'\subseteq \vis$. Therefore, it remains
        to prove that $\arb';(\vis_{\mbox{\sf new}}\setminus\sse) \subseteq \vis$. For all $(\vertice[x],\vertice[y])$ in $\arb';(\vis_{\mbox{\sf new}}\setminus\sse)$,
        there exists $\vertice[z]$ s.t. $(\vertice[x],\vertice[z])\in\arb'$ and  $(\vertice[z],\vertice[y])\in(\vis_{\mbox{\sf new}}\setminus\sse)$. Moreover,
        $\vertice[y] = \vertice$ because $(\vertice[z],\vertice[y])\in\vis_{\mbox{\sf new}}$ and  $\vertice[y]\not\in\sse(\cid)$ because
         $(\vertice[z],\vertice[y])\notin\sse$. By \lemref{lemma:update-ever-belong-different-client}, there exists $\anupd_0$ s.t.
         $\anupd_0^{\vertice[z]}\in\queuemessage[0..\tknown_{\,\cid}-1]$. By \lemref{lem:server-reflect-arb}
         there exists $\anupd_1$ s.t. $\anupd_1^{\vertice[x]}\in\queuemessage[0..\tknown_{\,\cid}-2]$. 
         Consequently, $(\vertice[x],\vertice[y]) \in \vis_{\mbox{\sf new}} \subseteq \vis$.   
        
        
        

 

        \item rule \ruleName{a-update}. $\so = \so'\cup (\sse(\cid)\times\{\vertice\})$, $\op(\vertice)\in\updatesets$, 
        $\vis'=\vis$, $\sse = \sse'\upd \cid {\sse(\cid)\cup\{\vertice\}}$, and
        $\arb = \arb' \cup \arb_{\mbox{\sf new}}$ with  $\arb_{\mbox{\sf new}}= (\{ \vertice[w] \ |\ \udec[@][{\vertice[w]}]\in \queuemessage' \}\times\{\vertice\})$.  
	Hence, 
        \[\arb;(\vis\setminus\sse) =  (\arb' \cup \arb_{\mbox{\sf new}});(\vis'\setminus\sse) =  \arb' ;(\vis'\setminus\sse)\cup \arb_{\mbox{\sf new}};(\vis'\setminus\sse) \]
	By inductive hypothesis,    $\arb';(\vis'\setminus\sse) \subseteq \vis'= \vis$. It remains
        to prove that $\arb_{\mbox{\sf new}};(\vis'\setminus\sse) \subseteq \vis$. By definition, if $(\vertice[x],\vertice[y])\in\arb_{\mbox{\sf new}}$,
        then $\vertice[y]=\vertice$, which is fresh and does not appear in $\vis'$. Therefore, $\arb_{\mbox{\sf new}};(\vis'\setminus\sse) = \emptyset \subseteq \vis$.
       	Hence, 
        \[\arb;(\vis\setminus\sse) =  (\arb' \cup \arb_{\mbox{\sf new}});(\vis'\setminus\sse) =  \arb' ;(\vis'\setminus\sse)\cup \arb_{\mbox{\sf new}};(\vis'\setminus\sse) \]

    
       \item rule \ruleName{a-arb}. Then, $\queuemessage'= \queuemessage''\cdot\ublock[{\udec[{\anupd[_0]}][{\vertice[@][_0]}]}\cdots{\udec[{\anupd[_n]}][{\vertice[@][_n]}]}]$, $\vis = \vis'$, $\sse = \sse'$  and
        $\arb = \arb' \cup \arb_{\mbox{\sf new}}$ with 
         $\arb_{\mbox{\sf new}}= \{ (\vertice, \vertice[w])\ \ |\ \  \vertice[w]\in\dom{\op}, \ \op(\vertice[w])\in\updatesets, \
		     \forall\anupd_0. \anupd_0^{\vertice[w]}\not\in\queuemessage\cdot \ublock[{\udec[{\anupd[_0]}][{\vertice[@][_0]}]}\cdots{\udec[{\anupd[_n]}][{\vertice[@][_n]}]}]\} \cup\  \{(\vertice[@][_i],\vertice[@][_j]) \ |\ \vertice[@][_i],\vertice[@][_j]\in \{\vertice[@][_0],\ldots,\vertice[@][_n]\}, i<j\}$.
		     
    	Hence, 
        \[\arb;(\vis\setminus\sse) =  (\arb' \cup \arb_{\mbox{\sf new}});(\vis'\setminus\sse) =  \arb' ;(\vis'\setminus\sse)\cup \arb_{\mbox{\sf new}};(\vis'\setminus\sse) \]
     Then,         
        \[\arb;\restUU{\so} = \arb';\restUU{\so'}\  \cup\ \arb_{\mbox{\sf new}};\restUU{\so'} 
        \]
        \[\arb;(\vis\setminus\sse) =  (\arb' \cup \arb_{\mbox{\sf new}});(\vis'\setminus\sse) =  \arb' ;(\vis'\setminus\sse)\cup \arb_{\mbox{\sf new}};(\vis'\setminus\sse) \]
	By inductive hypothesis,    $\arb';(\vis'\setminus\sse) \subseteq \vis'= \vis$. It remains
        to prove that $\arb_{\mbox{\sf new}};(\vis'\setminus\sse) \subseteq \vis$.
        If $(\vertice[x],\vertice[y])\in\arb_{\mbox{\sf new}}$ then $\vertice[x]=\vertice$ and
        \begin{equation}
        \label{eq:cst-pref-arb-a}
        \forall\anupd_0. \anupd_0^{\vertice[y]}\not\in\queuemessage''\cdot \ublock[{\udec[{\anupd[_0]}][{\vertice[@][_0]}]}\cdots{\udec[{\anupd[_n]}][{\vertice[@][_n]}]}]
        \end{equation}
        On the other hand, $(\vertice[y],\vertice[z])\in (\vis'\setminus\sse)$ implies
         $\vertice[y]\in\sse(\cid_0)$
         and $\vertice[z]\in\sse(\cid_1)$ with
        $\cid_0\neq\cid_1$. By \lemref{lemma:update-ever-belong-different-client},
        \begin{equation}
        \label{eq:cst-pref-arb-b}
        \exists\anupd_0. \anupd_0^{\vertice[y]}\in \queuemessage''\cdot\ublock[{\udec[{\anupd[_0]}][{\vertice[@][_0]}]}\cdots{\udec[{\anupd[_n]}][{\vertice[@][_n]}]}]
        \end{equation}
        From \equref{eq:cst-pref-arb-a} and  \equref{eq:cst-pref-arb-b}, we conclude that 
        $\arb_{\mbox{\sf new}};(\vis'\setminus\sse) = \emptyset \subseteq \vis$.

       \item rule \ruleName{a-int} follows straightforwardly.
       \qed
    \end{itemize}
    \end{itemize}
	
\end{proof}


