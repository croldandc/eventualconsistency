% !TEX root = main.tex

\section{Proof of results in \secref{sec:transactions}}

\begin{definition}
A program $P$ is {\em free from asynchronous updates} if either
\begin{enumerate}
\item  no subterm of the form $\updcmd$ occurs in $P$; or 
\item  $P=\updcmd;\pushcmd;\waitcmd;P'$ with $P'$ free from asynchronous updates.
\end{enumerate}

$P$ is {\em engaged in an update} if for some $P'$ free from asynchronous updates 
one of the following conditions hold:
\begin{enumerate}
\item $P = \pushcmd;\waitcmd;P'$;
\item $P = \waitcmd;P'$; 
\item $P = {\tconfirmedins{x}[{\tguarded[x][\pullcmd;\waitcmd;P'][P']}]}$;
\item $P = \tguarded[x][\pullcmd;\waitcmd;P'][P']$;
\item $P = \pullcmd;\waitcmd;P' $  
\end{enumerate}

A system $\systemterm =  \Absclient_0\bigpar \ldots\bigpar \Absclient_m\bigpar\queuemessage$
is free from asynchronous update if for all  $l\in \{0,\ldots,m\}$, 
$\Absclient_l = \tclienti{\tprogram_l}{\tknown_l}{\tpending_l}{\ttransactionbuffer_l}{\tsent_l}{\treceivebuffer_l}[\cid_l][\trounds_l]$ 
 the following conditions holds
 \begin{enumerate}
  \item $\tpending_l\cdot{\ttransactionbuffer_l} = \epsilon$ and $P_l$ is free from asynchronous updates.
  \item $\tpending_l\cdot{\ttransactionbuffer_l} \neq \epsilon$ and $P_l$ is engaged in an update.
\end{enumerate}
\end{definition}
 

\begin{lemma}\label{lemma:empty_queue} 
Let $\systemterm$  be a \gsp\ system free from asynchronous update. If $\systemterm\arroi{\lambda}\systemterm[']$ 
then $\systemterm[']$ if free from asynchronous update. 
\end{lemma}

\begin{proof}The proof follows by case analysis on the applied rule.
\end{proof}

\begin{corollary}\label{corollary:async-emptyqueue} Let $\systemterm$  be a \gsp\ system free from asynchronous update. If $\systemterm\arroi{\readtran{\aread}}\systemterm[']$,
the $\Absclient_{\cid}= \tclienti{\tprogram_{\cid}}{\tknown_{\cid}}{\tpending_{\cid}}{\ttransactionbuffer_{\cid}}{\tsent_{\cid}}{\treceivebuffer_{\cid}}[\cid][\trounds_{\cid}]$ and
$\tpending_{\cid}\cdot{\ttransactionbuffer_{\cid}} = \epsilon$.
\end{corollary}



%First, we prove a useful lemma: 
%
%\begin{lemma}\label{lemma:empty_queue} 
%
%Let $\systemterm$ a \gsp\ system. If $\systemterm\Arro^*\systemterm[']$ then $\systemterm[']\Arro^*\systemterm['']$, where
%\[\begin{array}{r@{=}l}
%					\systemterm &  \tclienti{\tprogram^0}{\tknown}{\tpending}{\ttransactionbuffer}{\tsent}{\treceivebuffer} \ \bigpar \ \systemterm[^0]
%					\\
%					\systemterm['] & \tclienti{\tsyncupd{u};\treadins{x}{r}}{\tknown}{\tpending}{\ttransactionbuffer}{\tsent}{\treceivebuffer}  \ \bigpar \   \systemterm[^1]	\\
%										\systemterm[''] & \tclienti{\treadins{x}{r}}{\tknown}{\emptyset}{\epsilon}{\tsent}{\treceivebuffer} \ \bigpar \   \systemterm[^2]	
%
%				  \end{array}
%				\]				
%
%
%\end{lemma}
%
%\begin{proof} Confirmed is evaluated to true when $\tpending$ and [$\ttransactionbuffer$] are empty so that, after the loop, the read operation only return values which belong to the $\queuemessage$.
%\end{proof}	
%
%\begin{lemma} $\ovis = \roarb$ iff 
%
%\begin{enumerate}
%   \item $\roarb;\ovis \subseteq \ovis$; and
%   \item $\neg\ovis;\neg\roarb \subseteq \neg\ovis$.
%\end{enumerate}
%\end{lemma}
%
%\begin{proof} $\Rightarrow$) 
%\begin{enumerate}
%\item
%	\[ \begin{array}{l@{\ =\ }l@{\qquad}l}
%		\roarb;\ovis &  \roarb;\roarb & \roarb =\ovis\\
%		& \roarb & \roarb \mbox{is a partial order (i.e., transitive)}\\
%		&  \ovis & \roarb =\ovis
%   	\end{array}
%	\]
%\item Note that  $\roarb;\ovis =\ovis$ implies $\neg(\roarb;\ovis) =\neg\ovis$. By property of complement and composition, $\neg(\roarb;\ovis) =\neg\ovis;\neg\roarb =\neg\ovis$. 
%\end{enumerate}
%
%$\Leftarrow)$ We divided the proof in two implications
% \begin{enumerate}
%    \item  $\roarb;\ovis \subseteq \ovis$ implies $\ovis\subseteq \roarb$. We proceed by contradiction. Assume $(a,b)\in\ovis$ but $(a,b)\not\in\roarb$.
%    Since $\roarb$ is a total order, $(b,a)\in\roarb$. By assumption, $(b,a)\in\roarb$ and $(a,b)\in\ovis$ implies $(b,b)\in\ovis$, but this is in contradiction 
%    with the fact that $\ovis$ is acyclic. 
%        
%    \item $\neg\ovis;\neg\roarb \subseteq \neg\ovis$ implies $\roarb\subseteq\ovis$. We proceed by contradiction. 
%    Assume $(a,b)\in\roarb$ and $(a,b)\not\in\ovis$, and hence,  $(a,b)\in\neg\ovis$. Since,
%    $\roarb$ is a strict total order $(b,a)\not\in\roarb$. Hence, $(b,a)\in\neg\roarb$. By assumption, $(a,b)\in\neg\ovis$ and $(b,a)\in\neg\roarb$ implies $(a,a)\in\neg\ovis$.
%    Again, by assumption, 
% \end{enumerate}
%
%\end{proof}

	


%\begin{lemma}\label{lemma:empty_queue} 
%Let $\systemterm$  be a \gsp\ system free from asynchronous update. If $\systemterm\arroi{\lambda}\systemterm[']$ 
%then $\systemterm[']$ if free from asynchronous update. 
%\end{lemma}
%\chnote{esto no se si es viejo. Creo que no se usa.}

\begin{lemma}\label{lemma:vertice_arb_store} 

Let $\systemterm$ be a \gsp\ system free from asynchronous update and%For all  $\systemterm[']$ and $u\in\updatesets$, 
\linebreak $\environment{\systemterm}{\emptyset}{\emptyset}{\emptyset}{\emptyset}[\emptyset] \tr{} ^*\ \environment{\systemterm[']}{\op}{\so}{\vis}{\arb}$.
If $(\vertice,\vertice[w])\in\arb$ and $\anupd[^{\vertice[w]}] \in \queuemessage[0 .. \tknown_{\cid} - 1]$ then $\exists \anupd[_0]$ s.t. $\anupd[_0]^{\vertice} \in \queuemessage[0 .. \tknown_{\cid}-1]$.
\end{lemma}
\begin{proof}
We proceed by induction on the length of the derivation 
$\environment{\systemterm}{\emptyset}{\emptyset}{\emptyset}{\emptyset}[\emptyset] \tr{} ^n\ \environment{\systemterm[']}{\op}{\so}{\vis}{\arb}$.

\begin{itemize}
	    \item {\bf n=0}. Trivially, because $\arb=\vis$ and $\queuemessage=\epsilon$.
    \item{\bf n=k+1}. Then 
      \[\environment{\systemterm}{\emptyset}{\emptyset}{\emptyset}{\emptyset}[\emptyset] 
         \tr{} ^k\ 
         \environment{\systemterm['']}{\op'}{\so'}{\vis'}{\arb'}[\sse'] 
         \arro{\lambda}[\cid]
 	\environment{\systemterm[']}{\op}{\so}{\vis}{\arb}
	 \]
	 
	By inductive hypothesis, If $(\vertice,\vertice[w])\in\arb'$ and $\anupd[^{\vertice[w]}] \in \queuemessage'[0 .. \tknown_{\cid}' - 1]$ then $\exists \anupd[_0]$ s.t. $ \anupd[_0]^{\vertice} \in \queuemessage[0 .. \tknown_{\cid}'-1]$
. We proceed by 
case analysis on the inference rule applied for the last transition:

	\begin{itemize}
      
        \item rule \ruleName{a-read}. Follow immediately by inductive hypothesis since $\arb = \arb'$.      
       
        \item rule \ruleName{a-update}. Follow immediately by inductive hypothesis since $\vertice[w] \in \queuemessage[0 .. |\queuemessage| - 1]$.     
        
       \item rule \ruleName{a-arb}. $\arb = \arb' \cup \arb_{\mbox{\sf new}}$ with 
         $\arb_{\mbox{\sf new}}= \{ (\vertice, \vertice[w])\ \ |\ \  \vertice[w]\in\dom{\op}, \ \op(\vertice[w])\in\updatesets, \
		     \forall\anupd_0. \anupd_0^{\vertice[w]}\not\in\queuemessage\cdot \ublock[{\udec[{\anupd[_0]}][{\vertice[@][_0]}]}\cdots{\udec[{\anupd[_n]}][{\vertice[@][_n]}]}]\} \cup\  \{(\vertice[@][_i],\vertice[@][_j]) \ |\ \vertice[@][_i],\vertice[@][_j]\in \{\vertice[@][_0],\ldots,\vertice[@][_n]\}, i<j\}$. 
	By inductive hypothesis $(\vertice,\vertice[w])\in\arb'$, it remains to prove that $(\vertice,\vertice[w])\in\arb_{\mbox{\sf new}}$. There are two cases:
	\begin{itemize}
	\item $(\vertice,\vertice[w])\in \{ (\vertice', \vertice[w'])\ \ |\ \  \vertice[w']\in\dom{\op}, \ \op(\vertice[w])\in\updatesets, \
		     \forall\anupd_0. \anupd_0^{\vertice[w']}\not\in\queuemessage\cdot \ublock[{\udec[{\anupd[_0]}][{\vertice[@][_0]}]}\cdots{\udec[{\anupd[_n]}][{\vertice[@][_n]}]}]\} $. It is immediate since $\anupd_0^{\vertice[w']}\not\in\queuemessage$.
	\item $(\vertice,\vertice[w])\in \{(\vertice[@][_i],\vertice[@][_j]) \ |\ \vertice[@][_i],\vertice[@][_j]\in \{\vertice[@][_0],\ldots,\vertice[@][_n]\}, i<j\}$. It is immediate.
		    \end{itemize}
     
       \item rule \ruleName{a-int} follows straightforwardly.
       \qed
    \end{itemize}
    \end{itemize}
	
\end{proof}
%\chnote{esto sale todo por hipotesis inductiva. Los casos arb son inmediatos. No se si vale la pena poner la prueba}

\begin{lemma}\label{lemma:vertice_vis_store} 
Let $\systemterm$ be a \gsp\ system free from asynchronous update and%For all  $\systemterm[']$ and $u\in\updatesets$, 
\linebreak $\environment{\systemterm}{\emptyset}{\emptyset}{\emptyset}{\emptyset}[\emptyset] \tr{} ^*\ \environment{\systemterm[']}{\op}{\so}{\vis}{\arb}$.
If $(\vertice,\vertice[w])\in\vis$ then $\anupd[^{\vertice}] \in \queuemessage$.
\end{lemma}
\begin{proof}
The proof follows by case analysis on the applied rule. The only interesting rule is \ruleName{a-read}. We rely on \corref{corollary:async-emptyqueue} to conclude that $\tpending_{\cid}\cdot{\ttransactionbuffer_{\cid}} = \epsilon$, hence, $\anupd[^{\vertice}] \in \queuemessage$.
\end{proof}

\subsubsection{Proof of Theorem Single Order \ref{thm:single-order}}

We proceed by induction on the length of the derivation 
$\environment{\systemterm}{\emptyset}{\emptyset}{\emptyset}{\emptyset}[\emptyset] \tr{} ^n\ \environment{\systemterm[']}{\op}{\so}{\vis}{\arb}$.

\begin{itemize}
	    \item {\bf n=0}. Trivially, because $\arb=\vis=\emptyset$.
    \item{\bf n=k+1}. Then 
      \[\environment{\systemterm}{\emptyset}{\emptyset}{\emptyset}{\emptyset}[\emptyset] 
         \tr{} ^k\ 
         \environment{\systemterm['']}{\op'}{\so'}{\vis'}{\arb'}[\sse'] 
         \arro{\lambda}[\cid]
 	\environment{\systemterm[']}{\op}{\so}{\vis}{\arb}
	 \]
	 
	By inductive hypothesis, $\arb';\vis' \subseteq \vis'$ and $\arb'^{-1};\neg\vis'\subseteq\neg\vis'$. We proceed by 
case analysis on the inference rule applied for the last transition:

	\begin{itemize}
      
        \item rule \ruleName{a-read}. Then, $\arb=\arb'$, $\arb^{-1}=\arb'^{-1}$,
        		 $\vis = \vis' \cup \vis_{\mbox{\sf new}}$ and $\neg\vis=\neg(\vis'\cup\vis_{\mbox{\sf new}}) = (\neg\vis'\cap\neg\vis_{\mbox{\sf new}})$.
         
        \begin{itemize}
        \item $\arb;\vis \subseteq \vis$
        
        Hence, 
        \[\arb;\vis = \arb';(\vis'\cup\vis_{\mbox{\sf new}}) =  \arb';\vis'\cup \arb';\vis_{\mbox{\sf new}} \]
        	By inductive hypothesis,    $\arb';\vis' \subseteq \vis' \subseteq \vis$. It remains
        to prove that $\arb';\vis_{\mbox{\sf new}} \subseteq \vis$. 
        By \corref{corollary:async-emptyqueue} $\tpending\cdot{\ttransactionbuffer} = \epsilon$, hence, $(\vertice[w],\vertice) \in \vis_{\mbox{\sf new}}$
        implies $\anupd[^{\vertice[w]}] \in \queuemessage[0..\tknown-1]$. Then, if  $(\vertice[w],\vertice[w']) \in \arb$, by \lemref{lemma:vertice_arb_store}, $\anupd[{\vertice[w']}] \in \queuemessage[0..\tknown-1]$.
        Therefore, $(\vertice[w'],\vertice) \in \vis_{\mbox{\sf new}} \subseteq \vis$ .
	
	\item $\arb^{-1};\neg\vis\subseteq\neg\vis$
	
        Hence, 
        \[\arb^{-1};\neg\vis= (\arb^{-1};\neg\vis')\cap(\arb^{-1};\vis_{\mbox{\sf new}})\subseteq \vis_{\mbox{\sf new}} \subseteq \vis \]

	\end{itemize}
        
       
        \item rule \ruleName{a-update}. $\vis = \vis'$ and 
        $\arb = \arb' \cup \arb_{\mbox{\sf new}}$ with  $\arb_{\mbox{\sf new}}= (\{ \vertice[w] \ |\ \udec[@][{\vertice[w]}]\in \queuemessage' \}\times\{\vertice\})$.  
        \begin{itemize}
        \item $\arb;\vis \subseteq \vis$
	
	Hence,
	        \[\arb;\vis = \arb';\vis\cup\arb_{\mbox{\sf new}};\vis' \]

	\end{itemize}
        
        Since $\vertice$ is fresh, $\vertice \notin \vis'$ then $\arb_{\mbox{\sf new}};\vis' = \emptyset$ and $\emptyset \subseteq \vis$. 
        By inductive hypothesis,    $\arb';\vis' \subseteq \vis' \subseteq \vis$.

	\item $\arb^{-1};\neg\vis\subseteq\neg\vis$
	
        Hence, 
	\[\arb^{-1};\neg\vis = (\arb\cup\arb_{\mbox{\sf new}})^{-1};\neg\vis = (\arb^{-1} \cup\arb_{\mbox{\sf new}}^{-1});\vis' \]
	By inductive hypothesis,    $\arb^{-1};\neg\vis'\subseteq\neg\vis'=\vis$. It remains
        to prove that $\arb_{\mbox{\sf new}};\neg\vis \subseteq \vis$. Since $\vertice$ is fresh, $\forall\vertice[w'].(\vertice,\vertice[w'])\notin\vis$ then 
        $(\vertice,\vertice[w'])\in\neg\vis$ and $\arb^{-1};\neg\vis'\subseteq\neg\vis'$.
        
       \item rule \ruleName{a-arb}. $\vis = \vis'$, and
        $\arb = \arb' \cup \arb_{\mbox{\sf new}}$ with 
         $\arb_{\mbox{\sf new}}= \{ (\vertice, \vertice[w])\ \ |\ \  \vertice[w]\in\dom{\op}, \ \op(\vertice[w])\in\updatesets, \
		     \forall\anupd_0. \anupd_0^{\vertice[w]}\not\in\queuemessage\cdot \ublock[{\udec[{\anupd[_0]}][{\vertice[@][_0]}]}\cdots{\udec[{\anupd[_n]}][{\vertice[@][_n]}]}]\} \cup\  \{(\vertice[@][_i],\vertice[@][_j]) \ |\ \vertice[@][_i],\vertice[@][_j]\in \{\vertice[@][_0],\ldots,\vertice[@][_n]\}, i<j\}$.
	\begin{itemize}
        \item $\arb;\vis \subseteq \vis$
	
	Hence, 
        \[\arb;\vis =  (\arb' \cup \arb_{\mbox{\sf new}});\vis = \arb';\vis \cup \arb_{\mbox{\sf new}};\vis' \]
	If $(\vertice[w],\vertice) \in \arb_{\mbox{\sf new}}$ implies that $\vertice\notin\queuemessage$ by \lemref{lemma:vertice_vis_store}. There is no $(\vertice,\vertice') \in \vis$ then $\arb_{\mbox{\sf new}};\vis = \emptyset$
		
	\item $\arb^{-1};\neg\vis\subseteq\neg\vis$
	
	The case follows analogously to the previous case, we have to prove that  $\arb_{\mbox{\sf new}};\neg\vis \subseteq \vis$. There are two cases:
	\begin{itemize}
		\item $(\vertice,\vertice[w]) \in \{ (\vertice, \vertice[w])\ \ |\ \  \vertice[w]\in\dom{\op}, \ \op(\vertice[w])\in\updatesets, \
		     \forall\anupd_0. \anupd_0^{\vertice[w]}\not\in\queuemessage\cdot\anupdseq\}$ with $\anupdseq = \ublock[{\udec[{\anupd[_0]}][{\vertice[@][_0]}]}\cdots{\udec[{\anupd[_n]}][{\vertice[@][_n]}]}]$ then $(\vertice[w],\vertice) \in \arb^{-1}$ implies $\forall \vertice[w'].(\vertice[w],\vertice[w'])\notin\vis$ then $(\vertice,\vertice[w]) \in \vis$. 
		     Hence $\arb_{\mbox{\sf new}}';\neg\vis\subseteq \neg\vis$ trivially holds.
		
		\item $(\vertice,\vertice[w]) \in \{(\vertice[@][_i],\vertice[@][_j]) \ |\ \vertice[@][_i],\vertice[@][_j]\in \{\vertice[@][_0],\ldots,\vertice[@][_n]\}, i<j\}$. Then $\vertice[w]\notin\queuemessage$. Hence $\forall \vertice[w'].(\vertice[w],\vertice[w]) \in \neg\vis$. Hence, $\arb_{\mbox{\sf new}}^{-1};\neg\vis \subseteq \neg\vis$
	\end{itemize}
%	\chnote{el caso de arb no me termina de convencer.}
	\end{itemize}	     
    	
       \item rule \ruleName{a-int} follows straightforwardly.
    \end{itemize}
    \end{itemize}
\QEDB


